\documentclass[color=usenames,dvipsnames]{beamer}\usepackage[]{graphicx}\usepackage[]{color}
% maxwidth is the original width if it is less than linewidth
% otherwise use linewidth (to make sure the graphics do not exceed the margin)
\makeatletter
\def\maxwidth{ %
  \ifdim\Gin@nat@width>\linewidth
    \linewidth
  \else
    \Gin@nat@width
  \fi
}
\makeatother

\definecolor{fgcolor}{rgb}{0, 0, 0}
\newcommand{\hlnum}[1]{\textcolor[rgb]{0.69,0.494,0}{#1}}%
\newcommand{\hlstr}[1]{\textcolor[rgb]{0.749,0.012,0.012}{#1}}%
\newcommand{\hlcom}[1]{\textcolor[rgb]{0.514,0.506,0.514}{\textit{#1}}}%
\newcommand{\hlopt}[1]{\textcolor[rgb]{0,0,0}{#1}}%
\newcommand{\hlstd}[1]{\textcolor[rgb]{0,0,0}{#1}}%
\newcommand{\hlkwa}[1]{\textcolor[rgb]{0,0,0}{\textbf{#1}}}%
\newcommand{\hlkwb}[1]{\textcolor[rgb]{0,0.341,0.682}{#1}}%
\newcommand{\hlkwc}[1]{\textcolor[rgb]{0,0,0}{\textbf{#1}}}%
\newcommand{\hlkwd}[1]{\textcolor[rgb]{0.004,0.004,0.506}{#1}}%
\let\hlipl\hlkwb

\usepackage{framed}
\makeatletter
\newenvironment{kframe}{%
 \def\at@end@of@kframe{}%
 \ifinner\ifhmode%
  \def\at@end@of@kframe{\end{minipage}}%
  \begin{minipage}{\columnwidth}%
 \fi\fi%
 \def\FrameCommand##1{\hskip\@totalleftmargin \hskip-\fboxsep
 \colorbox{shadecolor}{##1}\hskip-\fboxsep
     % There is no \\@totalrightmargin, so:
     \hskip-\linewidth \hskip-\@totalleftmargin \hskip\columnwidth}%
 \MakeFramed {\advance\hsize-\width
   \@totalleftmargin\z@ \linewidth\hsize
   \@setminipage}}%
 {\par\unskip\endMakeFramed%
 \at@end@of@kframe}
\makeatother

\definecolor{shadecolor}{rgb}{.97, .97, .97}
\definecolor{messagecolor}{rgb}{0, 0, 0}
\definecolor{warningcolor}{rgb}{1, 0, 1}
\definecolor{errorcolor}{rgb}{1, 0, 0}
\newenvironment{knitrout}{}{} % an empty environment to be redefined in TeX

\usepackage{alltt}
%\documentclass[color=usenames,dvipsnames,handout]{beamer}

\usepackage[roman]{../lectures}
%\usepackage[sans]{../lectures}


\hypersetup{pdfpagemode=UseNone,pdfstartview={FitV}}


%\title{Lecture 6 -- Binomial $N$-mixture models: simulation, fitting, and prediction }
%\author{Richard Chandler}


% Load function to compile and open PDF


% Compile and open PDF







%<<knitr-setup,include=FALSE,purl=FALSE>>=
%##opts_chunk$set(comment=NA)
%@


%% New command for inline code that isn't to be evaluated
\definecolor{inlinecolor}{rgb}{0.878, 0.918, 0.933}
\newcommand{\inr}[1]{\colorbox{inlinecolor}{\texttt{#1}}}
\IfFileExists{upquote.sty}{\usepackage{upquote}}{}
\begin{document}




\begin{frame}[plain]
  \LARGE
%  \maketitle
  \centering
  {\LARGE Lecture 6 -- Binomial $N$-mixture models: \\ simulation, fitting, and prediction} \\
  {\color{default} \rule{\textwidth}{0.1pt}}
  \vfill
  \large
  WILD(FISH) 8390 \\
  Estimation of Fish and Wildlife Population Parameters \\
  \vfill
  \large
  Richard Chandler \\
  University of Georgia \\
\end{frame}






\section{Overview}



\begin{frame}[plain]
  \frametitle{Outline}
  \Large
  \only<1>{\tableofcontents}%[hideallsubsections]}
  \only<2 | handout:0>{\tableofcontents[currentsection]}%,hideallsubsections]}
\end{frame}



\begin{frame}
  \frametitle{Overview}
  We're often interested in abundance more than occupancy. \\
  \pause
  \vfill
  Even if you're interested in occupancy, you get it for free by modeling abundance. \\
  \pause
  \vfill
  Occupancy isn't useful if all sites are occupied. \\
  \pause
  \vfill
  Binomial $N$-mixture models are useful in this context if we have
  repeated count data, instead of repeated detection data. 
\end{frame}




\begin{frame}
  \frametitle{Binomial $N$-mixture model}
  \small
  State model (with Poisson assumption)
  \begin{gather*}
    \mathrm{log}(\lambda_i) = \beta_0 + \beta_1 {\color{blue} x_{i1}} +
    \beta_2 {\color{blue} x_{i2}} + \cdots \\
    N_i \sim \mathrm{Poisson}(\lambda_i)
  \end{gather*}
  \pause
  \vfill
  Observation model
  \begin{gather*}
    \mathrm{logit}(p_{ij}) = \alpha_0 + \alpha_1 {\color{blue} x_{i1}}
    + \alpha_2 {\color{Purple} w_{ij}} + \cdots \\
    y_{ij} \sim \mathrm{Binomial}(N_i, p_{ij})
  \end{gather*}
  \pause
  \vfill
  \small
  Definitions \\
  $\lambda_i$ -- Expected value of abundance at site $i$ \\
  $N_i$ -- Realized value of abundance at site $i$ \\
  $p_{ij}$ -- Probability of detecting \alert{an individual} at site $i$ on occasion $j$ \\
  $y_{ij}$ -- Count data \\
%  \vfill
  $\color{blue} x_1$ and $\color{blue} x_2$ -- site covariates \\
%  \vspace{12pt}
  $\color{Purple} w$ -- observation covariate
\end{frame}


\section{Simulation}



\begin{frame}
  \frametitle{Outline}
  \Large
  \tableofcontents[currentsection]
\end{frame}




\begin{frame}[fragile]
  \frametitle{Simulation -- No covariates}
  \small
%  \begin{enumerate}[<+->]
%  \item  
  Abundance
\begin{knitrout}\scriptsize
\definecolor{shadecolor}{rgb}{0.878, 0.918, 0.933}\color{fgcolor}\begin{kframe}
\begin{alltt}
\hlstd{nSites} \hlkwb{<-} \hlnum{100}
\hlstd{nVisits} \hlkwb{<-} \hlnum{4}
\hlkwd{set.seed}\hlstd{(}\hlnum{3439}\hlstd{)}  \hlcom{## Make it reproducible}
\hlstd{lambda1} \hlkwb{<-} \hlnum{2.6}  \hlcom{## Expected value of N}
\hlstd{N1} \hlkwb{<-} \hlkwd{rpois}\hlstd{(}\hlkwc{n}\hlstd{=nSites,} \hlkwc{lambda}\hlstd{=lambda1)}
\end{alltt}
\end{kframe}
\end{knitrout}
% \item
  \pause
  \vfill
  Detection probability and data
\begin{knitrout}\scriptsize
\definecolor{shadecolor}{rgb}{0.878, 0.918, 0.933}\color{fgcolor}\begin{kframe}
\begin{alltt}
\hlstd{p1} \hlkwb{<-} \hlnum{0.3}
\hlstd{y1} \hlkwb{<-} \hlkwd{matrix}\hlstd{(}\hlnum{NA}\hlstd{,} \hlkwc{nrow}\hlstd{=nSites,} \hlkwc{ncol}\hlstd{=nVisits)}
\hlkwa{for}\hlstd{(i} \hlkwa{in} \hlnum{1}\hlopt{:}\hlstd{nSites) \{}
    \hlstd{y1[i,]} \hlkwb{<-} \hlkwd{rbinom}\hlstd{(nVisits,} \hlkwc{size}\hlstd{=N1[i],} \hlkwc{prob}\hlstd{=p1)}
\hlstd{\}}
\end{alltt}
\end{kframe}
\end{knitrout}
%\end{enumerate}
  \pause
  \vfill
  Data and latent abundance
\begin{knitrout}\scriptsize
\definecolor{shadecolor}{rgb}{0.878, 0.918, 0.933}\color{fgcolor}\begin{kframe}
\begin{alltt}
\hlkwd{cbind}\hlstd{(y1, N1)[}\hlnum{1}\hlopt{:}\hlnum{5}\hlstd{,]}
\end{alltt}
\begin{verbatim}
##              N1
## [1,] 2 2 1 0  3
## [2,] 0 0 0 0  1
## [3,] 2 1 2 2  6
## [4,] 0 0 0 0  1
## [5,] 0 0 0 0  0
\end{verbatim}
\end{kframe}
\end{knitrout}
\end{frame}



\begin{frame}[fragile]
  \frametitle{Simulation -- Covariates}
  \small
%  Two continuous covariates and one categorical covariate with 2 levels
%  \vfill
%  \begin{enumerate}[<+->]
%  \item
  Covariates
  \vspace{-6pt}
\begin{knitrout}\scriptsize
\definecolor{shadecolor}{rgb}{0.878, 0.918, 0.933}\color{fgcolor}\begin{kframe}
\begin{alltt}
\hlstd{forest} \hlkwb{<-} \hlkwd{factor}\hlstd{(}\hlkwd{sample}\hlstd{(}\hlkwd{c}\hlstd{(}\hlstr{"Hardwood"}\hlstd{,} \hlstr{"Mixed"}\hlstd{,} \hlstr{"Pine"}\hlstd{), nSites,} \hlkwc{replace}\hlstd{=}\hlnum{TRUE}\hlstd{))}
\hlstd{forestMixed} \hlkwb{<-} \hlkwd{ifelse}\hlstd{(forest}\hlopt{==}\hlstr{"Mixed"}\hlstd{,} \hlnum{1}\hlstd{,} \hlnum{0}\hlstd{)}        \hlcom{## Dummy}
\hlstd{forestPine} \hlkwb{<-} \hlkwd{ifelse}\hlstd{(forest}\hlopt{==}\hlstr{"Pine"}\hlstd{,} \hlnum{1}\hlstd{,} \hlnum{0}\hlstd{)}          \hlcom{## Dummy}
\hlstd{temp} \hlkwb{<-} \hlkwd{matrix}\hlstd{(}\hlkwd{rnorm}\hlstd{(nSites}\hlopt{*}\hlstd{nVisits),} \hlkwc{nrow}\hlstd{=nSites,} \hlkwc{ncol}\hlstd{=nVisits)}
\end{alltt}
\end{kframe}
\end{knitrout}
% \item
\vfill
  Coefficients, $\lambda$, and $p$
  \vspace{-6pt}
\begin{knitrout}\scriptsize
\definecolor{shadecolor}{rgb}{0.878, 0.918, 0.933}\color{fgcolor}\begin{kframe}
\begin{alltt}
\hlstd{beta0} \hlkwb{<-} \hlnum{0}\hlstd{; beta1} \hlkwb{<-} \hlopt{-}\hlnum{1}\hlstd{; beta2} \hlkwb{<-} \hlnum{1}
\hlstd{lambda2} \hlkwb{<-} \hlkwd{exp}\hlstd{(beta0} \hlopt{+} \hlstd{beta1}\hlopt{*}\hlstd{forestMixed} \hlopt{+} \hlstd{beta2}\hlopt{*}\hlstd{forestPine)}
\hlstd{alpha0} \hlkwb{<-} \hlopt{-}\hlnum{2}\hlstd{; alpha1} \hlkwb{<-} \hlnum{1}
\hlstd{p2} \hlkwb{<-} \hlkwd{plogis}\hlstd{(alpha0} \hlopt{+} \hlstd{alpha1}\hlopt{*}\hlstd{temp)}
\end{alltt}
\end{kframe}
\end{knitrout}
% \item
\vfill
  Simulate occupancy and detection data
  \vspace{-6pt}
\begin{knitrout}\scriptsize
\definecolor{shadecolor}{rgb}{0.878, 0.918, 0.933}\color{fgcolor}\begin{kframe}
\begin{alltt}
\hlstd{N2} \hlkwb{<-} \hlkwd{rpois}\hlstd{(nSites,} \hlkwc{lambda}\hlstd{=lambda2)}         \hlcom{## local abundance }
\hlstd{y2} \hlkwb{<-} \hlkwd{matrix}\hlstd{(}\hlnum{NA}\hlstd{,} \hlkwc{nrow}\hlstd{=nSites,} \hlkwc{ncol}\hlstd{=nVisits)}
\hlkwa{for}\hlstd{(i} \hlkwa{in} \hlnum{1}\hlopt{:}\hlstd{nSites) \{}
    \hlstd{y2[i,]} \hlkwb{<-} \hlkwd{rbinom}\hlstd{(nVisits,} \hlkwc{size}\hlstd{=N2[i],} \hlkwc{prob}\hlstd{=p2[i,])}
\hlstd{\}}
\end{alltt}
\end{kframe}
\end{knitrout}
%\end{enumerate}
\end{frame}




\begin{frame}[fragile]
  \frametitle{Simulated data}
  \begin{columns}
    \begin{column}{0.4\textwidth}
      \small
      Observations
%      \tiny
\begin{knitrout}\scriptsize
\definecolor{shadecolor}{rgb}{0.878, 0.918, 0.933}\color{fgcolor}\begin{kframe}
\begin{alltt}
\hlstd{y2[}\hlnum{1}\hlopt{:}\hlnum{20}\hlstd{,]}
\end{alltt}
\begin{verbatim}
##       [,1] [,2] [,3] [,4]
##  [1,]    1    1    0    0
##  [2,]    0    0    3    2
##  [3,]    0    1    0    0
##  [4,]    0    1    0    0
##  [5,]    0    0    0    0
##  [6,]    0    0    0    1
##  [7,]    0    0    0    0
##  [8,]    0    2    0    0
##  [9,]    0    1    0    1
## [10,]    0    1    0    0
## [11,]    0    0    0    1
## [12,]    0    0    1    0
## [13,]    0    0    0    0
## [14,]    0    0    0    0
## [15,]    0    0    0    0
## [16,]    0    0    0    0
## [17,]    0    0    0    0
## [18,]    0    0    0    0
## [19,]    0    0    1    0
## [20,]    0    0    0    1
\end{verbatim}
\end{kframe}
\end{knitrout}
  \end{column}
  \begin{column}{0.6\textwidth}
    \pause
%    \scriptsize
    {\centering Summary stats \\}
    \vspace{24pt}
  Detections at each site \\
\begin{knitrout}\scriptsize
\definecolor{shadecolor}{rgb}{0.878, 0.918, 0.933}\color{fgcolor}\begin{kframe}
\begin{alltt}
\hlcom{# Max count at each site}
\hlstd{maxCounts} \hlkwb{<-} \hlkwd{apply}\hlstd{(y2,} \hlnum{1}\hlstd{, max)}
\hlkwd{table}\hlstd{(maxCounts)}
\end{alltt}
\begin{verbatim}
## maxCounts
##  0  1  2  3 
## 55 33  9  3
\end{verbatim}
\end{kframe}
\end{knitrout}
\pause
\vfill
\small
Proportion of sites known to be occupied \\
\begin{knitrout}\scriptsize
\definecolor{shadecolor}{rgb}{0.878, 0.918, 0.933}\color{fgcolor}\begin{kframe}
\begin{alltt}
\hlstd{naiveOccupancy} \hlkwb{<-} \hlkwd{sum}\hlstd{(maxCounts}\hlopt{>}\hlnum{0}\hlstd{)}\hlopt{/}\hlstd{nSites}
\hlstd{naiveOccupancy}
\end{alltt}
\begin{verbatim}
## [1] 0.45
\end{verbatim}
\end{kframe}
\end{knitrout}

  \end{column}
  \end{columns}
\end{frame}



\section{Prediction}



\begin{frame}
  \frametitle{Outline}
  \Large
  \tableofcontents[currentsection]
\end{frame}



\subsection{Likelihood-based methods}



\begin{frame}[fragile]
  \frametitle{Prepare data in `unmarked'}
  \small
\begin{knitrout}\tiny
\definecolor{shadecolor}{rgb}{0.878, 0.918, 0.933}\color{fgcolor}\begin{kframe}
\begin{alltt}
\hlstd{umf} \hlkwb{<-} \hlkwd{unmarkedFramePCount}\hlstd{(}\hlkwc{y}\hlstd{=y2,} \hlkwc{siteCovs}\hlstd{=}\hlkwd{data.frame}\hlstd{(forest),} \hlkwc{obsCovs}\hlstd{=}\hlkwd{list}\hlstd{(}\hlkwc{temp}\hlstd{=temp))}
\end{alltt}
\end{kframe}
\end{knitrout}
\pause
\begin{knitrout}\tiny
\definecolor{shadecolor}{rgb}{0.878, 0.918, 0.933}\color{fgcolor}\begin{kframe}
\begin{alltt}
\hlkwd{summary}\hlstd{(umf)}
\end{alltt}
\begin{verbatim}
## unmarkedFrame Object
## 
## 100 sites
## Maximum number of observations per site: 4 
## Mean number of observations per site: 4 
## Sites with at least one detection: 45 
## 
## Tabulation of y observations:
##   0   1   2   3 
## 332  52  13   3 
## 
## Site-level covariates:
##       forest  
##  Hardwood:31  
##  Mixed   :32  
##  Pine    :37  
## 
## Observation-level covariates:
##       temp          
##  Min.   :-3.229012  
##  1st Qu.:-0.603218  
##  Median :-0.019813  
##  Mean   :-0.001029  
##  3rd Qu.: 0.663820  
##  Max.   : 3.001200
\end{verbatim}
\end{kframe}
\end{knitrout}
\end{frame}




\begin{frame}[fragile]
  \frametitle{Fit the model}
  \footnotesize
  \inr{pcount} is similar to \inr{occu}, but there's a new argument
  (\texttt{K}) that should be an integer much 
  greater than the highest possible value of local abundance. 
\begin{knitrout}\tiny
\definecolor{shadecolor}{rgb}{0.878, 0.918, 0.933}\color{fgcolor}\begin{kframe}
\begin{alltt}
\hlstd{fm} \hlkwb{<-} \hlkwd{pcount}\hlstd{(}\hlopt{~}\hlstd{temp} \hlopt{~}\hlstd{forest, umf,} \hlkwc{K}\hlstd{=}\hlnum{100}\hlstd{)}
\hlstd{fm}
\end{alltt}
\begin{verbatim}
## 
## Call:
## pcount(formula = ~temp ~ forest, data = umf, K = 100)
## 
## Abundance:
##             Estimate    SE      z  P(>|z|)
## (Intercept)   -0.326 0.366 -0.891 0.372706
## forestMixed   -0.326 0.461 -0.706 0.480137
## forestPine     1.207 0.336  3.595 0.000324
## 
## Detection:
##             Estimate    SE     z  P(>|z|)
## (Intercept)    -1.96 0.313 -6.26 3.79e-10
## temp            1.18 0.205  5.73 9.83e-09
## 
## AIC: 373.6761
\end{verbatim}
\end{kframe}
\end{knitrout}
\pause
\vfill
Compare to actual parameter values:
\begin{knitrout}\tiny
\definecolor{shadecolor}{rgb}{0.878, 0.918, 0.933}\color{fgcolor}\begin{kframe}
\begin{alltt}
\hlkwd{c}\hlstd{(}\hlkwc{beta0}\hlstd{=beta0,} \hlkwc{beta1}\hlstd{=beta1,} \hlkwc{beta2}\hlstd{=beta2);} \hlkwd{c}\hlstd{(}\hlkwc{alpha0}\hlstd{=alpha0,} \hlkwc{alpha1}\hlstd{=alpha1)}
\end{alltt}
\begin{verbatim}
## beta0 beta1 beta2 
##     0    -1     1
## alpha0 alpha1 
##     -2      1
\end{verbatim}
\end{kframe}
\end{knitrout}
\end{frame}




\begin{frame}[fragile]
  \frametitle{Make sure $K$ is high enough}
%  \begin{columns}
%    \begin{column}{0.95\paperwidth}

  Estimates should not change when you increase $K$.
  \vspace{-6pt}
\begin{knitrout}\tiny
\definecolor{shadecolor}{rgb}{0.878, 0.918, 0.933}\color{fgcolor}\begin{kframe}
\begin{alltt}
\hlkwd{round}\hlstd{(}\hlkwd{coef}\hlstd{(fm),} \hlkwc{digits}\hlstd{=}\hlnum{4}\hlstd{)}
\end{alltt}
\begin{verbatim}
##         lam(Int) lam(forestMixed)  lam(forestPine)           p(Int)          p(temp) 
##          -0.3260          -0.3258           1.2073          -1.9615           1.1760
\end{verbatim}
\end{kframe}
\end{knitrout}
\pause
\vfill
  Looks good:
  \vspace{-6pt}
\begin{knitrout}\tiny
\definecolor{shadecolor}{rgb}{0.878, 0.918, 0.933}\color{fgcolor}\begin{kframe}
\begin{alltt}
\hlstd{fm.test} \hlkwb{<-} \hlkwd{pcount}\hlstd{(}\hlopt{~}\hlstd{temp} \hlopt{~}\hlstd{forest, umf,} \hlkwc{K}\hlstd{=}\hlnum{150}\hlstd{)}
\hlkwd{round}\hlstd{(}\hlkwd{coef}\hlstd{(fm.test),} \hlkwc{digits}\hlstd{=}\hlnum{4}\hlstd{)}
\end{alltt}
\begin{verbatim}
##         lam(Int) lam(forestMixed)  lam(forestPine)           p(Int)          p(temp) 
##          -0.3260          -0.3258           1.2073          -1.9615           1.1760
\end{verbatim}
\end{kframe}
\end{knitrout}
\pause
\vfill
If the estimates do change, increase $K$ until they stabilize.
%    \end{column}
%  \end{columns}
\end{frame}



\begin{frame}[fragile]
  \frametitle{\normalsize Empirical Bayes -- Site-level abundance}
\begin{knitrout}\scriptsize
\definecolor{shadecolor}{rgb}{0.878, 0.918, 0.933}\color{fgcolor}\begin{kframe}
\begin{alltt}
\hlstd{re} \hlkwb{<-} \hlkwd{ranef}\hlstd{(fm)}
\hlkwd{plot}\hlstd{(re,} \hlkwc{layout}\hlstd{=}\hlkwd{c}\hlstd{(}\hlnum{4}\hlstd{,}\hlnum{3}\hlstd{),} \hlkwc{subset}\hlstd{=site}\hlopt\hlnum{1}\hlopt{:}\hlnum{12}\hlstd{,} \hlkwc{xlim}\hlstd{=}\hlkwd{c}\hlstd{(}\hlopt{-}\hlnum{1}\hlstd{,} \hlnum{11}\hlstd{),} \hlkwc{lwd}\hlstd{=}\hlnum{5}\hlstd{)}
\end{alltt}
\end{kframe}

{\centering \includegraphics[width=0.8\linewidth]{figure/ranef-1} 

}



\end{knitrout}
\end{frame}





\begin{frame}[fragile]
  \frametitle{Total abundance (in surveyed region)}
\begin{knitrout}\scriptsize
\definecolor{shadecolor}{rgb}{0.878, 0.918, 0.933}\color{fgcolor}\begin{kframe}
\begin{alltt}
\hlstd{N.total.post} \hlkwb{<-} \hlkwd{predict}\hlstd{(re,} \hlkwc{func}\hlstd{=sum,} \hlkwc{nsim}\hlstd{=}\hlnum{1000}\hlstd{)}
\hlkwd{hist}\hlstd{(N.total.post,} \hlkwc{freq}\hlstd{=}\hlnum{FALSE}\hlstd{,} \hlkwc{main}\hlstd{=}\hlstr{""}\hlstd{,} \hlkwc{xlab}\hlstd{=}\hlstr{"N total"}\hlstd{,} \hlkwc{ylab}\hlstd{=}\hlstr{"Probability"}\hlstd{)}
\end{alltt}
\end{kframe}

{\centering \includegraphics[width=0.6\linewidth]{figure/Ntotal-1} 

}



\end{knitrout}
\end{frame}






\begin{frame}[fragile]
  \frametitle{Prediction in `unmarked'}
  \small
  Create \texttt{data.frame} with prediction covariates. 
  \vspace{-6pt}
\begin{knitrout}\footnotesize
\definecolor{shadecolor}{rgb}{0.878, 0.918, 0.933}\color{fgcolor}\begin{kframe}
\begin{alltt}
\hlstd{pred.data} \hlkwb{<-} \hlkwd{data.frame}\hlstd{(}\hlkwc{forest}\hlstd{=}\hlkwd{c}\hlstd{(}\hlstr{"Hardwood"}\hlstd{,} \hlstr{"Mixed"}\hlstd{,} \hlstr{"Pine"}\hlstd{),}
                        \hlkwc{temp}\hlstd{=}\hlnum{0}\hlstd{)}
\end{alltt}
\end{kframe}
\end{knitrout}
\pause
\vfill
Get predictions of $\lambda$ for each row of prediction data.
  \vspace{-6pt}
\begin{knitrout}\footnotesize
\definecolor{shadecolor}{rgb}{0.878, 0.918, 0.933}\color{fgcolor}\begin{kframe}
\begin{alltt}
\hlstd{lambda.pred} \hlkwb{<-} \hlkwd{predict}\hlstd{(fm,} \hlkwc{newdata}\hlstd{=pred.data,}
                       \hlkwc{type}\hlstd{=}\hlstr{'state'}\hlstd{,} \hlkwc{append}\hlstd{=}\hlnum{TRUE}\hlstd{)}
\end{alltt}
\end{kframe}
\end{knitrout}
\pause
\vfill
  View $\lambda$ predictions
  \vspace{-6pt}
\begin{knitrout}\footnotesize
\definecolor{shadecolor}{rgb}{0.878, 0.918, 0.933}\color{fgcolor}\begin{kframe}
\begin{alltt}
\hlkwd{print}\hlstd{(}\hlkwd{head}\hlstd{(lambda.pred),} \hlkwc{digits}\hlstd{=}\hlnum{2}\hlstd{)}
\end{alltt}
\begin{verbatim}
##   Predicted   SE lower upper   forest temp
## 1      0.72 0.26  0.35   1.5 Hardwood    0
## 2      0.52 0.22  0.23   1.2    Mixed    0
## 3      2.41 0.64  1.44   4.0     Pine    0
\end{verbatim}
\end{kframe}
\end{knitrout}
\end{frame}



% \begin{frame}[fragile]
%   \frametitle{Prediction in `unmarked'}
%   \small
%   View $\lambda$ predictions
% <<psi-head,size='footnotesize'>>=
% print(head(lambda.pred), digits=2)
% @
% \end{frame}






\begin{frame}[fragile]
  \frametitle{Prediction in `unmarked'}
\begin{knitrout}\tiny
\definecolor{shadecolor}{rgb}{0.878, 0.918, 0.933}\color{fgcolor}\begin{kframe}
\begin{alltt}
\hlstd{bpx} \hlkwb{<-} \hlkwd{barplot}\hlstd{(lambda.pred}\hlopt{$}\hlstd{Predicted,} \hlkwc{ylab}\hlstd{=}\hlstr{"Expected value of abundance"}\hlstd{,} \hlcom{#col="blue",}
               \hlkwc{ylim}\hlstd{=}\hlkwd{c}\hlstd{(}\hlnum{0}\hlstd{,}\hlnum{3.5}\hlstd{),} \hlkwc{names}\hlstd{=lambda.pred}\hlopt{$}\hlstd{forest,} \hlkwc{xlab}\hlstd{=}\hlstr{"Forest type"}\hlstd{);} \hlkwd{box}\hlstd{()}
\hlkwd{arrows}\hlstd{(bpx, lambda.pred}\hlopt{$}\hlstd{Predicted, bpx, lambda.pred}\hlopt{$}\hlstd{Predicted}\hlopt{+}\hlstd{lambda.pred}\hlopt{$}\hlstd{SE,}
       \hlkwc{angle}\hlstd{=}\hlnum{90}\hlstd{,} \hlkwc{length}\hlstd{=}\hlnum{0.1}\hlstd{)}
\end{alltt}
\end{kframe}

{\centering \includegraphics[width=0.8\linewidth]{figure/pred-psi1-1} 

}



\end{knitrout}
\end{frame}







\begin{frame}
  \frametitle{In-class exercise}
  % \small
  % \begin{enumerate}
  %   \item Predict
  %   \end{enumerate}
  \centering
  Graph $p$ as a function of temperature. Include 95\% confidence intervals. \\
\end{frame}




\subsection{Bayesian methods: posterior prediction}



\begin{frame}[plain]
  \frametitle{Outline}
  \Large
  \tableofcontents[currentsection,currentsubsection]
\end{frame}




\begin{frame}[fragile]
  \frametitle{The BUGS model}
\begin{knitrout}\scriptsize
\definecolor{shadecolor}{rgb}{0.878, 0.918, 0.933}\color{fgcolor}\begin{kframe}
\begin{verbatim}
## model {
## 
## # Priors for occupancy coefficients
## beta0 ~ dnorm(0, 0.5)  # variance=1/0.5
## beta1 ~ dnorm(0, 0.5)
## beta2 ~ dnorm(0, 0.5)
## 
## # Priors for detection coefficients
## alpha0 ~ dnorm(0, 0.5)  
## alpha1 ~ dnorm(0, 0.5)
## 
## for(i in 1:nSites) {
##   log(lambda[i]) <- beta0 + beta1*forestMixed[i] + beta2*forestPine[i]
##   N[i] ~ dpois(lambda[i])         # Latent local abundance
##   for(j in 1:nOccasions) {
##     logit(p[i,j]) <- alpha0 + alpha1*temp[i,j]
##     y[i,j] ~ dbin(p[i,j], N[i])   # Data
##   }
## }
## 
## totalAbundance <- sum(N)
## 
## }
\end{verbatim}
\end{kframe}
\end{knitrout}

\end{frame}





\begin{frame}[fragile]
  \frametitle{Data, inits, and parameters}
  Put data in a named list
  \vspace{-12pt}
\begin{knitrout}\small
\definecolor{shadecolor}{rgb}{0.878, 0.918, 0.933}\color{fgcolor}\begin{kframe}
\begin{alltt}
\hlstd{jags.data} \hlkwb{<-} \hlkwd{list}\hlstd{(}\hlkwc{y}\hlstd{=y2,} \hlkwc{temp}\hlstd{=temp,}
                  \hlkwc{forestMixed}\hlstd{=forestMixed,}
                  \hlkwc{forestPine}\hlstd{=forestPine,}
                  \hlkwc{nSites}\hlstd{=nSites,} \hlkwc{nOccasions}\hlstd{=nVisits)}
\end{alltt}
\end{kframe}
\end{knitrout}
\pause
\vfill
  Initial values
  \vspace{-12pt}
\begin{knitrout}\small
\definecolor{shadecolor}{rgb}{0.878, 0.918, 0.933}\color{fgcolor}\begin{kframe}
\begin{alltt}
\hlstd{jags.inits} \hlkwb{<-} \hlkwa{function}\hlstd{() \{}
    \hlkwd{list}\hlstd{(}\hlkwc{beta0}\hlstd{=}\hlkwd{rnorm}\hlstd{(}\hlnum{1}\hlstd{),} \hlkwc{alpha0}\hlstd{=}\hlkwd{rnorm}\hlstd{(}\hlnum{1}\hlstd{),} \hlkwc{N}\hlstd{=maxCounts)}
\hlstd{\}}
\end{alltt}
\end{kframe}
\end{knitrout}
\pause
\vfill
  Parameters to monitor
  \vspace{-12pt}
\begin{knitrout}\small
\definecolor{shadecolor}{rgb}{0.878, 0.918, 0.933}\color{fgcolor}\begin{kframe}
\begin{alltt}
\hlstd{jags.pars} \hlkwb{<-} \hlkwd{c}\hlstd{(}\hlstr{"beta0"}\hlstd{,} \hlstr{"beta1"}\hlstd{,} \hlstr{"beta2"}\hlstd{,}
               \hlstr{"alpha0"}\hlstd{,} \hlstr{"alpha1"}\hlstd{,} \hlstr{"totalAbundance"}\hlstd{)}
\end{alltt}
\end{kframe}
\end{knitrout}
\end{frame}





\begin{frame}[fragile]
  \frametitle{MCMC}
  \small
\begin{knitrout}\scriptsize
\definecolor{shadecolor}{rgb}{0.878, 0.918, 0.933}\color{fgcolor}\begin{kframe}
\begin{alltt}
\hlkwd{library}\hlstd{(jagsUI)}
\hlstd{jags.post.samples} \hlkwb{<-} \hlkwd{jags.basic}\hlstd{(}\hlkwc{data}\hlstd{=jags.data,} \hlkwc{inits}\hlstd{=jags.inits,}
                                \hlkwc{parameters.to.save}\hlstd{=}\hlkwd{c}\hlstd{(jags.pars,} \hlstr{"N"}\hlstd{),} \hlcom{## NOTE "N"!}
                                \hlkwc{model.file}\hlstd{=}\hlstr{"Nmix-model-covs.jag"}\hlstd{,}
                                \hlkwc{n.chains}\hlstd{=}\hlnum{3}\hlstd{,} \hlkwc{n.adapt}\hlstd{=}\hlnum{100}\hlstd{,} \hlkwc{n.burnin}\hlstd{=}\hlnum{0}\hlstd{,}
                                \hlkwc{n.iter}\hlstd{=}\hlnum{2000}\hlstd{,} \hlkwc{parallel}\hlstd{=}\hlnum{TRUE}\hlstd{)}
\end{alltt}
\begin{verbatim}
## 
## Processing function input....... 
## 
## Done. 
##  
## Beginning parallel processing using 3 cores. Console output will be suppressed.
## 
## Parallel processing completed.
## 
## MCMC took 0.028 minutes.
\end{verbatim}
\end{kframe}
\end{knitrout}
\end{frame}



\begin{frame}[fragile]
  \frametitle{Summarize output}
\begin{knitrout}\tiny
\definecolor{shadecolor}{rgb}{0.878, 0.918, 0.933}\color{fgcolor}\begin{kframe}
\begin{alltt}
\hlkwd{summary}\hlstd{(jags.post.samples[,jags.pars])}
\end{alltt}
\begin{verbatim}
## 
## Iterations = 1:2000
## Thinning interval = 1 
## Number of chains = 3 
## Sample size per chain = 2000 
## 
## 1. Empirical mean and standard deviation for each variable,
##    plus standard error of the mean:
## 
##                    Mean      SD Naive SE Time-series SE
## beta0           -0.3948  0.2944 0.003801        0.02864
## beta1           -0.3543  0.3968 0.005122        0.03468
## beta2            1.1813  0.3063 0.003954        0.02476
## alpha0          -1.8616  0.2481 0.003203        0.04997
## alpha1           1.2205  0.2232 0.002881        0.03982
## totalAbundance 121.6455 22.9389 0.296140        4.43671
## 
## 2. Quantiles for each variable:
## 
##                   2.5%      25%      50%       75%    97.5%
## beta0          -1.0191  -0.5847  -0.3840  -0.20292   0.1777
## beta1          -1.1408  -0.6357  -0.3381  -0.07623   0.3893
## beta2           0.6295   0.9658   1.1736   1.37667   1.8204
## alpha0         -2.2896  -2.0223  -1.8690  -1.72943  -1.3283
## alpha1          0.8375   1.0623   1.2013   1.37039   1.7243
## totalAbundance 84.0000 106.0000 119.0000 134.00000 177.0000
\end{verbatim}
\end{kframe}
\end{knitrout}
\end{frame}


\begin{frame}[fragile]
  \frametitle{Local abundance}
\begin{knitrout}\scriptsize
\definecolor{shadecolor}{rgb}{0.878, 0.918, 0.933}\color{fgcolor}\begin{kframe}
\begin{alltt}
\hlkwd{plot}\hlstd{(jags.post.samples[,}\hlkwd{paste0}\hlstd{(}\hlstr{"N["}\hlstd{,} \hlnum{1}\hlopt{:}\hlnum{4}\hlstd{,} \hlstr{"]"}\hlstd{)])}
\end{alltt}
\end{kframe}

{\centering \includegraphics[width=0.7\linewidth]{figure/localN-1} 

}



\end{knitrout}
\end{frame}



\begin{frame}[fragile]
  \frametitle{Traceplots and density plots}
\begin{knitrout}\footnotesize
\definecolor{shadecolor}{rgb}{0.878, 0.918, 0.933}\color{fgcolor}\begin{kframe}
\begin{alltt}
\hlkwd{plot}\hlstd{(jags.post.samples[,jags.pars[}\hlnum{1}\hlopt{:}\hlnum{3}\hlstd{]])}
\end{alltt}
\end{kframe}

{\centering \includegraphics[width=0.7\textwidth]{figure/bugs-plot1-1} 

}



\end{knitrout}
\end{frame}



\begin{frame}[fragile]
  \frametitle{Traceplots and density plots}
\begin{knitrout}\footnotesize
\definecolor{shadecolor}{rgb}{0.878, 0.918, 0.933}\color{fgcolor}\begin{kframe}
\begin{alltt}
\hlkwd{plot}\hlstd{(jags.post.samples[,jags.pars[}\hlnum{4}\hlopt{:}\hlnum{5}\hlstd{]])}
\end{alltt}
\end{kframe}

{\centering \includegraphics[width=0.7\textwidth]{figure/bugs-plot2-1} 

}



\end{knitrout}
\end{frame}




\begin{frame}[fragile]
  \frametitle{Bayesian prediction}
  \small
  First, extract the $p$ coefficients
  \vspace{-6pt}
\begin{knitrout}\scriptsize
\definecolor{shadecolor}{rgb}{0.878, 0.918, 0.933}\color{fgcolor}\begin{kframe}
\begin{alltt}
\hlstd{p.coef.post} \hlkwb{<-} \hlkwd{as.matrix}\hlstd{(jags.post.samples[,}\hlkwd{c}\hlstd{(}\hlstr{"alpha0"}\hlstd{,}\hlstr{"alpha1"}\hlstd{)])}
\hlkwd{head}\hlstd{(p.coef.post,} \hlkwc{n}\hlstd{=}\hlnum{4}\hlstd{)}
\end{alltt}
\begin{verbatim}
##         alpha0   alpha1
## [1,] -1.536070 1.229850
## [2,] -1.542794 1.238841
## [3,] -1.537215 1.251539
## [4,] -1.599808 1.232748
\end{verbatim}
\end{kframe}
\end{knitrout}
  \pause
  \vfill
  Create prediction matrix, one row for each MCMC iteration.
  \vspace{-6pt}
%  Columns represent covariate values. 
\begin{knitrout}\scriptsize
\definecolor{shadecolor}{rgb}{0.878, 0.918, 0.933}\color{fgcolor}\begin{kframe}
\begin{alltt}
\hlstd{n.iter} \hlkwb{<-} \hlkwd{nrow}\hlstd{(p.coef.post)}
\hlstd{temp.pred} \hlkwb{<-} \hlkwd{seq}\hlstd{(}\hlopt{-}\hlnum{3}\hlstd{,} \hlnum{3}\hlstd{,} \hlkwc{length}\hlstd{=}\hlnum{50}\hlstd{)}
\hlstd{p.post.pred} \hlkwb{<-} \hlkwd{matrix}\hlstd{(}\hlnum{NA}\hlstd{,} \hlkwc{nrow}\hlstd{=n.iter,} \hlkwc{ncol}\hlstd{=}\hlkwd{length}\hlstd{(temp.pred))}
\end{alltt}
\end{kframe}
\end{knitrout}
  \pause
  \vfill
  Predict $p$ for each MCMC iteration.
  \vspace{-6pt}
%  using covariate values from \inr{pred.data}. 
\begin{knitrout}\scriptsize
\definecolor{shadecolor}{rgb}{0.878, 0.918, 0.933}\color{fgcolor}\begin{kframe}
\begin{alltt}
\hlkwa{for}\hlstd{(i} \hlkwa{in} \hlnum{1}\hlopt{:}\hlstd{n.iter) \{}
    \hlstd{p.post.pred[i,]} \hlkwb{<-} \hlkwd{plogis}\hlstd{(p.coef.post[i,}\hlstr{"alpha0"}\hlstd{]} \hlopt{+}
                              \hlstd{p.coef.post[i,}\hlstr{"alpha1"}\hlstd{]}\hlopt{*}\hlstd{temp.pred)}
\hlstd{\}}
\end{alltt}
\end{kframe}
\end{knitrout}
\end{frame}





\begin{frame}[fragile]
  \frametitle{Bayesian prediction}
%  Now with posterior mean and 95\% CI
\begin{knitrout}\tiny
\definecolor{shadecolor}{rgb}{0.878, 0.918, 0.933}\color{fgcolor}\begin{kframe}
\begin{alltt}
\hlkwd{plot}\hlstd{(temp.pred, p.post.pred[}\hlnum{1}\hlstd{,],} \hlkwc{type}\hlstd{=}\hlstr{"l"}\hlstd{,} \hlkwc{xlab}\hlstd{=}\hlstr{"Temperature (standardized)"}\hlstd{,}
     \hlkwc{ylab}\hlstd{=}\hlstr{"Detection probability"}\hlstd{,} \hlkwc{ylim}\hlstd{=}\hlkwd{c}\hlstd{(}\hlnum{0}\hlstd{,} \hlnum{1}\hlstd{),} \hlkwc{col}\hlstd{=}\hlkwd{gray}\hlstd{(}\hlnum{0.8}\hlstd{))}
\hlkwa{for}\hlstd{(i} \hlkwa{in} \hlkwd{seq}\hlstd{(}\hlnum{1}\hlstd{, n.iter,} \hlkwc{by}\hlstd{=}\hlnum{10}\hlstd{)) \{}  \hlcom{## Thin by 10}
    \hlkwd{lines}\hlstd{(temp.pred, p.post.pred[i,],} \hlkwc{col}\hlstd{=}\hlkwd{gray}\hlstd{(}\hlnum{0.8}\hlstd{))  \}}
\hlstd{pred.post.mean} \hlkwb{<-} \hlkwd{colMeans}\hlstd{(p.post.pred)}
\hlstd{pred.post.lower} \hlkwb{<-} \hlkwd{apply}\hlstd{(p.post.pred,} \hlnum{2}\hlstd{, quantile,} \hlkwc{prob}\hlstd{=}\hlnum{0.025}\hlstd{)}
\hlstd{pred.post.upper} \hlkwb{<-} \hlkwd{apply}\hlstd{(p.post.pred,} \hlnum{2}\hlstd{, quantile,} \hlkwc{prob}\hlstd{=}\hlnum{0.975}\hlstd{)}
\hlkwd{lines}\hlstd{(temp.pred, pred.post.mean,} \hlkwc{col}\hlstd{=}\hlstr{"blue"}\hlstd{)}
\hlkwd{lines}\hlstd{(temp.pred, pred.post.lower,} \hlkwc{col}\hlstd{=}\hlstr{"blue"}\hlstd{,} \hlkwc{lty}\hlstd{=}\hlnum{2}\hlstd{)}
\hlkwd{lines}\hlstd{(temp.pred, pred.post.upper,} \hlkwc{col}\hlstd{=}\hlstr{"blue"}\hlstd{,} \hlkwc{lty}\hlstd{=}\hlnum{2}\hlstd{)}
\end{alltt}
\end{kframe}

{\centering \includegraphics[width=0.7\linewidth]{figure/psi-pred-post-meanCI-1} 

}



\end{knitrout}
\end{frame}





\subsection{Bayesian methods: prior prediction}



\begin{frame}[plain]
  \frametitle{Outline}
  \Large
  \tableofcontents[currentsection,currentsubsection]
\end{frame}






\begin{frame}
  \frametitle{Prior sensitivity}
  Questions
  \begin{itemize}
    \item What are the implied priors on $\lambda$ and $p$?
    \item What would be the consequence of changing the priors?
  \end{itemize}
  \pause
  \vfill
  \alert{Prior predictive checks} can help us answer these questions. \\
  \pause
  \vfill
  Prior prediction can be acheived by replacing the data with missing values \\
\end{frame}






\begin{frame}[fragile]
  \frametitle{Prior sampling}
  Replace data with missing values
  \vspace{-12pt}
\begin{knitrout}\small
\definecolor{shadecolor}{rgb}{0.878, 0.918, 0.933}\color{fgcolor}\begin{kframe}
\begin{alltt}
\hlstd{jags.data2} \hlkwb{<-} \hlstd{jags.data}
\hlstd{jags.data2}\hlopt{$}\hlstd{y[]} \hlkwb{<-} \hlnum{NA}      \hlcom{## Replace data with missing values}
\end{alltt}
\end{kframe}
\end{knitrout}
  \pause
  \vfill
  Fit model (Really, we're just sampling from the prior)
\begin{knitrout}\scriptsize
\definecolor{shadecolor}{rgb}{0.878, 0.918, 0.933}\color{fgcolor}\begin{kframe}
\begin{alltt}
\hlstd{jags.prior.samples} \hlkwb{<-} \hlkwd{jags.basic}\hlstd{(}\hlkwc{data}\hlstd{=jags.data2,} \hlkwc{inits}\hlstd{=jags.inits,}
                                 \hlkwc{parameters.to.save}\hlstd{=jags.pars,}
                                 \hlkwc{model.file}\hlstd{=}\hlstr{"Nmix-model-covs.jag"}\hlstd{,}
                                 \hlkwc{n.chains}\hlstd{=}\hlnum{3}\hlstd{,} \hlkwc{n.adapt}\hlstd{=}\hlnum{100}\hlstd{,} \hlkwc{n.burnin}\hlstd{=}\hlnum{0}\hlstd{,}
                                 \hlkwc{n.iter}\hlstd{=}\hlnum{2000}\hlstd{,} \hlkwc{parallel}\hlstd{=}\hlnum{TRUE}\hlstd{)}
\end{alltt}
\begin{verbatim}
## 
## Processing function input....... 
## 
## Done. 
##  
## Beginning parallel processing using 3 cores. Console output will be suppressed.
## 
## Parallel processing completed.
## 
## MCMC took 0.005 minutes.
\end{verbatim}
\end{kframe}
\end{knitrout}
\end{frame}




\begin{frame}[fragile]
  \frametitle{Prior summary}
\begin{knitrout}\tiny
\definecolor{shadecolor}{rgb}{0.878, 0.918, 0.933}\color{fgcolor}\begin{kframe}
\begin{alltt}
\hlkwd{summary}\hlstd{(jags.prior.samples)}
\end{alltt}
\begin{verbatim}
## 
## Iterations = 1:2000
## Thinning interval = 1 
## Number of chains = 3 
## Sample size per chain = 2000 
## 
## 1. Empirical mean and standard deviation for each variable,
##    plus standard error of the mean:
## 
##                      Mean       SD Naive SE Time-series SE
## alpha0          -0.014595    1.407  0.01817        0.01817
## alpha1           0.002061    1.415  0.01827        0.01876
## beta0            0.011140    1.407  0.01816        0.01845
## beta1           -0.005991    1.408  0.01818        0.01818
## beta2            0.032224    1.427  0.01842        0.01874
## totalAbundance 582.423833 1911.608 24.67876       26.41232
## 
## 2. Quantiles for each variable:
## 
##                  2.5%     25%        50%      75%    97.5%
## alpha0         -2.793 -0.9539  -0.017139   0.9193    2.762
## alpha1         -2.779 -0.9556   0.004390   0.9408    2.782
## beta0          -2.796 -0.9281   0.011999   0.9457    2.790
## beta1          -2.745 -0.9542  -0.008274   0.9474    2.770
## beta2          -2.794 -0.9322   0.050079   0.9664    2.810
## totalAbundance  6.000 49.0000 145.000000 428.0000 3824.450
\end{verbatim}
\end{kframe}
\end{knitrout}
\end{frame}






\begin{frame}[fragile]
  \frametitle{Prior prediction}
  \small
  Push the prior samples through the model to predict $p$
\begin{knitrout}\scriptsize
\definecolor{shadecolor}{rgb}{0.878, 0.918, 0.933}\color{fgcolor}\begin{kframe}
\begin{alltt}
\hlstd{p.coef.prior} \hlkwb{<-} \hlkwd{as.matrix}\hlstd{(jags.prior.samples[,}\hlkwd{c}\hlstd{(}\hlstr{"alpha0"}\hlstd{,}\hlstr{"alpha1"}\hlstd{)])}
\hlstd{p.prior.pred} \hlkwb{<-} \hlkwd{matrix}\hlstd{(}\hlnum{NA}\hlstd{,} \hlkwc{nrow}\hlstd{=n.iter,} \hlkwc{ncol}\hlstd{=}\hlkwd{length}\hlstd{(temp.pred))}
\hlkwa{for}\hlstd{(i} \hlkwa{in} \hlnum{1}\hlopt{:}\hlstd{n.iter) \{}
    \hlstd{p.prior.pred[i,]} \hlkwb{<-} \hlkwd{plogis}\hlstd{(p.coef.prior[i,}\hlstr{"alpha0"}\hlstd{]} \hlopt{+}
                               \hlstd{p.coef.prior[i,}\hlstr{"alpha1"}\hlstd{]}\hlopt{*}\hlstd{temp.pred)}
\hlstd{\}}
\end{alltt}
\end{kframe}
\end{knitrout}
\end{frame}





\begin{frame}[fragile]
  \frametitle{Bayesian prediction}
  Compute prior mean and 95\% CI for $p$ predictions
\begin{knitrout}\scriptsize
\definecolor{shadecolor}{rgb}{0.878, 0.918, 0.933}\color{fgcolor}\begin{kframe}
\begin{alltt}
\hlstd{pred.prior.mean} \hlkwb{<-} \hlkwd{colMeans}\hlstd{(p.prior.pred)}
\hlstd{pred.prior.lower} \hlkwb{<-} \hlkwd{apply}\hlstd{(p.prior.pred,} \hlnum{2}\hlstd{, quantile,} \hlkwc{prob}\hlstd{=}\hlnum{0.025}\hlstd{)}
\hlstd{pred.prior.upper} \hlkwb{<-} \hlkwd{apply}\hlstd{(p.prior.pred,} \hlnum{2}\hlstd{, quantile,} \hlkwc{prob}\hlstd{=}\hlnum{0.975}\hlstd{)}
\end{alltt}
\end{kframe}
\end{knitrout}
  \pause
  \vfill
  Show credible regions using shaded polygons. Include a few prior prediction lines.
\begin{knitrout}\scriptsize
\definecolor{shadecolor}{rgb}{0.878, 0.918, 0.933}\color{fgcolor}\begin{kframe}
\begin{alltt}
\hlkwd{plot}\hlstd{(temp.pred, p.post.pred[}\hlnum{1}\hlstd{,],} \hlkwc{type}\hlstd{=}\hlstr{"n"}\hlstd{,} \hlkwc{xlab}\hlstd{=}\hlstr{"Temperature (standardized)"}\hlstd{,}
     \hlkwc{ylab}\hlstd{=}\hlstr{"Detection probability"}\hlstd{,} \hlkwc{ylim}\hlstd{=}\hlkwd{c}\hlstd{(}\hlnum{0}\hlstd{,} \hlnum{1.3}\hlstd{),} \hlkwc{col}\hlstd{=}\hlkwd{gray}\hlstd{(}\hlnum{0.8}\hlstd{))}
\hlkwd{polygon}\hlstd{(}\hlkwc{x}\hlstd{=}\hlkwd{c}\hlstd{(temp.pred,} \hlkwd{rev}\hlstd{(temp.pred)),}
        \hlkwc{y}\hlstd{=}\hlkwd{c}\hlstd{(pred.post.lower,} \hlkwd{rev}\hlstd{(pred.post.upper)),}
        \hlkwc{col}\hlstd{=}\hlkwd{rgb}\hlstd{(}\hlnum{0}\hlstd{,}\hlnum{0}\hlstd{,}\hlnum{1}\hlstd{,}\hlnum{0.5}\hlstd{),} \hlkwc{border}\hlstd{=}\hlnum{NA}\hlstd{)}                   \hlcom{# Post CI}
\hlkwd{polygon}\hlstd{(}\hlkwc{x}\hlstd{=}\hlkwd{c}\hlstd{(temp.pred,} \hlkwd{rev}\hlstd{(temp.pred)),}
        \hlkwc{y}\hlstd{=}\hlkwd{c}\hlstd{(pred.prior.lower,} \hlkwd{rev}\hlstd{(pred.prior.upper)),}
        \hlkwc{col}\hlstd{=}\hlkwd{rgb}\hlstd{(}\hlnum{0}\hlstd{,}\hlnum{1}\hlstd{,}\hlnum{0}\hlstd{,}\hlnum{0.5}\hlstd{),} \hlkwc{border}\hlstd{=}\hlnum{NA}\hlstd{)}                   \hlcom{# Prior CI}
\hlkwa{for}\hlstd{(i} \hlkwa{in} \hlkwd{seq}\hlstd{(}\hlnum{1}\hlstd{, n.iter,} \hlkwc{by}\hlstd{=}\hlnum{100}\hlstd{)) \{}  \hlcom{## Thin by 100}
    \hlkwd{lines}\hlstd{(temp.pred, p.prior.pred[i,],} \hlkwc{col}\hlstd{=}\hlkwd{gray}\hlstd{(}\hlnum{0.8}\hlstd{))  \}} \hlcom{# Prior preds}
\hlkwd{lines}\hlstd{(temp.pred, pred.post.mean,} \hlkwc{col}\hlstd{=}\hlstr{"blue"}\hlstd{,} \hlkwc{lwd}\hlstd{=}\hlnum{2}\hlstd{)}
\hlkwd{lines}\hlstd{(temp.pred, pred.prior.mean,} \hlkwc{col}\hlstd{=}\hlstr{"darkgreen"}\hlstd{,} \hlkwc{lwd}\hlstd{=}\hlnum{2}\hlstd{)}
\hlkwd{legend}\hlstd{(}\hlopt{-}\hlnum{3}\hlstd{,} \hlnum{1.3}\hlstd{,} \hlkwd{c}\hlstd{(}\hlstr{"Prior mean"}\hlstd{,} \hlstr{"Prior samples"}\hlstd{,} \hlstr{"Posterior mean"}\hlstd{),}
       \hlkwc{col}\hlstd{=}\hlkwd{c}\hlstd{(}\hlstr{"darkgreen"}\hlstd{,} \hlstr{"grey"}\hlstd{,} \hlstr{"blue"}\hlstd{),} \hlkwc{lwd}\hlstd{=}\hlnum{2}\hlstd{)}
\end{alltt}
\end{kframe}
\end{knitrout}
\end{frame}



\begin{frame}
  \frametitle{Prior and posterior prediction}
  \vspace{-3pt}
  \centering
  \includegraphics[width=0.9\textwidth]{figure/prior-post-pred-2-1}  \\
\end{frame}



\begin{frame}
  \frametitle{Prior predictive recap}
  There are two common strategies for specifying priors:
  \begin{enumerate}
    \item Use uniform/flat priors to obtain results that will often be
      similar to likelihood-based methods.
    \item Use priors that reflect available information.
  \end{enumerate}
  \pause
  \vfill
  The second approach can be used even if there isn't much prior
  information available. Priors can allow for the
  possibility of being ``surprised'' by the data, but without
  putting most of the prior weight on extreme values. \\
  \pause
  \vfill
  Often, highly diffuse priors on the link scales result in most of the
  prior weight being on extreme values on the natural scales. \\
  \pause
  \vfill
  Prior predictive checks can be used to assess this possibility, and
  should be a standard component of Bayesian analysis. \\
\end{frame}



\section{Assignment}




\begin{frame}[fragile]
  \frametitle{Assignment}
  % \small
  \footnotesize
  Create a self-contained R script or Rmarkdown file
  to do the following:
  \vfill
  \begin{enumerate}
%    \small
    \footnotesize
    \item Using the simulated data, compare the prior and posterior
      predictive distributions of $\lambda$ for each of the 3 forest
      types. 
    \item Change the priors for $\alpha_0$ and $\alpha_1$ from
      \inr{dnorm(0,0.5)} to \inr{dnorm(0, 0.001)} and then compare the
      prior and posterior predictions of $p$ as a function of
      temperature. Make the same graph as we made above. How
      sensitive is the posterior to the prior?
    \item Fit a binomial $N$-mixture model to the Canada warbler data
      using `unmarked'. The data include: 
      \begin{itemize}
        \footnotesize
        \item Response: \texttt{cawa1, cawa2, cawa3, cawa4}
        \item Site covs: \texttt{Elevation, Wind, Noise}
      \end{itemize}
    \item Graph the predictions of $\lambda$ over the 
      elevation range, along with 95\% CIs.
  \end{enumerate}
  \vfill
  Upload your {\tt .R} or {\tt .Rmd} file to ELC before Monday. 
\end{frame}





\end{document}

