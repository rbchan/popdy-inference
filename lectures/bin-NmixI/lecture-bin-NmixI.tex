\documentclass[color=usenames,dvipsnames]{beamer}\usepackage[]{graphicx}\usepackage[]{color}
% maxwidth is the original width if it is less than linewidth
% otherwise use linewidth (to make sure the graphics do not exceed the margin)
\makeatletter
\def\maxwidth{ %
  \ifdim\Gin@nat@width>\linewidth
    \linewidth
  \else
    \Gin@nat@width
  \fi
}
\makeatother

\definecolor{fgcolor}{rgb}{0, 0, 0}
\newcommand{\hlnum}[1]{\textcolor[rgb]{0.69,0.494,0}{#1}}%
\newcommand{\hlstr}[1]{\textcolor[rgb]{0.749,0.012,0.012}{#1}}%
\newcommand{\hlcom}[1]{\textcolor[rgb]{0.514,0.506,0.514}{\textit{#1}}}%
\newcommand{\hlopt}[1]{\textcolor[rgb]{0,0,0}{#1}}%
\newcommand{\hlstd}[1]{\textcolor[rgb]{0,0,0}{#1}}%
\newcommand{\hlkwa}[1]{\textcolor[rgb]{0,0,0}{\textbf{#1}}}%
\newcommand{\hlkwb}[1]{\textcolor[rgb]{0,0.341,0.682}{#1}}%
\newcommand{\hlkwc}[1]{\textcolor[rgb]{0,0,0}{\textbf{#1}}}%
\newcommand{\hlkwd}[1]{\textcolor[rgb]{0.004,0.004,0.506}{#1}}%
\let\hlipl\hlkwb

\usepackage{framed}
\makeatletter
\newenvironment{kframe}{%
 \def\at@end@of@kframe{}%
 \ifinner\ifhmode%
  \def\at@end@of@kframe{\end{minipage}}%
  \begin{minipage}{\columnwidth}%
 \fi\fi%
 \def\FrameCommand##1{\hskip\@totalleftmargin \hskip-\fboxsep
 \colorbox{shadecolor}{##1}\hskip-\fboxsep
     % There is no \\@totalrightmargin, so:
     \hskip-\linewidth \hskip-\@totalleftmargin \hskip\columnwidth}%
 \MakeFramed {\advance\hsize-\width
   \@totalleftmargin\z@ \linewidth\hsize
   \@setminipage}}%
 {\par\unskip\endMakeFramed%
 \at@end@of@kframe}
\makeatother

\definecolor{shadecolor}{rgb}{.97, .97, .97}
\definecolor{messagecolor}{rgb}{0, 0, 0}
\definecolor{warningcolor}{rgb}{1, 0, 1}
\definecolor{errorcolor}{rgb}{1, 0, 0}
\newenvironment{knitrout}{}{} % an empty environment to be redefined in TeX

\usepackage{alltt}
%\documentclass[color=usenames,dvipsnames,handout]{beamer}

\usepackage[roman]{../lectures}
%\usepackage[sans]{../lectures}


\hypersetup{pdfpagemode=UseNone,pdfstartview={FitV}}


\title{Lecture 6 -- Binomial $N$-mixture models: simulation, fitting, and prediction }
\author{Richard Chandler}


% Load function to compile and open PDF


% Compile and open PDF







%<<knitr-setup,include=FALSE,purl=FALSE>>=
%##opts_chunk$set(comment=NA)
%@


%% New command for inline code that isn't to be evaluated
\definecolor{inlinecolor}{rgb}{0.878, 0.918, 0.933}
\newcommand{\inr}[1]{\colorbox{inlinecolor}{\texttt{#1}}}
\IfFileExists{upquote.sty}{\usepackage{upquote}}{}
\begin{document}




\begin{frame}[plain]
  \LARGE
%  \maketitle
  \centering
  {\huge Lecture 6 -- Binomial $N$-mixture models: simulation, fitting, and predicting} \\
  {\color{default} \rule{\textwidth}{0.1pt}}
  \vfill
  \large
  WILD(FISH) 8390 \\
  Estimation of Fish and Wildlife Population Parameters \\
  \vfill
  \large
  Richard Chandler \\
  University of Georgia \\
\end{frame}






\section{Overview}



\begin{frame}[plain]
  \frametitle{Outline}
  \Large
  \only<1>{\tableofcontents}%[hideallsubsections]}
  \only<2 | handout:0>{\tableofcontents[currentsection]}%,hideallsubsections]}
\end{frame}



\begin{frame}
  \frametitle{Overview}
\end{frame}




\begin{frame}
  \frametitle{Binomial $N$-mixture model}
  \small
  State model (with Poisson assumption)
  \begin{gather*}
    \mathrm{log}(\lambda_i) = \beta_0 + \beta_1 {\color{blue} x_{i1}} +
    \beta_2 {\color{blue} x_{i2}} + \cdots \\
    N_i \sim \mathrm{Pois}(\lambda_i)
  \end{gather*}
  \pause
  \vfill
  Observation model
  \begin{gather*}
    \mathrm{logit}(p_{ij}) = \alpha_0 + \alpha_1 {\color{blue} x_{i1}}
    + \alpha_2 {\color{Purple} w_{ij}} + \cdots \\
    y_{ij} \sim \mathrm{Binomial}(N_i, p_{ij})
  \end{gather*}
  \pause
  \vfill
  \small
  Definitions \\
  $\lambda_i$ -- Expected value of abundance at site $i$ \\
  $N_i$ -- Realized value of abundance at site $i$ \\
  $p_{ij}$ -- Probability of detecting \alert{an individual} at site $i$ on occasion $j$ \\
  $y_{ij}$ -- Count data
%  \vfill
%  $\color{blue} x_1$ and $\color{blue} x_2$ are site covariates \\
%  \vspace{12pt}
%  $\color{Purple} w$ is an observation covariate
\end{frame}


\section{Simulation}



\begin{frame}
  \frametitle{Outline}
  \Large
  \tableofcontents[currentsection]
\end{frame}




\begin{frame}[fragile]
  \frametitle{Simulation -- No covariates}
  \begin{enumerate}[<+->]
  \item Abundance
\begin{knitrout}\scriptsize
\definecolor{shadecolor}{rgb}{0.878, 0.918, 0.933}\color{fgcolor}\begin{kframe}
\begin{alltt}
\hlstd{nSites} \hlkwb{<-} \hlnum{100}
\hlstd{nVisits} \hlkwb{<-} \hlnum{4}
\hlkwd{set.seed}\hlstd{(}\hlnum{3439}\hlstd{)} \hlcom{## Make it reproducible}
\hlstd{lambda1} \hlkwb{<-} \hlnum{2.6}  \hlcom{## Expected value of N}
\hlstd{N1} \hlkwb{<-} \hlkwd{rpois}\hlstd{(}\hlkwc{n}\hlstd{=nSites,} \hlkwc{lambda}\hlstd{=lambda1)}
\end{alltt}
\end{kframe}
\end{knitrout}
  \item Detection probability and data
\begin{knitrout}\scriptsize
\definecolor{shadecolor}{rgb}{0.878, 0.918, 0.933}\color{fgcolor}\begin{kframe}
\begin{alltt}
\hlstd{p1} \hlkwb{<-} \hlnum{0.3}
\hlstd{y1} \hlkwb{<-} \hlkwd{matrix}\hlstd{(}\hlnum{NA}\hlstd{,} \hlkwc{nrow}\hlstd{=nSites,} \hlkwc{ncol}\hlstd{=nVisits)}
\hlkwa{for}\hlstd{(i} \hlkwa{in} \hlnum{1}\hlopt{:}\hlstd{nSites) \{}
    \hlstd{y1[i,]} \hlkwb{<-} \hlkwd{rbinom}\hlstd{(nVisits,} \hlkwc{size}\hlstd{=N1[i],} \hlkwc{prob}\hlstd{=p1)}
\hlstd{\}}
\end{alltt}
\end{kframe}
\end{knitrout}
\end{enumerate}
\end{frame}



\begin{frame}[fragile]
  \frametitle{Simulation -- Covariates}
  \small
  Two continuous covariates and one categorical covariate with 2 levels
  \vfill
  \begin{enumerate}[<+->]
  \item Covariates
\begin{knitrout}\scriptsize
\definecolor{shadecolor}{rgb}{0.878, 0.918, 0.933}\color{fgcolor}\begin{kframe}
\begin{alltt}
\hlstd{nSites} \hlkwb{<-} \hlnum{100}\hlstd{; nVisits} \hlkwb{<-} \hlnum{4}\hlstd{;} \hlkwd{set.seed}\hlstd{(}\hlnum{3439}\hlstd{)} \hlcom{## Make it reproducible}
\hlstd{x1} \hlkwb{<-} \hlkwd{rnorm}\hlstd{(nSites,}\hlnum{0}\hlstd{,}\hlnum{0.5}\hlstd{); x2} \hlkwb{<-} \hlkwd{rnorm}\hlstd{(nSites,}\hlnum{100}\hlstd{,}\hlnum{10}\hlstd{)} \hlcom{## Continuous covs}
\hlstd{w} \hlkwb{<-} \hlkwd{matrix}\hlstd{(}\hlkwd{sample}\hlstd{(}\hlkwd{c}\hlstd{(}\hlstr{"Cold"}\hlstd{,} \hlstr{"Hot"}\hlstd{),} \hlkwc{size}\hlstd{=nSites}\hlopt{*}\hlstd{nVisits,} \hlkwc{replace}\hlstd{=T),}
            \hlkwc{nrow}\hlstd{=nSites,} \hlkwc{ncol}\hlstd{=nVisits)}
\hlstd{wHot} \hlkwb{<-} \hlkwd{ifelse}\hlstd{(w}\hlopt{==}\hlstr{"Hot"}\hlstd{,} \hlnum{1}\hlstd{,} \hlnum{0}\hlstd{)}              \hlcom{## Dummy variable}
\end{alltt}
\end{kframe}
\end{knitrout}
  \item Coefficients, $\psi$, and $p$
\begin{knitrout}\scriptsize
\definecolor{shadecolor}{rgb}{0.878, 0.918, 0.933}\color{fgcolor}\begin{kframe}
\begin{alltt}
\hlstd{beta0} \hlkwb{<-} \hlnum{1}\hlstd{; beta1} \hlkwb{<-} \hlopt{-}\hlnum{1}\hlstd{; beta2} \hlkwb{<-} \hlnum{0}
\hlstd{psi} \hlkwb{<-} \hlkwd{plogis}\hlstd{(beta0} \hlopt{+} \hlstd{beta1}\hlopt{*}\hlstd{x1} \hlopt{+} \hlstd{beta2}\hlopt{*}\hlstd{x2)}
\hlstd{alpha0} \hlkwb{<-} \hlnum{0}\hlstd{; alpha1} \hlkwb{<-} \hlnum{1}\hlstd{; alpha2} \hlkwb{<-} \hlnum{0.5}
\hlstd{p} \hlkwb{<-} \hlkwd{plogis}\hlstd{(alpha0} \hlopt{+} \hlstd{alpha1}\hlopt{*}\hlstd{x1} \hlopt{+} \hlstd{alpha2}\hlopt{*}\hlstd{wHot)}
\end{alltt}
\end{kframe}
\end{knitrout}
  \item Simulate occupancy and detection data
\begin{knitrout}\scriptsize
\definecolor{shadecolor}{rgb}{0.878, 0.918, 0.933}\color{fgcolor}\begin{kframe}
\begin{alltt}
\hlstd{z} \hlkwb{<-} \hlkwd{rbinom}\hlstd{(nSites,} \hlkwc{size}\hlstd{=}\hlnum{1}\hlstd{, psi)}            \hlcom{## pres/absence}
\hlstd{y} \hlkwb{<-} \hlkwd{matrix}\hlstd{(}\hlnum{NA}\hlstd{,} \hlkwc{nrow}\hlstd{=nSites,} \hlkwc{ncol}\hlstd{=nVisits)}
\hlkwa{for}\hlstd{(i} \hlkwa{in} \hlnum{1}\hlopt{:}\hlstd{nSites) \{}
    \hlstd{y[i,]} \hlkwb{<-} \hlkwd{rbinom}\hlstd{(nVisits,} \hlkwc{size}\hlstd{=}\hlnum{1}\hlstd{,} \hlkwc{prob}\hlstd{=z[i]}\hlopt{*}\hlstd{p[i,])}
\hlstd{\}}
\end{alltt}
\end{kframe}
\end{knitrout}
\end{enumerate}
\end{frame}




\begin{frame}[fragile]
  \frametitle{Simulated data}
  \begin{columns}
    \begin{column}{0.4\textwidth}
      \small
      Observations
%      \tiny
\begin{knitrout}\scriptsize
\definecolor{shadecolor}{rgb}{0.878, 0.918, 0.933}\color{fgcolor}\begin{kframe}
\begin{alltt}
\hlstd{y[}\hlnum{1}\hlopt{:}\hlnum{20}\hlstd{,]}
\end{alltt}
\begin{verbatim}
##       [,1] [,2] [,3] [,4]
##  [1,]    1    1    1    1
##  [2,]    1    1    1    1
##  [3,]    0    0    0    0
##  [4,]    1    1    0    1
##  [5,]    0    0    0    0
##  [6,]    0    1    0    0
##  [7,]    0    0    0    0
##  [8,]    0    0    0    0
##  [9,]    1    0    0    0
## [10,]    1    1    1    1
## [11,]    1    1    0    1
## [12,]    0    1    0    0
## [13,]    1    1    1    1
## [14,]    0    1    1    1
## [15,]    0    0    0    0
## [16,]    1    0    1    1
## [17,]    0    1    1    0
## [18,]    0    1    0    1
## [19,]    0    1    1    0
## [20,]    0    0    0    0
\end{verbatim}
\end{kframe}
\end{knitrout}
  \end{column}
  \begin{column}{0.6\textwidth}
    \pause
%    \scriptsize
    {\centering Summary stats \\}
    \vspace{24pt}
  Detections at each site \\
\begin{knitrout}\scriptsize
\definecolor{shadecolor}{rgb}{0.878, 0.918, 0.933}\color{fgcolor}\begin{kframe}
\begin{alltt}
\hlstd{siteDets} \hlkwb{<-} \hlkwd{rowSums}\hlstd{(y)} \hlcom{# Dets at each site}
\hlkwd{table}\hlstd{(siteDets)}        \hlcom{# Frequency}
\end{alltt}
\begin{verbatim}
## siteDets
##  0  1  2  3  4 
## 30 12 17 26 15
\end{verbatim}
\end{kframe}
\end{knitrout}
\pause
\vfill
\small
Proportion of sites known to be occupied \\
\begin{knitrout}\scriptsize
\definecolor{shadecolor}{rgb}{0.878, 0.918, 0.933}\color{fgcolor}\begin{kframe}
\begin{alltt}
\hlstd{naiveOccupancy} \hlkwb{<-} \hlkwd{sum}\hlstd{(siteDets}\hlopt{>}\hlnum{0}\hlstd{)}\hlopt{/}\hlstd{nSites}
\hlstd{naiveOccupancy}
\end{alltt}
\begin{verbatim}
## [1] 0.7
\end{verbatim}
\end{kframe}
\end{knitrout}

  \end{column}
  \end{columns}
\end{frame}



\section{Prediction}



\begin{frame}
  \frametitle{Outline}
  \Large
  \tableofcontents[currentsection]
\end{frame}



\subsection{Likelihood-based methods}



\begin{frame}[fragile]
  \frametitle{Prepare data in `unmarked'}
  \small
Notice the two new arguments \inr{siteCovs} and \inr{obsCovs}: 
\begin{knitrout}\tiny
\definecolor{shadecolor}{rgb}{0.878, 0.918, 0.933}\color{fgcolor}\begin{kframe}
\begin{alltt}
\hlstd{umf} \hlkwb{<-} \hlkwd{unmarkedFrameOccu}\hlstd{(}\hlkwc{y}\hlstd{=y,} \hlkwc{siteCovs}\hlstd{=}\hlkwd{data.frame}\hlstd{(x1,x2),} \hlkwc{obsCovs}\hlstd{=}\hlkwd{list}\hlstd{(}\hlkwc{w}\hlstd{=w))}
\end{alltt}


{\ttfamily\noindent\color{warningcolor}{\#\# Warning: obsCovs contains characters. Converting them to factors.}}\end{kframe}
\end{knitrout}
\pause
%\vfill
%Reformat $w$ as a factor: %, but it's formatted as a matrix of
%characters, we have to reformat it:
%Summary
\begin{knitrout}\tiny
\definecolor{shadecolor}{rgb}{0.878, 0.918, 0.933}\color{fgcolor}\begin{kframe}
\begin{alltt}
\hlkwd{summary}\hlstd{(umf)}
\end{alltt}
\begin{verbatim}
## unmarkedFrame Object
## 
## 100 sites
## Maximum number of observations per site: 4 
## Mean number of observations per site: 4 
## Sites with at least one detection: 70 
## 
## Tabulation of y observations:
##   0   1 
## 216 184 
## 
## Site-level covariates:
##        x1                 x2        
##  Min.   :-0.98075   Min.   : 64.05  
##  1st Qu.:-0.38819   1st Qu.: 91.10  
##  Median :-0.05609   Median : 97.47  
##  Mean   :-0.02434   Mean   : 98.88  
##  3rd Qu.: 0.29402   3rd Qu.:107.18  
##  Max.   : 1.64324   Max.   :127.15  
## 
## Observation-level covariates:
##     w      
##  Cold:208  
##  Hot :192
\end{verbatim}
\end{kframe}
\end{knitrout}
\end{frame}



\begin{frame}[fragile]
  \frametitle{Standardizing}
  It's almost always a good idea to standardize \alert{continuous} covariates. \\
  \pause
  \vfill
  Standardizing involves subtracting the mean and then dividing by the standard deviation. \\
  \pause
%  \vfill
\begin{knitrout}\small
\definecolor{shadecolor}{rgb}{0.878, 0.918, 0.933}\color{fgcolor}\begin{kframe}
\begin{alltt}
\hlkwd{siteCovs}\hlstd{(umf)}\hlopt{$}\hlstd{x1s} \hlkwb{<-} \hlstd{(x1}\hlopt{-}\hlkwd{mean}\hlstd{(x1))}\hlopt{/}\hlkwd{sd}\hlstd{(x1)}
\hlkwd{siteCovs}\hlstd{(umf)}\hlopt{$}\hlstd{x2s} \hlkwb{<-} \hlstd{(x2}\hlopt{-}\hlkwd{mean}\hlstd{(x2))}\hlopt{/}\hlkwd{sd}\hlstd{(x2)}
\end{alltt}
\end{kframe}
\end{knitrout}
%  \pause
%  \vfill
%  If all of your site covariates are continuous, you can use this
%  shortcut with the \inr{scale} function:
%<<umf-zcovs>>=
%siteCovs(umf) <- scale(siteCovs(umf))
%@
  \pause
  \vfill
  Standardizing makes it easier to find the maximum likelihood
  estimates (MLEs) and it facilitates comparison of estimates.  \\
  \pause
  \vfill
  We just have to remember to back-transform covariates when graphing
  predictions.
\end{frame}




\begin{frame}[fragile]
  \frametitle{Fit the model}
  \footnotesize
\begin{knitrout}\tiny
\definecolor{shadecolor}{rgb}{0.878, 0.918, 0.933}\color{fgcolor}\begin{kframe}
\begin{alltt}
\hlstd{fm} \hlkwb{<-} \hlkwd{occu}\hlstd{(}\hlopt{~}\hlstd{x1s}\hlopt{+}\hlstd{w} \hlopt{~}\hlstd{x1s}\hlopt{+}\hlstd{x2s, umf)}    \hlcom{## Notice standardized covariates}
\hlstd{fm}
\end{alltt}
\begin{verbatim}
## 
## Call:
## occu(formula = ~x1s + w ~ x1s + x2s, data = umf)
## 
## Occupancy:
##             Estimate    SE      z  P(>|z|)
## (Intercept)    1.139 0.292  3.898 9.68e-05
## x1s           -0.748 0.312 -2.401 1.63e-02
## x2s           -0.147 0.262 -0.561 5.75e-01
## 
## Detection:
##             Estimate    SE    z  P(>|z|)
## (Intercept)    0.482 0.187 2.57 1.01e-02
## x1s            0.862 0.170 5.06 4.23e-07
## wHot           0.584 0.265 2.20 2.76e-02
## 
## AIC: 455.6924
\end{verbatim}
\end{kframe}
\end{knitrout}
\pause
\vfill
Compare to actual parameter values:
\begin{knitrout}\tiny
\definecolor{shadecolor}{rgb}{0.878, 0.918, 0.933}\color{fgcolor}\begin{kframe}
\begin{alltt}
\hlkwd{c}\hlstd{(}\hlkwc{beta0}\hlstd{=beta0,} \hlkwc{beta1}\hlstd{=beta1,} \hlkwc{beta2}\hlstd{=beta2)}
\end{alltt}
\begin{verbatim}
## beta0 beta1 beta2 
##     1    -1     0
\end{verbatim}
\begin{alltt}
\hlkwd{c}\hlstd{(}\hlkwc{alpha0}\hlstd{=alpha0,} \hlkwc{alpha1}\hlstd{=alpha1,} \hlkwc{alpha2}\hlstd{=alpha2)}
\end{alltt}
\begin{verbatim}
## alpha0 alpha1 alpha2 
##    0.0    1.0    0.5
\end{verbatim}
\end{kframe}
\end{knitrout}
\end{frame}






\begin{frame}[fragile]
  \frametitle{Prediction in `unmarked'}
  Create \texttt{data.frame} with prediction covariates. We'll let $x_1$
  vary while holding other two covariates constant. Important that we
  use the standardized version of the continuous covariates.
\begin{knitrout}\small
\definecolor{shadecolor}{rgb}{0.878, 0.918, 0.933}\color{fgcolor}\begin{kframe}
\begin{alltt}
\hlstd{pred.data} \hlkwb{<-} \hlkwd{data.frame}\hlstd{(}\hlkwc{x1s}\hlstd{=}\hlkwd{seq}\hlstd{(}\hlkwc{from}\hlstd{=}\hlopt{-}\hlnum{3}\hlstd{,} \hlkwc{to}\hlstd{=}\hlnum{3}\hlstd{,} \hlkwc{length}\hlstd{=}\hlnum{50}\hlstd{),}
                        \hlkwc{x2s}\hlstd{=}\hlnum{0}\hlstd{,} \hlkwc{w}\hlstd{=}\hlstr{'Hot'}\hlstd{)}
\end{alltt}
\end{kframe}
\end{knitrout}
\pause
\vfill
Get predictions of $\psi$ for each row of prediction data.
\begin{knitrout}\small
\definecolor{shadecolor}{rgb}{0.878, 0.918, 0.933}\color{fgcolor}\begin{kframe}
\begin{alltt}
\hlstd{psi.pred} \hlkwb{<-} \hlkwd{predict}\hlstd{(fm,} \hlkwc{newdata}\hlstd{=pred.data,}
                    \hlkwc{type}\hlstd{=}\hlstr{'state'}\hlstd{,} \hlkwc{append}\hlstd{=}\hlnum{TRUE}\hlstd{)}
\end{alltt}
\end{kframe}
\end{knitrout}
\pause
\vfill
Get predictions of $p$ for each row of prediction data.
\begin{knitrout}\small
\definecolor{shadecolor}{rgb}{0.878, 0.918, 0.933}\color{fgcolor}\begin{kframe}
\begin{alltt}
\hlstd{p.pred} \hlkwb{<-} \hlkwd{predict}\hlstd{(fm,} \hlkwc{newdata}\hlstd{=pred.data,}
                  \hlkwc{type}\hlstd{=}\hlstr{'det'}\hlstd{,} \hlkwc{append}\hlstd{=}\hlnum{TRUE}\hlstd{)}
\end{alltt}
\end{kframe}
\end{knitrout}
\end{frame}



\begin{frame}[fragile]
  \frametitle{Prediction in `unmarked'}
  \small
  View $\psi$ predictions
\begin{knitrout}\footnotesize
\definecolor{shadecolor}{rgb}{0.878, 0.918, 0.933}\color{fgcolor}\begin{kframe}
\begin{alltt}
\hlkwd{print}\hlstd{(}\hlkwd{head}\hlstd{(psi.pred),} \hlkwc{digits}\hlstd{=}\hlnum{2}\hlstd{)}
\end{alltt}
\begin{verbatim}
##   Predicted    SE lower upper  x1s x2s   w
## 1      0.97 0.035  0.77  1.00 -3.0   0 Hot
## 2      0.96 0.037  0.77  1.00 -2.9   0 Hot
## 3      0.96 0.039  0.76  0.99 -2.8   0 Hot
## 4      0.96 0.041  0.76  0.99 -2.6   0 Hot
## 5      0.95 0.043  0.76  0.99 -2.5   0 Hot
## 6      0.95 0.045  0.75  0.99 -2.4   0 Hot
\end{verbatim}
\end{kframe}
\end{knitrout}
\pause
\vfill
  View $p$ predictions
\begin{knitrout}\footnotesize
\definecolor{shadecolor}{rgb}{0.878, 0.918, 0.933}\color{fgcolor}\begin{kframe}
\begin{alltt}
\hlkwd{print}\hlstd{(}\hlkwd{head}\hlstd{(p.pred),} \hlkwc{digits}\hlstd{=}\hlnum{2}\hlstd{)}
\end{alltt}
\begin{verbatim}
##   Predicted    SE lower upper  x1s x2s   w
## 1      0.18 0.073 0.076  0.37 -3.0   0 Hot
## 2      0.20 0.075 0.087  0.38 -2.9   0 Hot
## 3      0.21 0.077 0.099  0.40 -2.8   0 Hot
## 4      0.23 0.078 0.112  0.42 -2.6   0 Hot
## 5      0.25 0.079 0.127  0.43 -2.5   0 Hot
## 6      0.27 0.080 0.144  0.45 -2.4   0 Hot
\end{verbatim}
\end{kframe}
\end{knitrout}
\end{frame}






\begin{frame}[fragile]
  \frametitle{Prediction in `unmarked'}
\begin{knitrout}\tiny
\definecolor{shadecolor}{rgb}{0.878, 0.918, 0.933}\color{fgcolor}\begin{kframe}
\begin{alltt}
\hlkwd{plot}\hlstd{(Predicted} \hlopt{~} \hlstd{x1s, psi.pred,} \hlkwc{type}\hlstd{=}\hlstr{"l"}\hlstd{,} \hlkwc{ylab}\hlstd{=}\hlstr{"Occurrence probability"}\hlstd{,} \hlkwc{col}\hlstd{=}\hlstr{"blue"}\hlstd{,}
     \hlkwc{xlab}\hlstd{=}\hlstr{"Standardized covariate (x1s)"}\hlstd{,} \hlkwc{ylim}\hlstd{=}\hlnum{0}\hlopt{:}\hlnum{1}\hlstd{)}
\hlkwd{lines}\hlstd{(lower} \hlopt{~} \hlstd{x1s, psi.pred,} \hlkwc{col}\hlstd{=}\hlstr{"grey"}\hlstd{);} \hlkwd{lines}\hlstd{(upper} \hlopt{~} \hlstd{x1s, psi.pred,} \hlkwc{col}\hlstd{=}\hlstr{"grey"}\hlstd{)}
\end{alltt}
\end{kframe}

{\centering \includegraphics[width=0.8\linewidth]{figure/pred-psi1-1} 

}



\end{knitrout}
\end{frame}







\begin{frame}
  \frametitle{In-class exercise}
  \small
  \begin{enumerate}
    \item Fit this model (to the simulated data):
      \begin{gather*}
        \mathrm{logit}(\psi_i) = \beta_0 + \beta_1 {\color{blue} x_{i1}} \\
        z_i \sim \mathrm{Bern}(\psi_i) \\
%      \end{gather*}
%      \begin{gather*}
        \mathrm{logit}(p_{ij}) = \alpha_0 + \alpha_1 {\color{blue} x_{i1}} +
        \alpha_2 {\color{Purple} w_{ij}} \\
        y_{ij} \sim \mathrm{Bern}(z_i\times p_{ij})
      \end{gather*}
  \end{enumerate}
\end{frame}




\subsection{Bayesian methods}



\begin{frame}[fragile]
  \frametitle{The BUGS model}

\end{frame}





\begin{frame}[fragile]
  \frametitle{Data, inits, and parameters}
  Put data in a named list
  \vspace{-12pt}
\begin{knitrout}\small
\definecolor{shadecolor}{rgb}{0.878, 0.918, 0.933}\color{fgcolor}\begin{kframe}
\begin{alltt}
\hlstd{jags.data} \hlkwb{<-} \hlkwd{list}\hlstd{(}\hlkwc{y}\hlstd{=y,} \hlkwc{x1}\hlstd{=(x1}\hlopt{-}\hlkwd{mean}\hlstd{(x1))}\hlopt{/}\hlkwd{sd}\hlstd{(x1),}
                  \hlkwc{x2}\hlstd{=(x2}\hlopt{-}\hlkwd{mean}\hlstd{(x2))}\hlopt{/}\hlkwd{sd}\hlstd{(x2),} \hlkwc{wHot}\hlstd{=wHot,}
                  \hlkwc{nSites}\hlstd{=nSites,} \hlkwc{nOccasions}\hlstd{=nVisits)}
\end{alltt}
\end{kframe}
\end{knitrout}
\pause
\vfill
  Initial values
  \vspace{-12pt}
\begin{knitrout}\small
\definecolor{shadecolor}{rgb}{0.878, 0.918, 0.933}\color{fgcolor}\begin{kframe}
\begin{alltt}
\hlstd{jags.inits} \hlkwb{<-} \hlkwa{function}\hlstd{() \{}
    \hlkwd{list}\hlstd{(}\hlkwc{beta0}\hlstd{=}\hlkwd{rnorm}\hlstd{(}\hlnum{1}\hlstd{),} \hlkwc{alpha0}\hlstd{=}\hlkwd{rnorm}\hlstd{(}\hlnum{1}\hlstd{),} \hlkwc{z}\hlstd{=}\hlkwd{rep}\hlstd{(}\hlnum{1}\hlstd{, nSites))}
\hlstd{\}}
\end{alltt}
\end{kframe}
\end{knitrout}
\pause
\vfill
  Parameters to monitor
  \vspace{-12pt}
\begin{knitrout}\small
\definecolor{shadecolor}{rgb}{0.878, 0.918, 0.933}\color{fgcolor}\begin{kframe}
\begin{alltt}
\hlstd{jags.pars} \hlkwb{<-} \hlkwd{c}\hlstd{(}\hlstr{"beta0"}\hlstd{,} \hlstr{"beta1"}\hlstd{,} \hlstr{"beta2"}\hlstd{,}
               \hlstr{"alpha0"}\hlstd{,} \hlstr{"alpha1"}\hlstd{,} \hlstr{"alpha2"}\hlstd{,} \hlstr{"sitesOccupied"}\hlstd{)}
\end{alltt}
\end{kframe}
\end{knitrout}
\end{frame}





\begin{frame}[fragile]
  \frametitle{MCMC}
  \small
\begin{knitrout}\scriptsize
\definecolor{shadecolor}{rgb}{0.878, 0.918, 0.933}\color{fgcolor}\begin{kframe}
\begin{alltt}
\hlkwd{library}\hlstd{(jagsUI)}
\hlcom{#jags.post.samples <- jags.basic(data=jags.data, inits=jags.inits,}
\hlcom{#                                parameters.to.save=jags.pars,}
\hlcom{#                                model.file="occupancy-model-covs.jag",}
\hlcom{#                                n.chains=3, n.adapt=100, n.burnin=0,}
\hlcom{#                                n.iter=2000, parallel=TRUE)}
\end{alltt}
\end{kframe}
\end{knitrout}
\end{frame}



\begin{frame}[fragile]
  \frametitle{Summarize output}
\begin{knitrout}\tiny
\definecolor{shadecolor}{rgb}{0.878, 0.918, 0.933}\color{fgcolor}\begin{kframe}
\begin{alltt}
\hlcom{#summary(jags.post.samples)}
\end{alltt}
\end{kframe}
\end{knitrout}
\end{frame}




\begin{frame}[fragile]
  \frametitle{Traceplots and density plots}
\begin{knitrout}\footnotesize
\definecolor{shadecolor}{rgb}{0.878, 0.918, 0.933}\color{fgcolor}\begin{kframe}
\begin{alltt}
\hlcom{#plot(jags.post.samples[,1:3])}
\end{alltt}
\end{kframe}
\end{knitrout}
\end{frame}



\begin{frame}[fragile]
  \frametitle{Traceplots and density plots}
\begin{knitrout}\footnotesize
\definecolor{shadecolor}{rgb}{0.878, 0.918, 0.933}\color{fgcolor}\begin{kframe}
\begin{alltt}
\hlcom{#plot(jags.post.samples[,c(4:6,8)])}
\end{alltt}
\end{kframe}
\end{knitrout}
\end{frame}


\begin{frame}
  \frametitle{Bayesian prediction}
  Every MCMC iteration represents a sample from the posterior
  distribution. \\
  \pause
  \vfill
  Each sample of the occupancy and detection coefficients can be used
  to make a prediction. \\
  \pause
  \vfill
  The collection of predictions can be used to summarize the posterior 
  predictive distribution. \\
  \pause
  \vfill
  We will look at the distribution of prediction lines. \\  
\end{frame}


\begin{frame}[fragile]
  \frametitle{Bayesian prediction}
  \small
  First, extract the $\psi$ coefficients
\begin{knitrout}\scriptsize
\definecolor{shadecolor}{rgb}{0.878, 0.918, 0.933}\color{fgcolor}\begin{kframe}
\begin{alltt}
\hlcom{#psi.coef.post <- as.matrix(jags.post.samples[,c("beta0","beta1","beta2")])}
\hlcom{#head(psi.coef.post, n=4)}
\end{alltt}
\end{kframe}
\end{knitrout}
  \pause
  \vfill
  Create prediction matrix, one row for each MCMC iteration.
%  Columns represent covariate values. 
\begin{knitrout}\scriptsize
\definecolor{shadecolor}{rgb}{0.878, 0.918, 0.933}\color{fgcolor}\begin{kframe}
\begin{alltt}
\hlcom{#n.iter <- nrow(psi.coef.post)  }
\hlcom{#psi.post.pred <- matrix(NA, nrow=n.iter, ncol=nrow(pred.data))}
\end{alltt}
\end{kframe}
\end{knitrout}
  \pause
  \vfill
  Predict $\psi$ for each MCMC iteration.
%  using covariate values from \inr{pred.data}. 
\begin{knitrout}\scriptsize
\definecolor{shadecolor}{rgb}{0.878, 0.918, 0.933}\color{fgcolor}\begin{kframe}
\begin{alltt}
\hlcom{#for(i in 1:n.iter) \{}
\hlcom{#    psi.post.pred[i,] <- plogis(psi.coef.post[i,"beta0"] +}
\hlcom{#                                psi.coef.post[i,"beta1"]*pred.data$x1s)}
\hlcom{#\}}
\end{alltt}
\end{kframe}
\end{knitrout}
\end{frame}



\begin{frame}[fragile]
  \frametitle{Bayesian prediction}
  Prediction line for first posterior sample
\begin{knitrout}\scriptsize
\definecolor{shadecolor}{rgb}{0.878, 0.918, 0.933}\color{fgcolor}\begin{kframe}
\begin{alltt}
\hlcom{#plot(pred.data$x1s, psi.post.pred[1,], type="l", xlab="x1s",}
\hlcom{#     ylab="Occurrence probability", ylim=c(0, 1), col=gray(0.8))}
\end{alltt}
\end{kframe}
\end{knitrout}
\end{frame}



\begin{frame}[fragile]
  \frametitle{Bayesian prediction}
  All samples from the posterior predictive distribution
\begin{knitrout}\scriptsize
\definecolor{shadecolor}{rgb}{0.878, 0.918, 0.933}\color{fgcolor}\begin{kframe}
\begin{alltt}
\hlcom{#     ylab="Occurrence probability", ylim=c(0, 1), col=gray(0.8))}
\hlcom{#for(i in 1:n.iter) \{}
\hlcom{#    lines(pred.data$x1s, psi.post.pred[i,], col=gray(0.8))}
\hlcom{#\}}
\end{alltt}
\end{kframe}
\end{knitrout}
\end{frame}





\begin{frame}[fragile]
  \frametitle{Bayesian prediction}
  Now with posterior mean and 95\% CI
\begin{knitrout}\tiny
\definecolor{shadecolor}{rgb}{0.878, 0.918, 0.933}\color{fgcolor}\begin{kframe}
\begin{alltt}
\hlcom{#for(i in 1:n.iter) \{}
\hlcom{#    lines(pred.data$x1s, psi.post.pred[i,], col=gray(0.8))}
\hlcom{#\}}
\hlcom{#pred.post.mean <- colMeans(psi.post.pred)}
\hlcom{#pred.post.lower <- apply(psi.post.pred, 2, quantile, prob=0.025)}
\hlcom{#pred.post.upper <- apply(psi.post.pred, 2, quantile, prob=0.975)}
\hlcom{#lines(pred.data$x1, pred.post.mean, col="blue")}
\hlcom{#lines(pred.data$x1, pred.post.lower, col="blue", lty=2)}
\hlcom{#lines(pred.data$x1, pred.post.upper, col="blue", lty=2)}
\end{alltt}
\end{kframe}
\end{knitrout}
\end{frame}




\begin{frame}
  \frametitle{Recap}
\end{frame}



\section{Assignment}




\begin{frame}
  \frametitle{Assignment}
  % \small
  \footnotesize
  Create a self-contained R script or Rmarkdown file
  to do the following:
  \vfill
  \begin{enumerate}
%    \small
    \footnotesize
    \item 
      \begin{itemize}
        \footnotesize
        \item 
        \item 
      \end{itemize}
    \end{enumerate}
    \vfill
    Upload your {\tt .R} or {\tt .Rmd} file to ELC before Monday. 
\end{frame}





\end{document}

