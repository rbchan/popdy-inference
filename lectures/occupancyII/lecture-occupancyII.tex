\documentclass[color=usenames,dvipsnames]{beamer}\usepackage[]{graphicx}\usepackage[]{color}
% maxwidth is the original width if it is less than linewidth
% otherwise use linewidth (to make sure the graphics do not exceed the margin)
\makeatletter
\def\maxwidth{ %
  \ifdim\Gin@nat@width>\linewidth
    \linewidth
  \else
    \Gin@nat@width
  \fi
}
\makeatother

\definecolor{fgcolor}{rgb}{0, 0, 0}
\newcommand{\hlnum}[1]{\textcolor[rgb]{0.69,0.494,0}{#1}}%
\newcommand{\hlstr}[1]{\textcolor[rgb]{0.749,0.012,0.012}{#1}}%
\newcommand{\hlcom}[1]{\textcolor[rgb]{0.514,0.506,0.514}{\textit{#1}}}%
\newcommand{\hlopt}[1]{\textcolor[rgb]{0,0,0}{#1}}%
\newcommand{\hlstd}[1]{\textcolor[rgb]{0,0,0}{#1}}%
\newcommand{\hlkwa}[1]{\textcolor[rgb]{0,0,0}{\textbf{#1}}}%
\newcommand{\hlkwb}[1]{\textcolor[rgb]{0,0.341,0.682}{#1}}%
\newcommand{\hlkwc}[1]{\textcolor[rgb]{0,0,0}{\textbf{#1}}}%
\newcommand{\hlkwd}[1]{\textcolor[rgb]{0.004,0.004,0.506}{#1}}%
\let\hlipl\hlkwb

\usepackage{framed}
\makeatletter
\newenvironment{kframe}{%
 \def\at@end@of@kframe{}%
 \ifinner\ifhmode%
  \def\at@end@of@kframe{\end{minipage}}%
  \begin{minipage}{\columnwidth}%
 \fi\fi%
 \def\FrameCommand##1{\hskip\@totalleftmargin \hskip-\fboxsep
 \colorbox{shadecolor}{##1}\hskip-\fboxsep
     % There is no \\@totalrightmargin, so:
     \hskip-\linewidth \hskip-\@totalleftmargin \hskip\columnwidth}%
 \MakeFramed {\advance\hsize-\width
   \@totalleftmargin\z@ \linewidth\hsize
   \@setminipage}}%
 {\par\unskip\endMakeFramed%
 \at@end@of@kframe}
\makeatother

\definecolor{shadecolor}{rgb}{.97, .97, .97}
\definecolor{messagecolor}{rgb}{0, 0, 0}
\definecolor{warningcolor}{rgb}{1, 0, 1}
\definecolor{errorcolor}{rgb}{1, 0, 0}
\newenvironment{knitrout}{}{} % an empty environment to be redefined in TeX

\usepackage{alltt}
%\documentclass[color=usenames,dvipsnames,handout]{beamer}

\usepackage[roman]{../lectures}
%\usepackage[sans]{../lectures}


\hypersetup{pdfpagemode=UseNone,pdfstartview={FitV}}

%%% TODO: Add something about missing values

\title{Occupancy models: covariates and prediction }
\author{Richard Chandler}


% Load function to compile and open PDF


% Compile and open PDF







%<<knitr-setup,include=FALSE,purl=FALSE>>=
%##opts_chunk$set(comment=NA)
%@


%% New command for inline code that isn't to be evaluated
\definecolor{inlinecolor}{rgb}{0.878, 0.918, 0.933}
\newcommand{\inr}[1]{\colorbox{inlinecolor}{\texttt{#1}}}
\IfFileExists{upquote.sty}{\usepackage{upquote}}{}
\begin{document}






\begin{frame}[plain]
  \LARGE
  \centering
  {\huge Occupancy models: covariates and prediction} \\
  {\color{default} \rule{\textwidth}{0.1pt}}
  \vfill
  \large
  WILD(FISH) 8390 \\
  Estimation of Fish and Wildlife Population Parameters \\
  \vfill
  \large
  Richard Chandler \\
  University of Georgia \\
\end{frame}






\section{Overview}



\begin{frame}[plain]
  \frametitle{Outline}
  \Large
  \only<1>{\tableofcontents}%[hideallsubsections]}
  \only<2 | handout:0>{\tableofcontents[currentsection]}%,hideallsubsections]}
\end{frame}



\begin{frame}
  \frametitle{Overview}
  We almost always have covariates, which are also called explanatory
  variables, predictors, independent variables, etc\dots \\
  \pause
  \vfill
  Explaining variation in occupancy and detection probability with
  covariates is central to reliable prediction and scientific
  investigation. \\  
  \pause
  \vfill
  In static occupancy models, there are two types of covariates:
  \begin{enumerate}
    \item \textcolor{blue}{Site covariates} vary among sites
      but not among occasions
    \item \textcolor{Purple}{Observation covariates} may vary among
      sites and occasions
  \end{enumerate}
  \pause
  \vfill
  In both cases, covariates can be continuous or categorical. \\
  \pause
  \vfill
  \centering
  {Observation covariates cannot be used to describe variation
    in occupancy because occupancy is assumed to be \\ constant over
    time.} \\ 
\end{frame}

\begin{frame}
  \frametitle{Example}
  \begin{enumerate}
    \item \textcolor{blue}{Site covariates} vary among sites
      but not among occasions
    \item \textcolor{Purple}{Observation covariates} may vary among
      sites and occasions
  \end{enumerate}
  
  Define the following as Site or Observation covariates (or both?):
  \begin{itemize}
    \item{Temperature}
    \item{Distance to Stream}
    \item{Wind}
    \item{Others?}
  \end{itemize}  
\end{frame}
    

% \begin{frame}
%   \frametitle{Ignoring detection probability}
%   Show some examples
% \end{frame}



\begin{frame}
  \frametitle{Covariates}
  \small
  State model
  \begin{gather*}
    \mathrm{logit}(\psi_i) = \beta_0 + \beta_1 {\color{blue} x_{i1}} +
    \beta_2 {\color{blue} x_{i2}} + \cdots \\
    z_i \sim \mathrm{Bern}(\psi_i)
  \end{gather*}
  \pause
  \vfill
  Observation model
  \begin{gather*}
    \mathrm{logit}(p_{ij}) = \alpha_0 + \alpha_1 {\color{blue} x_{i1}}
    + \alpha_2 {\color{Purple} w_{ij}} + \cdots \\
    y_{ij} \sim \mathrm{Bern}(z_i\times p_{ij})
  \end{gather*}
  \pause
  %  \vfill
  %  \small
  % Definitions \\
  % $\psi_i$ -- probability that the species occurs at site $i$ \\
  % $z_i$ -- binary presence/absence variable at site $i$ \\
  % $p_{ij}$ -- probability of detecting the species at site $i$ on occasion $j$ \\
  % $y_{ij}$ -- binary detection/non-detection data
  \vfill
  $\color{blue} x_1$ and $\color{blue} x_2$ are site covariates \\
  \vspace{12pt}
  $\color{Purple} w$ is an observation covariate
\end{frame}


\section{Simulation}



\begin{frame}
  \frametitle{Outline}
  \Large
  \tableofcontents[currentsection]
\end{frame}


\begin{frame}[fragile]
  \frametitle{Simulation}
  \small
  Two continuous covariates and one categorical covariate with 2 levels
  \vfill
  \begin{enumerate}[<+->]
  \item Covariates
\begin{knitrout}\scriptsize
\definecolor{shadecolor}{rgb}{0.878, 0.918, 0.933}\color{fgcolor}\begin{kframe}
\begin{alltt}
\hlstd{nSites} \hlkwb{<-} \hlnum{100}\hlstd{; nVisits} \hlkwb{<-} \hlnum{4}\hlstd{;} \hlkwd{set.seed}\hlstd{(}\hlnum{83}\hlstd{)} \hlcom{## Make it reproducible}
\hlstd{x1} \hlkwb{<-} \hlkwd{rnorm}\hlstd{(nSites,}\hlnum{0}\hlstd{,}\hlnum{0.5}\hlstd{); x2} \hlkwb{<-} \hlkwd{rnorm}\hlstd{(nSites,}\hlnum{100}\hlstd{,}\hlnum{10}\hlstd{)} \hlcom{## Continuous covs}
\hlstd{w} \hlkwb{<-} \hlkwd{matrix}\hlstd{(}\hlkwd{sample}\hlstd{(}\hlkwd{c}\hlstd{(}\hlstr{"Cold"}\hlstd{,} \hlstr{"Hot"}\hlstd{),} \hlkwc{size}\hlstd{=nSites}\hlopt{*}\hlstd{nVisits,} \hlkwc{replace}\hlstd{=T),}
            \hlkwc{nrow}\hlstd{=nSites,} \hlkwc{ncol}\hlstd{=nVisits)}
\hlstd{wHot} \hlkwb{<-} \hlkwd{ifelse}\hlstd{(w}\hlopt{==}\hlstr{"Hot"}\hlstd{,} \hlnum{1}\hlstd{,} \hlnum{0}\hlstd{)}              \hlcom{## Dummy variable}
\end{alltt}
\end{kframe}
\end{knitrout}
  \item Coefficients, $\psi$, and $p$
\begin{knitrout}\scriptsize
\definecolor{shadecolor}{rgb}{0.878, 0.918, 0.933}\color{fgcolor}\begin{kframe}
\begin{alltt}
\hlstd{beta0} \hlkwb{<-} \hlnum{0}\hlstd{; beta1} \hlkwb{<-} \hlopt{-}\hlnum{1}\hlstd{; beta2} \hlkwb{<-} \hlnum{0}
\hlstd{psi} \hlkwb{<-} \hlkwd{plogis}\hlstd{(beta0} \hlopt{+} \hlstd{beta1}\hlopt{*}\hlstd{x1} \hlopt{+} \hlstd{beta2}\hlopt{*}\hlstd{x2)}
\hlstd{alpha0} \hlkwb{<-} \hlopt{-}\hlnum{0.5}\hlstd{; alpha1} \hlkwb{<-} \hlnum{1}\hlstd{; alpha2} \hlkwb{<-} \hlnum{2}
\hlstd{p} \hlkwb{<-} \hlkwd{plogis}\hlstd{(alpha0} \hlopt{+} \hlstd{alpha1}\hlopt{*}\hlstd{x1} \hlopt{+} \hlstd{alpha2}\hlopt{*}\hlstd{wHot)}
\end{alltt}
\end{kframe}
\end{knitrout}
  \item Simulate occupancy and detection data
\begin{knitrout}\scriptsize
\definecolor{shadecolor}{rgb}{0.878, 0.918, 0.933}\color{fgcolor}\begin{kframe}
\begin{alltt}
\hlstd{z} \hlkwb{<-} \hlkwd{rbinom}\hlstd{(nSites,} \hlkwc{size}\hlstd{=}\hlnum{1}\hlstd{, psi)}            \hlcom{## pres/absence}
\hlstd{y} \hlkwb{<-} \hlkwd{matrix}\hlstd{(}\hlnum{NA}\hlstd{,} \hlkwc{nrow}\hlstd{=nSites,} \hlkwc{ncol}\hlstd{=nVisits)}
\hlkwa{for}\hlstd{(i} \hlkwa{in} \hlnum{1}\hlopt{:}\hlstd{nSites) \{}
    \hlstd{y[i,]} \hlkwb{<-} \hlkwd{rbinom}\hlstd{(nVisits,} \hlkwc{size}\hlstd{=}\hlnum{1}\hlstd{,} \hlkwc{prob}\hlstd{=z[i]}\hlopt{*}\hlstd{p[i,])}
\hlstd{\}}
\end{alltt}
\end{kframe}
\end{knitrout}
\end{enumerate}
\end{frame}




\begin{frame}[fragile]
  \frametitle{Simulated data}
  \begin{columns}
    \begin{column}{0.4\textwidth}
      \small
      Observations
%      \tiny
\begin{knitrout}\scriptsize
\definecolor{shadecolor}{rgb}{0.878, 0.918, 0.933}\color{fgcolor}\begin{kframe}
\begin{alltt}
\hlstd{y[}\hlnum{1}\hlopt{:}\hlnum{20}\hlstd{,]}
\end{alltt}
\begin{verbatim}
##       [,1] [,2] [,3] [,4]
##  [1,]    1    1    1    0
##  [2,]    1    0    0    1
##  [3,]    0    0    1    1
##  [4,]    1    1    0    1
##  [5,]    0    0    0    0
##  [6,]    0    0    0    0
##  [7,]    0    0    0    0
##  [8,]    1    1    1    1
##  [9,]    0    0    0    0
## [10,]    0    0    0    0
## [11,]    0    0    0    0
## [12,]    1    1    0    1
## [13,]    0    0    0    0
## [14,]    0    0    0    0
## [15,]    0    0    0    0
## [16,]    0    0    0    0
## [17,]    1    0    1    1
## [18,]    1    1    1    0
## [19,]    0    0    0    0
## [20,]    0    0    0    0
\end{verbatim}
\end{kframe}
\end{knitrout}
  \end{column}
  \begin{column}{0.6\textwidth}
    \pause
%    \scriptsize
    {\centering Summary stats \\}
    \vspace{24pt}
  Detections at each site \\
\begin{knitrout}\scriptsize
\definecolor{shadecolor}{rgb}{0.878, 0.918, 0.933}\color{fgcolor}\begin{kframe}
\begin{alltt}
\hlstd{siteDets} \hlkwb{<-} \hlkwd{rowSums}\hlstd{(y)} \hlcom{# Dets at each site}
\hlkwd{table}\hlstd{(siteDets)}        \hlcom{# Frequency}
\end{alltt}
\begin{verbatim}
## siteDets
##  0  1  2  3  4 
## 52  6 15 22  5
\end{verbatim}
\end{kframe}
\end{knitrout}
\pause
\vfill
\small
Proportion of sites known to be occupied \\
\begin{knitrout}\scriptsize
\definecolor{shadecolor}{rgb}{0.878, 0.918, 0.933}\color{fgcolor}\begin{kframe}
\begin{alltt}
\hlstd{naiveOccupancy} \hlkwb{<-} \hlkwd{sum}\hlstd{(siteDets}\hlopt{>}\hlnum{0}\hlstd{)}\hlopt{/}\hlstd{nSites}
\hlstd{naiveOccupancy}
\end{alltt}
\begin{verbatim}
## [1] 0.48
\end{verbatim}
\end{kframe}
\end{knitrout}

  \end{column}
  \end{columns}
\end{frame}



\section{Prediction}



\begin{frame}
  \frametitle{Outline}
  \Large
  \tableofcontents[currentsection]
\end{frame}



\subsection{Likelihood-based methods}



\begin{frame}[fragile]
  \frametitle{Prepare data in `unmarked'}
  \small
Notice the two new arguments \inr{siteCovs} and \inr{obsCovs}: 
\begin{knitrout}\tiny
\definecolor{shadecolor}{rgb}{0.878, 0.918, 0.933}\color{fgcolor}\begin{kframe}
\begin{alltt}
\hlstd{umf} \hlkwb{<-} \hlkwd{unmarkedFrameOccu}\hlstd{(}\hlkwc{y}\hlstd{=y,} \hlkwc{siteCovs}\hlstd{=}\hlkwd{data.frame}\hlstd{(x1,x2),} \hlkwc{obsCovs}\hlstd{=}\hlkwd{list}\hlstd{(}\hlkwc{w}\hlstd{=w))}
\end{alltt}
\end{kframe}
\end{knitrout}
\pause
%\vfill
%Reformat $w$ as a factor: %, but it's formatted as a matrix of
%characters, we have to reformat it:
%Summary
\begin{knitrout}\tiny
\definecolor{shadecolor}{rgb}{0.878, 0.918, 0.933}\color{fgcolor}\begin{kframe}
\begin{alltt}
\hlkwd{summary}\hlstd{(umf)}
\end{alltt}
\begin{verbatim}
## unmarkedFrame Object
## 
## 100 sites
## Maximum number of observations per site: 4 
## Mean number of observations per site: 4 
## Sites with at least one detection: 48 
## 
## Tabulation of y observations:
##   0   1 
## 278 122 
## 
## Site-level covariates:
##        x1                 x2        
##  Min.   :-1.19272   Min.   : 68.38  
##  1st Qu.:-0.26889   1st Qu.: 95.43  
##  Median : 0.12981   Median : 99.59  
##  Mean   : 0.07007   Mean   :100.26  
##  3rd Qu.: 0.37525   3rd Qu.:107.57  
##  Max.   : 1.25305   Max.   :123.12  
## 
## Observation-level covariates:
##     w      
##  Cold:200  
##  Hot :200
\end{verbatim}
\end{kframe}
\end{knitrout}
\end{frame}



\begin{frame}[fragile]
  \frametitle{Standardizing}
  It's almost always a good idea to standardize \alert{continuous} covariates. \\
  \pause
  \vfill
  Standardizing involves subtracting the mean and then dividing by the standard deviation. \\
  \pause
%  \vfill
\begin{knitrout}\footnotesize
\definecolor{shadecolor}{rgb}{0.878, 0.918, 0.933}\color{fgcolor}\begin{kframe}
\begin{alltt}
\hlstd{mean.x1} \hlkwb{<-} \hlkwd{mean}\hlstd{(x1); mean.x2} \hlkwb{<-} \hlkwd{mean}\hlstd{(x2)}
\hlstd{sd.x1} \hlkwb{<-} \hlkwd{sd}\hlstd{(x1); sd.x2} \hlkwb{<-} \hlkwd{sd}\hlstd{(x2)}
\hlkwd{siteCovs}\hlstd{(umf)}\hlopt{$}\hlstd{x1s} \hlkwb{<-} \hlstd{(x1}\hlopt{-}\hlstd{mean.x1)}\hlopt{/}\hlstd{sd.x1}
\hlkwd{siteCovs}\hlstd{(umf)}\hlopt{$}\hlstd{x2s} \hlkwb{<-} \hlstd{(x2}\hlopt{-}\hlstd{mean.x2)}\hlopt{/}\hlstd{sd.x2}
\end{alltt}
\end{kframe}
\end{knitrout}
%  \pause
%  \vfill
%  If all of your site covariates are continuous, you can use this
%  shortcut with the \inr{scale} function:
%<<umf-zcovs>>=
%siteCovs(umf) <- scale(siteCovs(umf))
%@
  \pause
  \vfill
  Standardizing makes it easier to find the maximum likelihood
  estimates (MLEs) and it facilitates comparison of estimates.  \\
  \pause
  \vfill
  We just have to remember to back-transform covariates when graphing
  predictions.
\end{frame}




\begin{frame}[fragile]
  \frametitle{Fit the model}
  \footnotesize
\begin{knitrout}\tiny
\definecolor{shadecolor}{rgb}{0.878, 0.918, 0.933}\color{fgcolor}\begin{kframe}
\begin{alltt}
\hlstd{fm} \hlkwb{<-} \hlkwd{occu}\hlstd{(}\hlopt{~}\hlstd{x1s}\hlopt{+}\hlstd{w} \hlopt{~}\hlstd{x1s}\hlopt{+}\hlstd{x2s, umf)}    \hlcom{## Notice standardized covariates}
\hlstd{fm}
\end{alltt}
\begin{verbatim}
## 
## Call:
## occu(formula = ~x1s + w ~ x1s + x2s, data = umf)
## 
## Occupancy:
##             Estimate    SE      z P(>|z|)
## (Intercept)  -0.0236 0.214 -0.110  0.9122
## x1s          -0.4991 0.236 -2.115  0.0345
## x2s           0.1580 0.214  0.738  0.4606
## 
## Detection:
##             Estimate    SE     z  P(>|z|)
## (Intercept)   -0.311 0.217 -1.43 1.52e-01
## x1s            0.510 0.195  2.61 8.96e-03
## wHot           2.104 0.365  5.76 8.17e-09
## 
## AIC: 352.8098
\end{verbatim}
\end{kframe}
\end{knitrout}
\pause
\vfill
Compare to actual parameter values:
\begin{knitrout}\tiny
\definecolor{shadecolor}{rgb}{0.878, 0.918, 0.933}\color{fgcolor}\begin{kframe}
\begin{alltt}
\hlkwd{rbind}\hlstd{(}\hlstr{'Occupancy coefs (beta)'}\hlstd{=}\hlkwd{c}\hlstd{(beta0, beta1, beta2),}
      \hlstr{'Detection coefs (alpha)'}\hlstd{=}\hlkwd{c}\hlstd{(alpha0, alpha1, alpha2))}
\end{alltt}
\begin{verbatim}
##                         [,1] [,2] [,3]
## Occupancy coefs (beta)   0.0   -1    0
## Detection coefs (alpha) -0.5    1    2
\end{verbatim}
\end{kframe}
\end{knitrout}
\end{frame}






\begin{frame}[fragile]
  \frametitle{Prediction in `unmarked'}
  Create \texttt{data.frame} with prediction covariates. We'll let $x_1$
  vary while holding other two covariates constant. Important that we
  use the standardized version of the continuous covariates.
\begin{knitrout}\footnotesize
\definecolor{shadecolor}{rgb}{0.878, 0.918, 0.933}\color{fgcolor}\begin{kframe}
\begin{alltt}
\hlstd{pred.data} \hlkwb{<-} \hlkwd{data.frame}\hlstd{(}\hlkwc{x1s}\hlstd{=}\hlkwd{seq}\hlstd{(}\hlkwc{from}\hlstd{=}\hlopt{-}\hlnum{3}\hlstd{,} \hlkwc{to}\hlstd{=}\hlnum{3}\hlstd{,} \hlkwc{length}\hlstd{=}\hlnum{50}\hlstd{),}
                        \hlkwc{x2s}\hlstd{=}\hlnum{0}\hlstd{,} \hlkwc{w}\hlstd{=}\hlstr{'Hot'}\hlstd{)}
\hlstd{pred.data}\hlopt{$}\hlstd{x1} \hlkwb{<-} \hlstd{pred.data}\hlopt{$}\hlstd{x1s}\hlopt{*}\hlstd{sd.x1}\hlopt{+}\hlstd{mean.x1} \hlcom{## Back-transform x1s}
\hlstd{pred.data}\hlopt{$}\hlstd{x2} \hlkwb{<-} \hlstd{pred.data}\hlopt{$}\hlstd{x2s}\hlopt{*}\hlstd{sd.x2}\hlopt{+}\hlstd{mean.x2} \hlcom{## Back-transform x2s}
\end{alltt}
\end{kframe}
\end{knitrout}
\pause
\vfill
Get predictions of $\psi$ for each row of prediction data.
\begin{knitrout}\footnotesize
\definecolor{shadecolor}{rgb}{0.878, 0.918, 0.933}\color{fgcolor}\begin{kframe}
\begin{alltt}
\hlstd{psi.pred} \hlkwb{<-} \hlkwd{predict}\hlstd{(fm,} \hlkwc{newdata}\hlstd{=pred.data,}
                    \hlkwc{type}\hlstd{=}\hlstr{'state'}\hlstd{,} \hlkwc{append}\hlstd{=}\hlnum{TRUE}\hlstd{)}
\end{alltt}
\end{kframe}
\end{knitrout}
\pause
\vfill
Get predictions of $p$ for each row of prediction data.
\begin{knitrout}\footnotesize
\definecolor{shadecolor}{rgb}{0.878, 0.918, 0.933}\color{fgcolor}\begin{kframe}
\begin{alltt}
\hlstd{p.pred} \hlkwb{<-} \hlkwd{predict}\hlstd{(fm,} \hlkwc{newdata}\hlstd{=pred.data,}
                  \hlkwc{type}\hlstd{=}\hlstr{'det'}\hlstd{,} \hlkwc{append}\hlstd{=}\hlnum{TRUE}\hlstd{)}
\end{alltt}
\end{kframe}
\end{knitrout}
\end{frame}



\begin{frame}[fragile]
  \frametitle{Prediction in `unmarked'}
  \small
  View $\psi$ predictions
\begin{knitrout}\footnotesize
\definecolor{shadecolor}{rgb}{0.878, 0.918, 0.933}\color{fgcolor}\begin{kframe}
\begin{alltt}
\hlkwd{print}\hlstd{(}\hlkwd{head}\hlstd{(psi.pred),} \hlkwc{digits}\hlstd{=}\hlnum{2}\hlstd{)}
\end{alltt}
\begin{verbatim}
##   Predicted   SE lower upper  x1s x2s   w   x1  x2
## 1      0.81 0.11  0.50  0.95 -3.0   0 Hot -1.5 100
## 2      0.80 0.11  0.50  0.94 -2.9   0 Hot -1.4 100
## 3      0.79 0.11  0.50  0.94 -2.8   0 Hot -1.4 100
## 4      0.78 0.11  0.49  0.93 -2.6   0 Hot -1.3 100
## 5      0.77 0.11  0.49  0.92 -2.5   0 Hot -1.2 100
## 6      0.76 0.11  0.49  0.92 -2.4   0 Hot -1.2 100
\end{verbatim}
\end{kframe}
\end{knitrout}
\pause
\vfill
  View $p$ predictions
\begin{knitrout}\footnotesize
\definecolor{shadecolor}{rgb}{0.878, 0.918, 0.933}\color{fgcolor}\begin{kframe}
\begin{alltt}
\hlkwd{print}\hlstd{(}\hlkwd{head}\hlstd{(p.pred),} \hlkwc{digits}\hlstd{=}\hlnum{2}\hlstd{)}
\end{alltt}
\begin{verbatim}
##   Predicted   SE lower upper  x1s x2s   w   x1  x2
## 1      0.57 0.14  0.30  0.80 -3.0   0 Hot -1.5 100
## 2      0.58 0.13  0.32  0.80 -2.9   0 Hot -1.4 100
## 3      0.60 0.13  0.35  0.80 -2.8   0 Hot -1.4 100
## 4      0.61 0.12  0.37  0.81 -2.6   0 Hot -1.3 100
## 5      0.63 0.11  0.39  0.81 -2.5   0 Hot -1.2 100
## 6      0.64 0.11  0.42  0.81 -2.4   0 Hot -1.2 100
\end{verbatim}
\end{kframe}
\end{knitrout}
\end{frame}






\begin{frame}[fragile]
  \frametitle{Standardized covariate}
\begin{knitrout}\tiny
\definecolor{shadecolor}{rgb}{0.878, 0.918, 0.933}\color{fgcolor}\begin{kframe}
\begin{alltt}
\hlkwd{plot}\hlstd{(Predicted} \hlopt{~} \hlstd{x1s, psi.pred,} \hlkwc{type}\hlstd{=}\hlstr{"l"}\hlstd{,} \hlkwc{ylab}\hlstd{=}\hlstr{"Occurrence probability"}\hlstd{,} \hlkwc{col}\hlstd{=}\hlstr{"blue"}\hlstd{,}
     \hlkwc{xlab}\hlstd{=}\hlstr{"Standardized covariate (x1s)"}\hlstd{,} \hlkwc{ylim}\hlstd{=}\hlnum{0}\hlopt{:}\hlnum{1}\hlstd{)}
\hlkwd{lines}\hlstd{(lower} \hlopt{~} \hlstd{x1s, psi.pred,} \hlkwc{col}\hlstd{=}\hlstr{"grey"}\hlstd{);} \hlkwd{lines}\hlstd{(upper} \hlopt{~} \hlstd{x1s, psi.pred,} \hlkwc{col}\hlstd{=}\hlstr{"grey"}\hlstd{)}
\end{alltt}
\end{kframe}

{\centering \includegraphics[width=0.8\linewidth]{figure/pred-psi1-1} 

}


\end{knitrout}
\end{frame}




\begin{frame}[fragile]
  \frametitle{Covariate on original scale}
\begin{knitrout}\tiny
\definecolor{shadecolor}{rgb}{0.878, 0.918, 0.933}\color{fgcolor}\begin{kframe}
\begin{alltt}
\hlkwd{plot}\hlstd{(Predicted} \hlopt{~} \hlstd{x1, psi.pred,} \hlkwc{type}\hlstd{=}\hlstr{"l"}\hlstd{,} \hlkwc{ylab}\hlstd{=}\hlstr{"Occurrence probability"}\hlstd{,} \hlkwc{col}\hlstd{=}\hlstr{"blue"}\hlstd{,}
     \hlkwc{xlab}\hlstd{=}\hlstr{"Original scale covariate (x1)"}\hlstd{,} \hlkwc{ylim}\hlstd{=}\hlnum{0}\hlopt{:}\hlnum{1}\hlstd{)}
\hlkwd{lines}\hlstd{(lower} \hlopt{~} \hlstd{x1, psi.pred,} \hlkwc{col}\hlstd{=}\hlstr{"grey"}\hlstd{);} \hlkwd{lines}\hlstd{(upper} \hlopt{~} \hlstd{x1, psi.pred,} \hlkwc{col}\hlstd{=}\hlstr{"grey"}\hlstd{)}
\end{alltt}
\end{kframe}

{\centering \includegraphics[width=0.8\linewidth]{figure/pred-psi1s-1} 

}


\end{knitrout}
\end{frame}




% \begin{frame}[fragile]
%   \frametitle{Prediction in `unmarked'}
% <<pred-p1,fig.width=7,fig.height=5,size='tiny',out.width='80%',fig.align='center',echo=-1>>=
% par(mai=c(0.9,0.9,0.1,0.1))  
% plot(Predicted ~ x1s, p.pred, type="l", ylab="Detection probability", col="purple",
%      xlab="Original scale covariate (x1)", ylim=0:1, xaxt="n")
% axis(side=1, at=x1s.ticks, labels=round(x1s.ticks*sd(x1)+mean(x1),1)) ## Backtransform x1s
% lines(lower ~ x1s, p.pred, col="grey")
% lines(upper ~ x1s, p.pred, col="grey")
% @   
% \end{frame}


% \begin{frame}[fragile]
%   \frametitle{Prediction in `unmarked'}
% <<pred-plot1,fig.width=7,fig.height=5.5,size='scriptsize',out.width='80%',fig.align='center'>>=
% plot(Predicted ~ x1, psi.pred, type="l", ylab="Probability", col="blue")
% ## lines(Predicted ~ x1, p.pred, col="grey")
% legend(-3, 0.75, c("psi", ""), lty=c(1, NA), col=c("blue", NA))
% @   
% \end{frame}


\begin{frame}[fragile]
  \frametitle{Prediction in `unmarked'}
\begin{knitrout}\tiny
\definecolor{shadecolor}{rgb}{0.878, 0.918, 0.933}\color{fgcolor}\begin{kframe}
\begin{alltt}
\hlkwd{plot}\hlstd{(Predicted} \hlopt{~} \hlstd{x1, psi.pred,} \hlkwc{type}\hlstd{=}\hlstr{"l"}\hlstd{,} \hlkwc{ylab}\hlstd{=}\hlstr{"Probability"}\hlstd{,} \hlkwc{col}\hlstd{=}\hlstr{"blue"}\hlstd{,} \hlkwc{ylim}\hlstd{=}\hlnum{0}\hlopt{:}\hlnum{1}\hlstd{,}
     \hlkwc{xlab}\hlstd{=}\hlstr{"x1"}\hlstd{,} \hlkwc{lwd}\hlstd{=}\hlnum{2}\hlstd{)}
\hlkwd{lines}\hlstd{(Predicted} \hlopt{~} \hlstd{x1, p.pred,} \hlkwc{col}\hlstd{=}\hlstr{"orange"}\hlstd{,} \hlkwc{lwd}\hlstd{=}\hlnum{2}\hlstd{)}
\hlkwd{legend}\hlstd{(}\hlopt{-}\hlnum{3}\hlstd{,} \hlnum{0.3}\hlstd{,} \hlkwd{c}\hlstd{(}\hlstr{"Occupancy (psi)"}\hlstd{,} \hlstr{"Detection (p)"}\hlstd{),} \hlkwc{lty}\hlstd{=}\hlkwd{c}\hlstd{(}\hlnum{1}\hlstd{,} \hlnum{1}\hlstd{),} \hlkwc{col}\hlstd{=}\hlkwd{c}\hlstd{(}\hlstr{"blue"}\hlstd{,} \hlstr{"orange"}\hlstd{),} \hlkwc{lwd}\hlstd{=}\hlnum{2}\hlstd{)}
\end{alltt}
\end{kframe}

{\centering \includegraphics[width=0.7\linewidth]{figure/pred-plot2-1} 

}


\end{knitrout}
%\pause
\small
\centering
Major problems if you ignore imperfect detection in this case \\
\end{frame}





% \begin{frame}[fragile]
%   \frametitle{R package `unmarked'}
%   How many sites were occupied? \pause
%   We can answer this using posterior prediction.
% <<bup-hist,size='tiny',fig.width=7,fig.height=5,out.width='0.7\\textwidth',fig.align='center'>>=
% nsim <- 1000
% sites.occupied.post <- predict(z.post, func=function(x) sum(x), nsim=nsim)
% par(mai=c(0.9,0.9,0.1,0.1))
% plot(table(sites.occupied.post)/nsim, lwd=5, xlab="Sites occupied",
%     ylab="Empirical Bayes posterior probability")
% abline(v=sum(z), col="red") ## Actual number occupied
% @   
% \end{frame}






\begin{frame}
  \frametitle{In-class exercise}
  \small
  \begin{enumerate}
    \item Fit this model (to the simulated data):
      \begin{gather*}
        \mathrm{logit}(\psi_i) = \beta_0 + \beta_1 {\color{blue} x_{i1}} \\
        z_i \sim \mathrm{Bern}(\psi_i) \\
%      \end{gather*}
%      \begin{gather*}
        \mathrm{logit}(p_{ij}) = \alpha_0 + \alpha_1 {\color{blue} x_{i1}} +
        \alpha_2 {\color{Purple} w_{ij}} \\
        y_{ij} \sim \mathrm{Bern}(z_i\times p_{ij})
      \end{gather*}
%      \pause
    \item Predict $p$ again by letting (standardized) $x_1$ vary, but
      this time, predict for the case where $w$=`Cold'.
    \item Graph $p$ as a function of $x_1$.
    \item (Bonus) What happens when a covariate is missing for a site or observation?
  \end{enumerate}
\end{frame}




\subsection{Bayesian methods}



\begin{frame}[fragile]
  \frametitle{The BUGS model}
\begin{knitrout}\scriptsize
\definecolor{shadecolor}{rgb}{0.678, 0.847, 0.902}\color{fgcolor}\begin{kframe}
\begin{verbatim}
model {

# Priors for occupancy coefficients
beta0 ~ dnorm(0, 0.5)  # 0.5 is 1/variance
beta1 ~ dnorm(0, 0.5)
beta2 ~ dnorm(0, 0.5)

# Priors for detection coefficients
alpha0 ~ dnorm(0, 0.5)  
alpha1 ~ dnorm(0, 0.5)
alpha2 ~ dnorm(0, 0.5)

for(i in 1:nSites) {
  logit(psi[i]) <- beta0 + beta1*x1[i] + beta2*x2[i]
  z[i] ~ dbern(psi[i])            # Latent presence/absence
  for(j in 1:nOccasions) {
    logit(p[i,j]) <- alpha0 + alpha1*x1[i] + alpha2*wHot[i,j]
    y[i,j] ~ dbern(z[i]*p[i,j])   # Data
  }
}

sitesOccupied <- sum(z)

}
\end{verbatim}
\end{kframe}
\end{knitrout}
\end{frame}



%% TODO: Add something here about priors and prior predictive checks



\begin{frame}[fragile]
  \frametitle{Data, inits, and parameters}
  Put data in a named list
  \vspace{-12pt}
\begin{knitrout}\small
\definecolor{shadecolor}{rgb}{0.878, 0.918, 0.933}\color{fgcolor}\begin{kframe}
\begin{alltt}
\hlstd{jags.data} \hlkwb{<-} \hlkwd{list}\hlstd{(}\hlkwc{y}\hlstd{=y,} \hlkwc{x1}\hlstd{=(x1}\hlopt{-}\hlkwd{mean}\hlstd{(x1))}\hlopt{/}\hlkwd{sd}\hlstd{(x1),}
                  \hlkwc{x2}\hlstd{=(x2}\hlopt{-}\hlkwd{mean}\hlstd{(x2))}\hlopt{/}\hlkwd{sd}\hlstd{(x2),} \hlkwc{wHot}\hlstd{=wHot,}
                  \hlkwc{nSites}\hlstd{=nSites,} \hlkwc{nOccasions}\hlstd{=nVisits)}
\end{alltt}
\end{kframe}
\end{knitrout}
\pause
\vfill
  Initial values
  \vspace{-12pt}
\begin{knitrout}\small
\definecolor{shadecolor}{rgb}{0.878, 0.918, 0.933}\color{fgcolor}\begin{kframe}
\begin{alltt}
\hlstd{jags.inits} \hlkwb{<-} \hlkwa{function}\hlstd{() \{}
    \hlkwd{list}\hlstd{(}\hlkwc{beta0}\hlstd{=}\hlkwd{rnorm}\hlstd{(}\hlnum{1}\hlstd{),} \hlkwc{alpha0}\hlstd{=}\hlkwd{rnorm}\hlstd{(}\hlnum{1}\hlstd{),} \hlkwc{z}\hlstd{=}\hlkwd{rep}\hlstd{(}\hlnum{1}\hlstd{, nSites))}
\hlstd{\}}
\end{alltt}
\end{kframe}
\end{knitrout}
\pause
\vfill
  Parameters to monitor
  \vspace{-12pt}
\begin{knitrout}\small
\definecolor{shadecolor}{rgb}{0.878, 0.918, 0.933}\color{fgcolor}\begin{kframe}
\begin{alltt}
\hlstd{jags.pars} \hlkwb{<-} \hlkwd{c}\hlstd{(}\hlstr{"beta0"}\hlstd{,} \hlstr{"beta1"}\hlstd{,} \hlstr{"beta2"}\hlstd{,}
               \hlstr{"alpha0"}\hlstd{,} \hlstr{"alpha1"}\hlstd{,} \hlstr{"alpha2"}\hlstd{,} \hlstr{"sitesOccupied"}\hlstd{)}
\end{alltt}
\end{kframe}
\end{knitrout}
\end{frame}





\begin{frame}[fragile]
  \frametitle{MCMC}
  \small
\begin{knitrout}\scriptsize
\definecolor{shadecolor}{rgb}{0.878, 0.918, 0.933}\color{fgcolor}\begin{kframe}
\begin{alltt}
\hlkwd{library}\hlstd{(jagsUI)}
\hlstd{jags.post.samples} \hlkwb{<-} \hlkwd{jags.basic}\hlstd{(}\hlkwc{data}\hlstd{=jags.data,} \hlkwc{inits}\hlstd{=jags.inits,}
                                \hlkwc{parameters.to.save}\hlstd{=jags.pars,}
                                \hlkwc{model.file}\hlstd{=}\hlstr{"occupancy-model-covs.jag"}\hlstd{,}
                                \hlkwc{n.chains}\hlstd{=}\hlnum{3}\hlstd{,} \hlkwc{n.adapt}\hlstd{=}\hlnum{100}\hlstd{,} \hlkwc{n.burnin}\hlstd{=}\hlnum{0}\hlstd{,}
                                \hlkwc{n.iter}\hlstd{=}\hlnum{2000}\hlstd{,} \hlkwc{parallel}\hlstd{=}\hlnum{TRUE}\hlstd{)}
\end{alltt}
\begin{verbatim}
## 
## Processing function input....... 
## 
## Done. 
##  
## Beginning parallel processing using 3 cores. Console output will be suppressed.
## 
## Parallel processing completed.
## 
## MCMC took 0.032 minutes.
\end{verbatim}
\end{kframe}
\end{knitrout}
\end{frame}



\begin{frame}[fragile]
  \frametitle{Summarize output}
\begin{knitrout}\tiny
\definecolor{shadecolor}{rgb}{0.878, 0.918, 0.933}\color{fgcolor}\begin{kframe}
\begin{alltt}
\hlkwd{summary}\hlstd{(jags.post.samples)}
\end{alltt}
\begin{verbatim}
## 
## Iterations = 1:2000
## Thinning interval = 1 
## Number of chains = 3 
## Sample size per chain = 2000 
## 
## 1. Empirical mean and standard deviation for each variable,
##    plus standard error of the mean:
## 
##                    Mean     SD Naive SE Time-series SE
## alpha0         -0.26914 0.2135 0.002756       0.005381
## alpha1          0.49913 0.1908 0.002463       0.004038
## alpha2          1.99823 0.3482 0.004495       0.009214
## beta0          -0.01505 0.2151 0.002777       0.003597
## beta1          -0.53200 0.2458 0.003173       0.004934
## beta2           0.16682 0.2210 0.002854       0.003620
## deviance      222.01755 8.2574 0.106603       0.150209
## sitesOccupied  49.60950 1.3475 0.017396       0.024891
## 
## 2. Quantiles for each variable:
## 
##                   2.5%       25%       50%      75%     97.5%
## alpha0         -0.6920  -0.41247  -0.26462  -0.1225   0.15099
## alpha1          0.1322   0.36710   0.49489   0.6275   0.88459
## alpha2          1.3163   1.76663   1.99315   2.2274   2.69170
## beta0          -0.4343  -0.16327  -0.01771   0.1299   0.41113
## beta1          -1.0529  -0.68947  -0.52150  -0.3643  -0.07513
## beta2          -0.2737   0.02489   0.16755   0.3148   0.60205
## deviance      210.7703 215.77769 220.64195 226.7527 241.24989
## sitesOccupied  48.0000  49.00000  49.00000  50.0000  53.00000
\end{verbatim}
\end{kframe}
\end{knitrout}
\end{frame}




\begin{frame}[fragile]
  \frametitle{Traceplots and density plots}
\begin{knitrout}\footnotesize
\definecolor{shadecolor}{rgb}{0.878, 0.918, 0.933}\color{fgcolor}\begin{kframe}
\begin{alltt}
\hlkwd{plot}\hlstd{(jags.post.samples[,}\hlnum{1}\hlopt{:}\hlnum{3}\hlstd{])}
\end{alltt}
\end{kframe}

{\centering \includegraphics[width=0.7\textwidth]{figure/bugs-plot1-1} 

}


\end{knitrout}
\end{frame}



\begin{frame}[fragile]
  \frametitle{Traceplots and density plots}
\begin{knitrout}\footnotesize
\definecolor{shadecolor}{rgb}{0.878, 0.918, 0.933}\color{fgcolor}\begin{kframe}
\begin{alltt}
\hlkwd{plot}\hlstd{(jags.post.samples[,}\hlkwd{c}\hlstd{(}\hlnum{4}\hlopt{:}\hlnum{6}\hlstd{,}\hlnum{8}\hlstd{)])}
\end{alltt}
\end{kframe}

{\centering \includegraphics[width=0.7\textwidth]{figure/bugs-plot2-1} 

}


\end{knitrout}
\end{frame}


\begin{frame}
  \frametitle{Bayesian prediction}
  Every MCMC iteration represents a sample from the posterior
  distribution. \\
  \pause
  \vfill
  Each sample of the occupancy and detection coefficients can be used
  to make a prediction. \\
  \pause
  \vfill
  The collection of predictions can be used to summarize the posterior 
  predictive distribution. \\
  \pause
  \vfill
  We will look at the distribution of prediction lines. \\  
\end{frame}


\begin{frame}[fragile]
  \frametitle{Bayesian prediction}
  \small
  First, extract the $\psi$ coefficients
\begin{knitrout}\scriptsize
\definecolor{shadecolor}{rgb}{0.878, 0.918, 0.933}\color{fgcolor}\begin{kframe}
\begin{alltt}
\hlstd{psi.coef.post} \hlkwb{<-} \hlkwd{as.matrix}\hlstd{(jags.post.samples[,}\hlkwd{c}\hlstd{(}\hlstr{"beta0"}\hlstd{,}\hlstr{"beta1"}\hlstd{,}\hlstr{"beta2"}\hlstd{)])}
\hlkwd{head}\hlstd{(psi.coef.post,} \hlkwc{n}\hlstd{=}\hlnum{4}\hlstd{)}
\end{alltt}
\begin{verbatim}
##            beta0      beta1       beta2
## [1,]  0.09274697 -0.2716474  0.15535692
## [2,] -0.19173696 -0.6637521 -0.05146374
## [3,]  0.14213546 -0.6215908  0.14574879
## [4,]  0.16840311 -0.9722204  0.27184547
\end{verbatim}
\end{kframe}
\end{knitrout}
  \pause
  \vfill
  Create prediction matrix, one row for each MCMC iteration.
%  Columns represent covariate values. 
\begin{knitrout}\scriptsize
\definecolor{shadecolor}{rgb}{0.878, 0.918, 0.933}\color{fgcolor}\begin{kframe}
\begin{alltt}
\hlstd{n.iter} \hlkwb{<-} \hlkwd{nrow}\hlstd{(psi.coef.post)}
\hlstd{psi.post.pred} \hlkwb{<-} \hlkwd{matrix}\hlstd{(}\hlnum{NA}\hlstd{,} \hlkwc{nrow}\hlstd{=n.iter,} \hlkwc{ncol}\hlstd{=}\hlkwd{nrow}\hlstd{(pred.data))}
\end{alltt}
\end{kframe}
\end{knitrout}
  \pause
  \vfill
  Predict $\psi$ for each MCMC iteration.
%  using covariate values from \inr{pred.data}. 
\begin{knitrout}\scriptsize
\definecolor{shadecolor}{rgb}{0.878, 0.918, 0.933}\color{fgcolor}\begin{kframe}
\begin{alltt}
\hlkwa{for}\hlstd{(i} \hlkwa{in} \hlnum{1}\hlopt{:}\hlstd{n.iter) \{}
    \hlstd{psi.post.pred[i,]} \hlkwb{<-} \hlkwd{plogis}\hlstd{(psi.coef.post[i,}\hlstr{"beta0"}\hlstd{]} \hlopt{+}
                                \hlstd{psi.coef.post[i,}\hlstr{"beta1"}\hlstd{]}\hlopt{*}\hlstd{pred.data}\hlopt{$}\hlstd{x1s)}
\hlstd{\}}
\end{alltt}
\end{kframe}
\end{knitrout}
\end{frame}



\begin{frame}[fragile]
  \frametitle{Bayesian prediction}
  Prediction line for first posterior sample
\begin{knitrout}\scriptsize
\definecolor{shadecolor}{rgb}{0.878, 0.918, 0.933}\color{fgcolor}\begin{kframe}
\begin{alltt}
\hlkwd{plot}\hlstd{(pred.data}\hlopt{$}\hlstd{x1, psi.post.pred[}\hlnum{1}\hlstd{,],} \hlkwc{type}\hlstd{=}\hlstr{"l"}\hlstd{,} \hlkwc{xlab}\hlstd{=}\hlstr{"x1s"}\hlstd{,}
     \hlkwc{ylab}\hlstd{=}\hlstr{"Occurrence probability"}\hlstd{,} \hlkwc{ylim}\hlstd{=}\hlkwd{c}\hlstd{(}\hlnum{0}\hlstd{,} \hlnum{1}\hlstd{),} \hlkwc{col}\hlstd{=}\hlkwd{gray}\hlstd{(}\hlnum{0.8}\hlstd{))}
\end{alltt}
\end{kframe}

{\centering \includegraphics[width=0.8\linewidth]{figure/psi-pred1-1} 

}


\end{knitrout}
\end{frame}



\begin{frame}[fragile]
  \frametitle{Bayesian prediction}
  All samples from the posterior predictive distribution
\begin{knitrout}\scriptsize
\definecolor{shadecolor}{rgb}{0.878, 0.918, 0.933}\color{fgcolor}\begin{kframe}
\begin{alltt}
\hlkwa{for}\hlstd{(i} \hlkwa{in} \hlnum{1}\hlopt{:}\hlstd{n.iter) \{}
    \hlkwd{lines}\hlstd{(pred.data}\hlopt{$}\hlstd{x1, psi.post.pred[i,],} \hlkwc{col}\hlstd{=}\hlkwd{gray}\hlstd{(}\hlnum{0.8}\hlstd{))}
\hlstd{\}}
\end{alltt}
\end{kframe}

{\centering \includegraphics[width=0.8\linewidth]{figure/psi-pred-post-1} 

}


\end{knitrout}
\end{frame}





\begin{frame}[fragile]
  \frametitle{Bayesian prediction}
  Now with posterior mean and 95\% CI
\begin{knitrout}\tiny
\definecolor{shadecolor}{rgb}{0.878, 0.918, 0.933}\color{fgcolor}\begin{kframe}
\begin{alltt}
\hlstd{pred.post.mean} \hlkwb{<-} \hlkwd{colMeans}\hlstd{(psi.post.pred)}
\hlstd{pred.post.lower} \hlkwb{<-} \hlkwd{apply}\hlstd{(psi.post.pred,} \hlnum{2}\hlstd{, quantile,} \hlkwc{prob}\hlstd{=}\hlnum{0.025}\hlstd{)}
\hlstd{pred.post.upper} \hlkwb{<-} \hlkwd{apply}\hlstd{(psi.post.pred,} \hlnum{2}\hlstd{, quantile,} \hlkwc{prob}\hlstd{=}\hlnum{0.975}\hlstd{)}
\hlkwd{lines}\hlstd{(pred.data}\hlopt{$}\hlstd{x1, pred.post.mean,} \hlkwc{col}\hlstd{=}\hlstr{"blue"}\hlstd{)}
\hlkwd{lines}\hlstd{(pred.data}\hlopt{$}\hlstd{x1, pred.post.lower,} \hlkwc{col}\hlstd{=}\hlstr{"blue"}\hlstd{,} \hlkwc{lty}\hlstd{=}\hlnum{2}\hlstd{)}
\hlkwd{lines}\hlstd{(pred.data}\hlopt{$}\hlstd{x1, pred.post.upper,} \hlkwc{col}\hlstd{=}\hlstr{"blue"}\hlstd{,} \hlkwc{lty}\hlstd{=}\hlnum{2}\hlstd{)}
\end{alltt}
\end{kframe}

{\centering \includegraphics[width=0.8\linewidth]{figure/psi-pred-post-meanCI-1} 

}


\end{knitrout}
\end{frame}




\begin{frame}
  \frametitle{Summary of static occupancy models}
  \small
  Assumptions are made explicit by the model description, but it's worth emphasizing that: 
  \begin{itemize}[<+->]
    \item<2-> Occupancy is assumed to remain constant at each site during the sampling period
    \item<3-> The definition of a `site' is critical, but can difficult in
      some situations, such as in camera studies.
    \item<4-> Abundance is assumed to be unimportant
    \item<5-> Observations are assumed to be independent, conditional
      on latent variables and covariates
  \end{itemize}
  \vfill
  \uncover<6->{  
    These assumptions can be relaxed if we have enough data. 
  }
  \begin{itemize}
    \item<7-> Dynamic occupancy models
    \begin{itemize}
      \item Occupancy evolves as a function of local colonization
        and extinction
    \end{itemize}
    \item<8-> $N$-mixture models
    \begin{itemize}
      \item Focus is on local abundance, rather than presence/absence
    \end{itemize}
  \end{itemize}
  \vfill
  \uncover<9->{We'll talk more about design issues later}
\end{frame}



\section{Assignment}




\begin{frame}
  \frametitle{Assignment}
  % \small
  \footnotesize
  Create a self-contained R script or Rmarkdown file
  to do the following:
  \vfill
  \begin{enumerate}
%    \small
    \footnotesize
    % \item Fit 3 covariate models in `unmarked' to the Canada Warbler data. 
    %   \begin{itemize}
    %     \footnotesize
    %     \item Response: \texttt{cawa1, cawa2, cawa3, cawa4}
    %     \item Site covs: \texttt{Elevation, Wind, Noise}
    %   \end{itemize}
    \item Fit 3 covariate models in `unmarked' to the Ruffed Grouse data ({\tt grouse\_data.csv}). 
      \begin{itemize}
        \footnotesize
        \item Response: \texttt{grouse1, grouse2, grouse3}
        \item Site covs: Elevation (\texttt{elev}), forest cover ({\tt fc})
        \item Obs cov: Day of the year (\texttt{doy.1, doy.2, doy.3})
      \end{itemize}
    \item Make one prediction graph for occupancy and one for detection. 
    \item Create the same graphs as above, but using Bayesian methods.
    \item What differences do you see between likelihood and Bayesian graphs?
    \end{enumerate}
    \vfill
    Upload your {\tt .R} or {\tt .Rmd} file to ELC before Monday. 
\end{frame}





\end{document}

