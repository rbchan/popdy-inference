\documentclass[color=usenames,dvipsnames]{beamer}\usepackage[]{graphicx}\usepackage[]{color}
% maxwidth is the original width if it is less than linewidth
% otherwise use linewidth (to make sure the graphics do not exceed the margin)
\makeatletter
\def\maxwidth{ %
  \ifdim\Gin@nat@width>\linewidth
    \linewidth
  \else
    \Gin@nat@width
  \fi
}
\makeatother

\definecolor{fgcolor}{rgb}{0, 0, 0}
\newcommand{\hlnum}[1]{\textcolor[rgb]{0.69,0.494,0}{#1}}%
\newcommand{\hlstr}[1]{\textcolor[rgb]{0.749,0.012,0.012}{#1}}%
\newcommand{\hlcom}[1]{\textcolor[rgb]{0.514,0.506,0.514}{\textit{#1}}}%
\newcommand{\hlopt}[1]{\textcolor[rgb]{0,0,0}{#1}}%
\newcommand{\hlstd}[1]{\textcolor[rgb]{0,0,0}{#1}}%
\newcommand{\hlkwa}[1]{\textcolor[rgb]{0,0,0}{\textbf{#1}}}%
\newcommand{\hlkwb}[1]{\textcolor[rgb]{0,0.341,0.682}{#1}}%
\newcommand{\hlkwc}[1]{\textcolor[rgb]{0,0,0}{\textbf{#1}}}%
\newcommand{\hlkwd}[1]{\textcolor[rgb]{0.004,0.004,0.506}{#1}}%
\let\hlipl\hlkwb

\usepackage{framed}
\makeatletter
\newenvironment{kframe}{%
 \def\at@end@of@kframe{}%
 \ifinner\ifhmode%
  \def\at@end@of@kframe{\end{minipage}}%
  \begin{minipage}{\columnwidth}%
 \fi\fi%
 \def\FrameCommand##1{\hskip\@totalleftmargin \hskip-\fboxsep
 \colorbox{shadecolor}{##1}\hskip-\fboxsep
     % There is no \\@totalrightmargin, so:
     \hskip-\linewidth \hskip-\@totalleftmargin \hskip\columnwidth}%
 \MakeFramed {\advance\hsize-\width
   \@totalleftmargin\z@ \linewidth\hsize
   \@setminipage}}%
 {\par\unskip\endMakeFramed%
 \at@end@of@kframe}
\makeatother

\definecolor{shadecolor}{rgb}{.97, .97, .97}
\definecolor{messagecolor}{rgb}{0, 0, 0}
\definecolor{warningcolor}{rgb}{1, 0, 1}
\definecolor{errorcolor}{rgb}{1, 0, 0}
\newenvironment{knitrout}{}{} % an empty environment to be redefined in TeX

\usepackage{alltt}
%\documentclass[color=usenames,dvipsnames,handout]{beamer}

\usepackage[roman]{../lectures}
%\usepackage[sans]{../lectures}


\hypersetup{pdfpagemode=UseNone,pdfstartview={FitV}}



% Load function to compile and open PDF


% Compile and open PDF




% New command for inline code that isn't to be evaluated
\definecolor{inlinecolor}{rgb}{0.878, 0.918, 0.933}
\newcommand{\inr}[1]{\colorbox{inlinecolor}{\texttt{#1}}}




\newcommand{\bxt}{${\bm x}_j$}
\newcommand{\bx}{{\bm x}}
\newcommand{\bxj}{{\bm x}_j}
\newcommand{\bst}{${\bm s}_i$}
\newcommand{\bs}{{\bm s}}
\newcommand{\bsi}{{\bm s}_i}
\newcommand{\ed}{\|\bx - \bs\|}
\newcommand{\cs}{\mathcal{S} }
\newcommand{\dsixj}{\|\bsi - \bxj\|}
\IfFileExists{upquote.sty}{\usepackage{upquote}}{}
\begin{document}




\begin{frame}[plain]
  \LARGE
  \centering
  {
    \LARGE Lecture 12 -- Spatial capture-recapture \\
    for closed populations: \\
    \Large Part II: mapping density surfaces \\
  }
  {\color{default} \rule{\textwidth}{0.1pt} }
  \vfill
  \large
  WILD(FISH) 8390 \\
  Estimation of Fish and Wildlife Population Parameters \\
  \vfill
  \large
  Richard Chandler \\
  University of Georgia \\
\end{frame}






\section{Overview}



\begin{frame}[plain]
  \frametitle{Outline}
  \Large
  \only<1>{\tableofcontents}%[hideallsubsections]}
  \only<2 | handout:0>{\tableofcontents[currentsection]}%,hideallsubsections]}
\end{frame}




\begin{frame}
  \frametitle{SCR overview}
  Last time, we focused on simulation and basic model fitting. \\
  \pause \vfill
  This time, we're going to talk about estimating and mapping density surfaces. \\
%  \pause \vfill
%  And you'll learn how to create maps of 
\end{frame}



\begin{frame}
  \frametitle{SCR overview}
  {\centering Two motivations for SCR \\}
  \vfill
  \begin{enumerate}
    \item Improved inference
    \begin{itemize}
      \item<1-> Non-spatial models can't properly account for sources
        of variation in $p$ that can cause bias.
        \begin{itemize}
          \item<1-> Distance to traps
          \item<1-> Trap-specific covariates
        \end{itemize}
      \item<1-> SCR makes it possible to estimate \alert{density}, not
        just $N$ in an unknown region. 
    \end{itemize}
%    \pause
    \vfill
  \item<1-> Improved science
  \begin{itemize}
    \item<1-> We can ask new questions, such as:
      \begin{itemize}
        \item<1-> What influences spatial variation in density?
        \item<1-> How do survival and recruitment vary in space and time?
        \item<1-> How does movement influence density and detectability?
      \end{itemize}
    \item<1-> Rather than think of SCR as a new estimation tool, you
      can think of it as an individual-based framework for inference on
      spatial population dynamics.
    \end{itemize}
  \end{enumerate}
\end{frame}




% \begin{frame}
%   \frametitle{In-class exercise}
%   Building off the previous example\dots
%   \begin{enumerate}
%     \item Compute $\bar{p}$ for line-transect sampling when
%       $\sigma=50, 100, \mathrm{and}\, 200$, instead of $\sigma=25$.  
%     \item Repeat, but for point-transect sampling. 
%   \end{enumerate}
% \end{frame}





\begin{frame}
  \frametitle{\large Closed population model ($N$ known hypothetically) }
  \footnotesize
  State model (a spatial point process model) %\\
  \begin{gather*}
    \lambda(\bs) = \beta_0 + \beta_1 w_1(\bs) + \beta_2 w_2(\bs) \dots \\
    \Lambda = \int_{\mathcal{S}} \lambda(\bs) \; \mathrm{d}\bs \\
    N \sim \mathrm{Pois}(\Lambda) \\
    \bsi \propto p(\lambda(\bs)) \;\; \mathrm{for}\; i=1,\dots,N 
  \end{gather*}
%  \pause
%  \vfill
  Observation model (supposing $N$ was known)
  \begin{gather*}
    p_{ij} = g_0\exp(-\|\bsi - \bxj\|^2/(2\sigma^2))  \;\; \mathrm{for}\, j=1,\dots,J  \\
    y_{ijk} \sim \mathrm{Bernoulli}(p_{ij})
  \end{gather*}
%  \pause
%  \vfill
%  \footnotesize
  \scriptsize
  Definitions \\
  \hangindent=0.9cm $\lambda(\bs)$ -- The ``intensity function'' %or ``density surface''
  describing the density of individuals at location $\bs$ \\ 
  $\Lambda$ -- Expected number of individuals \\
  $N$ -- Realized number of individuals (ie, population size) \\
  $\bsi$ -- Location of the $i$th activity center \\
  $\bxj$ -- Location of trap $j$ \\
  $\dsixj$ -- Euclidean distance between $\bsi$ and $\bxj$ \\
  $g_0$ -- Capture probability when distance between activity centers
  and traps is 0 \\
  $\sigma$ -- Scale parameter of encounter function \\
  $p_{ij}$ -- Capture probability \\
  $y_{ijk}$ -- Spatial capture histories \\
\end{frame}







\section{Simulation}


\begin{frame}
  \frametitle{Outline}
  \Large
%  \tableofcontents[currentsection,currentsubsection]
  \tableofcontents[currentsection]
\end{frame}






\begin{frame}[fragile]
  \frametitle{Inhomogeneous Poisson point process}
  First, let's import a raster layer
\begin{knitrout}\footnotesize
\definecolor{shadecolor}{rgb}{0.878, 0.918, 0.933}\color{fgcolor}\begin{kframe}
\begin{alltt}
\hlkwd{library}\hlstd{(raster)}
\hlstd{elevation} \hlkwb{<-} \hlkwd{raster}\hlstd{(}\hlstr{"elevation.tif"}\hlstd{)}
\hlkwd{plot}\hlstd{(elevation,} \hlkwc{col}\hlstd{=}\hlkwd{topo.colors}\hlstd{(}\hlnum{100}\hlstd{),} \hlkwc{main}\hlstd{=}\hlstr{"Elevation"}\hlstd{)}
\end{alltt}
\end{kframe}

{\centering \includegraphics[width=0.6\linewidth]{figure/ippp1-1} 

}



\end{knitrout}
\end{frame}




\begin{frame}[fragile]
  \frametitle{Inhomogeneous Poisson point process}
  \small
  Second, let's pick some coefficients and create a density surface
\begin{knitrout}\footnotesize
\definecolor{shadecolor}{rgb}{0.878, 0.918, 0.933}\color{fgcolor}\begin{kframe}
\begin{alltt}
\hlstd{beta0} \hlkwb{<-} \hlopt{-}\hlnum{15}
\hlstd{beta1} \hlkwb{<-} \hlnum{0.01} \hlcom{#0.005}
\hlstd{lambda} \hlkwb{<-} \hlkwd{exp}\hlstd{(beta0} \hlopt{+} \hlstd{beta1}\hlopt{*}\hlstd{elevation)} \hlcom{# Intensity function}
\hlkwd{plot}\hlstd{(lambda,} \hlkwc{col}\hlstd{=}\hlkwd{terrain.colors}\hlstd{(}\hlnum{100}\hlstd{),} \hlkwc{main}\hlstd{=}\hlstr{"Density surface"}\hlstd{)}
\end{alltt}
\end{kframe}

{\centering \includegraphics[width=0.6\linewidth]{figure/ippp2-1} 

}



\end{knitrout}
\end{frame}




\begin{frame}[fragile]
  \frametitle{Inhomogeneous Poisson point process}
  \small
  Third, simulate $N$
  \vspace{-6pt}
\begin{knitrout}\footnotesize
\definecolor{shadecolor}{rgb}{0.878, 0.918, 0.933}\color{fgcolor}\begin{kframe}
\begin{alltt}
\hlkwd{set.seed}\hlstd{(}\hlnum{538}\hlstd{)}
\hlstd{ds} \hlkwb{<-} \hlnum{1}                            \hlcom{## Pixel area is 1 ha}
\hlstd{lambda.values} \hlkwb{<-} \hlkwd{values}\hlstd{(lambda)}    \hlcom{## Convert raster to vector}
\hlstd{Lambda} \hlkwb{<-} \hlkwd{sum}\hlstd{(lambda.values}\hlopt{*}\hlstd{ds)}    \hlcom{## E(N)}
\hlstd{(N} \hlkwb{<-} \hlkwd{rpois}\hlstd{(}\hlnum{1}\hlstd{, Lambda))}            \hlcom{## Realized N}
\end{alltt}
\begin{verbatim}
## [1] 98
\end{verbatim}
\end{kframe}
\end{knitrout}
\pause
\vfill
Fourth, simulate and $\bs_1, \dots, \bs_N$. To do this, we'll pick
pixels proportional to density. Then we'll jitter each point
inside its pixel. 
  \vspace{-6pt}
\begin{knitrout}\footnotesize
\definecolor{shadecolor}{rgb}{0.878, 0.918, 0.933}\color{fgcolor}\begin{kframe}
\begin{alltt}
\hlstd{n.pixels} \hlkwb{<-} \hlkwd{length}\hlstd{(lambda)}
\hlstd{jitter} \hlkwb{<-} \hlnum{0.005}                    \hlcom{## Half width of pixel }
\hlstd{s.pixels} \hlkwb{<-} \hlkwd{sample}\hlstd{(n.pixels,} \hlkwc{size}\hlstd{=N,} \hlkwc{replace}\hlstd{=}\hlnum{TRUE}\hlstd{,}
                   \hlkwc{prob}\hlstd{=lambda.values}\hlopt{/}\hlstd{Lambda)}
\hlstd{elevation.xyz} \hlkwb{<-} \hlkwd{as.data.frame}\hlstd{(elevation,} \hlkwc{xy}\hlstd{=}\hlnum{TRUE}\hlstd{)}
\hlstd{s} \hlkwb{<-} \hlstd{elevation.xyz[s.pixels,}\hlkwd{c}\hlstd{(}\hlstr{"x"}\hlstd{,}\hlstr{"y"}\hlstd{)]} \hlopt{+}
    \hlkwd{cbind}\hlstd{(}\hlkwd{runif}\hlstd{(N,} \hlopt{-}\hlstd{jitter, jitter),}\hlkwd{runif}\hlstd{(N,} \hlopt{-}\hlstd{jitter, jitter))}
\end{alltt}
\end{kframe}
\end{knitrout}
\end{frame}







\begin{frame}[fragile]
  \frametitle{Inhomogeneous Poisson point process}
\begin{knitrout}\scriptsize
\definecolor{shadecolor}{rgb}{0.878, 0.918, 0.933}\color{fgcolor}\begin{kframe}
\begin{alltt}
\hlkwd{plot}\hlstd{(lambda,} \hlkwc{col}\hlstd{=}\hlkwd{terrain.colors}\hlstd{(}\hlnum{100}\hlstd{),}
     \hlkwc{main}\hlstd{=}\hlstr{"Density surface with activity centers"}\hlstd{)}
\hlkwd{points}\hlstd{(s,} \hlkwc{pch}\hlstd{=}\hlnum{16}\hlstd{,} \hlkwc{cex}\hlstd{=}\hlnum{1}\hlstd{,} \hlkwc{col}\hlstd{=}\hlstr{"blue"}\hlstd{)}
\end{alltt}
\end{kframe}

{\centering \includegraphics[width=0.7\linewidth]{figure/ippp5-1} 

}



\end{knitrout}
\end{frame}






\begin{frame}[fragile]
  \frametitle{Traps}
\begin{knitrout}\scriptsize
\definecolor{shadecolor}{rgb}{0.878, 0.918, 0.933}\color{fgcolor}\begin{kframe}
\begin{alltt}
\hlstd{x} \hlkwb{<-} \hlkwd{cbind}\hlstd{(}\hlkwd{rep}\hlstd{(}\hlkwd{seq}\hlstd{(}\hlnum{0.15}\hlstd{,} \hlnum{0.85}\hlstd{,} \hlkwc{by}\hlstd{=}\hlnum{0.1}\hlstd{),} \hlkwc{each}\hlstd{=}\hlnum{8}\hlstd{),}
           \hlkwd{rep}\hlstd{(}\hlkwd{seq}\hlstd{(}\hlnum{0.15}\hlstd{,} \hlnum{0.85}\hlstd{,} \hlkwc{by}\hlstd{=}\hlnum{0.1}\hlstd{),} \hlkwc{times}\hlstd{=}\hlnum{8}\hlstd{))}  \hlcom{## Trap locations}
\hlkwd{plot}\hlstd{(lambda,} \hlkwc{col}\hlstd{=}\hlkwd{terrain.colors}\hlstd{(}\hlnum{100}\hlstd{),}
     \hlkwc{main}\hlstd{=}\hlstr{"Density surface with activity centers and traps"}\hlstd{)}
\hlkwd{points}\hlstd{(s,} \hlkwc{pch}\hlstd{=}\hlnum{16}\hlstd{,} \hlkwc{col}\hlstd{=}\hlstr{"blue"}\hlstd{)} \hlcom{## Activity center locations}
\hlkwd{points}\hlstd{(x,} \hlkwc{pch}\hlstd{=}\hlnum{3}\hlstd{)}              \hlcom{## Trap locations}
\end{alltt}
\end{kframe}

{\centering \includegraphics[width=0.6\linewidth]{figure/traps1-1} 

}



\end{knitrout}
\end{frame}





\begin{frame}[fragile]
  \frametitle{Distance between traps and activity centers}
  Compute distances between activity centers ($\bs_1, \dots, \bs_N$)
  and traps ($\bx_1, \dots, \bx_J$).
\begin{knitrout}\footnotesize
\definecolor{shadecolor}{rgb}{0.878, 0.918, 0.933}\color{fgcolor}\begin{kframe}
\begin{alltt}
\hlstd{J} \hlkwb{<-} \hlkwd{nrow}\hlstd{(x)}                 \hlcom{## nTraps}
\hlstd{dist.sx} \hlkwb{<-} \hlkwd{matrix}\hlstd{(}\hlnum{NA}\hlstd{, N, J)}
\hlkwa{for}\hlstd{(i} \hlkwa{in} \hlnum{1}\hlopt{:}\hlstd{N) \{}
    \hlstd{dist.sx[i,]} \hlkwb{<-} \hlkwd{sqrt}\hlstd{((s[i,}\hlnum{1}\hlstd{]}\hlopt{-}\hlstd{x[,}\hlnum{1}\hlstd{])}\hlopt{^}\hlnum{2} \hlopt{+} \hlstd{(s[i,}\hlnum{2}\hlstd{]}\hlopt{-}\hlstd{x[,}\hlnum{2}\hlstd{])}\hlopt{^}\hlnum{2}\hlstd{)}
\hlstd{\}}
\end{alltt}
\end{kframe}
\end{knitrout}
\pause
\vfill
  Look at distances between first 4 individuals and first 5 traps.
\begin{knitrout}\footnotesize
\definecolor{shadecolor}{rgb}{0.878, 0.918, 0.933}\color{fgcolor}\begin{kframe}
\begin{alltt}
\hlstd{dist.sx[}\hlnum{1}\hlopt{:}\hlnum{4}\hlstd{,}\hlnum{1}\hlopt{:}\hlnum{5}\hlstd{]}
\end{alltt}
\begin{verbatim}
##           [,1]      [,2]      [,3]      [,4]      [,5]
## [1,] 0.1218625 0.1175070 0.1810119 0.2678109 0.3614970
## [2,] 0.9587743 0.8922040 0.8323508 0.7807610 0.7391668
## [3,] 0.7162980 0.6912027 0.6800289 0.6834600 0.7012815
## [4,] 0.6222620 0.5569781 0.5032288 0.4650310 0.4463949
\end{verbatim}
\end{kframe}
\end{knitrout}

\end{frame}






\begin{frame}[fragile]
  \frametitle{Capture probability}
  Compute capture probability
\begin{knitrout}\footnotesize
\definecolor{shadecolor}{rgb}{0.878, 0.918, 0.933}\color{fgcolor}\begin{kframe}
\begin{alltt}
\hlstd{g0} \hlkwb{<-} \hlnum{0.2}
\hlstd{sigma} \hlkwb{<-} \hlnum{0.05}
\hlstd{p} \hlkwb{<-} \hlstd{g0}\hlopt{*}\hlkwd{exp}\hlstd{(}\hlopt{-}\hlstd{dist.sx}\hlopt{^}\hlnum{2}\hlopt{/}\hlstd{(}\hlnum{2}\hlopt{*}\hlstd{sigma}\hlopt{^}\hlnum{2}\hlstd{))}
\end{alltt}
\end{kframe}
\end{knitrout}
\pause
\vfill
  Look at capture probs for first 4 individuals and first 5 traps.
\begin{knitrout}\footnotesize
\definecolor{shadecolor}{rgb}{0.878, 0.918, 0.933}\color{fgcolor}\begin{kframe}
\begin{alltt}
\hlkwd{print}\hlstd{(p[}\hlnum{1}\hlopt{:}\hlnum{4}\hlstd{,}\hlnum{1}\hlopt{:}\hlnum{5}\hlstd{],} \hlkwc{digits}\hlstd{=}\hlnum{3}\hlstd{)}
\end{alltt}
\begin{verbatim}
##          [,1]     [,2]     [,3]     [,4]     [,5]
## [1,] 1.03e-02 1.26e-02 2.85e-04 1.18e-07 8.92e-13
## [2,] 2.86e-81 1.44e-70 1.33e-61 2.25e-54 6.98e-49
## [3,] 5.44e-46 6.36e-43 1.36e-41 5.34e-42 3.84e-44
## [4,] 4.66e-35 2.27e-28 2.02e-23 3.29e-20 9.84e-19
\end{verbatim}
\end{kframe}
\end{knitrout}

\end{frame}





\begin{frame}[fragile]
  \frametitle{Capture histories}
  Simulate capture histories for all $N$ individuals
\begin{knitrout}\footnotesize
\definecolor{shadecolor}{rgb}{0.878, 0.918, 0.933}\color{fgcolor}\begin{kframe}
\begin{alltt}
\hlstd{K} \hlkwb{<-} \hlnum{5}                          \hlcom{# nOccasions}
\hlstd{y.all} \hlkwb{<-} \hlkwd{array}\hlstd{(}\hlnum{NA}\hlstd{,} \hlkwd{c}\hlstd{(N, J, K))}
\hlkwa{for}\hlstd{(i} \hlkwa{in} \hlnum{1}\hlopt{:}\hlstd{N) \{}
    \hlkwa{for}\hlstd{(j} \hlkwa{in} \hlnum{1}\hlopt{:}\hlstd{J) \{}
        \hlstd{y.all[i,j,]} \hlkwb{<-} \hlkwd{rbinom}\hlstd{(K,} \hlnum{1}\hlstd{,} \hlkwc{prob}\hlstd{=p[i,j])}
    \hlstd{\}}
\hlstd{\}}
\end{alltt}
\end{kframe}
\end{knitrout}
\pause
\vfill
  Discard individuals not captured
\begin{knitrout}\footnotesize
\definecolor{shadecolor}{rgb}{0.878, 0.918, 0.933}\color{fgcolor}\begin{kframe}
\begin{alltt}
\hlstd{captured} \hlkwb{<-} \hlkwd{rowSums}\hlstd{(y.all)}\hlopt{>}\hlnum{0}
\hlstd{y} \hlkwb{<-} \hlstd{y.all[captured,,]}
\end{alltt}
\end{kframe}
\end{knitrout}
\pause
\vfill
  Capture histories for first 2 individuals and first 5 traps
  on first occasion.
\begin{knitrout}\footnotesize
\definecolor{shadecolor}{rgb}{0.878, 0.918, 0.933}\color{fgcolor}\begin{kframe}
\begin{alltt}
\hlstd{y[}\hlnum{1}\hlopt{:}\hlnum{2}\hlstd{,}\hlnum{1}\hlopt{:}\hlnum{5}\hlstd{,}\hlnum{1}\hlstd{]}
\end{alltt}
\begin{verbatim}
##      [,1] [,2] [,3] [,4] [,5]
## [1,]    0    0    0    0    0
## [2,]    0    0    0    0    0
\end{verbatim}
\end{kframe}
\end{knitrout}
\end{frame}







\section{Likelihood}



\begin{frame}
  \frametitle{Outline}
  \Large
  \tableofcontents[currentsection]
\end{frame}









\begin{frame}[fragile]
  \frametitle{R package `secr'}
  \footnotesize
  Import data from two text files.
\begin{knitrout}\scriptsize
\definecolor{shadecolor}{rgb}{0.878, 0.918, 0.933}\color{fgcolor}\begin{kframe}
\begin{alltt}
\hlkwd{library}\hlstd{(secr)}
\hlstd{sch} \hlkwb{<-} \hlkwd{read.capthist}\hlstd{(}\hlkwc{captfile}\hlstd{=}\hlstr{"encounter_data_file.csv"}\hlstd{,}
                   \hlkwc{trapfile}\hlstd{=}\hlstr{"trap_data_file.csv"}\hlstd{,}
                   \hlkwc{detector}\hlstd{=}\hlstr{"proximity"}\hlstd{,} \hlkwc{fmt}\hlstd{=}\hlstr{"trapID"}\hlstd{)}
\end{alltt}
\end{kframe}
\end{knitrout}
  \pause
  Create the ``habitat mask'', which defines the state-space and the
  spatial covariates. First, prepare the elevation covariate. 
\begin{knitrout}\scriptsize
\definecolor{shadecolor}{rgb}{0.878, 0.918, 0.933}\color{fgcolor}\begin{kframe}
\begin{alltt}
\hlstd{elevation.xyz} \hlkwb{<-} \hlkwd{as.data.frame}\hlstd{(elevation,} \hlkwc{xy}\hlstd{=}\hlnum{TRUE}\hlstd{)}
\hlstd{elevation.xyz.m} \hlkwb{<-} \hlstd{elevation.xyz}
\hlstd{elevation.xyz.m}\hlopt{$}\hlstd{x} \hlkwb{<-} \hlstd{elevation.xyz}\hlopt{$}\hlstd{x}\hlopt{*}\hlnum{1000}  \hlcom{## Convert units to meters}
\hlstd{elevation.xyz.m}\hlopt{$}\hlstd{y} \hlkwb{<-} \hlstd{elevation.xyz}\hlopt{$}\hlstd{y}\hlopt{*}\hlnum{1000}
\hlstd{elevation.xyz.m}\hlopt{$}\hlstd{elevation} \hlkwb{<-} \hlkwd{scale}\hlstd{(elevation.xyz}\hlopt{$}\hlstd{elevation)} \hlcom{## Standardize}
\hlstd{elevation.m} \hlkwb{<-} \hlkwd{rasterFromXYZ}\hlstd{(elevation.xyz.m)}
\end{alltt}
\end{kframe}
\end{knitrout}
  \pause
  Now create the mask.
\begin{knitrout}\scriptsize
\definecolor{shadecolor}{rgb}{0.878, 0.918, 0.933}\color{fgcolor}\begin{kframe}
\begin{alltt}
\hlkwd{library}\hlstd{(sp)}
\hlstd{elev.sp} \hlkwb{<-} \hlkwd{as}\hlstd{(elevation.m,} \hlstr{"SpatialGridDataFrame"}\hlstd{)}
\hlstd{trp} \hlkwb{<-} \hlkwd{traps}\hlstd{(sch)}
\hlstd{mask} \hlkwb{<-} \hlkwd{make.mask}\hlstd{(trp,} \hlkwc{buffer}\hlstd{=}\hlnum{150}\hlstd{,} \hlkwc{spacing}\hlstd{=}\hlnum{15}\hlstd{)}
\hlstd{mask} \hlkwb{<-} \hlkwd{addCovariates}\hlstd{(mask,} \hlkwc{spatialdata}\hlstd{=elev.sp)}
\end{alltt}
\end{kframe}
\end{knitrout}
\end{frame}


\begin{frame}[fragile]
  \frametitle{Habitat mask}
\begin{knitrout}\scriptsize
\definecolor{shadecolor}{rgb}{0.878, 0.918, 0.933}\color{fgcolor}\begin{kframe}
\begin{alltt}
\hlkwd{plot}\hlstd{(mask,} \hlkwc{covariate}\hlstd{=}\hlstr{"elevation"}\hlstd{)}
\hlkwd{plot}\hlstd{(trp,} \hlkwc{add}\hlstd{=}\hlnum{TRUE}\hlstd{)}
\end{alltt}
\end{kframe}

{\centering \includegraphics[width=0.95\linewidth]{figure/plot-mask-1} 

}



\end{knitrout}
\end{frame}




\begin{frame}[fragile]
  \frametitle{R package `secr'}
  Model density as a function of elevation. 
\begin{knitrout}\scriptsize
\definecolor{shadecolor}{rgb}{0.878, 0.918, 0.933}\color{fgcolor}\begin{kframe}
\begin{alltt}
\hlstd{fm.elev} \hlkwb{<-} \hlkwd{secr.fit}\hlstd{(sch,} \hlkwc{model}\hlstd{=}\hlkwd{list}\hlstd{(}\hlkwc{D}\hlstd{=}\hlopt{~}\hlstd{elevation,} \hlkwc{g0}\hlstd{=}\hlopt{~}\hlnum{1}\hlstd{,} \hlkwc{sigma}\hlstd{=}\hlopt{~}\hlnum{1}\hlstd{),}
                    \hlkwc{mask}\hlstd{=mask,} \hlkwc{trace}\hlstd{=}\hlnum{FALSE}\hlstd{)} \hlcom{## Don't use 'buffer'}
\hlkwd{coef}\hlstd{(fm.elev)}
\end{alltt}
\begin{verbatim}
##                   beta    SE.beta        lcl         ucl
## D           -0.5498612 0.23019036 -1.0010260 -0.09869634
## D.elevation  0.9301088 0.18698334  0.5636282  1.29658945
## g0          -1.1809462 0.21373511 -1.5998593 -0.76203311
## sigma        3.9099026 0.06563899  3.7812525  4.03855261
\end{verbatim}
\end{kframe}
\end{knitrout}
\pause
\vfill
Estimates on original scale (at average covariate values).
\begin{knitrout}\scriptsize
\definecolor{shadecolor}{rgb}{0.878, 0.918, 0.933}\color{fgcolor}\begin{kframe}
\begin{alltt}
\hlkwd{predict}\hlstd{(fm.elev)}
\end{alltt}
\begin{verbatim}
##        link   estimate SE.estimate        lcl       ucl
## D       log  1.4125652  0.19635565  1.0770892  1.852530
## g0    logit  0.2348821  0.03841087  0.1680013  0.318205
## sigma   log 49.8940897  3.27852830 43.8709558 56.744152
\end{verbatim}
\end{kframe}
\end{knitrout}
\end{frame}



\begin{frame}[fragile]
  \frametitle{Predicted density surface}
\begin{knitrout}\scriptsize
\definecolor{shadecolor}{rgb}{0.878, 0.918, 0.933}\color{fgcolor}\begin{kframe}
\begin{alltt}
\hlstd{dsurf} \hlkwb{<-} \hlkwd{predictDsurface}\hlstd{(fm.elev)}
\hlstd{dsurf.r} \hlkwb{<-} \hlkwd{raster}\hlstd{(dsurf,} \hlkwc{covariate}\hlstd{=}\hlstr{"D.0"}\hlstd{)}
\hlkwd{plot}\hlstd{(dsurf.r,} \hlkwc{col}\hlstd{=}\hlkwd{terrain.colors}\hlstd{(}\hlnum{100}\hlstd{),} \hlkwc{main}\hlstd{=}\hlstr{"Estimated density surface"}\hlstd{)}
\end{alltt}
\end{kframe}
\end{knitrout}
\centering
\includegraphics[width=0.49\textwidth]{figure/secr-dsurf-1}
\includegraphics[width=0.49\textwidth]{figure/ippp2-1} \\
\pause \vfill
The units differ because we transformed to meters. \\
\end{frame}





\begin{frame}[fragile]
  \frametitle{R package `secr'}
  We can still estimate $N$ as before. \\
  \vfill
  \inr{E.N} is the expected value of
  $N$. \inr{R.N} is the realized value of $N$. 
  \vfill
\begin{knitrout}\small
\definecolor{shadecolor}{rgb}{0.878, 0.918, 0.933}\color{fgcolor}\begin{kframe}
\begin{alltt}
\hlkwd{region.N}\hlstd{(fm.elev)}
\end{alltt}
\begin{verbatim}
##     estimate SE.estimate      lcl      ucl  n
## E.N 88.05875   12.336690 67.00411 115.7294 59
## R.N 94.33840    8.008443 81.79063 113.7945 59
\end{verbatim}
\end{kframe}
\end{knitrout}
\end{frame}




\section{Data augmentation}


%\section{Prediction}
%\subsection{Likelihood-based inference}


\begin{frame}
  \frametitle{Outline}
  \Large
  \tableofcontents[currentsection]
\end{frame}





\begin{frame}
  \frametitle{Data augmentation model}
  The DA version of a basic SCR model is:
  \begin{gather*}
    \bsi \sim \mathrm{Unif}(\mathcal{S}) \\
    z_i \sim \mathrm{Bern}(\psi) \\
    p_{ij} = g_0\exp(-\|\bsi-\bxj\|^2/(2\sigma^2)) \\
    y_{ijk} \sim \mathrm{Bern}(z_i p_{ij}) \\
    N=\sum_{i=1}^M z_i
  \end{gather*}
  % A uniform prior on $\psi$ results in a discrete uniform prior on
  % $N$. We can change the prior for $N$ by changing the prior on
  % $\psi$, recognizing that $E(N)=M\psi$.
  % But why bother with augmentation?
  % \begin{itemize}
  %   \item DA works for \alert{all} varieties of mark-recapture models
  %   \item Make it easy to incorporate
  %     individual-covariates\dots\pause including distance and
  %     location!   
  % \end{itemize}
\end{frame}






\begin{frame}[fragile]
  \frametitle{Model SCR$_0$ -- data augmentation}
\vspace{-3pt}
\begin{knitrout}\scriptsize
\definecolor{shadecolor}{rgb}{0.878, 0.918, 0.933}\color{fgcolor}\begin{kframe}
\begin{alltt}
\hlkwd{writeLines}\hlstd{(}\hlkwd{readLines}\hlstd{(}\hlstr{"SCR0.jag"}\hlstd{))}
\end{alltt}


{\ttfamily\noindent\color{warningcolor}{\#\# Warning in file(con, "{}r"{}): cannot open file 'SCR0.jag': No such file or directory}}

{\ttfamily\noindent\bfseries\color{errorcolor}{\#\# Error in file(con, "{}r"{}): cannot open the connection}}\end{kframe}
\end{knitrout}
\end{frame}





\begin{frame}[fragile]
  \frametitle{Model SCR$_0$ -- data augmentation}
  Data
  \vspace{-6pt}
\begin{knitrout}\scriptsize
\definecolor{shadecolor}{rgb}{0.878, 0.918, 0.933}\color{fgcolor}\begin{kframe}
\begin{alltt}
\hlstd{M} \hlkwb{<-} \hlnum{150}
\hlstd{y.aug} \hlkwb{<-} \hlkwd{array}\hlstd{(}\hlnum{0}\hlstd{,} \hlkwd{c}\hlstd{(M, J, K))}
\hlstd{y.aug[}\hlnum{1}\hlopt{:}\hlkwd{nrow}\hlstd{(y),,]} \hlkwb{<-} \hlstd{y}
\hlstd{jags.data.SCR0} \hlkwb{<-} \hlkwd{list}\hlstd{(}\hlkwc{y}\hlstd{=y.aug,} \hlkwc{M}\hlstd{=M,} \hlkwc{J}\hlstd{=J,} \hlkwc{K}\hlstd{=K,}
                       \hlkwc{x}\hlstd{=x,} \hlkwc{xlim}\hlstd{=}\hlkwd{c}\hlstd{(}\hlnum{0}\hlstd{,}\hlnum{1}\hlstd{),} \hlkwc{ylim}\hlstd{=}\hlkwd{c}\hlstd{(}\hlnum{0}\hlstd{,}\hlnum{1}\hlstd{))}
\end{alltt}
\end{kframe}
\end{knitrout}
\pause
\vfill
  Inits and parameters
  \vspace{-6pt}
\begin{knitrout}\scriptsize
\definecolor{shadecolor}{rgb}{0.878, 0.918, 0.933}\color{fgcolor}\begin{kframe}
\begin{alltt}
\hlstd{ji.SCR0} \hlkwb{<-} \hlkwa{function}\hlstd{() \{}
    \hlkwd{list}\hlstd{(}\hlkwc{z}\hlstd{=}\hlkwd{rep}\hlstd{(}\hlnum{1}\hlstd{,M),} \hlkwc{psi}\hlstd{=}\hlkwd{runif}\hlstd{(}\hlnum{1}\hlstd{),}
         \hlkwc{s}\hlstd{=}\hlkwd{cbind}\hlstd{(}\hlkwd{runif}\hlstd{(M),} \hlkwd{runif}\hlstd{(M)),}
         \hlkwc{g0}\hlstd{=}\hlkwd{runif}\hlstd{(}\hlnum{1}\hlstd{),} \hlkwc{sigma}\hlstd{=}\hlkwd{runif}\hlstd{(}\hlnum{1}\hlstd{,} \hlnum{0.05}\hlstd{,} \hlnum{0.1}\hlstd{)) \}}
\hlstd{jp.SCR0} \hlkwb{<-} \hlkwd{c}\hlstd{(}\hlstr{"g0"}\hlstd{,} \hlstr{"sigma"}\hlstd{,} \hlstr{"EN"}\hlstd{,} \hlstr{"N"}\hlstd{)}
\hlkwd{library}\hlstd{(jagsUI)}
\end{alltt}
\end{kframe}
\end{knitrout}
\pause
\vfill
MCMC
  \vspace{-6pt}
\begin{knitrout}\scriptsize
\definecolor{shadecolor}{rgb}{0.878, 0.918, 0.933}\color{fgcolor}\begin{kframe}
\begin{alltt}
\hlstd{jags.post.SCR0} \hlkwb{<-} \hlkwd{jags.basic}\hlstd{(}\hlkwc{data}\hlstd{=jags.data.SCR0,} \hlkwc{inits}\hlstd{=ji.SCR0,}
                             \hlkwc{parameters.to.save}\hlstd{=jp.SCR0,}
                             \hlkwc{model.file}\hlstd{=}\hlstr{"SCR0.jag"}\hlstd{,}
                             \hlkwc{n.chains}\hlstd{=}\hlnum{3}\hlstd{,} \hlkwc{n.adapt}\hlstd{=}\hlnum{100}\hlstd{,} \hlkwc{n.burnin}\hlstd{=}\hlnum{0}\hlstd{,}
                             \hlkwc{n.iter}\hlstd{=}\hlnum{2000}\hlstd{,} \hlkwc{parallel}\hlstd{=}\hlnum{TRUE}\hlstd{)}
\end{alltt}


{\ttfamily\noindent\bfseries\color{errorcolor}{\#\# Error in checkForRemoteErrors(val): 3 nodes produced errors; first error: Cannot open model file "{}SCR0.jag"{}}}\end{kframe}
\end{knitrout}
\end{frame}




\begin{frame}[fragile]
  \frametitle{Posterior summaries}
\begin{knitrout}\tiny
\definecolor{shadecolor}{rgb}{0.878, 0.918, 0.933}\color{fgcolor}\begin{kframe}
\begin{alltt}
\hlkwd{summary}\hlstd{(jags.post.SCR0)}
\end{alltt}


{\ttfamily\noindent\bfseries\color{errorcolor}{\#\# Error in h(simpleError(msg, call)): error in evaluating the argument 'object' in selecting a method for function 'summary': object 'jags.post.SCR0' not found}}\end{kframe}
\end{knitrout}
\end{frame}



\begin{frame}[fragile]
  \frametitle{Traceplots and density plots}
\begin{knitrout}\footnotesize
\definecolor{shadecolor}{rgb}{0.878, 0.918, 0.933}\color{fgcolor}\begin{kframe}
\begin{alltt}
\hlkwd{plot}\hlstd{(jags.post.SCR0[,jp.SCR0])}
\end{alltt}


{\ttfamily\noindent\bfseries\color{errorcolor}{\#\# Error in h(simpleError(msg, call)): error in evaluating the argument 'x' in selecting a method for function 'plot': object 'jags.post.SCR0' not found}}\end{kframe}
\end{knitrout}
\end{frame}


\begin{frame}
  \frametitle{How do we speed up Bayesian inference?}
  \begin{itemize}
    \item Use joint likelihood $p(y|n)p(0|n,N)p(N)$ approach as shown in
      non-spatial lecture.
      \begin{itemize}
        \item This can work well, but not when there are other
          individual-level covariates.
      \end{itemize}
    \item Treat $z_i=1$ as data for first $n$ individuals.
    \item If there are no occasion-specific covariates, collapse data
      and use binomial instead of Bernoulli distribution.
    \item Use a single zero for each augmented individual, instead of
      an array of zeros. Then compute probability of detecting an
      individual at least once.
  \end{itemize}
\end{frame}




\begin{frame}[fragile]
  \frametitle{Model SCR$_0$ -- data augmentation}
\vspace{-3pt}
\begin{knitrout}\tiny
\definecolor{shadecolor}{rgb}{0.878, 0.918, 0.933}\color{fgcolor}\begin{kframe}
\begin{alltt}
\hlkwd{writeLines}\hlstd{(}\hlkwd{readLines}\hlstd{(}\hlstr{"SCR0-faster.jag"}\hlstd{))}
\end{alltt}
\begin{verbatim}
## model {
## 
## #psi ~ dunif(0, 1)
## beta0 ~ dnorm(0, 0.1)
## beta1 ~ dnorm(0, 0.1)
## g0 ~ dunif(0, 1)
## sigma ~ dunif(0, 0.5)
## 
## for(g in 1:G) { ## Loop over pixels
##   lambda[g] <- exp(beta0 + beta1*elevation[g])*pixelArea
##   pi[g] <- lambda[g]/EN
## }
## EN <- sum(lambda)
## 
## for(i in 1:M) {
##   s[i,1] ~ dunif(xlim[1], xlim[2])
##   s[i,2] ~ dunif(ylim[1], ylim[2])
##   pixel[i] <- lookup[]
##   logProb[i] <- pi[pixel[i]]
##   zeros[i] ~ dpois(-logProb[i])
##   z[i] ~ dbern(psi)
##   for(j in 1:J) {
##     dist[i,j] <- sqrt((s[i,1]-x[j,1])^2 + (s[i,2]-x[j,2])^2)
##     p[i,j] <- g0*exp(-dist[i,j]^2/(2*sigma^2))
##   }
## }
## for(i in 1:n) {  ## Model for observed capture histories
##   for(j in 1:J) {
##     y.tilde[i,j] ~ dbinom(p[i,j], K)
##   }
## }
## for(i in (n+1):M) { ## Model for augmented guys
##   PrAtLeastOneCap[i] <- 1-prod(1-p[i,])^K
##   zero[i] ~ dbern(PrAtLeastOneCap[i]*z[i])
## }
## 
## ## EN <- M*psi
## N <- sum(z)
## }
\end{verbatim}
\end{kframe}
\end{knitrout}
\end{frame}




\begin{frame}[fragile]
  \frametitle{Model SCR$_0$ -- faster}
  Data
  \vspace{-6pt}
\begin{knitrout}\scriptsize
\definecolor{shadecolor}{rgb}{0.878, 0.918, 0.933}\color{fgcolor}\begin{kframe}
\begin{alltt}
\hlstd{y.tilde} \hlkwb{<-} \hlkwd{apply}\hlstd{(y,} \hlkwd{c}\hlstd{(}\hlnum{1}\hlstd{,}\hlnum{2}\hlstd{), sum)}
\hlstd{n} \hlkwb{<-} \hlkwd{nrow}\hlstd{(y)}
\hlstd{jags.data.SCR0.faster} \hlkwb{<-} \hlkwd{list}\hlstd{(}\hlkwc{y.tilde}\hlstd{=y.tilde,} \hlkwc{n}\hlstd{=n,} \hlkwc{M}\hlstd{=M,} \hlkwc{J}\hlstd{=J,}
                              \hlkwc{z}\hlstd{=}\hlkwd{c}\hlstd{(}\hlkwd{rep}\hlstd{(}\hlnum{1}\hlstd{, n),} \hlkwd{rep}\hlstd{(}\hlnum{NA}\hlstd{, M}\hlopt{-}\hlstd{n)),}
                              \hlkwc{K}\hlstd{=K,} \hlkwc{zero}\hlstd{=}\hlkwd{rep}\hlstd{(}\hlnum{0}\hlstd{, M),} \hlkwc{x}\hlstd{=x,}
                              \hlkwc{xlim}\hlstd{=}\hlkwd{c}\hlstd{(}\hlnum{0}\hlstd{,}\hlnum{1}\hlstd{),} \hlkwc{ylim}\hlstd{=}\hlkwd{c}\hlstd{(}\hlnum{0}\hlstd{,}\hlnum{1}\hlstd{))}
\end{alltt}
\end{kframe}
\end{knitrout}
\pause
\vfill
  Inits and parameters (same as before)
\pause
\vfill
\begin{knitrout}\scriptsize
\definecolor{shadecolor}{rgb}{0.878, 0.918, 0.933}\color{fgcolor}\begin{kframe}
\begin{alltt}
\hlstd{ji.SCR0.faster} \hlkwb{<-} \hlkwa{function}\hlstd{() \{}
    \hlkwd{list}\hlstd{(}\hlkwc{z}\hlstd{=}\hlkwd{c}\hlstd{(}\hlkwd{rep}\hlstd{(}\hlnum{NA}\hlstd{, n),} \hlkwd{rep}\hlstd{(}\hlnum{0}\hlstd{,M}\hlopt{-}\hlstd{n)),} \hlkwc{psi}\hlstd{=}\hlkwd{runif}\hlstd{(}\hlnum{1}\hlstd{),}
         \hlkwc{s}\hlstd{=}\hlkwd{cbind}\hlstd{(}\hlkwd{runif}\hlstd{(M),} \hlkwd{runif}\hlstd{(M)),}
         \hlkwc{g0}\hlstd{=}\hlkwd{runif}\hlstd{(}\hlnum{1}\hlstd{),} \hlkwc{sigma}\hlstd{=}\hlkwd{runif}\hlstd{(}\hlnum{1}\hlstd{,} \hlnum{0.05}\hlstd{,} \hlnum{0.1}\hlstd{)) \}}
\end{alltt}
\end{kframe}
\end{knitrout}
MCMC
  \vspace{-6pt}
\begin{knitrout}\scriptsize
\definecolor{shadecolor}{rgb}{0.878, 0.918, 0.933}\color{fgcolor}\begin{kframe}
\begin{alltt}
\hlstd{jags.post.SCR0.faster} \hlkwb{<-} \hlkwd{jags.basic}\hlstd{(}\hlkwc{data}\hlstd{=jags.data.SCR0.faster,}
                                    \hlkwc{inits}\hlstd{=ji.SCR0.faster,}
                                    \hlkwc{parameters.to.save}\hlstd{=jp.SCR0,}
                                    \hlkwc{model.file}\hlstd{=}\hlstr{"SCR0-faster.jag"}\hlstd{,}
                                    \hlkwc{n.chains}\hlstd{=}\hlnum{3}\hlstd{,} \hlkwc{n.adapt}\hlstd{=}\hlnum{100}\hlstd{,} \hlkwc{n.burnin}\hlstd{=}\hlnum{0}\hlstd{,}
                                    \hlkwc{n.iter}\hlstd{=}\hlnum{2000}\hlstd{,} \hlkwc{parallel}\hlstd{=}\hlnum{TRUE}\hlstd{)}
\end{alltt}
\end{kframe}
\end{knitrout}
\end{frame}




\begin{frame}[fragile]
  \frametitle{Posterior summaries}
\begin{knitrout}\tiny
\definecolor{shadecolor}{rgb}{0.878, 0.918, 0.933}\color{fgcolor}\begin{kframe}
\begin{alltt}
\hlkwd{summary}\hlstd{(jags.post.SCR0.faster)}
\end{alltt}
\begin{verbatim}
## 
## Iterations = 1:2000
## Thinning interval = 1 
## Number of chains = 3 
## Sample size per chain = 2000 
## 
## 1. Empirical mean and standard deviation for each variable,
##    plus standard error of the mean:
## 
##               Mean        SD  Naive SE Time-series SE
## EN        98.00326 11.040638 1.425e-01      0.4079510
## N         98.42683  9.631378 1.243e-01      0.3684158
## deviance 520.76092 25.430234 3.283e-01      0.9887463
## g0         0.23958  0.039458 5.094e-04      0.0017107
## sigma      0.04975  0.003275 4.227e-05      0.0001427
## 
## 2. Quantiles for each variable:
## 
##               2.5%       25%      50%       75%     97.5%
## EN        77.00001  90.53685  97.4968 105.29392 120.71776
## N         82.00000  92.00000  98.0000 105.00000 119.00000
## deviance 473.26223 502.97915 520.0376 537.43704 572.82357
## g0         0.17076   0.21131   0.2366   0.26443   0.32633
## sigma      0.04386   0.04748   0.0496   0.05188   0.05645
\end{verbatim}
\end{kframe}
\end{knitrout}
\end{frame}



\begin{frame}[fragile]
  \frametitle{Traceplots and density plots}
\begin{knitrout}\footnotesize
\definecolor{shadecolor}{rgb}{0.878, 0.918, 0.933}\color{fgcolor}\begin{kframe}
\begin{alltt}
\hlkwd{plot}\hlstd{(jags.post.SCR0.faster[,jp.SCR0])}
\end{alltt}
\end{kframe}

{\centering \includegraphics[width=0.7\textwidth]{figure/plot-mcmc-SCR0-faster-1} 

}



\end{knitrout}
\end{frame}




\begin{frame}
  \frametitle{SCR study design}
  SCR uses model-based, rather than design-based,
  inference (see Ch. 10). \\ 
  \pause
  \vfill
  Random placement of traps is not required, but it's a good idea
  randomly sample locations along the environmental gradients you're
  interested in. \\
  \pause \vfill
  The other two key design considerations are:
  \begin{enumerate}
    \item<3-> Capture as many individuals as you can (i.e., maximize $n$)
    \item<4-> Obtain as many {\it spatial} recaptures as possible
  \end{enumerate}
  \vfill
  \uncover<5->{
    There is a tradeoff between these two objectives. Simulation is
    often the best option for finding the right balance.
  }
\end{frame}

%\section{Summary}


\begin{frame}
  \frametitle{SCR summary}
  We assume that variation in $p$ arises from distance between animals
  and traps. \\
  \pause \vfill
  We can estimate abundance and model distribution (i.e., spatial
  variation in density) \\
  \pause \vfill
  Next time, we'll see how to do that using secr and JAGS, and we'll
  make a bunch of maps. \\
\end{frame}




\section{Assignment}




\begin{frame}[fragile]
  \frametitle{Assignment}
  Create a self-contained R script or Rmarkdown file to do the
  following: 
  \vfill
  \begin{enumerate}
    \item Fit a ``local behavioral response'' model,
      rather than a ``global behavioral response'' model in secr.
    \item Fit a model in JAGS where $g_0$ varies among
      occasions. Compare estimates of $E(N)$ and $N$ to the estimates
      from secr for the same model (which we fit earlier).
  \end{enumerate}
  \vfill
  Upload your {\tt .R} or {\tt .Rmd} file to ELC before Tuesday. 
\end{frame}





\end{document}

