\documentclass[color=usenames,dvipsnames]{beamer}
%\documentclass[color=usenames,dvipsnames,handout]{beamer}

%\usepackage[roman]{../../../pres1}
\usepackage[sans]{../../../pres1}
\usepackage{graphicx}
\usepackage{bm}
\usepackage{soul}
\usepackage{color}
\usepackage{Sweave}
\usepackage{pdfpages}

%\setbeameroption{notes on second screen}



\DefineVerbatimEnvironment{Sinput}{Verbatim}{fontshape=sl,formatcom=\color{red}}
\DefineVerbatimEnvironment{Soutput}{Verbatim}{formatcom=\color{MidnightBlue}}
\DefineVerbatimEnvironment{Scode}{Verbatim}{fontshape=sl}


\newcommand{\bxt}{${\bm x}_j$}
\newcommand{\bx}{{\bm x}}
\newcommand{\bxj}{{\bm x}_j}
\newcommand{\bst}{${\bm s}_i$}
\newcommand{\bs}{{\bm s}}
\newcommand{\bsi}{{\bm s}_i}
\newcommand{\ed}{\|\bx - \bs\|}
\newcommand{\cs}{\mathcal{S} }


\begin{document}




\begin{frame}[plain]
  \begin{center}
    \LARGE {\bf \color{RoyalBlue}{Spatial Jolly-Seber Model}} \par
    \vspace{0.8cm}
%    \color{Gray}{
%    \Large {Spatial Capture-Recapture Workshop} \\ % \par
%    \large Athens, GA -- March 2015 \\
    \large WILD 8300 -- Spatial Capture-Recapture \par
    \vspace{0.2cm}
    March 30, 2016
%  \includegraphics[width=0.3\textwidth]{figs/scrbook} \\
%    }
  \end{center}
\end{frame}




%\section{Intro}



\begin{frame}[plain]
  \frametitle{Topics}
  \Large
  \only<1>{\tableofcontents}%[hideallsubsections]}
%  \only<2 | handout:0>{\tableofcontents[currentsection]}%,hideallsubsections]}
\end{frame}




\section{Intro}









\begin{frame}
  \frametitle{Overview}
  \large
  {\bf Jolly-Seber model}
  \begin{itemize}
    \item Extends CJS model to allow for recruitment
    \item We no longer ``condition on first capture''
    \item Typically, we use the robust design
  \end{itemize}
\end{frame}




\begin{frame}
  \frametitle{Stochastic population dynamics without dispersal}
  {\bf Initial Abundance}
  \[
    N_1 \sim \mbox{Poisson}(\lambda)
  \]
  \vfill
  {\bf Survival}
  \[
    S_t \sim \mbox{Binomial}(N_{t-1}, \phi)
  \]
  \vfill
  {\bf Recruitment}
  \[
    R_t \sim \mbox{Poisson}(N_{t-1} \gamma)
  \]
  \vfill
  {\bf Abundance}
  \[
    N_t = S_t + R_t
  \]
\end{frame}





%% \section{Non-spatial CJS}






%% \begin{frame}
%%   \frametitle{Non-spatial CJS model}
%%   {\bf State model}
%%   \[
%%     z_{i,t} \sim \mbox{Bernoulli}(z_{i,t-1} \times \phi)
%%   \]
%%   \vfill
%%   {\bf Observation model}
%%   \[
%%     y_{i,t} \sim \mbox{Bernoulli}(z_{i,t} \times p)
%%   \]
%%   \pause
%%   \vfill
%%   \small
%%   where \\
%%   \begin{itemize}
%%     \item $z_{i,t}$ is ``alive state'' of individual $i$ at time $t$
%%     \item $\phi$ is ``apparent survival''. Probability of being alive and not permanently emigrating.
%%     \item $y_{i,t}=1$ if individual was encountered. $y_{i,t}=0$ otherwise.
%%   \end{itemize}
%% \end{frame}








\section{Spatial JS}






\begin{frame}
  \frametitle{Spatial model}
  \large
  {\bf Extensions}
  \begin{itemize}
    \item Individuals heterogeneity in vital rates
    \item Spatial heterogeneity in vital rates
    \item Density dependence
    \item Dispersal
  \end{itemize}
\end{frame}




\begin{frame}
  \frametitle{Spatial Model using Data Augmentation}
  {\bf Initial State}
  \[
    z_{i,1} \sim \mbox{Bernoulli}(\psi)
  \]
  \vfill
  {\bf Survival and Recruitment}
  \[
    z_{i,t} \sim \mbox{Bernoulli}(z_{i,t-1}\phi + \mbox{recruitable}_{i,t-1}\gamma')
  \]
  \vfill
  {\bf Abundance}
  \[
    N_t = \sum_{i=1}^M z_{i,t}
  \]
  \pause
  \vfill
  {\centering \bf Dispersal could be added using approaches discussed last week \par}
\end{frame}





\begin{frame}[fragile]
  \frametitle{Simulating spatial JS data with robust design}
  \scriptsize % \tiny %\small
  {\bf Parameters and data dimensions}
\begin{Schunk}
\begin{Sinput}
> T <- 10      # years/primary periods
> K <- 3       # 3 secondary sampling occasion
> N0 <- 25     # Abundance in year 1
> M <- 500     # Easiest way to simulate data is using data augmentation
> phi <- 0.7   # Apparent survival
> gamma <- 0.3 # Per-capital recruitment rate
> p0 <- 0.4
> sigma <- 0.1
\end{Sinput}
\end{Schunk}
\pause
{\bf Traps, activity centers, and detection probability}
\begin{Schunk}
\begin{Sinput}
> set.seed(340)
> co <- seq(0.25, 0.75, length=5)
> X <- cbind(rep(co, each=5), rep(co, times=5))
> J <- nrow(X)
> xlim <- ylim <- c(0,1)
> s <- cbind(runif(M, xlim[1], xlim[2]), runif(M, ylim[1], ylim[2]))
> d <- p <- matrix(NA, M, J)
> for(i in 1:M) {
+     d[i,] <- sqrt((s[i,1]-X[,1])^2 + (s[i,2]-X[,2])^2)
+     p[i,] <- p0*exp(-d[i,]^2/(2*sigma^2))
+ }
\end{Sinput}
\end{Schunk}
\end{frame}





\begin{frame}[fragile]
  \frametitle{Simulating spatial JS data with robust design}
{\bf Generate $z$}
\scriptsize
\begin{Schunk}
\begin{Sinput}
> set.seed(034)
> z <- recruitable <- died <- recruited <- matrix(0, M, T)
> z[1:N0,1] <- 1 # First N0 are alive
> recruitable[(N0+1):M,1] <- 1
> for(t in 2:T) {
+     prevN <- sum(z[,t-1]) # number alive at t-1
+     ER <- prevN*gamma # expected number of recruits
+     prevA <- sum(recruitable[,t-1]) # Number available to be recruited
+     gammaPrime <- ER/prevA
+     if(gammaPrime > 1)
+         stop("M isn't big enough")
+     for(i in 1:M) {
+         z[i,t] <- rbinom(1, 1, z[i,t-1]*phi + recruitable[i,t-1]*gammaPrime)
+         recruitable[i,t] <- 1 - max(z[i,1:(t)]) # to be recruited
+         died[i,t] <- z[i,t-1]==1 & z[i,t]==0
+         recruited[i,t] <- z[i,t]==1 & z[i,t-1]==0
+     }
+ }
\end{Sinput}
\end{Schunk}
\pause
\vfill
{\bf \normalsize Populaton size, mortalities, and recruits}
\begin{Schunk}
\begin{Sinput}
> N <- colSums(z) # Population size
> Deaths <- colSums(died)
> Recruits <- colSums(recruited)
> everAlive <- sum(rowSums(z)>0)
\end{Sinput}
\end{Schunk}
\end{frame}









\begin{frame}[fragile]
  \frametitle{Simulating spatial JS data with robust design}
{\bf Generate encounter histories for all $M$ individuals}
\footnotesize
\begin{Schunk}
\begin{Sinput}
> yall <- array(NA, c(M, J, K, T))
> for(i in 1:M) {
+     for(t in 1:T) {
+         for(j in 1:J) {
+             yall[i,j,1:K,t] <- rbinom(K, 1, z[i,t]*p[i,j])
+         }
+     }
+ }
\end{Sinput}
\end{Schunk}
\pause
\vfill
{\bf \normalsize Discard individuals that were never captured}
\begin{Schunk}
\begin{Sinput}
> detected <- rowSums(yall) > 0
> y <- yall[detected,,,]
> str(y)
\end{Sinput}
\begin{Soutput}
 int [1:42, 1:25, 1:3, 1:10] 0 0 1 0 0 0 0 0 0 0 ...
\end{Soutput}
\end{Schunk}
\end{frame}






\begin{frame}[fragile]
  \frametitle{Time series}
  \tiny
\vspace{-3mm}
\begin{center}
  \includegraphics[width=0.8\textwidth]{Open-JS-NDR}
\end{center}
\end{frame}



\begin{frame}[fragile]
  \frametitle{Spatial CJS model in \jags}
  \vspace{-5mm}
  \tiny \fbox{\parbox{\linewidth}{\verbatiminput{JS-spatial.jag}}}
\end{frame}





\begin{frame}[fragile]
  \frametitle{\jags}
%  \footnotesize
  {\bf Data augmentation}
  \scriptsize
\begin{Schunk}
\begin{Sinput}
> M <- nrow(y) + 50
> yz <- array(0, c(M, J, K, T))
> yz[1:nrow(y),,,] <- y
\end{Sinput}
\end{Schunk}
\pause
\vfill
  {\bf \normalsize Initial values for $z$ matrix}
\begin{Schunk}
\begin{Sinput}
> zi <- matrix(0, M, T)
> zi[1:nrow(y),] <- 1
> ji1 <- function() list(phi=0.01, gamma=0.01, z=zi)
\end{Sinput}
\end{Schunk}
\pause
\vfill
  {\bf \normalsize Fit the model}
\begin{Schunk}
\begin{Sinput}
> jd1 <- list(y = yz, M = M, X = X, J = J, K = K, T = T, xlim = xlim, 
+     ylim = ylim)
> jp1 <- c("phi", "gamma", "p0", "sigma", "N", "Deaths", "Recruits", 
+     "Ntot")
> library(rjags)
> jm1 <- jags.model("JS-spatial.jag", jd1, ji1, n.chains = 1, n.adapt = 500)
> jc1 <- coda.samples(jm1, jp1, 1000)
\end{Sinput}
\end{Schunk}
\end{frame}




\begin{frame}
  \frametitle{Is $M$ big enough?}
\includegraphics{Open-JS-Ntot}
\end{frame}






\begin{frame}[fragile]
  \frametitle{Posterior distributions}
\begin{center}
  \fbox{\includegraphics[width=0.7\textwidth]{Open-JS-jc1}}
\end{center}
\end{frame}






\begin{frame}[fragile]
  \frametitle{Posterior distributions}
\begin{center}
  \fbox{\includegraphics[width=0.45\textwidth]{Open-JS-jcN1-4}}
  \fbox{\includegraphics[width=0.45\textwidth]{Open-JS-jcN5-8}}
\end{center}
\end{frame}
















\section{Spatial JS with density-dependence}









\begin{frame}
  \frametitle{Density-dependent recruitment}
  \large
  {\bf Logistic}
  \[
     \gamma_t = \gamma_{max}(1 - N_{t-1}/K)
  \]
  \vfill
  {\bf Log-linear}
  \[
     \gamma_t = \nu_0\exp(-\nu_1 N_{t-1})
  \]
  \vfill
  {\bf Log-linear (alt version)}
  \[
    \log(\gamma_t) = \nu_0 + \nu_1 N_{t-1}
  \]
%<<>>=
%gompertz <- function(N, a, b, c) {
%    x <- numeric(length(N))
%    for(i in N) {
%        x[i] <- a*exp(-b*exp(-c*i))
%    }
%    return(x)
%}
%gompertz(1:10, a=1, b=1, c=1)
%##plot(gompertz(1:10, a=2, b=1, c=.11))
%@
\end{frame}






\begin{frame}[fragile]
  \frametitle{Density-dependent recruitment}
  \tiny
\vspace{-4mm}
\begin{center}
  \includegraphics[width=\textwidth]{Open-JS-dd1}
\end{center}
\end{frame}
















\begin{frame}[fragile]
  \frametitle{Simulating spatial JS data with robust design}
  \scriptsize % \tiny %\small
  {\bf Parameters and data dimensions}
\begin{Schunk}
\begin{Sinput}
> T <- 10      # years/primary periods
> K <- 3       # 3 secondary sampling occasion
> ## Is it necessary to be far from equilibrium to detect density-dependence?
> ## Equilibrium here is where (1-phi) == gamma, where gamma is function of N
> N0 <- 10     # Abundance in year 1
> M <- 500     # Easiest way to simulate data is using data augmentation
> phi <- 0.7   # Apparent survival
> ##gamma <- 0.3 # Per-capital recruitment rate
> nu0 <- 2
> nu1 <- 0.05
> p0 <- 0.4
> sigma <- 0.1
\end{Sinput}
\end{Schunk}
\pause
{\bf Traps, activity centers, and detection probability}
\begin{Schunk}
\begin{Sinput}
> set.seed(3479)
> co <- seq(0.25, 0.75, length=5)
> X <- cbind(rep(co, each=5), rep(co, times=5))
> J <- nrow(X)
> xlim <- ylim <- c(0,1)
> s <- cbind(runif(M, xlim[1], xlim[2]), runif(M, ylim[1], ylim[2]))
> d <- p <- matrix(NA, M, J)
> for(i in 1:M) {
+     d[i,] <- sqrt((s[i,1]-X[,1])^2 + (s[i,2]-X[,2])^2)
+     p[i,] <- p0*exp(-d[i,]^2/(2*sigma^2))
+ }
\end{Sinput}
\end{Schunk}
\end{frame}





\begin{frame}[fragile]
  \frametitle{Simulating spatial JS data with robust design}
{\bf Generate $z$}
\scriptsize
\begin{Schunk}
\begin{Sinput}
> set.seed(3401)
> z2 <- recruitable <- died <- recruited <- matrix(0, M, T)
> z2[1:N0,1] <- 1 # First N0 are alive
> recruitable[(N0+1):M,1] <- 1
> for(t in 2:T) {
+     prevN <- sum(z2[,t-1]) # number alive at t-1
+     gamma <- nu0*exp(-nu1*prevN) ## Density dependent recruitment rate
+     ER <- prevN*gamma # expected number of recruits
+     prevA <- sum(recruitable[,t-1]) # Number available to be recruited
+     gammaPrime <- ER/prevA
+     if(gammaPrime > 1)
+         stop("M isn't big enough")
+     for(i in 1:M) {
+         z2[i,t] <- rbinom(1, 1, z2[i,t-1]*phi + recruitable[i,t-1]*gammaPrime)
+         recruitable[i,t] <- 1 - max(z2[i,1:(t)]) # to be recruited
+         died[i,t] <- z2[i,t-1]==1 & z2[i,t]==0
+         recruited[i,t] <- z2[i,t]==1 & z2[i,t-1]==0
+     }
+ }
\end{Sinput}
\end{Schunk}
\pause
\vfill
{\bf \normalsize Populaton size, mortalities, and recruits}
\begin{Schunk}
\begin{Sinput}
> N2 <- colSums(z2) # Population size
> Deaths2 <- colSums(died)
> Recruits2 <- colSums(recruited)
> everAlive2 <- sum(rowSums(z2)>0)
\end{Sinput}
\end{Schunk}
\end{frame}









\begin{frame}[fragile]
  \frametitle{Simulating spatial JS data with robust design}
{\bf Generate encounter histories for all $M$ individuals}
\footnotesize
\begin{Schunk}
\begin{Sinput}
> yall <- array(NA, c(M, J, K, T))
> for(i in 1:M) {
+     for(t in 1:T) {
+         for(j in 1:J) {
+             yall[i,j,1:K,t] <- rbinom(K, 1, z2[i,t]*p[i,j])
+         }
+     }
+ }
\end{Sinput}
\end{Schunk}
\pause
\vfill
{\bf \normalsize Discard individuals that were never captured}
\begin{Schunk}
\begin{Sinput}
> detected <- rowSums(yall) > 0
> y2 <- yall[detected,,,]
> str(y2)
\end{Sinput}
\begin{Soutput}
 int [1:85, 1:25, 1:3, 1:10] 0 0 0 0 0 0 0 0 0 0 ...
\end{Soutput}
\end{Schunk}
\end{frame}






\begin{frame}[fragile]
  \frametitle{Time series}
  \tiny
\vspace{-3mm}
\begin{center}
  \includegraphics[width=\textwidth]{Open-JS-NDR-DD}
\end{center}
\end{frame}



\begin{frame}[fragile]
  \frametitle{Spatial CJS model in \jags}
  \vspace{-5mm}
  \tiny \fbox{\parbox{\linewidth}{\verbatiminput{JS-spatial-DD.jag}}}
\end{frame}





\begin{frame}[fragile]
  \frametitle{\jags}
%  \footnotesize
  {\bf Data augmentation}
  \scriptsize
\begin{Schunk}
\begin{Sinput}
> M2 <- nrow(y2) + 75
> yz2 <- array(0, c(M2, J, K, T))
> yz2[1:nrow(y2),,,] <- y2
\end{Sinput}
\end{Schunk}
\pause
\vfill
  {\bf \normalsize Initial values for $z$ matrix}
\begin{Schunk}
\begin{Sinput}
> zi <- matrix(0, M2, T)
> ##zi[1:nrow(y2),] <- 1
> zi[1:nrow(y2),] <- z2[detected,] ## cheating
> ji2 <- function() list(phi=0.01, z=zi)
\end{Sinput}
\end{Schunk}
\pause
\vfill
  {\bf \normalsize Fit the model}
\begin{Schunk}
\begin{Sinput}
> jd2 <- list(y = yz2, M = M2, X = X, J = J, K = K, T = T, xlim = xlim, 
+     ylim = ylim)
> jp2 <- c("phi", "nu0", "nu1", "p0", "sigma", "N", "Deaths", "Recruits", 
+     "Ntot")
> library(rjags)
> jm2 <- jags.model("JS-spatial-DD.jag", jd2, ji2, n.chains = 1, 
+     n.adapt = 200)
> jc2 <- coda.samples(jm2, jp2, 5000)
\end{Sinput}
\end{Schunk}
\end{frame}




\begin{frame}[fragile]
  \frametitle{Posterior distributions}
\begin{center}
  \fbox{\includegraphics[width=0.7\textwidth]{Open-JS-jc1}}
\end{center}
\end{frame}










\begin{frame}[fragile]
  \frametitle{Actual and estimated abundance}
  {\bf Extract and summarize posterior samples of $N_t$}
  \footnotesize
\begin{Schunk}
\begin{Sinput}
> Npost <- as.matrix(jc2[,paste("N[", 1:10, "]", sep="")])
> Nmed <- apply(Npost, 2, median)
> Nupper <- apply(Npost, 2, quantile, prob=0.975)
> Nlower <- apply(Npost, 2, quantile, prob=0.025)
\end{Sinput}
\end{Schunk}
  \pause
  \vfill
  {\bf \normalsize Plot}
\begin{Schunk}
\begin{Sinput}
> plot(1:T, N2, type="o", col="blue", ylim=c(0, 100), xlab="Time",
+      ylab="Abundance")
> points(1:T, Nmed)
> arrows(1:T, Nlower, 1:T, Nupper, angle=90, code=3, length=0.05)
> legend(1, 100, c("Actual abundance", "Estimated abundance"),
+        col=c("blue", "black"), lty=c(1,1), pch=c(1,1))
\end{Sinput}
\end{Schunk}
\end{frame}





\begin{frame}
  \frametitle{Actual and estimated abundance}
  \vspace{-4mm}
  \begin{center}
    \includegraphics[width=0.8\textwidth]{Open-JS-Npost}
  \end{center}
\end{frame}










\begin{frame}
  \frametitle{Summary}
  \large
  {\bf Key points}
  \begin{itemize}[<+->]
    \item Spatial Jolly-Seber models make it possible to fit
      spatio-temporal models of population dynamics to standard data
    \item We could have movement just like we did with CJS models
  \end{itemize}
\end{frame}



\begin{frame}
  \frametitle{Assignment}
  {\bf \large For next week}
  \begin{enumerate}[\bf (1)]
    \item Work on analysis of your own data and your final paper, which should include:
      \begin{itemize}
        \item Introduction
        \item Methods (including model description)
        \item Results
        \item Discussion
      \end{itemize}
    \item Paper should be a minimum of 4 pages, single-spaced, 12-pt font
  \end{enumerate}
\end{frame}






\end{document}








