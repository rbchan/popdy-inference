\documentclass[color=usenames,dvipsnames]{beamer}\usepackage[]{graphicx}\usepackage[]{color}
% maxwidth is the original width if it is less than linewidth
% otherwise use linewidth (to make sure the graphics do not exceed the margin)
\makeatletter
\def\maxwidth{ %
  \ifdim\Gin@nat@width>\linewidth
    \linewidth
  \else
    \Gin@nat@width
  \fi
}
\makeatother

\definecolor{fgcolor}{rgb}{0, 0, 0}
\newcommand{\hlnum}[1]{\textcolor[rgb]{0.69,0.494,0}{#1}}%
\newcommand{\hlstr}[1]{\textcolor[rgb]{0.749,0.012,0.012}{#1}}%
\newcommand{\hlcom}[1]{\textcolor[rgb]{0.514,0.506,0.514}{\textit{#1}}}%
\newcommand{\hlopt}[1]{\textcolor[rgb]{0,0,0}{#1}}%
\newcommand{\hlstd}[1]{\textcolor[rgb]{0,0,0}{#1}}%
\newcommand{\hlkwa}[1]{\textcolor[rgb]{0,0,0}{\textbf{#1}}}%
\newcommand{\hlkwb}[1]{\textcolor[rgb]{0,0.341,0.682}{#1}}%
\newcommand{\hlkwc}[1]{\textcolor[rgb]{0,0,0}{\textbf{#1}}}%
\newcommand{\hlkwd}[1]{\textcolor[rgb]{0.004,0.004,0.506}{#1}}%
\let\hlipl\hlkwb

\usepackage{framed}
\makeatletter
\newenvironment{kframe}{%
 \def\at@end@of@kframe{}%
 \ifinner\ifhmode%
  \def\at@end@of@kframe{\end{minipage}}%
  \begin{minipage}{\columnwidth}%
 \fi\fi%
 \def\FrameCommand##1{\hskip\@totalleftmargin \hskip-\fboxsep
 \colorbox{shadecolor}{##1}\hskip-\fboxsep
     % There is no \\@totalrightmargin, so:
     \hskip-\linewidth \hskip-\@totalleftmargin \hskip\columnwidth}%
 \MakeFramed {\advance\hsize-\width
   \@totalleftmargin\z@ \linewidth\hsize
   \@setminipage}}%
 {\par\unskip\endMakeFramed%
 \at@end@of@kframe}
\makeatother

\definecolor{shadecolor}{rgb}{.97, .97, .97}
\definecolor{messagecolor}{rgb}{0, 0, 0}
\definecolor{warningcolor}{rgb}{1, 0, 1}
\definecolor{errorcolor}{rgb}{1, 0, 0}
\newenvironment{knitrout}{}{} % an empty environment to be redefined in TeX

\usepackage{alltt}
%\documentclass[color=usenames,dvipsnames,handout]{beamer}

\usepackage[roman]{../lectures}
%\usepackage[sans]{../lectures}
%\usepackage{mathrsfs}

\hypersetup{pdfpagemode=UseNone,pdfstartview={FitV}}



% Load function to compile and open PDF


% Compile and open PDF






% New command for inline code that isn't to be evaluated
\definecolor{inlinecolor}{rgb}{0.878, 0.918, 0.933}
\newcommand{\inr}[1]{\colorbox{inlinecolor}{\texttt{#1}}}
\IfFileExists{upquote.sty}{\usepackage{upquote}}{}
\begin{document}




\begin{frame}[plain]
  \LARGE
  \centering
  {
    \LARGE Lecture 9 -- Hierarchical distance sampling: \\
    \Large simulation, fitting, prediction, and random effects \\
  }
  {\color{default} \rule{\textwidth}{0.1pt} }
  \vfill
  \large
  WILD(FISH) 8390 \\
  Estimation of Fish and Wildlife Population Parameters \\
  \vfill
  \large
  Richard Chandler \\
  University of Georgia \\
\end{frame}






\section{Overview}



\begin{frame}[plain]
  \frametitle{Outline}
  \Large
  \only<1>{\tableofcontents}%[hideallsubsections]}
  \only<2 | handout:0>{\tableofcontents[currentsection]}%,hideallsubsections]}
\end{frame}



\begin{frame}
  \frametitle{Distance sampling overview}
  Distance sampling is one of the oldest wildlife sampling methods. \\
  \pause
  \vfill
  It's based on the simple idea that detection probability should
  decline with distance between an an animal and a transect. \\
  \pause
  \vfill
  If we can estimate the function describing how $p$ declines with
  distance $x$, we can estimate abundance \pause (if certain assumptions
  hold, as always). \\
\end{frame}



\begin{frame}
  \frametitle{Distance sampling overview}
  The simplest estimator of abundance is 
  \[
    \hat{N} = \frac{n}{\hat{p}}
  \]
  where $n$ is the number of individuals detected, $p$ is detection
  probability, and $E(n)=Np$. \\
  \pause
  \vfill
  In distance sampling, detection probability is a \alert{function} of
  distance, rather than a constant, such that all individuals have
  unique detection probabilities. \\
  \pause
  \vfill
  As a result, we have to replace
  $p$ with \alert{average} detection probability:
  \[
    \hat{N} = \frac{n}{\hat{\bar{p}}}
  \]
  \pause
  \vfill
  How do we compute average detection probability ($\bar{p}$)?
\end{frame}



\begin{frame}
  \frametitle{Detection functions}
  To estimate average detection probability ($\bar{p}$), we need:
  \begin{itemize}
    \item A detection function $g(x)$ describing how $p$ declines with
      distance.
    \item A probability distribution $p(x)$ describing the
      distances of all animals (detected and not detected). 
  \end{itemize}
  \pause
  \vfill
  \centering
  The most common functions are: \\
  \vspace{6pt}
  \begin{tabular}{lc}
%    \centering
    \hline
    Detection function & $g(x)$ \\
    \hline
    Half normal & $\exp(-x^2 / (2\sigma^2))$ \\
    Negative exponential & $\exp(-x/\sigma)$ \\
    Hazard rate & $1-\exp(-(x/a)^{-b})$ \\
    \hline
  \end{tabular}
\end{frame}


\begin{frame}[fragile]
  \frametitle{Half-normal}
  \footnotesize
  \[
    g(x,\sigma) = \exp(-x^2/(2\sigma^2))
  \]
  \vspace{-12pt}
  \centering
\begin{knitrout}\scriptsize
\definecolor{shadecolor}{rgb}{0.878, 0.918, 0.933}\color{fgcolor}\begin{kframe}
\begin{alltt}
\hlstd{sigma1} \hlkwb{<-} \hlnum{25}\hlstd{; sigma2} \hlkwb{<-} \hlnum{50}
\hlkwd{plot}\hlstd{(}\hlkwa{function}\hlstd{(}\hlkwc{x}\hlstd{)} \hlkwd{exp}\hlstd{(}\hlopt{-}\hlstd{x}\hlopt{^}\hlnum{2}\hlopt{/}\hlstd{(}\hlnum{2}\hlopt{*}\hlstd{sigma1}\hlopt{^}\hlnum{2}\hlstd{)),} \hlkwc{from}\hlstd{=}\hlnum{0}\hlstd{,} \hlkwc{to}\hlstd{=}\hlnum{100}\hlstd{,}
     \hlkwc{xlab}\hlstd{=}\hlstr{"Distance (x)"}\hlstd{,} \hlkwc{ylab}\hlstd{=}\hlstr{"Detection probability (p)"}\hlstd{)}
\hlkwd{plot}\hlstd{(}\hlkwa{function}\hlstd{(}\hlkwc{x}\hlstd{)} \hlkwd{exp}\hlstd{(}\hlopt{-}\hlstd{x}\hlopt{^}\hlnum{2}\hlopt{/}\hlstd{(}\hlnum{2}\hlopt{*}\hlstd{sigma2}\hlopt{^}\hlnum{2}\hlstd{)),} \hlkwc{from}\hlstd{=}\hlnum{0}\hlstd{,} \hlkwc{to}\hlstd{=}\hlnum{100}\hlstd{,} \hlkwc{add}\hlstd{=}\hlnum{TRUE}\hlstd{,} \hlkwc{col}\hlstd{=}\hlnum{4}\hlstd{)}
\hlkwd{legend}\hlstd{(}\hlnum{70}\hlstd{,} \hlnum{1}\hlstd{,} \hlkwd{c}\hlstd{(}\hlstr{"sigma=25"}\hlstd{,} \hlstr{"sigma=50"}\hlstd{),} \hlkwc{lty}\hlstd{=}\hlnum{1}\hlstd{,} \hlkwc{col}\hlstd{=}\hlkwd{c}\hlstd{(}\hlstr{"black"}\hlstd{,}\hlstr{"blue"}\hlstd{))}
\end{alltt}
\end{kframe}
\includegraphics[width=0.7\linewidth]{figure/hn-1} 

\end{knitrout}
\end{frame}




\begin{frame}[fragile]
  \frametitle{Negative exponential}
  \footnotesize
  \[
    g(x,\sigma) = \exp(-x/\sigma)
  \]
  \vspace{-12pt}
  \centering
\begin{knitrout}\scriptsize
\definecolor{shadecolor}{rgb}{0.878, 0.918, 0.933}\color{fgcolor}\begin{kframe}
\begin{alltt}
\hlstd{sigma1} \hlkwb{<-} \hlnum{25}\hlstd{; sigma2} \hlkwb{<-} \hlnum{50}
\hlkwd{plot}\hlstd{(}\hlkwa{function}\hlstd{(}\hlkwc{x}\hlstd{)} \hlkwd{exp}\hlstd{(}\hlopt{-}\hlstd{x}\hlopt{/}\hlstd{sigma1),} \hlkwc{from}\hlstd{=}\hlnum{0}\hlstd{,} \hlkwc{to}\hlstd{=}\hlnum{100}\hlstd{,}
     \hlkwc{xlab}\hlstd{=}\hlstr{"Distance (x)"}\hlstd{,} \hlkwc{ylab}\hlstd{=}\hlstr{"Detection probability (p)"}\hlstd{)}
\hlkwd{plot}\hlstd{(}\hlkwa{function}\hlstd{(}\hlkwc{x}\hlstd{)} \hlkwd{exp}\hlstd{(}\hlopt{-}\hlstd{x}\hlopt{/}\hlstd{sigma2),} \hlkwc{from}\hlstd{=}\hlnum{0}\hlstd{,} \hlkwc{to}\hlstd{=}\hlnum{100}\hlstd{,} \hlkwc{add}\hlstd{=}\hlnum{TRUE}\hlstd{,} \hlkwc{col}\hlstd{=}\hlnum{4}\hlstd{)}
\hlkwd{legend}\hlstd{(}\hlnum{70}\hlstd{,} \hlnum{1}\hlstd{,} \hlkwd{c}\hlstd{(}\hlstr{"sigma=25"}\hlstd{,} \hlstr{"sigma=50"}\hlstd{),} \hlkwc{lty}\hlstd{=}\hlnum{1}\hlstd{,} \hlkwc{col}\hlstd{=}\hlkwd{c}\hlstd{(}\hlstr{"black"}\hlstd{,}\hlstr{"blue"}\hlstd{))}
\end{alltt}
\end{kframe}
\includegraphics[width=0.7\linewidth]{figure/nexp-1} 

\end{knitrout}
\end{frame}





\begin{frame}[fragile]
  \frametitle{Hazard rate}
  \footnotesize
  \[
    g(x,a,b) = 1-\exp(-(x/a)^{-b})
  \]
  \vspace{-12pt}
  \centering
\begin{knitrout}\scriptsize
\definecolor{shadecolor}{rgb}{0.878, 0.918, 0.933}\color{fgcolor}\begin{kframe}
\begin{alltt}
\hlstd{a1} \hlkwb{<-} \hlnum{25}\hlstd{; a2} \hlkwb{<-} \hlnum{50}\hlstd{; b1} \hlkwb{<-} \hlnum{2}\hlstd{; b2} \hlkwb{<-} \hlnum{10}
\hlkwd{plot}\hlstd{(}\hlkwa{function}\hlstd{(}\hlkwc{x}\hlstd{)} \hlnum{1}\hlopt{-}\hlkwd{exp}\hlstd{(}\hlopt{-}\hlstd{(x}\hlopt{/}\hlstd{a1)}\hlopt{^}\hlstd{(}\hlopt{-}\hlstd{b1)),} \hlkwc{from}\hlstd{=}\hlnum{0}\hlstd{,} \hlkwc{to}\hlstd{=}\hlnum{100}\hlstd{,}
     \hlkwc{xlab}\hlstd{=}\hlstr{"Distance (x)"}\hlstd{,} \hlkwc{ylab}\hlstd{=}\hlstr{"Detection probability (p)"}\hlstd{,} \hlkwc{ylim}\hlstd{=}\hlnum{0}\hlopt{:}\hlnum{1}\hlstd{)}
\hlkwd{plot}\hlstd{(}\hlkwa{function}\hlstd{(}\hlkwc{x}\hlstd{)} \hlnum{1}\hlopt{-}\hlkwd{exp}\hlstd{(}\hlopt{-}\hlstd{(x}\hlopt{/}\hlstd{a2)}\hlopt{^}\hlstd{(}\hlopt{-}\hlstd{b2)),} \hlkwc{from}\hlstd{=}\hlnum{0}\hlstd{,} \hlkwc{to}\hlstd{=}\hlnum{100}\hlstd{,} \hlkwc{add}\hlstd{=}\hlnum{TRUE}\hlstd{,} \hlkwc{col}\hlstd{=}\hlnum{4}\hlstd{)}
\hlkwd{legend}\hlstd{(}\hlnum{70}\hlstd{,} \hlnum{1}\hlstd{,} \hlkwd{c}\hlstd{(}\hlstr{"a=25, b=2"}\hlstd{,} \hlstr{"a=50, b=10"}\hlstd{),} \hlkwc{lty}\hlstd{=}\hlnum{1}\hlstd{,} \hlkwc{col}\hlstd{=}\hlkwd{c}\hlstd{(}\hlstr{"black"}\hlstd{,}\hlstr{"blue"}\hlstd{))}
\end{alltt}
\end{kframe}
\includegraphics[width=0.7\linewidth]{figure/haz-1} 

\end{knitrout}
\end{frame}



\begin{frame}
  \frametitle{Average detection probability ($\bar{p}$)}
  Regardless of the chosen detection function, average detection
  probability is defined as: 
  \[
%     \bar{p} = \int g(x)p(x) \; \mathrm{d}x
     \bar{p} = \int_{b_1}^{b_2} g(x)p(x) \; \mathrm{d}x
   \]
   where $b_1$ and $b_2$ are the limits of the distance interval.
  \pause
  \vfill
  All that remains is the specification of $p(x)$, the
  distribution of distances (between animals and the transect).
  \pause
  \vfill
  To understand why $p(x)$ is needed, think about it this way:
  \begin{itemize}
    \item If most animals are close to the transect, $\bar{p}$ would
      be high
    \item If most animals are far from the transect, $\bar{p}$ would
      be low
  \end{itemize}
  % \pause
  % \vfill
  % The standard assumption (for line transects) is that animals are
  % uniformly distributed with respect to the transect
\end{frame}



\begin{frame}
  \frametitle{What should we use for $p(x)$?}
  What distribution should we use for the distances between animals
  and transects?
  \pause
  \vfill
  In \alert{line-transect sampling}, it is often assumed that animals
  are uniformly distributed with respect to the transect.
  \begin{itemize}
    \item Consequently, $p(x) = 1/B$ where $B$ is the width of the
      transect.
    \item This is guaranteed by random transect placement
    \item Can also be justified if animals are neither attracted to
      the transects or avoid them. 
  \end{itemize}
  \pause
  \vfill
  In \alert{point-transect sampling}, we make the same assumptions,
  but we recognize that area increases with distance from a point.
  \begin{itemize}
    \item Consequently, $p(x) = 2x/B^2$ (see pg. 408 in AHM)
  \end{itemize}
\end{frame}




\begin{frame}[fragile]
  \frametitle{Computing $\bar{p}$}
  Half-normal detection function and line-transect sampling.
  \vspace{-12pt}
\begin{knitrout}\footnotesize
\definecolor{shadecolor}{rgb}{0.878, 0.918, 0.933}\color{fgcolor}\begin{kframe}
\begin{alltt}
\hlstd{B} \hlkwb{<-} \hlnum{100}                                         \hlcom{## Transect width}
\hlstd{g} \hlkwb{<-} \hlkwa{function}\hlstd{(}\hlkwc{x}\hlstd{,} \hlkwc{sigma}\hlstd{=}\hlnum{25}\hlstd{)} \hlkwd{exp}\hlstd{(}\hlopt{-}\hlstd{x}\hlopt{^}\hlnum{2}\hlopt{/}\hlstd{(}\hlnum{2}\hlopt{*}\hlstd{sigma}\hlopt{^}\hlnum{2}\hlstd{))} \hlcom{## g(x)}
\hlstd{pdf} \hlkwb{<-} \hlkwa{function}\hlstd{(}\hlkwc{x}\hlstd{)} \hlnum{1}\hlopt{/}\hlstd{B}                           \hlcom{## p(x), constant}
\end{alltt}
\end{kframe}
\end{knitrout}
\pause
\vfill
  Do the integration
  \vspace{-12pt}
\begin{knitrout}\footnotesize
\definecolor{shadecolor}{rgb}{0.878, 0.918, 0.933}\color{fgcolor}\begin{kframe}
\begin{alltt}
\hlstd{gp} \hlkwb{<-} \hlkwa{function}\hlstd{(}\hlkwc{x}\hlstd{)} \hlkwd{g}\hlstd{(x)}\hlopt{*}\hlkwd{pdf}\hlstd{(x)}
\hlstd{(pbar} \hlkwb{<-} \hlkwd{integrate}\hlstd{(gp,} \hlkwc{lower}\hlstd{=}\hlnum{0}\hlstd{,} \hlkwc{upper}\hlstd{=B)}\hlopt{$}\hlstd{value)}
\end{alltt}
\begin{verbatim}
## [1] 0.3133087
\end{verbatim}
\end{kframe}
\end{knitrout}
\pause
\vfill
  Note the equivalence
  \vspace{-12pt}
\begin{knitrout}\footnotesize
\definecolor{shadecolor}{rgb}{0.878, 0.918, 0.933}\color{fgcolor}\begin{kframe}
\begin{alltt}
\hlstd{(pbar} \hlkwb{<-} \hlkwd{integrate}\hlstd{(g,} \hlkwc{lower}\hlstd{=}\hlnum{0}\hlstd{,} \hlkwc{upper}\hlstd{=B)}\hlopt{$}\hlstd{value} \hlopt{/} \hlstd{B)}
\end{alltt}
\begin{verbatim}
## [1] 0.3133087
\end{verbatim}
\end{kframe}
\end{knitrout}
\pause
\vfill
\centering
There's a 31.3\% chance of detecting an individual within 100 m of the 
transect. \\
\end{frame}




\begin{frame}
  \frametitle{In-class exercise}
  Building off of the previous example\dots
  \begin{enumerate}
    \item Compute $\bar{p}$ when $\sigma=50, 100, \mathrm{and}\, 200$, instead of
      $\sigma=25$. 
    \item Repeat step 1, but for point transect sampling where
      $p(x)=2x/B^2$, instead of $p(x)=1/B$. 
  \end{enumerate}
\end{frame}




% \begin{frame}
%   \frametitle{Distance sampling overview}
%   Distance sampling is one of the oldest wildlife sampling methods.
%   \pause
%   \vfill
%   \alert{Conventional} distance sampling focuses on estimating the
%   detection function, which describes how the probability of detecting
%   an individual decreases with distance from the observer. Abundance
%   and density are estimated, but there is no model for
%   spatial variation in abundance/density.
%   \pause
%   \vfill
%   \alert{Hierarchical} distance sampling is similar, but includes a
%   model for spatial variation in abundance/density -- it's really just
%   a multinomial $N$-mixture model with a unique function for computing
%   the multinomial cell probabilities.
% \end{frame}




\begin{frame}
  \frametitle{Hierarchical distance sampling}
  \small
  State model (with Poisson assumption)
  \begin{gather*}
    \mathrm{log}(\lambda_i) = \beta_0 + \beta_1 {\color{blue} w_{i1}} +
    \beta_2 {\color{blue} w_{i2}} + \cdots \\
    N_i \sim \mathrm{Poisson}(\lambda_i)
  \end{gather*}
  \pause
%  \vfill
  Observation model
  \begin{gather*}
    \mathrm{log}(\sigma_{i}) = \alpha_0 + \alpha_1 {\color{blue} w_{i1}}
    + \alpha_2 {\color{blue} w_{i3}} + \cdots \\
    \bar{p}_{ij} = \int_{c_j}^{c_{j+1}} g(x,\sigma_i)p(x) \; \mathrm{d}x \\
    \{y_{i1}, \dots, y_{iK}\}  \sim \mathrm{Multinomial}(N_i,
    \pi(\bar{p}_{i1}, \dots, \bar{p}_{iJ}))
  \end{gather*}
  \pause
  \vfill
  \small
  Definitions \\
  $\lambda_i$ -- Expected value of abundance at site $i$ \\
  $N_i$ -- Realized value of abundance at site $i$ \\
  $\sigma_{i}$ -- Scale parameter of detection function $g(x)$ at site $i$ \\
  $\pi(g(x))$ -- A function computing average detection probability in
  the $J$ distance bins \\
  $y_{ij}$ -- nAnimals detected in distance bin $j$ \\
%  \vfill
  $\color{blue} w_1$, $\color{blue} w_2$, $\color{blue} w_3$ -- site covariates %\hfill %\\
%  \vspace{12pt}
%  $\color{Purple} w$ -- observation covariate
\end{frame}












%\section{Simulation}

\section{Line transects}

\subsection{Likelihood-based methods}

\begin{frame}
  \frametitle{Outline}
  \Large
%  \tableofcontents[currentsection,currentsubsection]
  \tableofcontents[currentsection]
\end{frame}



\begin{frame}
  \frametitle{Bin-specific $\bar{p}$}
  \small
  dsf
  \pause
  \vfill
  Definitions
  \begin{itemize}
    \setlength\itemsep{1pt}
    \item $y_{ij}$ -- number of individuals removed at site $i$ on pass $j$
    \item $p$ -- probability of catching an individual on a single pass
  \end{itemize}
  \pause \vfill
  \footnotesize
  \begin{tabular}{lc}
    \hline
    \centering
    Description                       & Multinomial cell probability \\
    \hline
    $\bar{p}$ in first distance bin  & $\pi_1 = \int_0^b g(d)p(d) dd$                  \\
    {\centering $\cdots$}             & $\cdots$                     \\
    Pr(not detected)                  & $\pi_{J+1} = 1-\sum_1^J \pi_j$          \\
    \hline
  \end{tabular}
\end{frame}







\begin{frame}[fragile]
  \frametitle{Removal sampling, no covariates}
  \small
  Abundance
\begin{knitrout}\scriptsize
\definecolor{shadecolor}{rgb}{0.878, 0.918, 0.933}\color{fgcolor}\begin{kframe}
\begin{alltt}
\hlstd{nSites} \hlkwb{<-} \hlnum{100}
\hlstd{lambda1} \hlkwb{<-} \hlnum{2.6}  \hlcom{## Expected value of N}
\hlstd{N1} \hlkwb{<-} \hlkwd{rpois}\hlstd{(}\hlkwc{n}\hlstd{=nSites,} \hlkwc{lambda}\hlstd{=lambda1)}
\end{alltt}
\end{kframe}
\end{knitrout}
% \item
  \pause
  \vfill
  Capture probability and multinomial counts%, including individuals
%  \alert{not} detected
\begin{knitrout}\scriptsize
\definecolor{shadecolor}{rgb}{0.878, 0.918, 0.933}\color{fgcolor}\begin{kframe}
\begin{alltt}
\hlstd{nPasses} \hlkwb{<-} \hlnum{3}
\hlstd{K} \hlkwb{<-} \hlstd{nPasses}\hlopt{+}\hlnum{1}  \hlcom{# multinomial cells}
\hlstd{p1} \hlkwb{<-} \hlnum{0.3}
\hlstd{pi1} \hlkwb{<-} \hlkwd{c}\hlstd{(p1, (}\hlnum{1}\hlopt{-}\hlstd{p1)}\hlopt{*}\hlstd{p1, (}\hlnum{1}\hlopt{-}\hlstd{p1)}\hlopt{*}\hlstd{(}\hlnum{1}\hlopt{-}\hlstd{p1)}\hlopt{*}\hlstd{p1, (}\hlnum{1}\hlopt{-}\hlstd{p1)}\hlopt{^}\hlnum{3}\hlstd{)}
\hlstd{y1.all} \hlkwb{<-} \hlkwd{matrix}\hlstd{(}\hlnum{NA}\hlstd{,} \hlkwc{nrow}\hlstd{=nSites,} \hlkwc{ncol}\hlstd{=K)}
\hlkwa{for}\hlstd{(i} \hlkwa{in} \hlnum{1}\hlopt{:}\hlstd{nSites) \{}
    \hlstd{y1.all[i,]} \hlkwb{<-} \hlkwd{rmultinom}\hlstd{(}\hlkwc{n}\hlstd{=}\hlnum{1}\hlstd{,} \hlkwc{size}\hlstd{=N1[i],} \hlkwc{prob}\hlstd{=pi1)    \}}
\end{alltt}
\end{kframe}
\end{knitrout}
%\end{enumerate}
  \pause
  \vfill
  Discard final column of individuals not detected
\begin{knitrout}\scriptsize
\definecolor{shadecolor}{rgb}{0.878, 0.918, 0.933}\color{fgcolor}\begin{kframe}
\begin{alltt}
\hlstd{y1} \hlkwb{<-} \hlstd{y1.all[,}\hlopt{-}\hlstd{K]}
\hlkwd{head}\hlstd{(y1,} \hlkwc{n}\hlstd{=}\hlnum{3}\hlstd{)}
\end{alltt}
\begin{verbatim}
##      [,1] [,2] [,3]
## [1,]    0    0    0
## [2,]    2    1    2
## [3,]    2    0    1
\end{verbatim}
\end{kframe}
\end{knitrout}
\end{frame}



\begin{frame}[fragile]
  \frametitle{Removal model, covariates}
  \small
  Covariates
  \vspace{-6pt}
\begin{knitrout}\scriptsize
\definecolor{shadecolor}{rgb}{0.878, 0.918, 0.933}\color{fgcolor}\begin{kframe}
\begin{alltt}
\hlstd{streamDepth} \hlkwb{<-} \hlkwd{rnorm}\hlstd{(nSites)}
\end{alltt}
\end{kframe}
\end{knitrout}
% \item
\vfill
  Coefficients, $\lambda$, and $p$
  \vspace{-6pt}
\begin{knitrout}\scriptsize
\definecolor{shadecolor}{rgb}{0.878, 0.918, 0.933}\color{fgcolor}\begin{kframe}
\begin{alltt}
\hlstd{beta0} \hlkwb{<-} \hlnum{1}\hlstd{; beta1} \hlkwb{<-} \hlnum{0.5}
\hlstd{lambda2} \hlkwb{<-} \hlkwd{exp}\hlstd{(beta0} \hlopt{+} \hlstd{beta1}\hlopt{*}\hlstd{streamDepth)}
\hlstd{alpha0} \hlkwb{<-} \hlnum{0}\hlstd{; alpha1} \hlkwb{<-} \hlopt{-}\hlnum{1}
\hlstd{p2} \hlkwb{<-} \hlkwd{plogis}\hlstd{(alpha0} \hlopt{+} \hlstd{alpha1}\hlopt{*}\hlstd{streamDepth)}
\hlstd{pi2} \hlkwb{<-} \hlkwd{t}\hlstd{(}\hlkwd{sapply}\hlstd{(p2,} \hlkwa{function}\hlstd{(}\hlkwc{p}\hlstd{)} \hlkwd{c}\hlstd{(p, (}\hlnum{1}\hlopt{-}\hlstd{p)}\hlopt{*}\hlstd{p, (}\hlnum{1}\hlopt{-}\hlstd{p)}\hlopt{^}\hlnum{2}\hlopt{*}\hlstd{p, (}\hlnum{1}\hlopt{-}\hlstd{p)}\hlopt{^}\hlnum{3}\hlstd{)))}
\end{alltt}
\end{kframe}
\end{knitrout}
% \item
\vfill
  Simulate abundance and removal data
  \vspace{-6pt}
\begin{knitrout}\scriptsize
\definecolor{shadecolor}{rgb}{0.878, 0.918, 0.933}\color{fgcolor}\begin{kframe}
\begin{alltt}
\hlstd{N2} \hlkwb{<-} \hlkwd{rpois}\hlstd{(nSites,} \hlkwc{lambda}\hlstd{=lambda2)}         \hlcom{## local abundance }
\hlstd{y2.all} \hlkwb{<-} \hlkwd{matrix}\hlstd{(}\hlnum{NA}\hlstd{,} \hlkwc{nrow}\hlstd{=nSites,} \hlkwc{ncol}\hlstd{=K)}
\hlkwa{for}\hlstd{(i} \hlkwa{in} \hlnum{1}\hlopt{:}\hlstd{nSites) \{}
    \hlstd{y2.all[i,]} \hlkwb{<-} \hlkwd{rmultinom}\hlstd{(}\hlkwc{n}\hlstd{=}\hlnum{1}\hlstd{,} \hlkwc{size}\hlstd{=N2[i],} \hlkwc{prob}\hlstd{=pi2[i,])}
\hlstd{\}}
\hlstd{y2} \hlkwb{<-} \hlstd{y2.all[,}\hlopt{-}\hlstd{K]} \hlcom{## Discard final column... individuals not detected}
\end{alltt}
\end{kframe}
\end{knitrout}
%\end{enumerate}
\end{frame}




\begin{frame}[fragile]
  \frametitle{Simulated data}
  \begin{columns}
    \begin{column}{0.4\textwidth}
      \small
      Observations
%      \tiny
  \vspace{-6pt}
\begin{knitrout}\scriptsize
\definecolor{shadecolor}{rgb}{0.878, 0.918, 0.933}\color{fgcolor}\begin{kframe}
\begin{alltt}
\hlstd{y2[}\hlnum{1}\hlopt{:}\hlnum{19}\hlstd{,]}
\end{alltt}
\begin{verbatim}
##       [,1] [,2] [,3]
##  [1,]    3    0    0
##  [2,]    2    0    1
##  [3,]    2    0    0
##  [4,]    2    0    0
##  [5,]    0    0    0
##  [6,]    2    2    0
##  [7,]    3    0    0
##  [8,]    0    1    1
##  [9,]    0    0    0
## [10,]    0    1    0
## [11,]    0    0    0
## [12,]    2    0    0
## [13,]    0    2    0
## [14,]    1    1    0
## [15,]    2    2    1
## [16,]    4    0    0
## [17,]    0    0    2
## [18,]    3    3    0
## [19,]    1    0    0
\end{verbatim}
\end{kframe}
\end{knitrout}
  \end{column}
  \begin{column}{0.6\textwidth}
    \pause
%    \scriptsize
    {\centering Summary stats \\}
    \vspace{24pt}
    \small
    Proportion of sites known to be occupied
    \vspace{-6pt}
\begin{knitrout}\scriptsize
\definecolor{shadecolor}{rgb}{0.878, 0.918, 0.933}\color{fgcolor}\begin{kframe}
\begin{alltt}
\hlcom{# Max count at each site}
\hlstd{maxCounts} \hlkwb{<-} \hlkwd{apply}\hlstd{(y2,} \hlnum{1}\hlstd{, max)}
\hlstd{naiveOccupancy} \hlkwb{<-} \hlkwd{sum}\hlstd{(maxCounts}\hlopt{>}\hlnum{0}\hlstd{)}\hlopt{/}\hlstd{nSites}
\hlstd{naiveOccupancy}
\end{alltt}
\begin{verbatim}
## [1] 0.9
\end{verbatim}
\end{kframe}
\end{knitrout}
  \pause
  \vfill
  \small
  Captures on each pass
  \vspace{-6pt}
\begin{knitrout}\scriptsize
\definecolor{shadecolor}{rgb}{0.878, 0.918, 0.933}\color{fgcolor}\begin{kframe}
\begin{alltt}
\hlkwd{colSums}\hlstd{(y2)}
\end{alltt}
\begin{verbatim}
## [1] 117  55  28
\end{verbatim}
\end{kframe}
\end{knitrout}
  \pause
  \vfill
  Naive abundance
  \vspace{-6pt}
\begin{knitrout}\scriptsize
\definecolor{shadecolor}{rgb}{0.878, 0.918, 0.933}\color{fgcolor}\begin{kframe}
\begin{alltt}
\hlkwd{sum}\hlstd{(y2)}
\end{alltt}
\begin{verbatim}
## [1] 200
\end{verbatim}
\end{kframe}
\end{knitrout}

  \end{column}
  \end{columns}
\end{frame}









%\section{Prediction}



% \begin{frame}
%   \frametitle{Outline}
%   \Large
%   \tableofcontents[currentsection]
% \end{frame}






\begin{frame}[fragile]
  \frametitle{Prepare data in `unmarked'}
  \small
\begin{knitrout}\tiny
\definecolor{shadecolor}{rgb}{0.878, 0.918, 0.933}\color{fgcolor}\begin{kframe}
\begin{alltt}
\hlstd{umf} \hlkwb{<-} \hlkwd{unmarkedFrameMPois}\hlstd{(}\hlkwc{y}\hlstd{=y2,} \hlkwc{siteCovs}\hlstd{=}\hlkwd{data.frame}\hlstd{(streamDepth),} \hlkwc{type}\hlstd{=}\hlstr{"removal"}\hlstd{)}
\end{alltt}
\end{kframe}
\end{knitrout}
\pause
\begin{knitrout}\scriptsize
\definecolor{shadecolor}{rgb}{0.878, 0.918, 0.933}\color{fgcolor}\begin{kframe}
\begin{alltt}
\hlkwd{summary}\hlstd{(umf)}
\end{alltt}
\begin{verbatim}
## unmarkedFrame Object
## 
## 100 sites
## Maximum number of observations per site: 3 
## Mean number of observations per site: 3 
## Sites with at least one detection: 90 
## 
## Tabulation of y observations:
##   0   1   2   3   4 
## 166  82  39  12   1 
## 
## Site-level covariates:
##   streamDepth      
##  Min.   :-1.98353  
##  1st Qu.:-0.48806  
##  Median : 0.04055  
##  Mean   : 0.10013  
##  3rd Qu.: 0.78351  
##  Max.   : 2.64727
\end{verbatim}
\end{kframe}
\end{knitrout}
\end{frame}


% > fm <- multinomPois(~temp ~forest, umf)    

% error: Mat::operator(): index out of bounds
% terminate called after throwing an instance of 'std::logic_error'
%   what():  Mat::operator(): index out of bounds


\begin{frame}[fragile]
  \frametitle{Fit the model}
  \footnotesize
  \inr{multinomPois} has similar arguments as \inr{occu} and
  \inr{pcount}. 
\begin{knitrout}\tiny
\definecolor{shadecolor}{rgb}{0.878, 0.918, 0.933}\color{fgcolor}\begin{kframe}
\begin{alltt}
\hlstd{fm} \hlkwb{<-} \hlkwd{multinomPois}\hlstd{(}\hlopt{~}\hlstd{streamDepth} \hlopt{~}\hlstd{streamDepth, umf)}
\hlstd{fm}
\end{alltt}
\begin{verbatim}
## 
## Call:
## multinomPois(formula = ~streamDepth ~ streamDepth, data = umf)
## 
## Abundance:
##             Estimate     SE    z  P(>|z|)
## (Intercept)    0.853 0.0904 9.43 4.04e-21
## streamDepth    0.391 0.1269 3.08 2.09e-03
## 
## Detection:
##             Estimate    SE      z  P(>|z|)
## (Intercept)    0.198 0.205  0.968 3.33e-01
## streamDepth   -1.002 0.217 -4.623 3.78e-06
## 
## AIC: 590.9196
\end{verbatim}
\end{kframe}
\end{knitrout}
\pause
\vfill
Compare to actual parameter values:
\vspace{-6pt}
\begin{knitrout}\tiny
\definecolor{shadecolor}{rgb}{0.878, 0.918, 0.933}\color{fgcolor}\begin{kframe}
\begin{alltt}
\hlkwd{c}\hlstd{(}\hlkwc{beta0}\hlstd{=beta0,} \hlkwc{beta1}\hlstd{=beta1);} \hlkwd{c}\hlstd{(}\hlkwc{alpha0}\hlstd{=alpha0,} \hlkwc{alpha1}\hlstd{=alpha1)}
\end{alltt}
\begin{verbatim}
## beta0 beta1 
##   1.0   0.5
## alpha0 alpha1 
##      0     -1
\end{verbatim}
\end{kframe}
\end{knitrout}
\end{frame}








\begin{frame}[fragile]
  \frametitle{Prediction in `unmarked'}
  \small
  Create \texttt{data.frame} with prediction covariates. 
  \vspace{-6pt}
\begin{knitrout}\footnotesize
\definecolor{shadecolor}{rgb}{0.878, 0.918, 0.933}\color{fgcolor}\begin{kframe}
\begin{alltt}
\hlstd{pred.data} \hlkwb{<-} \hlkwd{data.frame}\hlstd{(}\hlkwc{streamDepth}\hlstd{=}\hlkwd{seq}\hlstd{(}\hlopt{-}\hlnum{3}\hlstd{,} \hlnum{3}\hlstd{,} \hlkwc{length}\hlstd{=}\hlnum{20}\hlstd{))}
\end{alltt}
\end{kframe}
\end{knitrout}
\pause
\vfill
Get predictions of $\lambda$ for each row of prediction data.
  \vspace{-6pt}
\begin{knitrout}\footnotesize
\definecolor{shadecolor}{rgb}{0.878, 0.918, 0.933}\color{fgcolor}\begin{kframe}
\begin{alltt}
\hlstd{lambda.pred} \hlkwb{<-} \hlkwd{predict}\hlstd{(fm,} \hlkwc{newdata}\hlstd{=pred.data,}
                       \hlkwc{type}\hlstd{=}\hlstr{'state'}\hlstd{,} \hlkwc{append}\hlstd{=}\hlnum{TRUE}\hlstd{)}
\end{alltt}
\end{kframe}
\end{knitrout}
\pause
\vfill
  View $\lambda$ predictions
  \vspace{-6pt}
\begin{knitrout}\footnotesize
\definecolor{shadecolor}{rgb}{0.878, 0.918, 0.933}\color{fgcolor}\begin{kframe}
\begin{alltt}
\hlkwd{print}\hlstd{(}\hlkwd{head}\hlstd{(lambda.pred),} \hlkwc{digits}\hlstd{=}\hlnum{2}\hlstd{)}
\end{alltt}
\begin{verbatim}
##   Predicted   SE lower upper streamDepth
## 1      0.73 0.27  0.36   1.5        -3.0
## 2      0.82 0.27  0.43   1.6        -2.7
## 3      0.93 0.27  0.53   1.6        -2.4
## 4      1.05 0.26  0.64   1.7        -2.1
## 5      1.19 0.25  0.78   1.8        -1.7
## 6      1.35 0.24  0.95   1.9        -1.4
\end{verbatim}
\end{kframe}
\end{knitrout}
\end{frame}





\begin{frame}[fragile]
  \frametitle{Prediction in `unmarked'}
\begin{knitrout}\tiny
\definecolor{shadecolor}{rgb}{0.878, 0.918, 0.933}\color{fgcolor}\begin{kframe}
\begin{alltt}
\hlkwd{plot}\hlstd{(Predicted} \hlopt{~} \hlstd{streamDepth, lambda.pred,} \hlkwc{ylab}\hlstd{=}\hlstr{"Expected value of abundance"}\hlstd{,}
     \hlkwc{ylim}\hlstd{=}\hlkwd{c}\hlstd{(}\hlnum{0}\hlstd{,}\hlnum{30}\hlstd{),} \hlkwc{xlab}\hlstd{=}\hlstr{"Stream depth"}\hlstd{,} \hlkwc{type}\hlstd{=}\hlstr{"l"}\hlstd{)}
\hlkwd{lines}\hlstd{(lower} \hlopt{~} \hlstd{streamDepth, lambda.pred,} \hlkwc{col}\hlstd{=}\hlstr{"grey"}\hlstd{)}
\hlkwd{lines}\hlstd{(upper} \hlopt{~} \hlstd{streamDepth, lambda.pred,} \hlkwc{col}\hlstd{=}\hlstr{"grey"}\hlstd{)}
\hlkwd{points}\hlstd{(}\hlkwd{rowSums}\hlstd{(y2)}\hlopt{~}\hlstd{streamDepth)}
\hlkwd{lines}\hlstd{(}\hlkwd{lowess}\hlstd{(}\hlkwd{rowSums}\hlstd{(y2)}\hlopt{~}\hlstd{streamDepth),} \hlkwc{col}\hlstd{=}\hlstr{"blue"}\hlstd{)}  \hlcom{## Loess line for fun (it's way off)}
\end{alltt}
\end{kframe}

{\centering \includegraphics[width=0.8\linewidth]{figure/pred-lam2-1} 

}



\end{knitrout}
\end{frame}







\begin{frame}[fragile]
  \frametitle{In-class exercise}
  % \small
  % \begin{enumerate}
  %   \item Predict
  %   \end{enumerate}
  %   \centering
%  \large
  Do the following using the fitted removal model above:
  \begin{enumerate}
    \normalsize
    \item Predict $p$ when \verb+streamDepth=-1+
    \item Use the prediction of $p$ to compute $\pi_1, \pi_2, \pi_3, \pi_4$
  \end{enumerate}
\end{frame}


\subsection{Bayesian methods}


\begin{frame}
  \frametitle{Outline}
  \Large
  \tableofcontents[currentsection,currentsubsection]
\end{frame}


\begin{frame}
  \frametitle{Bayesian multinomial $N$-mixture models}
  There are several equivalent formulations of the multinomial, some
  of which we can exploit to fit the model in JAGS.
  \begin{itemize}
    \item Conditional-on-$N$, missing $y_{iK}$
    \item Conditional-on-$N$, conditional on $n_i=\sum_{k=1}^{K-1} y_{i,k}$
    \item Conditional-on-$N$, sequential binomial
    \item Marginalized $N$
  \end{itemize}
  \pause
  \vfill
  These are just fun probability tricks that can help improve MCMC performance. \\
\end{frame}




% \begin{frame}[fragile]
%   \frametitle{Conditional-on-$N$, missing $y_k$}
% \end{frame}





\begin{frame}[fragile]
  \frametitle{\normalsize Conditional-on-$N$ and $n_i=\sum_{k=1}^{K-1} y_{i,k}$}
  Rather than treating the final multinomial cell as missing data:
  \[
    \{y_{i,1}, \dots, y_{i,K-1}, \mathtt{\color{red} NA}\} \sim
    \mathrm{Multinomial}(N_i, \{\pi_{i,1}, \dots, \pi_{i,K-1}, \pi_{i,K}\})
  \]
  \pause
  \vfill
  We can break the problem down into two steps by conditioning on
  $n_i$, the number of individuals captured at site $i$:
  \small
  \begin{gather*}
    n_i \sim \mathrm{Bin}(N_i, 1-\pi_K) \\
    \{y_{i,1}, \dots, y_{i,K-1}\} \sim \mathrm{Multinomial}(n_i,
    \{\pi_{i,1}, \dots, \pi_{i,K-1}\}/(1-\pi_K))
  \end{gather*}
  \pause
  \vfill
  Note that the modified cell probabilities still sum to 1. 
\end{frame}



\begin{frame}[fragile]
  \frametitle{\normalsize Conditional-on-$N$ and $n_i=\sum_{k=1}^{K-1} y_{i,k}$}
\vspace{-3pt}
\begin{knitrout}\scriptsize
\definecolor{shadecolor}{rgb}{0.878, 0.918, 0.933}\color{fgcolor}\begin{kframe}


{\ttfamily\noindent\color{warningcolor}{\#\# Warning in file(con, "{}r"{}): cannot open file 'removal-mod2.jag': No such file or directory}}

{\ttfamily\noindent\bfseries\color{errorcolor}{\#\# Error in file(con, "{}r"{}): cannot open the connection}}\end{kframe}
\end{knitrout}
\end{frame}





\begin{frame}[fragile]
  \frametitle{Data, inits, and parameters}
  Put data in a named list
  \vspace{-12pt}
\begin{knitrout}\small
\definecolor{shadecolor}{rgb}{0.878, 0.918, 0.933}\color{fgcolor}\begin{kframe}
\begin{alltt}
\hlstd{jags.data.rem2} \hlkwb{<-} \hlkwd{list}\hlstd{(}\hlkwc{y}\hlstd{=y2,} \hlkwc{n}\hlstd{=}\hlkwd{rowSums}\hlstd{(y2),}
                       \hlkwc{streamDepth}\hlstd{=streamDepth,}
                       \hlkwc{nSites}\hlstd{=nSites,} \hlkwc{nPasses}\hlstd{=nPasses)}
\end{alltt}
\end{kframe}
\end{knitrout}
\pause
\vfill
  Initial values
  \vspace{-12pt}
\begin{knitrout}\small
\definecolor{shadecolor}{rgb}{0.878, 0.918, 0.933}\color{fgcolor}\begin{kframe}
\begin{alltt}
\hlstd{jags.inits.rem} \hlkwb{<-} \hlkwa{function}\hlstd{() \{}
    \hlkwd{list}\hlstd{(}\hlkwc{lambda.intercept}\hlstd{=}\hlkwd{runif}\hlstd{(}\hlnum{1}\hlstd{),} \hlkwc{alpha0}\hlstd{=}\hlkwd{rnorm}\hlstd{(}\hlnum{1}\hlstd{),}
         \hlkwc{N}\hlstd{=}\hlkwd{rowSums}\hlstd{(y2)}\hlopt{+}\hlkwd{rpois}\hlstd{(}\hlkwd{nrow}\hlstd{(y2),} \hlnum{2}\hlstd{))}
\hlstd{\}}
\end{alltt}
\end{kframe}
\end{knitrout}
\pause
\vfill
  Parameters to monitor
  \vspace{-12pt}
\begin{knitrout}\small
\definecolor{shadecolor}{rgb}{0.878, 0.918, 0.933}\color{fgcolor}\begin{kframe}
\begin{alltt}
\hlstd{jags.pars.rem} \hlkwb{<-} \hlkwd{c}\hlstd{(}\hlstr{"beta0"}\hlstd{,} \hlstr{"beta1"}\hlstd{,}
                   \hlstr{"alpha0"}\hlstd{,} \hlstr{"alpha1"}\hlstd{,} \hlstr{"totalAbundance"}\hlstd{)}
\end{alltt}
\end{kframe}
\end{knitrout}
\end{frame}





\begin{frame}[fragile]
  \frametitle{MCMC}
  \small
\begin{knitrout}\tiny
\definecolor{shadecolor}{rgb}{0.878, 0.918, 0.933}\color{fgcolor}\begin{kframe}
\begin{alltt}
\hlkwd{library}\hlstd{(jagsUI)}
\hlstd{jags.post.rem2} \hlkwb{<-} \hlkwd{jags.basic}\hlstd{(}\hlkwc{data}\hlstd{=jags.data.rem2,} \hlkwc{inits}\hlstd{=jags.inits.rem,}
                             \hlkwc{parameters.to.save}\hlstd{=jags.pars.rem,} \hlkwc{model.file}\hlstd{=}\hlstr{"removal-mod2.jag"}\hlstd{,}
                             \hlkwc{n.chains}\hlstd{=}\hlnum{3}\hlstd{,} \hlkwc{n.adapt}\hlstd{=}\hlnum{100}\hlstd{,} \hlkwc{n.burnin}\hlstd{=}\hlnum{0}\hlstd{,} \hlkwc{n.iter}\hlstd{=}\hlnum{2000}\hlstd{,} \hlkwc{parallel}\hlstd{=}\hlnum{TRUE}\hlstd{)}
\end{alltt}


{\ttfamily\noindent\bfseries\color{errorcolor}{\#\# Error in checkForRemoteErrors(val): 3 nodes produced errors; first error: Cannot open model file "{}removal-mod2.jag"{}}}\end{kframe}
\end{knitrout}
%\end{frame}

\pause

%\begin{frame}[fragile]
%  \frametitle{Summarize output}
\begin{knitrout}\tiny
\definecolor{shadecolor}{rgb}{0.878, 0.918, 0.933}\color{fgcolor}\begin{kframe}
\begin{alltt}
\hlkwd{summary}\hlstd{(jags.post.rem2[,jags.pars.rem])}
\end{alltt}
\begin{verbatim}
## 
## Iterations = 1:2000
## Thinning interval = 1 
## Number of chains = 3 
## Sample size per chain = 2000 
## 
## 1. Empirical mean and standard deviation for each variable,
##    plus standard error of the mean:
## 
##                    Mean       SD Naive SE Time-series SE
## beta0            0.8665  0.09612 0.001241       0.004571
## beta1            0.4072  0.13444 0.001736       0.009476
## alpha0           0.1665  0.21570 0.002785       0.010526
## alpha1          -1.0127  0.22852 0.002950       0.013709
## totalAbundance 271.6515 34.81006 0.449396       2.830841
## 
## 2. Quantiles for each variable:
## 
##                    2.5%      25%      50%      75%    97.5%
## beta0            0.6854   0.8017   0.8628   0.9282   1.0668
## beta1            0.1734   0.3138   0.3982   0.4851   0.7073
## alpha0          -0.2614   0.0176   0.1777   0.3152   0.5747
## alpha1          -1.4810  -1.1607  -1.0107  -0.8636  -0.5596
## totalAbundance 226.0000 248.0000 264.0000 287.0000 360.0250
\end{verbatim}
\end{kframe}
\end{knitrout}
\end{frame}




\begin{frame}[fragile]
  \frametitle{Traceplots and density plots}
\begin{knitrout}\footnotesize
\definecolor{shadecolor}{rgb}{0.878, 0.918, 0.933}\color{fgcolor}\begin{kframe}
\begin{alltt}
\hlkwd{plot}\hlstd{(jags.post.rem2[,jags.pars.rem[}\hlnum{1}\hlopt{:}\hlnum{3}\hlstd{]])}
\end{alltt}
\end{kframe}

{\centering \includegraphics[width=0.7\textwidth]{figure/bugs-plot1-rem2-1} 

}



\end{knitrout}
\end{frame}



\begin{frame}[fragile]
  \frametitle{Traceplots and density plots}
\begin{knitrout}\footnotesize
\definecolor{shadecolor}{rgb}{0.878, 0.918, 0.933}\color{fgcolor}\begin{kframe}
\begin{alltt}
\hlkwd{plot}\hlstd{(jags.post.rem2[,jags.pars.rem[}\hlnum{4}\hlopt{:}\hlnum{5}\hlstd{]])}
\end{alltt}
\end{kframe}

{\centering \includegraphics[width=0.7\textwidth]{figure/bugs-plot2-rem2-1} 

}



\end{knitrout}
\end{frame}




\section{Point transects}



\section{Random effects}





\section{Assignment}




\begin{frame}[fragile]
  \frametitle{Assignment}
  % \small
  \footnotesize
  Create a self-contained R script or Rmarkdown file
  to do the following:
  \vfill
  \begin{enumerate}
%    \small
    \footnotesize
    \item Simulate \alert{independent} double observer data with the following
      properties:
      \begin{itemize}
        \item nSites=200
        \item $\lambda=3$
        \item $p_1=0.3$ and $p_2=0.5$
      \end{itemize}
    \item Fit the model using `unmarked'\footnote{\scriptsize You will
        need to create an observation covariate indicating observer A
        and B} and JAGS
    \item Use the point estimate of $p$ to compute the $\pi$ probabilities.
    \item Repeat steps 1-3 using the \alert{dependent} double observer
      method. 
  \end{enumerate}
  \vfill
  Upload your {\tt .R} or {\tt .Rmd} file to ELC before Monday. 
\end{frame}





\end{document}

