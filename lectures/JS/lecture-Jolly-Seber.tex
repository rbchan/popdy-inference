\documentclass[color=usenames,dvipsnames]{beamer}\usepackage[]{graphicx}\usepackage[]{color}
% maxwidth is the original width if it is less than linewidth
% otherwise use linewidth (to make sure the graphics do not exceed the margin)
\makeatletter
\def\maxwidth{ %
  \ifdim\Gin@nat@width>\linewidth
    \linewidth
  \else
    \Gin@nat@width
  \fi
}
\makeatother

\definecolor{fgcolor}{rgb}{0, 0, 0}
\newcommand{\hlnum}[1]{\textcolor[rgb]{0.69,0.494,0}{#1}}%
\newcommand{\hlstr}[1]{\textcolor[rgb]{0.749,0.012,0.012}{#1}}%
\newcommand{\hlcom}[1]{\textcolor[rgb]{0.514,0.506,0.514}{\textit{#1}}}%
\newcommand{\hlopt}[1]{\textcolor[rgb]{0,0,0}{#1}}%
\newcommand{\hlstd}[1]{\textcolor[rgb]{0,0,0}{#1}}%
\newcommand{\hlkwa}[1]{\textcolor[rgb]{0,0,0}{\textbf{#1}}}%
\newcommand{\hlkwb}[1]{\textcolor[rgb]{0,0.341,0.682}{#1}}%
\newcommand{\hlkwc}[1]{\textcolor[rgb]{0,0,0}{\textbf{#1}}}%
\newcommand{\hlkwd}[1]{\textcolor[rgb]{0.004,0.004,0.506}{#1}}%
\let\hlipl\hlkwb

\usepackage{framed}
\makeatletter
\newenvironment{kframe}{%
 \def\at@end@of@kframe{}%
 \ifinner\ifhmode%
  \def\at@end@of@kframe{\end{minipage}}%
  \begin{minipage}{\columnwidth}%
 \fi\fi%
 \def\FrameCommand##1{\hskip\@totalleftmargin \hskip-\fboxsep
 \colorbox{shadecolor}{##1}\hskip-\fboxsep
     % There is no \\@totalrightmargin, so:
     \hskip-\linewidth \hskip-\@totalleftmargin \hskip\columnwidth}%
 \MakeFramed {\advance\hsize-\width
   \@totalleftmargin\z@ \linewidth\hsize
   \@setminipage}}%
 {\par\unskip\endMakeFramed%
 \at@end@of@kframe}
\makeatother

\definecolor{shadecolor}{rgb}{.97, .97, .97}
\definecolor{messagecolor}{rgb}{0, 0, 0}
\definecolor{warningcolor}{rgb}{1, 0, 1}
\definecolor{errorcolor}{rgb}{1, 0, 0}
\newenvironment{knitrout}{}{} % an empty environment to be redefined in TeX

\usepackage{alltt}
%\documentclass[color=usenames,dvipsnames,handout]{beamer}

\usepackage[roman]{../lectures}
%\usepackage[sans]{../lectures}


% Compile and open PDF






%% New command for inline code that isn't to be evaluated
\definecolor{inlinecolor}{rgb}{0.878, 0.918, 0.933}
\newcommand{\inr}[1]{\colorbox{inlinecolor}{\texttt{#1}}}
\IfFileExists{upquote.sty}{\usepackage{upquote}}{}
\begin{document}




\begin{frame}[plain]
  \centering
  \LARGE
  % Lecture 17 \\
  Spatio-temporal variation in %\\
  survival, \\ recruitment, movement and abundance %with the \\
%  Jolly-Seber model \\
  \vfill
  \large
  WILD(FISH) 8390 \\
  Estimation of Fish and Wildlife Population Parameters \\
  \vfill
  Richard Chandler \\
  University of Georgia \\
\end{frame}




%\section{Intro}



\begin{frame}[plain]
  \frametitle{Topics}
  \Large
  \only<1>{\tableofcontents}%[hideallsubsections]}
%  \only<2 | handout:0>{\tableofcontents[currentsection]}%,hideallsubsections]}
\end{frame}




\section{Intro}






\begin{frame}
  \frametitle{Overview}
%  \large
  {Jolly-Seber model}
  \begin{itemize}
    \normalsize
    \item Extend CJS model to allow for recruitment
    \item We no longer ``condition on first capture''
    \item Typically, we use the robust design
  \end{itemize}
  \pause \vfill
  Bigger picture
  \begin{itemize}
    \normalsize
    \item<2-> We're interested in modeling population dynamics
    \item<3-> This framework allows for inference on \alert{all} of the
      key processes 
    \item<4-> Potential to model spatial, temporal, and individual-level
      variation vital rates
    \item<5-> Density-dependence and biotic interactions could be
      modeled too
  \end{itemize}
\end{frame}




% \begin{frame}
%   \frametitle{Spatial population dynamics}
%   \small
%   {Initial abundance at location $s$}
%   \[
%     N_{s,1} \sim \mbox{Poisson}(\lambda_{s,1})
%   \]
%   \pause \vfill
%   Survival and recruitment (local population dynamics)
%   \begin{gather*}
%     S_{s,t} \sim \mbox{Binomial}(N_{s,t-1}, \phi) \\
%     R_{s,t} \sim \mbox{Poisson}(N_{s,t-1} \gamma)
%   \end{gather*}
%   \pause \vfill
%   % {Recruitment}
%   % \[
%   %   R_{s,t} \sim \mbox{Poisson}(N_{s,t-1} \gamma)
%   % \]
%   % \pause \vfill
%   Movement (emigration and immigration)
%   \begin{gather*}
%     E_{s,t} \sim \mathrm{Binomial}(S_{s,t}+R_{s,t}, \kappa) \\
%     \{M_{1,s,t}, \dots, M_{J,s,t}\} \sim \mathrm{Multinomial}(E_{s,t}, \{\pi_1, \dots, \pi_J\}) \\
%     I_{s,t} = \sum_{j=1}^J M_{j,s,t}
%   \end{gather*}
%   % \vfill
%   % Immigration
%   % \[
%   % \]
%   % \vfill
%   \pause
%   {Abundance}
%   \[
%     N_{s,t} = S_{s,t} + R_{s,t} - E_{s,t} + I_{s,t}
%   \]
% \end{frame}



%% \begin{frame}
%%   \frametitle{Non-spatial CJS model}
%%   {\bf State model}
%%   \[
%%     z_{i,t} \sim \mbox{Bernoulli}(z_{i,t-1} \times \phi)
%%   \]
%%   \vfill
%%   {\bf Observation model}
%%   \[
%%     y_{i,t} \sim \mbox{Bernoulli}(z_{i,t} \times p)
%%   \]
%%   \pause
%%   \vfill
%%   \small
%%   where \\
%%   \begin{itemize}
%%     \item $z_{i,t}$ is ``alive state'' of individual $i$ at time $t$
%%     \item $\phi$ is ``apparent survival''. Probability of being alive and not permanently emigrating.
%%     \item $y_{i,t}=1$ if individual was encountered. $y_{i,t}=0$ otherwise.
%%   \end{itemize}
%% \end{frame}








% \section{Spatial JS}






% \begin{frame}
%   \frametitle{Spatial model}
%   \large
%   {Extensions}
%   \begin{itemize}
%     \item Individuals heterogeneity in vital rates
%     \item Spatial heterogeneity in vital rates
%     \item Density dependence
%     \item Dispersal
%   \end{itemize}
% \end{frame}




\begin{frame}
  \frametitle{\Large Individual-based spatial population dynamics}
  {Initial State}
  \begin{gather*}
    z_{i,1} \sim \mbox{Bernoulli}(\psi) \\
    {\bm s}_{i,1} \propto \lambda(\bs)
  \end{gather*}
  \vfill
  Survival and Recruitment
  \[
    z_{i,t} \sim
    \begin{cases}
      \mbox{Bernoulli}(z_{i,t-1}\phi) & \quad \text{if previously recruited} \\
      \mbox{Bernoulli}(\gamma'_t) & \quad \text{if not yet recruited} 
    \end{cases}
  \]
  \vfill
  Movement (Assuming random walk)
  \[
    \bsit \sim \mathrm{Norm}(\bsitp, \tau^2)  
  \]
  \vfill
  Abundance ($M$ is the upper bound of data augmentation)
  \[
    N_t = \sum_{i=1}^M z_{i,t}
  \]
\end{frame}





\begin{frame}[fragile]
  \frametitle{Simulating spatial JS data with robust design}
%  \scriptsize % \tiny %\small
  {Parameters and data dimensions}
\begin{knitrout}\scriptsize
\definecolor{shadecolor}{rgb}{0.878, 0.918, 0.933}\color{fgcolor}\begin{kframe}
\begin{alltt}
\hlstd{T} \hlkwb{<-} \hlnum{10}      \hlcom{# years/primary periods}
\hlstd{K} \hlkwb{<-} \hlnum{3}       \hlcom{# 3 secondary sampling occasion}
\hlstd{lambda} \hlkwb{<-} \hlnum{25} \hlcom{# Expected value of abundance in year 1}
\hlstd{M} \hlkwb{<-} \hlnum{500}     \hlcom{# Easiest way to simulate data is using data augmentation}
\hlstd{phi} \hlkwb{<-} \hlnum{0.7}   \hlcom{# Apparent survival}
\hlstd{gamma} \hlkwb{<-} \hlnum{0.3} \hlcom{# Per-capital recruitment rate}
\hlstd{p0} \hlkwb{<-} \hlnum{0.4}    \hlcom{# Baseline capture prob}
\hlstd{sigma} \hlkwb{<-} \hlnum{0.1} \hlcom{# Scale parameter of encounter rate function}
\end{alltt}
\end{kframe}
\end{knitrout}
\pause
{Traps, activity centers, and capture probability}
\begin{knitrout}\scriptsize
\definecolor{shadecolor}{rgb}{0.878, 0.918, 0.933}\color{fgcolor}\begin{kframe}
\begin{alltt}
\hlkwd{set.seed}\hlstd{(}\hlnum{340}\hlstd{)}
\hlstd{co} \hlkwb{<-} \hlkwd{seq}\hlstd{(}\hlnum{0.25}\hlstd{,} \hlnum{0.75}\hlstd{,} \hlkwc{length}\hlstd{=}\hlnum{5}\hlstd{)}
\hlstd{x} \hlkwb{<-} \hlkwd{cbind}\hlstd{(}\hlkwd{rep}\hlstd{(co,} \hlkwc{each}\hlstd{=}\hlnum{5}\hlstd{),} \hlkwd{rep}\hlstd{(co,} \hlkwc{times}\hlstd{=}\hlnum{5}\hlstd{))}
\hlstd{J} \hlkwb{<-} \hlkwd{nrow}\hlstd{(x)}  \hlcom{## nTraps}
\hlstd{xlim} \hlkwb{<-} \hlstd{ylim} \hlkwb{<-} \hlkwd{c}\hlstd{(}\hlnum{0}\hlstd{,}\hlnum{1}\hlstd{)}
\hlcom{## Activity centers, no dispersal}
\hlstd{s} \hlkwb{<-} \hlkwd{cbind}\hlstd{(}\hlkwd{runif}\hlstd{(M, xlim[}\hlnum{1}\hlstd{], xlim[}\hlnum{2}\hlstd{]),} \hlkwd{runif}\hlstd{(M, ylim[}\hlnum{1}\hlstd{], ylim[}\hlnum{2}\hlstd{]))}
\hlstd{d} \hlkwb{<-} \hlstd{p} \hlkwb{<-} \hlkwd{matrix}\hlstd{(}\hlnum{NA}\hlstd{, M, J)}
\hlkwa{for}\hlstd{(i} \hlkwa{in} \hlnum{1}\hlopt{:}\hlstd{M) \{}
    \hlstd{d[i,]} \hlkwb{<-} \hlkwd{sqrt}\hlstd{((s[i,}\hlnum{1}\hlstd{]}\hlopt{-}\hlstd{x[,}\hlnum{1}\hlstd{])}\hlopt{^}\hlnum{2} \hlopt{+} \hlstd{(s[i,}\hlnum{2}\hlstd{]}\hlopt{-}\hlstd{x[,}\hlnum{2}\hlstd{])}\hlopt{^}\hlnum{2}\hlstd{)}
    \hlstd{p[i,]} \hlkwb{<-} \hlstd{p0}\hlopt{*}\hlkwd{exp}\hlstd{(}\hlopt{-}\hlstd{d[i,]}\hlopt{^}\hlnum{2}\hlopt{/}\hlstd{(}\hlnum{2}\hlopt{*}\hlstd{sigma}\hlopt{^}\hlnum{2}\hlstd{)) \}} \hlcom{# capture prob }
\end{alltt}
\end{kframe}
\end{knitrout}
\end{frame}





\begin{frame}[fragile]
  \frametitle{Simulating spatial JS data with robust design}
{Generate $z$}
%\scriptsize
\begin{knitrout}\scriptsize
\definecolor{shadecolor}{rgb}{0.878, 0.918, 0.933}\color{fgcolor}\begin{kframe}
\begin{alltt}
\hlkwd{set.seed}\hlstd{(}\hlnum{034}\hlstd{)}
\hlstd{z} \hlkwb{<-} \hlstd{recruitable} \hlkwb{<-} \hlstd{died} \hlkwb{<-} \hlstd{recruited} \hlkwb{<-} \hlkwd{matrix}\hlstd{(}\hlnum{0}\hlstd{, M, T)}
\hlstd{z[,}\hlnum{1}\hlstd{]} \hlkwb{<-} \hlkwd{rbinom}\hlstd{(M,} \hlnum{1}\hlstd{, lambda}\hlopt{/}\hlstd{M)} \hlcom{# alive at t=1}
\hlstd{recruitable[,}\hlnum{1}\hlstd{]} \hlkwb{<-} \hlnum{1}\hlopt{-}\hlstd{z[,}\hlnum{1}\hlstd{]}
\hlstd{N} \hlkwb{<-} \hlkwd{integer}\hlstd{(T)}
\hlstd{N[}\hlnum{1}\hlstd{]} \hlkwb{<-} \hlkwd{sum}\hlstd{(z[,}\hlnum{1}\hlstd{])}
\hlkwa{for}\hlstd{(t} \hlkwa{in} \hlnum{2}\hlopt{:}\hlstd{T) \{}
    \hlstd{ER} \hlkwb{<-} \hlstd{N[t}\hlopt{-}\hlnum{1}\hlstd{]}\hlopt{*}\hlstd{gamma} \hlcom{# expected number of recruits}
    \hlstd{prevA} \hlkwb{<-} \hlkwd{sum}\hlstd{(recruitable[,t}\hlopt{-}\hlnum{1}\hlstd{])} \hlcom{# Number available to be recruited}
    \hlstd{gammaPrime} \hlkwb{<-} \hlstd{ER}\hlopt{/}\hlstd{prevA}
    \hlkwa{if}\hlstd{(gammaPrime} \hlopt{>} \hlnum{1}\hlstd{)} \hlkwd{stop}\hlstd{(}\hlstr{"M isn't big enough"}\hlstd{)}
    \hlstd{z[,t]} \hlkwb{<-} \hlkwd{rbinom}\hlstd{(M,} \hlnum{1}\hlstd{, (}\hlnum{1}\hlopt{-}\hlstd{recruitable[,t}\hlopt{-}\hlnum{1}\hlstd{])}\hlopt{*}\hlstd{z[,t}\hlopt{-}\hlnum{1}\hlstd{]}\hlopt{*}\hlstd{phi} \hlopt{+}
                          \hlstd{recruitable[,t}\hlopt{-}\hlnum{1}\hlstd{]}\hlopt{*}\hlstd{gammaPrime)}
    \hlstd{recruitable[,t]} \hlkwb{<-} \hlstd{recruitable[,t}\hlopt{-}\hlnum{1}\hlstd{]}\hlopt{*}\hlstd{(}\hlnum{1}\hlopt{-}\hlstd{z[,t])}
    \hlstd{N[t]} \hlkwb{<-} \hlkwd{sum}\hlstd{(z[,t])  \}}
\end{alltt}
\end{kframe}
\end{knitrout}
\pause
\vfill
{\normalsize Populaton size, mortalities, and recruits}
\begin{knitrout}\scriptsize
\definecolor{shadecolor}{rgb}{0.878, 0.918, 0.933}\color{fgcolor}\begin{kframe}
\begin{alltt}
\hlstd{died} \hlkwb{<-} \hlstd{(z[,}\hlnum{1}\hlopt{:}\hlstd{(T}\hlopt{-}\hlnum{1}\hlstd{)]}\hlopt{==}\hlnum{1}\hlstd{)} \hlopt{&} \hlstd{(z[,}\hlnum{2}\hlopt{:}\hlstd{T]}\hlopt{==}\hlnum{0}\hlstd{)}
\hlstd{recruited} \hlkwb{<-} \hlstd{(z[,}\hlnum{1}\hlopt{:}\hlstd{(T}\hlopt{-}\hlnum{1}\hlstd{)]}\hlopt{==}\hlnum{0}\hlstd{)} \hlopt{&} \hlstd{(z[,}\hlnum{2}\hlopt{:}\hlstd{T]}\hlopt{==}\hlnum{1}\hlstd{)}
\hlstd{Deaths} \hlkwb{<-} \hlkwd{colSums}\hlstd{(died)}
\hlstd{Recruits} \hlkwb{<-} \hlkwd{colSums}\hlstd{(recruited)}
\hlstd{everAlive} \hlkwb{<-} \hlkwd{sum}\hlstd{(}\hlkwd{rowSums}\hlstd{(z)}\hlopt{>}\hlnum{0}\hlstd{)}
\end{alltt}
\end{kframe}
\end{knitrout}
\end{frame}




\begin{frame}[fragile]
  \frametitle{Visualize dynamics}
\begin{knitrout}\tiny
\definecolor{shadecolor}{rgb}{0.878, 0.918, 0.933}\color{fgcolor}\begin{kframe}
\begin{alltt}
\hlkwd{plot}\hlstd{(}\hlnum{1}\hlopt{:}\hlstd{T, N,} \hlkwc{type}\hlstd{=}\hlstr{"b"}\hlstd{,} \hlkwc{xlab}\hlstd{=}\hlstr{"Year"}\hlstd{,} \hlkwc{ylab}\hlstd{=}\hlstr{""}\hlstd{,} \hlkwc{ylim}\hlstd{=}\hlkwd{c}\hlstd{(}\hlnum{0}\hlstd{,} \hlnum{50}\hlstd{),} \hlkwc{pch}\hlstd{=}\hlnum{16}\hlstd{)}
\hlkwd{lines}\hlstd{(}\hlnum{2}\hlopt{:}\hlstd{T, Deaths,} \hlkwc{type}\hlstd{=}\hlstr{"b"}\hlstd{,} \hlkwc{col}\hlstd{=}\hlstr{"red2"}\hlstd{,} \hlkwc{pch}\hlstd{=}\hlnum{16}\hlstd{);} \hlkwd{lines}\hlstd{(}\hlnum{2}\hlopt{:}\hlstd{T, Recruits,} \hlkwc{type}\hlstd{=}\hlstr{"b"}\hlstd{,} \hlkwc{col}\hlstd{=}\hlstr{"seagreen2"}\hlstd{,} \hlkwc{pch}\hlstd{=}\hlnum{16}\hlstd{)}
\hlkwd{legend}\hlstd{(}\hlnum{1}\hlstd{,} \hlnum{50}\hlstd{,} \hlkwd{c}\hlstd{(}\hlstr{"Abundance"}\hlstd{,}\hlstr{"Mortalities"}\hlstd{,}\hlstr{"Recruits"}\hlstd{),} \hlkwc{lty}\hlstd{=}\hlnum{1}\hlstd{,} \hlkwc{pch}\hlstd{=}\hlnum{16}\hlstd{,} \hlkwc{col}\hlstd{=}\hlkwd{c}\hlstd{(}\hlstr{"black"}\hlstd{,}\hlstr{"red2"}\hlstd{,}\hlstr{"seagreen2"}\hlstd{))}
\end{alltt}
\end{kframe}

{\centering \includegraphics[width=0.9\linewidth]{figure/dynamics1-1} 

}


\end{knitrout}
\end{frame}






\begin{frame}[fragile]
  \frametitle{Simulating spatial JS data with robust design}
{Generate encounter histories for all $M$ individuals}
\begin{knitrout}\footnotesize
\definecolor{shadecolor}{rgb}{0.878, 0.918, 0.933}\color{fgcolor}\begin{kframe}
\begin{alltt}
\hlstd{yall.bern} \hlkwb{<-} \hlkwd{array}\hlstd{(}\hlnum{NA}\hlstd{,} \hlkwd{c}\hlstd{(M, J, K, T))}   \hlcom{## For Bernoulli}
\hlstd{yall} \hlkwb{<-} \hlkwd{array}\hlstd{(}\hlnum{NA}\hlstd{,} \hlkwd{c}\hlstd{(M, J, T))}           \hlcom{## For Binomial}
\hlkwa{for}\hlstd{(i} \hlkwa{in} \hlnum{1}\hlopt{:}\hlstd{M) \{}
    \hlkwa{for}\hlstd{(t} \hlkwa{in} \hlnum{1}\hlopt{:}\hlstd{T) \{}
        \hlkwa{for}\hlstd{(j} \hlkwa{in} \hlnum{1}\hlopt{:}\hlstd{J) \{}
            \hlstd{yall.bern[i,j,}\hlnum{1}\hlopt{:}\hlstd{K,t]} \hlkwb{<-} \hlkwd{rbinom}\hlstd{(K,} \hlnum{1}\hlstd{, z[i,t]}\hlopt{*}\hlstd{p[i,j])}
            \hlstd{yall[i,j,t]} \hlkwb{<-} \hlkwd{rbinom}\hlstd{(}\hlnum{1}\hlstd{, K, z[i,t]}\hlopt{*}\hlstd{p[i,j])}
        \hlstd{\}}
    \hlstd{\}}
\hlstd{\}}
\end{alltt}
\end{kframe}
\end{knitrout}
\pause
\vfill
{\normalsize Discard individuals that were never captured}
\begin{knitrout}\footnotesize
\definecolor{shadecolor}{rgb}{0.878, 0.918, 0.933}\color{fgcolor}\begin{kframe}
\begin{alltt}
\hlstd{y.bern} \hlkwb{<-} \hlstd{yall.bern[}\hlkwd{rowSums}\hlstd{(yall.bern)}\hlopt{>}\hlnum{0}\hlstd{,,,]}
\hlstd{y} \hlkwb{<-} \hlstd{yall[}\hlkwd{rowSums}\hlstd{(yall)}\hlopt{>}\hlnum{0}\hlstd{,,]}
\hlkwd{str}\hlstd{(y)}
\end{alltt}
\begin{verbatim}
##  int [1:72, 1:25, 1:10] 0 0 0 0 0 0 0 0 0 0 ...
\end{verbatim}
\end{kframe}
\end{knitrout}
\end{frame}






% \begin{frame}[fragile]
%   \frametitle{Time series}
% %  \tiny
% <<NDR,size='tiny',out.width='60%',fig.align='center'>>=
% plot(1:T, N, ylim=c(0, 50), type="o", pch=16,
%      xlab="Year", ylab="")
% lines(2:T, Deaths[-1], col="red", type="o", pch=16)
% lines(2:T, Recruits[-1], col="blue", type="o", pch=16)
% legend(1, 50, c("Population size", "Deaths", "Recruits"),
%        col=c("black", "red", "blue"), pch=16, lty=1)
% @
% % \vspace{-3mm}
% % \begin{center}
% %   \includegraphics[width=0.8\textwidth]{Open-JS-NDR}
% % \end{center}
% \end{frame}




\section{Likelihood inference}



\begin{frame}[plain]
  \frametitle{Topics}
  \Large
  \tableofcontents[currentsection]
\end{frame}




\begin{frame}[fragile]
  \frametitle{Likelihood analysis of spatial JS model}
Begin by making a mask in `secr'
\begin{knitrout}\scriptsize
\definecolor{shadecolor}{rgb}{0.878, 0.918, 0.933}\color{fgcolor}\begin{kframe}
\begin{alltt}
\hlkwd{library}\hlstd{(openpopscr)}
\hlstd{trap.df} \hlkwb{<-} \hlkwd{data.frame}\hlstd{(x}\hlopt{*}\hlnum{1000}\hlstd{);} \hlkwd{colnames}\hlstd{(trap.df)} \hlkwb{<-} \hlkwd{c}\hlstd{(}\hlstr{"x"}\hlstd{,}\hlstr{"y"}\hlstd{)}
\hlstd{traps} \hlkwb{<-} \hlkwd{read.traps}\hlstd{(}\hlkwc{data}\hlstd{=trap.df,} \hlkwc{detector}\hlstd{=}\hlstr{"proximity"}\hlstd{)}
\hlstd{mask} \hlkwb{<-} \hlkwd{make.mask}\hlstd{(}\hlkwc{traps}\hlstd{=traps,} \hlkwc{buffer}\hlstd{=}\hlnum{250}\hlstd{)}
\hlkwd{plot}\hlstd{(mask);} \hlkwd{points}\hlstd{(traps,} \hlkwc{pch}\hlstd{=}\hlnum{3}\hlstd{,} \hlkwc{col}\hlstd{=}\hlstr{"blue"}\hlstd{,} \hlkwc{lwd}\hlstd{=}\hlnum{2}\hlstd{)}
\end{alltt}
\end{kframe}

{\centering \includegraphics[width=0.6\linewidth]{figure/mask-1} 

}


\end{knitrout}
\end{frame}


\begin{frame}[fragile]
  \frametitle{Likelihood analysis of spatial JS model}
Format for `secr' (with robust design)
\begin{knitrout}\scriptsize
\definecolor{shadecolor}{rgb}{0.878, 0.918, 0.933}\color{fgcolor}\begin{kframe}
\begin{alltt}
\hlstd{y.secr} \hlkwb{<-} \hlstd{y.bern}
\hlstd{year} \hlkwb{<-} \hlkwd{rep}\hlstd{(}\hlkwd{slice.index}\hlstd{(y.bern,} \hlnum{4}\hlstd{), y.secr)}  \hlcom{## Primary period}
\hlstd{day} \hlkwb{<-} \hlkwd{rep}\hlstd{(}\hlkwd{slice.index}\hlstd{(y.bern,} \hlnum{3}\hlstd{), y.secr)}   \hlcom{## Secondary period}
\hlstd{caps} \hlkwb{<-} \hlkwd{data.frame}\hlstd{(}\hlkwc{session}\hlstd{=}\hlnum{1}\hlstd{,}
                   \hlkwc{animal}\hlstd{=}\hlkwd{rep}\hlstd{(}\hlkwd{slice.index}\hlstd{(y.bern,} \hlnum{1}\hlstd{), y.secr),}
                   \hlkwc{occasion}\hlstd{=(year}\hlopt{-}\hlnum{1}\hlstd{)}\hlopt{*}\hlstd{K}\hlopt{+}\hlstd{day,}
                   \hlkwc{trap}\hlstd{=}\hlkwd{rep}\hlstd{(}\hlkwd{slice.index}\hlstd{(y.bern,} \hlnum{2}\hlstd{), y.secr))}
\hlstd{capthist} \hlkwb{<-} \hlkwd{make.capthist}\hlstd{(}\hlkwc{captures}\hlstd{=caps,} \hlkwc{traps}\hlstd{=traps,} \hlkwc{noccasions}\hlstd{=T}\hlopt{*}\hlstd{K)}
\end{alltt}
\end{kframe}
\end{knitrout}
\vfill
Then format for `openpopscr'
\begin{knitrout}\scriptsize
\definecolor{shadecolor}{rgb}{0.878, 0.918, 0.933}\color{fgcolor}\begin{kframe}
\begin{alltt}
\hlstd{js.data} \hlkwb{<-} \hlstd{ScrData}\hlopt{$}\hlkwd{new}\hlstd{(capthist, mask,} \hlkwc{primary}\hlstd{=}\hlkwd{rep}\hlstd{(}\hlnum{1}\hlopt{:}\hlstd{T,} \hlkwc{each}\hlstd{=K))}
\end{alltt}
\end{kframe}
\end{knitrout}
\end{frame}


\begin{frame}[fragile]
  \frametitle{Likelihood analysis}
Create the model object and then fit it
\begin{knitrout}\scriptsize
\definecolor{shadecolor}{rgb}{0.878, 0.918, 0.933}\color{fgcolor}\begin{kframe}
\begin{alltt}
start <- \hlkwd{get_start_values}(js.data, model = \hlstr{"JsModel"})
mod <- JsModel$\hlkwd{new}(\hlkwd{list}(lambda0~1, sigma~1, D~1, phi~1, beta~1), js.data,
                   start=start)
                   start=\hlkwd{list}(lambda0=0.1, sigma=100, D=100,
                              phi=0.8, beta=0.2))
mod$\hlkwd{fit}()
\hlcom{## mod}
\end{alltt}
\end{kframe}
\end{knitrout}
Back-transform
\begin{knitrout}\scriptsize
\definecolor{shadecolor}{rgb}{0.878, 0.918, 0.933}\color{fgcolor}\begin{kframe}
\begin{alltt}
\hlstd{mod}\hlopt{$}\hlkwd{get_par}\hlstd{(}\hlstr{"lambda0"}\hlstd{,} \hlkwc{k} \hlstd{=} \hlnum{1}\hlstd{,} \hlkwc{j} \hlstd{=} \hlnum{1}\hlstd{)}
\hlstd{mod}\hlopt{$}\hlkwd{get_par}\hlstd{(}\hlstr{"sigma"}\hlstd{,} \hlkwc{k} \hlstd{=} \hlnum{1}\hlstd{,} \hlkwc{j} \hlstd{=} \hlnum{1}\hlstd{)}
\hlstd{mod}\hlopt{$}\hlkwd{get_par}\hlstd{(}\hlstr{"D"}\hlstd{)}                 \hlcom{## Superpopulation density}
\hlstd{mod}\hlopt{$}\hlkwd{get_par}\hlstd{(}\hlstr{"beta"}\hlstd{,} \hlkwc{k} \hlstd{=} \hlnum{1}\hlstd{,} \hlkwc{m}\hlstd{=}\hlnum{1}\hlstd{)}  \hlcom{## Proportion of superpop alive in year 1}
\hlstd{mod}\hlopt{$}\hlkwd{get_par}\hlstd{(}\hlstr{"phi"}\hlstd{,} \hlkwc{k} \hlstd{=} \hlnum{1}\hlstd{,} \hlkwc{m}\hlstd{=}\hlnum{1}\hlstd{)}   \hlcom{## Survival}
\end{alltt}
\end{kframe}
\end{knitrout}
\end{frame}




\section{Bayesian inference}




\begin{frame}[plain]
  \frametitle{Topics}
  \Large
  \tableofcontents[currentsection]
\end{frame}



\begin{frame}[fragile]
  \frametitle{Spatial JS model in \jags}
%  \vspace{-3mm}
%  \tiny \fbox{\parbox{\linewidth}{\verbatiminput{JS-spatial.jag}}}
%  \tiny {\parbox{\linewidth}{\pagecolor{ProcessBlue}\verbatiminput{JS-spatial.jag}}}
  \tiny
\begin{knitrout}\tiny
\definecolor{shadecolor}{rgb}{0.678, 0.847, 0.902}\color{fgcolor}\begin{kframe}
\begin{verbatim}
model {
phi ~ dunif(0,1)      ## Survival
gamma ~ dunif(0, 5)   ## Per-capita recruitment rate
p0 ~ dunif(0,1)       ## Baseline capture probability
sigma ~ dunif(0, 2)   ## Scale parameter of capture function
psi ~ dunif(0,1)    
for(t in 1:T) {
  N[t] <- sum(z[,t])      ## Abundance
  ER[t] <- N[t]*gamma
  totalAvail[t] <- sum(recruitable[,t])
  gammaPrime[t] <- ER[t]/totalAvail[t] }
for(i in 1:M) {
  z[i,1] ~ dbern(psi)
  recruitable[i,1] <- 1 - z[i,1]
  s[i,1] ~ dunif(xlim[1], xlim[2]) # static activity centers
  s[i,2] ~ dunif(ylim[1], ylim[2])
  for(j in 1:J) {
    d[i,j] <- sqrt((s[i,1] - x[j,1])^2 + (s[i,2] - x[j,2])^2)
    p[i,j] <- p0*exp(-d[i,j]^2/(2*sigma^2))
    y[i,j,1] ~ dbin(z[i,1]*p[i,j], K)  }
  for(t in 2:T) {
    z[i,t] ~ dbern(z[i,t-1]*phi + recruitable[i,t-1]*gammaPrime[t-1])
    died[i,t] <- (z[i,t-1]==1) && (z[i,t]==0)
    recruited[i,t] <- (z[i,t-1]==0) && (z[i,t]==1)
    recruitable[i,t] <- recruitable[i,t-1]*(1-z[i,t])
    for(j in 1:J) {
      y[i,j,t] ~ dbin(z[i,t]*p[i,j], K) }  }
  everAlive[i] <- sum(z[i,]) > 0  }
for(t in 2:T) {
  Deaths[t-1] <- sum(died[,t])
  Recruits[t-1] <- sum(recruited[,t]) }
Ntot <- sum(everAlive)
}
\end{verbatim}
\end{kframe}
\end{knitrout}
\end{frame}






\begin{frame}[fragile]
  \frametitle{Bayesian analysis}
%  \footnotesize
  {Data augmentation}
\begin{knitrout}\scriptsize
\definecolor{shadecolor}{rgb}{0.878, 0.918, 0.933}\color{fgcolor}\begin{kframe}
\begin{alltt}
\hlstd{M} \hlkwb{<-} \hlkwd{nrow}\hlstd{(y)} \hlopt{+} \hlnum{50}   \hlcom{## Trial and error}
\hlstd{yz} \hlkwb{<-} \hlkwd{array}\hlstd{(}\hlnum{0}\hlstd{,} \hlkwd{c}\hlstd{(M, J, T))}
\hlstd{yz[}\hlnum{1}\hlopt{:}\hlkwd{nrow}\hlstd{(y),,]} \hlkwb{<-} \hlstd{y}
\hlstd{jags.data1} \hlkwb{<-} \hlkwd{list}\hlstd{(}\hlkwc{y}\hlstd{=yz,} \hlkwc{M}\hlstd{=M,} \hlkwc{x}\hlstd{=x,} \hlkwc{J}\hlstd{=J,} \hlkwc{K}\hlstd{=K,} \hlkwc{T}\hlstd{=T,} \hlkwc{xlim}\hlstd{=xlim,} \hlkwc{ylim}\hlstd{=ylim)}
\end{alltt}
\end{kframe}
\end{knitrout}
\pause
\vfill
  {\normalsize Initial values and parameters to monitor}
\begin{knitrout}\scriptsize
\definecolor{shadecolor}{rgb}{0.878, 0.918, 0.933}\color{fgcolor}\begin{kframe}
\begin{alltt}
\hlstd{zi} \hlkwb{<-} \hlkwd{matrix}\hlstd{(}\hlnum{0}\hlstd{, M, T)}
\hlstd{zi[}\hlnum{1}\hlopt{:}\hlkwd{nrow}\hlstd{(y),]} \hlkwb{<-} \hlnum{1}
\hlstd{ji1} \hlkwb{<-} \hlkwa{function}\hlstd{()} \hlkwd{list}\hlstd{(}\hlkwc{phi}\hlstd{=}\hlnum{0.01}\hlstd{,} \hlkwc{gamma}\hlstd{=}\hlnum{0.001}\hlstd{,} \hlkwc{z}\hlstd{=zi)}
\hlstd{jp1} \hlkwb{<-} \hlkwd{c}\hlstd{(}\hlstr{"phi"}\hlstd{,} \hlstr{"gamma"}\hlstd{,} \hlstr{"p0"}\hlstd{,} \hlstr{"sigma"}\hlstd{,} \hlstr{"N"}\hlstd{,} \hlstr{"Deaths"}\hlstd{,} \hlstr{"Recruits"}\hlstd{,} \hlstr{"Ntot"}\hlstd{)}
\end{alltt}
\end{kframe}
\end{knitrout}
\pause
\vfill
  {\normalsize Fit the model}
\begin{knitrout}\scriptsize
\definecolor{shadecolor}{rgb}{0.878, 0.918, 0.933}\color{fgcolor}\begin{kframe}
\begin{alltt}
\hlkwd{library}\hlstd{(jagsUI)}
\hlstd{jags.post.samples1} \hlkwb{<-} \hlkwd{jags.basic}\hlstd{(}\hlkwc{data}\hlstd{=jags.data1,} \hlkwc{inits}\hlstd{=ji1,}
                                 \hlkwc{parameters.to.save}\hlstd{=jp1,}
                                 \hlkwc{model.file}\hlstd{=}\hlstr{"JS-spatial.jag"}\hlstd{,}
                                 \hlkwc{n.chains}\hlstd{=}\hlnum{3}\hlstd{,} \hlkwc{n.adapt}\hlstd{=}\hlnum{100}\hlstd{,} \hlkwc{n.burnin}\hlstd{=}\hlnum{0}\hlstd{,}
                                 \hlkwc{n.iter}\hlstd{=}\hlnum{2000}\hlstd{,} \hlkwc{parallel}\hlstd{=}\hlnum{TRUE}\hlstd{)}
\end{alltt}
\end{kframe}
\end{knitrout}
\end{frame}







\begin{frame}[fragile]
  \frametitle{Is $M$ high enough?}
\begin{knitrout}\tiny
\definecolor{shadecolor}{rgb}{0.878, 0.918, 0.933}\color{fgcolor}\begin{kframe}
\begin{alltt}
\hlkwd{hist}\hlstd{(}\hlkwd{as.matrix}\hlstd{(jags.post.samples1[,}\hlstr{"Ntot"}\hlstd{]),} \hlkwc{xlab}\hlstd{=}\hlstr{"Total population size"}\hlstd{,}
     \hlkwc{ylab}\hlstd{=}\hlstr{""}\hlstd{,} \hlkwc{main}\hlstd{=}\hlstr{""}\hlstd{,} \hlkwc{freq}\hlstd{=}\hlnum{FALSE}\hlstd{,} \hlkwc{xlim}\hlstd{=}\hlkwd{c}\hlstd{(}\hlkwd{nrow}\hlstd{(y), M))}
\hlkwd{abline}\hlstd{(}\hlkwc{v}\hlstd{=M,} \hlkwc{lwd}\hlstd{=}\hlnum{3}\hlstd{,} \hlkwc{col}\hlstd{=}\hlstr{"blue"}\hlstd{)}
\end{alltt}
\end{kframe}

{\centering \includegraphics[width=0.7\linewidth]{figure/Ntot-1} 

}


\end{knitrout}
\end{frame}






\begin{frame}[fragile]
  \frametitle{Posterior distributions}
\begin{knitrout}\scriptsize
\definecolor{shadecolor}{rgb}{0.878, 0.918, 0.933}\color{fgcolor}\begin{kframe}
\begin{alltt}
\hlkwd{plot}\hlstd{(jags.post.samples1[,}\hlkwd{c}\hlstd{(}\hlstr{"phi"}\hlstd{,} \hlstr{"gamma"}\hlstd{,} \hlstr{"p0"}\hlstd{,} \hlstr{"sigma"}\hlstd{)])}
\end{alltt}
\end{kframe}

{\centering \includegraphics[width=0.6\linewidth]{figure/jc1-1} 

}


\end{knitrout}
% \begin{center}
%   \fbox{\includegraphics[width=0.7\textwidth]{Open-JS-jc1}}
% \end{center}
\end{frame}






\begin{frame}[fragile]
  \frametitle{Posterior distributions}


\begin{center}
  \fbox{\includegraphics[width=0.45\textwidth]{figure/jcN1-4-1}}
  \fbox{\includegraphics[width=0.45\textwidth]{figure/jcN5-8-1}}
\end{center}
\end{frame}





\begin{frame}[fragile]
  \frametitle{Actual and estimated abundance}
  Extract and summarize posterior samples of $N_t$
\begin{knitrout}\scriptsize
\definecolor{shadecolor}{rgb}{0.878, 0.918, 0.933}\color{fgcolor}\begin{kframe}
\begin{alltt}
\hlstd{Npost} \hlkwb{<-} \hlkwd{as.matrix}\hlstd{(jags.post.samples1[,}\hlkwd{paste}\hlstd{(}\hlstr{"N["}\hlstd{,} \hlnum{1}\hlopt{:}\hlnum{10}\hlstd{,} \hlstr{"]"}\hlstd{,} \hlkwc{sep}\hlstd{=}\hlstr{""}\hlstd{)])}
\hlstd{Nmed} \hlkwb{<-} \hlkwd{apply}\hlstd{(Npost,} \hlnum{2}\hlstd{, median)}
\hlstd{Nupper} \hlkwb{<-} \hlkwd{apply}\hlstd{(Npost,} \hlnum{2}\hlstd{, quantile,} \hlkwc{prob}\hlstd{=}\hlnum{0.975}\hlstd{)}
\hlstd{Nlower} \hlkwb{<-} \hlkwd{apply}\hlstd{(Npost,} \hlnum{2}\hlstd{, quantile,} \hlkwc{prob}\hlstd{=}\hlnum{0.025}\hlstd{)}
\end{alltt}
\end{kframe}
\end{knitrout}
  \pause
  \vfill
  Plot
\begin{knitrout}\scriptsize
\definecolor{shadecolor}{rgb}{0.878, 0.918, 0.933}\color{fgcolor}\begin{kframe}
\begin{alltt}
\hlkwd{plot}\hlstd{(}\hlnum{1}\hlopt{:}\hlstd{T, N,} \hlkwc{type}\hlstd{=}\hlstr{"b"}\hlstd{,} \hlkwc{ylim}\hlstd{=}\hlkwd{c}\hlstd{(}\hlnum{0}\hlstd{,} \hlnum{60}\hlstd{),} \hlkwc{xlab}\hlstd{=}\hlstr{"Time"}\hlstd{,}
     \hlkwc{ylab}\hlstd{=}\hlstr{"Abundance"}\hlstd{,} \hlkwc{pch}\hlstd{=}\hlnum{16}\hlstd{)}
\hlkwd{arrows}\hlstd{(}\hlnum{1}\hlopt{:}\hlstd{T, Nlower,} \hlnum{1}\hlopt{:}\hlstd{T, Nupper,} \hlkwc{angle}\hlstd{=}\hlnum{90}\hlstd{,} \hlkwc{code}\hlstd{=}\hlnum{3}\hlstd{,}
       \hlkwc{length}\hlstd{=}\hlnum{0.05}\hlstd{,} \hlkwc{col}\hlstd{=}\hlkwd{gray}\hlstd{(}\hlnum{0.7}\hlstd{))}
\hlkwd{points}\hlstd{(}\hlnum{1}\hlopt{:}\hlstd{T, Nmed,} \hlkwc{pch}\hlstd{=}\hlnum{16}\hlstd{,} \hlkwc{col}\hlstd{=}\hlkwd{gray}\hlstd{(}\hlnum{0.7}\hlstd{))}
\hlkwd{legend}\hlstd{(}\hlnum{1}\hlstd{,} \hlnum{60}\hlstd{,} \hlkwd{c}\hlstd{(}\hlstr{"Actual"}\hlstd{,} \hlstr{"Estimated"}\hlstd{),}
       \hlkwc{col}\hlstd{=}\hlkwd{c}\hlstd{(}\hlstr{"black"}\hlstd{,} \hlkwd{gray}\hlstd{(}\hlnum{0.7}\hlstd{)),} \hlkwc{lty}\hlstd{=}\hlkwd{c}\hlstd{(}\hlnum{NA}\hlstd{,}\hlnum{1}\hlstd{),} \hlkwc{pch}\hlstd{=}\hlkwd{c}\hlstd{(}\hlnum{16}\hlstd{,}\hlnum{16}\hlstd{))}
\end{alltt}
\end{kframe}
\end{knitrout}
\end{frame}





\begin{frame}
  \frametitle{Actual and estimated abundance}
%  \vspace{-4mm}
  \begin{center}
    \includegraphics[width=\textwidth]{figure/Npost-1}
  \end{center}
\end{frame}






\begin{frame}[fragile]
  \frametitle{Actual and estimated recruits}
  Extract and summarize posterior samples of $R_t$
\begin{knitrout}\scriptsize
\definecolor{shadecolor}{rgb}{0.878, 0.918, 0.933}\color{fgcolor}\begin{kframe}
\begin{alltt}
\hlstd{Rpost} \hlkwb{<-} \hlkwd{as.matrix}\hlstd{(jags.post.samples1[,}\hlkwd{paste}\hlstd{(}\hlstr{"Recruits["}\hlstd{,}\hlnum{1}\hlopt{:}\hlnum{9}\hlstd{,}\hlstr{"]"}\hlstd{,}\hlkwc{sep}\hlstd{=}\hlstr{""}\hlstd{)])}
\hlstd{Rmed} \hlkwb{<-} \hlkwd{apply}\hlstd{(Rpost,} \hlnum{2}\hlstd{, median)}
\hlstd{Rupper} \hlkwb{<-} \hlkwd{apply}\hlstd{(Rpost,} \hlnum{2}\hlstd{, quantile,} \hlkwc{prob}\hlstd{=}\hlnum{0.975}\hlstd{)}
\hlstd{Rlower} \hlkwb{<-} \hlkwd{apply}\hlstd{(Rpost,} \hlnum{2}\hlstd{, quantile,} \hlkwc{prob}\hlstd{=}\hlnum{0.025}\hlstd{)}
\end{alltt}
\end{kframe}
\end{knitrout}
  \pause
  \vfill
  Plot
\begin{knitrout}\scriptsize
\definecolor{shadecolor}{rgb}{0.878, 0.918, 0.933}\color{fgcolor}\begin{kframe}
\begin{alltt}
\hlkwd{plot}\hlstd{(}\hlnum{1}\hlopt{:}\hlstd{(T}\hlopt{-}\hlnum{1}\hlstd{), Recruits,} \hlkwc{type}\hlstd{=}\hlstr{"b"}\hlstd{,} \hlkwc{ylim}\hlstd{=}\hlkwd{c}\hlstd{(}\hlnum{0}\hlstd{,} \hlnum{30}\hlstd{),} \hlkwc{xlab}\hlstd{=}\hlstr{"Time"}\hlstd{,}
     \hlkwc{ylab}\hlstd{=}\hlstr{"Recruits"}\hlstd{,} \hlkwc{pch}\hlstd{=}\hlnum{16}\hlstd{,} \hlkwc{col}\hlstd{=}\hlstr{"seagreen2"}\hlstd{)}
\hlkwd{arrows}\hlstd{(}\hlnum{1}\hlopt{:}\hlstd{(T}\hlopt{-}\hlnum{1}\hlstd{), Rlower,} \hlnum{1}\hlopt{:}\hlstd{(T}\hlopt{-}\hlnum{1}\hlstd{), Rupper,} \hlkwc{angle}\hlstd{=}\hlnum{90}\hlstd{,} \hlkwc{code}\hlstd{=}\hlnum{3}\hlstd{,}
       \hlkwc{length}\hlstd{=}\hlnum{0.05}\hlstd{,} \hlkwc{col}\hlstd{=}\hlkwd{gray}\hlstd{(}\hlnum{0.7}\hlstd{))}
\hlkwd{points}\hlstd{(}\hlnum{1}\hlopt{:}\hlstd{(T}\hlopt{-}\hlnum{1}\hlstd{), Rmed,} \hlkwc{pch}\hlstd{=}\hlnum{16}\hlstd{,} \hlkwc{col}\hlstd{=}\hlkwd{gray}\hlstd{(}\hlnum{0.7}\hlstd{))}
\hlkwd{legend}\hlstd{(}\hlnum{1}\hlstd{,} \hlnum{30}\hlstd{,} \hlkwd{c}\hlstd{(}\hlstr{"Actual"}\hlstd{,} \hlstr{"Estimated"}\hlstd{),}
       \hlkwc{col}\hlstd{=}\hlkwd{c}\hlstd{(}\hlstr{"seagreen2"}\hlstd{,} \hlkwd{gray}\hlstd{(}\hlnum{0.7}\hlstd{)),} \hlkwc{lty}\hlstd{=}\hlkwd{c}\hlstd{(}\hlnum{NA}\hlstd{,}\hlnum{1}\hlstd{),} \hlkwc{pch}\hlstd{=}\hlkwd{c}\hlstd{(}\hlnum{16}\hlstd{,}\hlnum{16}\hlstd{))}
\end{alltt}
\end{kframe}
\end{knitrout}
\end{frame}





\begin{frame}
  \frametitle{Actual and estimated recruits}
%  \vspace{-4mm}
  \begin{center}
    \includegraphics[width=\textwidth]{figure/Rpost-1}
  \end{center}
\end{frame}




\begin{frame}[fragile]
  \frametitle{Actual and estimated mortalities}
  Extract and summarize posterior samples of $D_t=N_t-S_t$
\begin{knitrout}\scriptsize
\definecolor{shadecolor}{rgb}{0.878, 0.918, 0.933}\color{fgcolor}\begin{kframe}
\begin{alltt}
\hlstd{Dpost} \hlkwb{<-} \hlkwd{as.matrix}\hlstd{(jags.post.samples1[,}\hlkwd{paste}\hlstd{(}\hlstr{"Deaths["}\hlstd{,} \hlnum{1}\hlopt{:}\hlnum{9}\hlstd{,} \hlstr{"]"}\hlstd{,} \hlkwc{sep}\hlstd{=}\hlstr{""}\hlstd{)])}
\hlstd{Dmed} \hlkwb{<-} \hlkwd{apply}\hlstd{(Dpost,} \hlnum{2}\hlstd{, median)}
\hlstd{Dupper} \hlkwb{<-} \hlkwd{apply}\hlstd{(Dpost,} \hlnum{2}\hlstd{, quantile,} \hlkwc{prob}\hlstd{=}\hlnum{0.975}\hlstd{)}
\hlstd{Dlower} \hlkwb{<-} \hlkwd{apply}\hlstd{(Dpost,} \hlnum{2}\hlstd{, quantile,} \hlkwc{prob}\hlstd{=}\hlnum{0.025}\hlstd{)}
\end{alltt}
\end{kframe}
\end{knitrout}
  \pause
  \vfill
  Plot
\begin{knitrout}\scriptsize
\definecolor{shadecolor}{rgb}{0.878, 0.918, 0.933}\color{fgcolor}\begin{kframe}
\begin{alltt}
\hlkwd{plot}\hlstd{(}\hlnum{1}\hlopt{:}\hlstd{(T}\hlopt{-}\hlnum{1}\hlstd{), Deaths,} \hlkwc{type}\hlstd{=}\hlstr{"b"}\hlstd{,} \hlkwc{ylim}\hlstd{=}\hlkwd{c}\hlstd{(}\hlnum{0}\hlstd{,} \hlnum{30}\hlstd{),} \hlkwc{xlab}\hlstd{=}\hlstr{"Time"}\hlstd{,}
     \hlkwc{ylab}\hlstd{=}\hlstr{"Mortalities"}\hlstd{,} \hlkwc{pch}\hlstd{=}\hlnum{16}\hlstd{,} \hlkwc{col}\hlstd{=}\hlstr{"red2"}\hlstd{)}
\hlkwd{arrows}\hlstd{(}\hlnum{1}\hlopt{:}\hlstd{(T}\hlopt{-}\hlnum{1}\hlstd{), Dlower,} \hlnum{1}\hlopt{:}\hlstd{(T}\hlopt{-}\hlnum{1}\hlstd{), Dupper,} \hlkwc{angle}\hlstd{=}\hlnum{90}\hlstd{,} \hlkwc{code}\hlstd{=}\hlnum{3}\hlstd{,}
       \hlkwc{length}\hlstd{=}\hlnum{0.05}\hlstd{,} \hlkwc{col}\hlstd{=}\hlkwd{gray}\hlstd{(}\hlnum{0.7}\hlstd{))}
\hlkwd{points}\hlstd{(}\hlnum{1}\hlopt{:}\hlstd{(T}\hlopt{-}\hlnum{1}\hlstd{), Dmed,} \hlkwc{pch}\hlstd{=}\hlnum{16}\hlstd{,} \hlkwc{col}\hlstd{=}\hlkwd{gray}\hlstd{(}\hlnum{0.7}\hlstd{))}
\hlkwd{legend}\hlstd{(}\hlnum{1}\hlstd{,} \hlnum{30}\hlstd{,} \hlkwd{c}\hlstd{(}\hlstr{"Actual"}\hlstd{,} \hlstr{"Estimated"}\hlstd{),}
       \hlkwc{col}\hlstd{=}\hlkwd{c}\hlstd{(}\hlstr{"red2"}\hlstd{,} \hlkwd{gray}\hlstd{(}\hlnum{0.7}\hlstd{)),} \hlkwc{lty}\hlstd{=}\hlkwd{c}\hlstd{(}\hlnum{NA}\hlstd{,}\hlnum{1}\hlstd{),} \hlkwc{pch}\hlstd{=}\hlkwd{c}\hlstd{(}\hlnum{16}\hlstd{,}\hlnum{16}\hlstd{))}
\end{alltt}
\end{kframe}
\end{knitrout}
\end{frame}





\begin{frame}
  \frametitle{Actual and estimated mortalities}
%  \vspace{-4mm}
  \begin{center}
    \includegraphics[width=\textwidth]{figure/Dpost-1}
  \end{center}
\end{frame}








\begin{frame}
  \frametitle{Summary}
  This is perhaps the most general approach to modeling population
  dynamics. \\
  \pause \vfill
  Requires a lot of data on marked individuals, but it should be no
  surprise that we need a lot of data to understand complex
  processes. \\ 
  % \pause \vfill
  % Go out there and collect good data. \\
\end{frame}




\end{document}





% \section{Spatial JS with density-dependence}







% \begin{frame}
%   \frametitle{Density-dependent recruitment}
%   \large
%   {Logistic}
%   \[
%      \gamma_t = \gamma_{max}(1 - N_{t-1}/K)
%   \]
%   \vfill
%   {Log-linear}
%   \[
%      \gamma_t = \nu_0\exp(-\nu_1 N_{t-1})
%   \]
%   \vfill
%   {Log-linear (alt version)}
%   \[
%     \log(\gamma_t) = \nu_0 + \nu_1 N_{t-1}
%   \]
% \end{frame}






% \begin{frame}[fragile]
%   \frametitle{Density-dependent recruitment}
%   \tiny
% <<dd1,fig.height=5,out.width='90%',fig.align='center'>>=
% N <- 0:50
% nu0 <- 2
% nu1 <- 0.05
% plot(N, nu0*exp(-nu1*N), type="l", ylim=c(0,2), ylab="Per-capita Recruitment")
% @
% %\vspace{-4mm}
% %\begin{center}
% %  \includegraphics[width=\textwidth]{figure/dd1-1}
% %\end{center}
% \end{frame}




% \begin{frame}[fragile]
%   \frametitle{Simulating spatial JS data with robust design}
%   \scriptsize % \tiny %\small
%   {Parameters and data dimensions}
% <<sim-pars-robust>>=
% T <- 10      # years/primary periods
% K <- 3       # 3 secondary sampling occasion
% ## Is it necessary to be far from equilibrium to detect density-dependence?
% ## Equilibrium here is where (1-phi) == gamma, where gamma is function of N
% N0 <- 10     # Abundance in year 1
% M <- 500     # Easiest way to simulate data is using data augmentation
% phi <- 0.7   # Apparent survival
% ##gamma <- 0.3 # Per-capital recruitment rate
% nu0 <- 2
% nu1 <- 0.05
% p0 <- 0.4
% sigma <- 0.1
% @
% \pause
% {Traps, activity centers, and detection probability}
% <<sim-p-robust>>=
% set.seed(3479)
% co <- seq(0.25, 0.75, length=5)
% x <- cbind(rep(co, each=5), rep(co, times=5))
% J <- nrow(x)
% xlim <- ylim <- c(0,1)
% s <- cbind(runif(M, xlim[1], xlim[2]), runif(M, ylim[1], ylim[2]))
% d <- p <- matrix(NA, M, J)
% for(i in 1:M) {
%     d[i,] <- sqrt((s[i,1]-x[,1])^2 + (s[i,2]-x[,2])^2)
%     p[i,] <- p0*exp(-d[i,]^2/(2*sigma^2))
% }
% @
% \end{frame}





% \begin{frame}[fragile]
%   \frametitle{Simulating spatial JS data with robust design}
% {Generate $z$}
% \scriptsize
% <<sim-RS-robust>>=
% set.seed(3401)
% z2 <- recruitable <- died <- recruited <- matrix(0, M, T)
% z2[1:N0,1] <- 1 # First N0 are alive
% recruitable[(N0+1):M,1] <- 1
% for(t in 2:T) {
%     prevN <- sum(z2[,t-1]) # number alive at t-1
%     gamma <- nu0*exp(-nu1*prevN) ## Density dependent recruitment rate
%     ER <- prevN*gamma # expected number of recruits
%     prevA <- sum(recruitable[,t-1]) # Number available to be recruited
%     gammaPrime <- ER/prevA
%     if(gammaPrime > 1)
%         stop("M isn't big enough")
%     for(i in 1:M) {
%         z2[i,t] <- rbinom(1, 1, z2[i,t-1]*phi + recruitable[i,t-1]*gammaPrime)
%         recruitable[i,t] <- 1 - max(z2[i,1:(t)]) # to be recruited
%         died[i,t] <- z2[i,t-1]==1 & z2[i,t]==0
%         recruited[i,t] <- z2[i,t]==1 & z2[i,t-1]==0
%     }
% }
% @
% \pause
% \vfill
% {\normalsize Populaton size, mortalities, and recruits}
% <<sim-N-robust>>=
% N2 <- colSums(z2) # Population size
% Deaths2 <- colSums(died)
% Recruits2 <- colSums(recruited)
% everAlive2 <- sum(rowSums(z2)>0)
% @
% \end{frame}









% \begin{frame}[fragile]
%   \frametitle{Simulating spatial JS data with robust design}
% {Generate encounter histories for all $M$ individuals}
% \footnotesize
% <<sim-yall-robust>>=
% yall <- array(NA, c(M, J, K, T))
% for(i in 1:M) {
%     for(t in 1:T) {
%         for(j in 1:J) {
%             yall[i,j,1:K,t] <- rbinom(K, 1, z2[i,t]*p[i,j])
%         }
%     }
% }
% @
% \pause
% \vfill
% {\normalsize Discard individuals that were never captured}
% <<sim-y-robust>>=
% detected <- rowSums(yall) > 0
% y2 <- yall[detected,,,]
% str(y2)
% @
% \end{frame}






% \begin{frame}[fragile]
%   \frametitle{Time series}
%   \tiny
% <<NDR-DD,include=FALSE,echo=FALSE,fig.width=8,fig.height=6>>=
% plot(1:T, N2, ylim=c(0, 50), type="o", pch=16,
%      xlab="Year", ylab="")
% lines(2:T, Deaths2[-1], col="red", type="o", pch=16)
% lines(2:T, Recruits2[-1], col="blue", type="o", pch=16)
% legend(1, 50, c("Population size", "Deaths", "Recruits"),
%        col=c("black", "red", "blue"), pch=16, lty=1)
% @
% \vspace{-3mm}
% \begin{center}
%   \includegraphics[width=\textwidth]{figure/NDR-DD-1}
% \end{center}
% \end{frame}



% \begin{frame}[fragile]
%   \frametitle{Spatial CJS model in \jags}
%   \vspace{-5mm}
%   \tiny \fbox{\parbox{\linewidth}{\verbatiminput{JS-spatial-DD.jag}}}
% \end{frame}





% \begin{frame}[fragile]
%   \frametitle{\jags}
% %  \footnotesize
%   {Data augmentation}
%   \scriptsize
% <<aug-robust>>=
% M2 <- nrow(y2) + 75
% yz2 <- array(0, c(M2, J, K, T))
% yz2[1:nrow(y2),,,] <- y2
% @
% \pause
% \vfill
%   {\normalsize Initial values for $z$ matrix}
% <<zi-robust>>=
% zi <- matrix(0, M2, T)
% ##zi[1:nrow(y2),] <- 1
% zi[1:nrow(y2),] <- z2[detected,] ## cheating
% ji2 <- function() list(phi=0.01, z=zi)
% @
% \pause
% \vfill
%   {\normalsize Fit the model}
% <<jags-run-robust,results='hide',cache=TRUE>>=
% jd2 <- list(y=yz2, M=M2, x=x,
%             J=J, K=K, T=T, xlim=xlim, ylim=ylim)
% jp2 <- c("phi", "nu0", "nu1", "p0", "sigma", "N", "Deaths", "Recruits", "Ntot")
% library(rjags)
% jm2 <- jags.model("JS-spatial-DD.jag", jd2, ji2, n.chains=1, n.adapt=2)
% jc2 <- coda.samples(jm2, jp2, 5)
% ##jc2.2 <- coda.samples(jm2, jp2, 15000)
% @
% \end{frame}




% \begin{frame}[fragile]
%   \frametitle{Posterior distributions}
% <<jc2,include=FALSE,echo=FALSE>>=
% plot(jc2[,c("phi", "nu0", "nu1")])
% @
% \begin{center}
%   \fbox{\includegraphics[width=0.7\textwidth]{figure/jc2-1}}
% \end{center}
% \end{frame}










% \begin{frame}[fragile]
%   \frametitle{Actual and estimated abundance}
%   {Extract and summarize posterior samples of $N_t$}
%   \footnotesize
% <<N-post-samples>>=
% Npost <- as.matrix(jc2[,paste("N[", 1:10, "]", sep="")])
% Nmed <- apply(Npost, 2, median)
% Nupper <- apply(Npost, 2, quantile, prob=0.975)
% Nlower <- apply(Npost, 2, quantile, prob=0.025)
% @
%   \pause
%   \vfill
%   {\normalsize Plot}
% <<Npost,include=FALSE>>=
% plot(1:T, N2, type="o", col="blue", ylim=c(0, 100), xlab="Time",
%      ylab="Abundance")
% points(1:T, Nmed)
% arrows(1:T, Nlower, 1:T, Nupper, angle=90, code=3, length=0.05)
% legend(1, 100, c("Actual abundance", "Estimated abundance"),
%        col=c("blue", "black"), lty=c(1,1), pch=c(1,1))
% @
% \end{frame}





% \begin{frame}
%   \frametitle{Actual and estimated abundance}
%   \vspace{-4mm}
%   \begin{center}
%     \includegraphics[width=0.8\textwidth]{figure/Npost-1}
%   \end{center}
% \end{frame}










\begin{frame}
  \frametitle{Summary}
  \large
  {Key points}
  \begin{itemize}[<+->]
    \item Spatial Jolly-Seber models make it possible to fit
      spatio-temporal models of population dynamics to standard data
    \item We could have movement just like we did with CJS models
    \item Lots of tricks for speeding up MCMC
    \begin{itemize}
      \item Replace \inr{ER[t] <- N[t]*gamma} with \inr{ER[t] <- EN[t]*gamma} where \inr{EN} is the expected value of $N$.
      \item Repalce \inr{recruitable[i,t-1]*gammaPrime[t-1]} with \inr{Erecruitable[i,t-1]*gammaPrime[t-1]}, which again is expected value rather than the realized value
      \item These tricks result in approximations, but they should be very close to desired inferences. 
    \end{itemize}
  \end{itemize}
\end{frame}



% \begin{frame}
%   \frametitle{Assignment}
%   {\large For next week}
%   \begin{enumerate}[\bf (1)]
%     \item Work on analysis of your own data and your final paper, which should include:
%       \begin{itemize}
%         \item Introduction
%         \item Methods (including model description)
%         \item Results
%         \item Discussion
%       \end{itemize}
%     \item Paper should be a minimum of 4 pages, single-spaced, 12-pt font
%   \end{enumerate}
% \end{frame}






\end{document}








