\documentclass[color=usenames,dvipsnames]{beamer}\usepackage[]{graphicx}\usepackage[]{color}
% maxwidth is the original width if it is less than linewidth
% otherwise use linewidth (to make sure the graphics do not exceed the margin)
\makeatletter
\def\maxwidth{ %
  \ifdim\Gin@nat@width>\linewidth
    \linewidth
  \else
    \Gin@nat@width
  \fi
}
\makeatother

\definecolor{fgcolor}{rgb}{0, 0, 0}
\newcommand{\hlnum}[1]{\textcolor[rgb]{0.69,0.494,0}{#1}}%
\newcommand{\hlstr}[1]{\textcolor[rgb]{0.749,0.012,0.012}{#1}}%
\newcommand{\hlcom}[1]{\textcolor[rgb]{0.514,0.506,0.514}{\textit{#1}}}%
\newcommand{\hlopt}[1]{\textcolor[rgb]{0,0,0}{#1}}%
\newcommand{\hlstd}[1]{\textcolor[rgb]{0,0,0}{#1}}%
\newcommand{\hlkwa}[1]{\textcolor[rgb]{0,0,0}{\textbf{#1}}}%
\newcommand{\hlkwb}[1]{\textcolor[rgb]{0,0.341,0.682}{#1}}%
\newcommand{\hlkwc}[1]{\textcolor[rgb]{0,0,0}{\textbf{#1}}}%
\newcommand{\hlkwd}[1]{\textcolor[rgb]{0.004,0.004,0.506}{#1}}%
\let\hlipl\hlkwb

\usepackage{framed}
\makeatletter
\newenvironment{kframe}{%
 \def\at@end@of@kframe{}%
 \ifinner\ifhmode%
  \def\at@end@of@kframe{\end{minipage}}%
  \begin{minipage}{\columnwidth}%
 \fi\fi%
 \def\FrameCommand##1{\hskip\@totalleftmargin \hskip-\fboxsep
 \colorbox{shadecolor}{##1}\hskip-\fboxsep
     % There is no \\@totalrightmargin, so:
     \hskip-\linewidth \hskip-\@totalleftmargin \hskip\columnwidth}%
 \MakeFramed {\advance\hsize-\width
   \@totalleftmargin\z@ \linewidth\hsize
   \@setminipage}}%
 {\par\unskip\endMakeFramed%
 \at@end@of@kframe}
\makeatother

\definecolor{shadecolor}{rgb}{.97, .97, .97}
\definecolor{messagecolor}{rgb}{0, 0, 0}
\definecolor{warningcolor}{rgb}{1, 0, 1}
\definecolor{errorcolor}{rgb}{1, 0, 0}
\newenvironment{knitrout}{}{} % an empty environment to be redefined in TeX

\usepackage{alltt}
%\documentclass[color=usenames,dvipsnames,handout]{beamer}

\usepackage[roman]{../lectures}
%\usepackage[sans]{../lectures}


\hypersetup{pdfpagemode=UseNone,pdfstartview={FitV}}



% Load function to compile and open PDF


% Compile and open PDF






% New command for inline code that isn't to be evaluated
\definecolor{inlinecolor}{rgb}{0.878, 0.918, 0.933}
\newcommand{\inr}[1]{\colorbox{inlinecolor}{\texttt{#1}}}
\IfFileExists{upquote.sty}{\usepackage{upquote}}{}
\begin{document}




\begin{frame}[plain]
  \LARGE
  \centering
  {
    \LARGE Lecture 10 -- Non-spatial mark-recapture: \\
    \Large simulation, fitting, and prediction \\
  }
  {\color{default} \rule{\textwidth}{0.1pt} }
  \vfill
  \large
  WILD(FISH) 8390 \\
  Estimation of Fish and Wildlife Population Parameters \\
  \vfill
  \large
  Richard Chandler \\
  University of Georgia \\
\end{frame}






\section{Overview}



\begin{frame}[plain]
  \frametitle{Outline}
  \Large
  \only<1>{\tableofcontents}%[hideallsubsections]}
  \only<2 | handout:0>{\tableofcontents[currentsection]}%,hideallsubsections]}
\end{frame}



\begin{frame}
  \frametitle{Mark-recapture overview}
  Until now, we've focused on models for data on aggregated quantities
  like abundance and occupancy. \\ 
  \pause
  \vfill
  Now we will talk about data on uniquely identifable individuals. \\
  \pause
  \vfill
  This is nice because all individuals are different, and we're often
  interested in these differences, even when our primary goal is
  making inferences about population-level parameters like abundance
  and density. \\
\end{frame}



\begin{frame}
  \frametitle{Mark-recapture overview}
  The simplest estimator of abundance is 
  \[
    \hat{N} = \frac{n}{\hat{p}}
  \]
  where $n$ is the number of individuals detected, $p$ is detection
  probability, and $E(n)=Np$. \\
  \pause
  \vfill
  In distance sampling, detection probability is a \alert{function} of
  distance, rather than a constant, such that all individuals have
  unique detection probabilities. \\
  \pause
  \vfill
  As a result, we have to replace
  $p$ with \alert{average} detection probability:
  \[
    \hat{N} = \frac{n}{\hat{\bar{p}}}
  \]
  \pause
  \vfill
  How do we compute average detection probability ($\bar{p}$)?
\end{frame}










% \begin{frame}
%   \frametitle{In-class exercise}
%   Building off the previous example\dots
%   \begin{enumerate}
%     \item Compute $\bar{p}$ for line-transect sampling when
%       $\sigma=50, 100, \mathrm{and}\, 200$, instead of $\sigma=25$.  
%     \item Repeat, but for point-transect sampling. 
%   \end{enumerate}
% \end{frame}



\begin{frame}
  \frametitle{Hierarchical distance sampling}
  \small
  State model (with Poisson assumption)
  \begin{gather*}
    \mathrm{log}(\lambda_i) = \beta_0 + \beta_1 {\color{blue} w_{i1}} +
    \beta_2 {\color{blue} w_{i2}} + \cdots \\
    N_i \sim \mathrm{Poisson}(\lambda_i)
  \end{gather*}
  \pause
  Observation model
  \begin{gather*}
    \mathrm{log}(\sigma_{i}) = \alpha_0 + \alpha_1 {\color{blue} w_{i1}}
    + \alpha_2 {\color{blue} w_{i3}} + \cdots \\
    \{y_{i1}, \dots, y_{iK}\}  \sim \mathrm{Multinomial}(N_i,
    \pi(b_1, \dots, b_{J+1}, x, \sigma_i))
  \end{gather*}
  \pause
  \small
  Definitions \\
  $\lambda_i$ -- Expected value of abundance at site $i$ \\
  $N_i$ -- Realized value of abundance at site $i$ \\
  $\sigma_{i}$ -- Scale parameter of detection function $g(x)$ at site $i$ \\
  $\pi(x,b,\sigma_i)$ -- Function computing multinomial cell probs \\
  $y_{ij}$ -- count for distance bin $j$ (final count is unobserved) \\
  $\color{blue} w_1$, $\color{blue} w_2$, $\color{blue} w_3$ -- site covariates %\hfill %\\
\end{frame}











%\section{Simulation}

\section{Model $M_0$}






\subsection{Simulation}

\begin{frame}
  \frametitle{Outline}
  \Large
%  \tableofcontents[currentsection,currentsubsection]
  \tableofcontents[currentsection]
\end{frame}



\begin{frame}
  \frametitle{Multinomial cell probs}
  \small
  Definitions needed for computing \alert{bin-specific} $\bar{p}$ and
  multinomial cell probabilities. 
  \begin{itemize}
  \small
    \setlength\itemsep{1pt}
    \item $y_{ij}$ -- number of individuals detected at site $i$ in bin $j$
    \item $\sigma_i$ -- scale parameter of detection function $g(x)$
    \item $b_1, \dots, b_{J+1}$ -- Distance break points defining $J$ distance intervals
    \item $\bar{p}_j = \int_{b_j}^{b_{j+1}} g(x,\sigma)p(x|b_j\le x<b_{j+1})\, \mathrm{d}x$
    \item $p(x|b_j\le x<b_{j+1}) = 1/(b_{j+1}-b_j)$
    \item $\psi_j=\Pr(b_j\le x<b_{j+1})=(b_{j+1}-b_j)/B$ % -- Pr(occuring in distance bin $j$)
  \end{itemize}
  \pause \vfill
  \footnotesize
  \begin{columns}
    \column{0.9\paperwidth}
    \begin{tabular}{lc}
      \hline
      \centering
      Description                       & Multinomial cell probability \\
      \hline
      Pr(occurs and detected in first distance bin)  & $\pi_1 = \psi_1\bar{p}_1$   \\
      Pr(occurs and detected in second distance bin)  & $\pi_2 = \psi_2\bar{p}_2$   \\
      {\centering $\cdots$}             & $\cdots$                     \\
      Pr(occurs and detected in last distance bin)  & $\pi_J = \psi_J\bar{p}_J$   \\
      Pr(not detected)                  & $\pi_{K} = 1-\sum_{j=1}^J \pi_j$          \\
      \hline
    \end{tabular}
  \end{columns}
\end{frame}








\begin{frame}[fragile]
  \frametitle{Model $M_0$}
  \small
  Parameters
  \vspace{-6pt}
\begin{knitrout}\scriptsize
\definecolor{shadecolor}{rgb}{0.878, 0.918, 0.933}\color{fgcolor}\begin{kframe}
\begin{alltt}
\hlstd{N} \hlkwb{<-} \hlnum{100}
\hlstd{p} \hlkwb{<-} \hlnum{0.2}
\end{alltt}
\end{kframe}
\end{knitrout}
  \pause
  \vfill
  All capture histories (for captured and uncaptured individuals)
  \vspace{-6pt}
\begin{knitrout}\scriptsize
\definecolor{shadecolor}{rgb}{0.878, 0.918, 0.933}\color{fgcolor}\begin{kframe}
\begin{alltt}
\hlstd{J} \hlkwb{<-} \hlnum{4}  \hlcom{## Occasions}
\hlstd{y.all} \hlkwb{<-} \hlkwd{matrix}\hlstd{(}\hlnum{NA}\hlstd{, N, J)}
\hlkwa{for}\hlstd{(i} \hlkwa{in} \hlnum{1}\hlopt{:}\hlstd{N) \{}
    \hlstd{y.all[i,]} \hlkwb{<-} \hlkwd{rbinom}\hlstd{(J,} \hlnum{1}\hlstd{, p)}
\hlstd{\}}
\end{alltt}
\end{kframe}
\end{knitrout}
  \pause
  \vfill
  Observed capture histories (data)
  \vspace{-6pt}
\begin{knitrout}\scriptsize
\definecolor{shadecolor}{rgb}{0.878, 0.918, 0.933}\color{fgcolor}\begin{kframe}
\begin{alltt}
\hlstd{captured} \hlkwb{<-} \hlkwd{rowSums}\hlstd{(y.all)}\hlopt{>}\hlnum{0}
\hlstd{(n} \hlkwb{<-} \hlkwd{sum}\hlstd{(captured))}
\end{alltt}
\begin{verbatim}
## [1] 54
\end{verbatim}
\begin{alltt}
\hlstd{y} \hlkwb{<-} \hlstd{y.all[captured,]}
\hlstd{y[}\hlnum{1}\hlopt{:}\hlnum{3}\hlstd{,]}
\end{alltt}
\begin{verbatim}
##      [,1] [,2] [,3] [,4]
## [1,]    0    1    0    0
## [2,]    1    0    0    0
## [3,]    0    0    1    0
\end{verbatim}
\end{kframe}
\end{knitrout}
\end{frame}



\begin{frame}[fragile]
  \frametitle{Capture history frequencies}
  \centering
\begin{knitrout}\scriptsize
\definecolor{shadecolor}{rgb}{0.878, 0.918, 0.933}\color{fgcolor}\begin{kframe}
\begin{alltt}
\hlstd{histories} \hlkwb{<-} \hlkwd{apply}\hlstd{(y,} \hlnum{1}\hlstd{, paste,} \hlkwc{collapse}\hlstd{=}\hlstr{""}\hlstd{)}
\hlkwd{sort}\hlstd{(}\hlkwd{table}\hlstd{(histories))}
\end{alltt}
\begin{verbatim}
## histories
## 0011 0101 1100 1010 1101 0110 1001 0001 0010 0100 1000 
##    1    1    1    2    2    4    4    7   10   10   12
\end{verbatim}
\end{kframe}
\end{knitrout}
\begin{knitrout}\scriptsize
\definecolor{shadecolor}{rgb}{0.878, 0.918, 0.933}\color{fgcolor}\begin{kframe}
\begin{alltt}
\hlstd{frequencies} \hlkwb{<-} \hlkwd{table}\hlstd{(}\hlkwd{rowSums}\hlstd{(y))}
\hlstd{frequencies}
\end{alltt}
\begin{verbatim}
## 
##  1  2  3 
## 39 13  2
\end{verbatim}
\end{kframe}
\end{knitrout}
\end{frame}




\begin{frame}[fragile]
  \frametitle{Line transects, covariates}
  \small
  Abundance
  \vspace{-6pt}
\begin{knitrout}\scriptsize
\definecolor{shadecolor}{rgb}{0.878, 0.918, 0.933}\color{fgcolor}\begin{kframe}
\begin{alltt}
\hlstd{elevation} \hlkwb{<-} \hlkwd{rnorm}\hlstd{(nSites)}
\hlstd{beta0} \hlkwb{<-} \hlnum{2}\hlstd{; beta1} \hlkwb{<-} \hlnum{1}
\hlstd{lambda2} \hlkwb{<-} \hlkwd{exp}\hlstd{(beta0} \hlopt{+} \hlstd{beta1}\hlopt{*}\hlstd{elevation)}
\hlstd{N2} \hlkwb{<-} \hlkwd{rpois}\hlstd{(}\hlkwc{n}\hlstd{=nSites,} \hlkwc{lambda}\hlstd{=lambda2)}
\end{alltt}
\end{kframe}
\end{knitrout}
  \pause
  \vfill
  Multinomial cell probabilities
  \vspace{-6pt}
\begin{knitrout}\scriptsize
\definecolor{shadecolor}{rgb}{0.878, 0.918, 0.933}\color{fgcolor}\begin{kframe}
\begin{alltt}
\hlstd{noise} \hlkwb{<-} \hlkwd{rnorm}\hlstd{(nSites)}
\hlstd{alpha0} \hlkwb{<-} \hlnum{3}\hlstd{; alpha1} \hlkwb{<-} \hlopt{-}\hlnum{0.5}
\hlstd{sigma2} \hlkwb{<-} \hlkwd{exp}\hlstd{(alpha0} \hlopt{+} \hlstd{alpha1}\hlopt{*}\hlstd{noise)}
\hlstd{pi2} \hlkwb{<-} \hlkwd{matrix}\hlstd{(}\hlnum{NA}\hlstd{, nSites, J}\hlopt{+}\hlnum{1}\hlstd{)} \hlcom{# multinomial cell probs}
\hlkwa{for}\hlstd{(i} \hlkwa{in} \hlnum{1}\hlopt{:}\hlstd{nSites) \{}
  \hlkwa{for}\hlstd{(j} \hlkwa{in} \hlnum{1}\hlopt{:}\hlstd{J) \{}
      \hlstd{pi2[i,j]} \hlkwb{<-} \hlkwd{integrate}\hlstd{(}\hlkwa{function}\hlstd{(}\hlkwc{x}\hlstd{)} \hlkwd{exp}\hlstd{(}\hlopt{-}\hlstd{x}\hlopt{^}\hlnum{2}\hlopt{/}\hlstd{(}\hlnum{2}\hlopt{*}\hlstd{sigma2[i]}\hlopt{^}\hlnum{2}\hlstd{)),}
          \hlkwc{lower}\hlstd{=b[j],} \hlkwc{upper}\hlstd{=b[j}\hlopt{+}\hlnum{1}\hlstd{])}\hlopt{$}\hlstd{value} \hlopt{/} \hlstd{(b[j}\hlopt{+}\hlnum{1}\hlstd{]}\hlopt{-}\hlstd{b[j])} \hlopt{*} \hlstd{psi[j] \}}
  \hlstd{pi2[i,J}\hlopt{+}\hlnum{1}\hlstd{]} \hlkwb{<-} \hlnum{1}\hlopt{-}\hlkwd{sum}\hlstd{(pi2[i,}\hlnum{1}\hlopt{:}\hlstd{J]) \}}
\end{alltt}
\end{kframe}
\end{knitrout}
  \pause
  \vfill
  Detections in each distance interval
  \vspace{-6pt}
\begin{knitrout}\scriptsize
\definecolor{shadecolor}{rgb}{0.878, 0.918, 0.933}\color{fgcolor}\begin{kframe}
\begin{alltt}
\hlstd{y2.all} \hlkwb{<-} \hlkwd{matrix}\hlstd{(}\hlnum{NA}\hlstd{,} \hlkwc{nrow}\hlstd{=nSites,} \hlkwc{ncol}\hlstd{=J}\hlopt{+}\hlnum{1}\hlstd{)}
\hlkwa{for}\hlstd{(i} \hlkwa{in} \hlnum{1}\hlopt{:}\hlstd{nSites) \{}
    \hlstd{y2.all[i,]} \hlkwb{<-} \hlkwd{rmultinom}\hlstd{(}\hlkwc{n}\hlstd{=}\hlnum{1}\hlstd{,} \hlkwc{size}\hlstd{=N2[i],} \hlkwc{prob}\hlstd{=pi2[i,])    \}}
\hlstd{y2} \hlkwb{<-} \hlstd{y2.all[,}\hlnum{1}\hlopt{:}\hlstd{J]}
\end{alltt}
\end{kframe}
\end{knitrout}
\end{frame}






\begin{frame}[fragile]
  \frametitle{Simulated data}
  \begin{columns}
    \begin{column}{0.4\textwidth}
      \small
      Observations
%      \tiny
  \vspace{-6pt}
\begin{knitrout}\tiny
\definecolor{shadecolor}{rgb}{0.878, 0.918, 0.933}\color{fgcolor}\begin{kframe}
\begin{alltt}
\hlstd{y2[}\hlnum{1}\hlopt{:}\hlnum{25}\hlstd{,]}
\end{alltt}
\begin{verbatim}
##       [,1] [,2] [,3] [,4]
##  [1,]    1    0    1    0
##  [2,]    0    0    0    0
##  [3,]    2    0    0    0
##  [4,]    6    2    0    0
##  [5,]    0    1    0    0
##  [6,]   18    4    1    0
##  [7,]    5    3    1    1
##  [8,]    1    0    0    0
##  [9,]    0    0    0    0
## [10,]    3    1    0    0
## [11,]    4    1    1    0
## [12,]    0    0    0    0
## [13,]    3    1    0    0
## [14,]    1    2    4    1
## [15,]    1    1    0    0
## [16,]    2    0    0    0
## [17,]    0    0    0    0
## [18,]    0    0    0    0
## [19,]    2    0    0    0
## [20,]    0    0    0    0
## [21,]    1    1    0    0
## [22,]    0    0    0    0
## [23,]    4    0    0    0
## [24,]    1    0    0    0
## [25,]    0    1    0    0
\end{verbatim}
\end{kframe}
\end{knitrout}
  \end{column}
  \begin{column}{0.6\textwidth}
    \pause
%    \scriptsize
    {\centering Summary stats \\}
    \vspace{24pt}
    \small
    Proportion of sites known to be occupied
    \vspace{-6pt}
\begin{knitrout}\scriptsize
\definecolor{shadecolor}{rgb}{0.878, 0.918, 0.933}\color{fgcolor}\begin{kframe}
\begin{alltt}
\hlcom{# Max count at each site}
\hlstd{maxCounts} \hlkwb{<-} \hlkwd{apply}\hlstd{(y2,} \hlnum{1}\hlstd{, max)}
\hlstd{naiveOccupancy} \hlkwb{<-} \hlkwd{sum}\hlstd{(maxCounts}\hlopt{>}\hlnum{0}\hlstd{)}\hlopt{/}\hlstd{nSites}
\hlstd{naiveOccupancy}
\end{alltt}
\begin{verbatim}
## [1] 0.83
\end{verbatim}
\end{kframe}
\end{knitrout}
  \pause
  \vfill
  \small
  Total detections in each distance interval
  \vspace{-6pt}
\begin{knitrout}\scriptsize
\definecolor{shadecolor}{rgb}{0.878, 0.918, 0.933}\color{fgcolor}\begin{kframe}
\begin{alltt}
\hlkwd{colSums}\hlstd{(y2)}
\end{alltt}
\begin{verbatim}
## [1] 175  83  25  16
\end{verbatim}
\end{kframe}
\end{knitrout}
  \pause
  \vfill
  Naive abundance
  \vspace{-6pt}
\begin{knitrout}\scriptsize
\definecolor{shadecolor}{rgb}{0.878, 0.918, 0.933}\color{fgcolor}\begin{kframe}
\begin{alltt}
\hlkwd{sum}\hlstd{(y2)}
\end{alltt}
\begin{verbatim}
## [1] 299
\end{verbatim}
\end{kframe}
\end{knitrout}

  \end{column}
  \end{columns}
\end{frame}









%\section{Prediction}
\subsection{Likelihood-based inference}


\begin{frame}
  \frametitle{Outline}
  \Large
  \tableofcontents[currentsection]
\end{frame}






% \begin{frame}[fragile]
%   \frametitle{Prepare data in `unmarked'}
%   \small
%   Note the new arguments.
%   \vspace{-6pt}
% <<un-umf,size='tiny'>>=
% umf <- unmarkedFrameDS(y=y2, siteCovs=data.frame(elevation,noise), dist.breaks=b,
%                        tlength=rep(L, nSites), survey="line", unitsIn="m")
% @
% \pause
% <<wfac,size='tiny'>>=
% summary(umf)
% @ 
% \end{frame}



% \begin{frame}[fragile]
%   \frametitle{Fit the model}
%   \footnotesize
% <<un-fit,size='tiny'>>=
% ## fm <- distsamp(~noise ~elevation, umf, keyfun="exp")     # negative exp
% ## fm <- distsamp(~noise ~elevation, umf, keyfun="hazard")  # hazard rate
% fm <- distsamp(~noise ~elevation, umf, keyfun="halfnorm")   # half-normal
% fm
% @
% \pause
% \vfill
% Compare to actual parameter values:
% \vspace{-6pt}
% <<un-compare,size='tiny'>>=
% c(beta0=beta0, beta1=beta1); c(alpha0=alpha0, alpha1=alpha1)
% @ 
% \end{frame}







\subsection{Bayesian inference}


\begin{frame}
  \frametitle{Outline}
  \Large
  \tableofcontents[currentsection,currentsubsection]
\end{frame}





% \begin{frame}[fragile]
%   \frametitle{\normalsize Conditional-on-$N$ and $n_i=\sum_{j=1}^{J} y_{i,j}$}
% \vspace{-3pt}
% <<bugs-line,size='tiny',echo=FALSE>>=
% writeLines(readLines("distsamp-line-mod.jag"))
% @
% \end{frame}





% \begin{frame}[fragile]
%   \frametitle{Data, inits, and parameters}
%   Put data in a named list
%   \vspace{-12pt}
% <<bugs-data2,size='footnotesize'>>=
% jags.data.line <- list(y=y2, n=rowSums(y2),
%                        b=b,           # Distance break points
%                        psi=diff(b)/B, # Pr(occuring in bin j)
%                        elevation=elevation, noise=noise,
%                        nSites=nSites, nBins=J)
% @
% \pause
% \vfill
%   Initial values
%   \vspace{-12pt}
% <<bugs-inits,size='footnotesize'>>=
% jags.inits.line <- function() {
%     list(lambda.intercept=runif(1), alpha0=rnorm(1, 5),
%          N=rowSums(y2)+rpois(nrow(y2), 2))
% }
% @ 
% \pause
% \vfill
%   Parameters to monitor
%   \vspace{-12pt}
% <<bugs-pars,size='small'>>=
% jags.pars.line <- c("beta0", "beta1",
%                     "alpha0", "alpha1", "totalAbundance")
% @ 
% \end{frame}





% \begin{frame}[fragile]
%   \frametitle{MCMC}
%   \small
% <<bugs-mcmc-line,size='scriptsize',message=FALSE,cache=TRUE,results='hide'>>=
% library(jagsUI)
% jags.post.line <- jags.basic(data=jags.data.line, inits=jags.inits.line,
%                              parameters.to.save=jags.pars.line,
%                              model.file="distsamp-line-mod.jag",
%                              n.chains=3, n.adapt=100, n.burnin=0,
%                              n.iter=2000, parallel=TRUE)
% @ 
% \vfill
% <<jags-sum-line,size='scriptsize',cache=TRUE>>=
% round(summary(jags.post.line)$quantile, digits=3)
% @ 
% \end{frame}


% \begin{frame}[fragile]
%   \frametitle{Traceplots and density plots}
% <<bugs-plot1-rem2,size='footnotesize',out.width="0.7\\textwidth",fig.align='center',cache=TRUE>>=
% plot(jags.post.line[,jags.pars.line[1:3]])
% @ 
% \end{frame}



% \begin{frame}[fragile]
%   \frametitle{Traceplots and density plots}
% <<bugs-plot2-rem2,size='footnotesize',out.width="0.7\\textwidth",fig.align='center',cache=TRUE>>=
% plot(jags.post.line[,jags.pars.line[4:5]])
% @ 
% \end{frame}










\section{Summary}


\begin{frame}
  \frametitle{Distance sampling summary}
  Assumptions
  \begin{itemize}
    \small
    \item Animals don't move during the survey
    \item Animals are uniformly distributed with respect to the
      transects
    \item Detection is certain on the transect, i.e. $p=1$ when $x=0$. 
    \item Detections are independent
  \end{itemize}
  \pause
  \vfill
  \small
  If these assumptions can be met, distance sampling is a powerful
  method allowing for inference about abundance and density using data
  from a single visit. \\
\end{frame}




\section{Assignment}




\begin{frame}[fragile]
  \frametitle{Assignment}
  % \small
  \footnotesize
  Create a self-contained R script or Rmarkdown file to do the following:
  \vfill
  \begin{enumerate}
%    \small
    \footnotesize
    \item Fit a distance sampling model with a half-normal detection
      function and the following covariates to the black-throated blue
      warbler data ({\tt btbw\_data\_distsamp.csv}) in `unmarked' and
      `JAGS':   
      \begin{itemize}
        \footnotesize
        \item Density covariates: {\tt Elevation, UTM.N, UTM.W}
        \item Detection covariates: {\tt Wind, Noise}
        \item Response: {\scriptsize \tt btbw0\_20, btbw20\_40, btbw40\_60, btbw60\_80, btbw80\_100}
      \end{itemize}
    \item Using the model fitted in `unmarked', create two graphs of
      the predictions, one for density and the other for the scale
      parameter ($\sigma$).
    \item Compare the half-normal model to two other models with the
      same covariates, but with negative exponential and hazard
      rate detection functions. Which has the lowest AIC? 
  \end{enumerate}
  \pause
  \vfill
  Suggestions:
  \begin{itemize}
    \item Convert response variables to matrix with \inr{as.matrix}
    \item Standardize covariates
  \end{itemize}
  \pause
  \vfill
  Upload your {\tt .R} or {\tt .Rmd} file to ELC before Tuesday. 
\end{frame}





\end{document}

