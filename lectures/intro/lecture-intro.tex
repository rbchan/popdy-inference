\documentclass[color=usenames,dvipsnames]{beamer}\usepackage[]{graphicx}\usepackage[]{color}
% maxwidth is the original width if it is less than linewidth
% otherwise use linewidth (to make sure the graphics do not exceed the margin)
\makeatletter
\def\maxwidth{ %
  \ifdim\Gin@nat@width>\linewidth
    \linewidth
  \else
    \Gin@nat@width
  \fi
}
\makeatother

\definecolor{fgcolor}{rgb}{0, 0, 0}
\newcommand{\hlnum}[1]{\textcolor[rgb]{0.69,0.494,0}{#1}}%
\newcommand{\hlstr}[1]{\textcolor[rgb]{0.749,0.012,0.012}{#1}}%
\newcommand{\hlcom}[1]{\textcolor[rgb]{0.514,0.506,0.514}{\textit{#1}}}%
\newcommand{\hlopt}[1]{\textcolor[rgb]{0,0,0}{#1}}%
\newcommand{\hlstd}[1]{\textcolor[rgb]{0,0,0}{#1}}%
\newcommand{\hlkwa}[1]{\textcolor[rgb]{0,0,0}{\textbf{#1}}}%
\newcommand{\hlkwb}[1]{\textcolor[rgb]{0,0.341,0.682}{#1}}%
\newcommand{\hlkwc}[1]{\textcolor[rgb]{0,0,0}{\textbf{#1}}}%
\newcommand{\hlkwd}[1]{\textcolor[rgb]{0.004,0.004,0.506}{#1}}%
\let\hlipl\hlkwb

\usepackage{framed}
\makeatletter
\newenvironment{kframe}{%
 \def\at@end@of@kframe{}%
 \ifinner\ifhmode%
  \def\at@end@of@kframe{\end{minipage}}%
  \begin{minipage}{\columnwidth}%
 \fi\fi%
 \def\FrameCommand##1{\hskip\@totalleftmargin \hskip-\fboxsep
 \colorbox{shadecolor}{##1}\hskip-\fboxsep
     % There is no \\@totalrightmargin, so:
     \hskip-\linewidth \hskip-\@totalleftmargin \hskip\columnwidth}%
 \MakeFramed {\advance\hsize-\width
   \@totalleftmargin\z@ \linewidth\hsize
   \@setminipage}}%
 {\par\unskip\endMakeFramed%
 \at@end@of@kframe}
\makeatother

\definecolor{shadecolor}{rgb}{.97, .97, .97}
\definecolor{messagecolor}{rgb}{0, 0, 0}
\definecolor{warningcolor}{rgb}{1, 0, 1}
\definecolor{errorcolor}{rgb}{1, 0, 0}
\newenvironment{knitrout}{}{} % an empty environment to be redefined in TeX

\usepackage{alltt}
%\documentclass[color=usenames,dvipsnames,handout]{beamer}

%\usepackage[roman]{../lectures}
\usepackage[sans]{../lectures}




\hypersetup{pdfpagemode=UseNone,pdfstartview=FitH}



\title{Lecture 1 -- Introduction to }
\author{Richard Chandler}
\date{January 14, 2019}

%\newcommand{\R}{{\bf R}}


%% Switching from Sweave to knitr
%\DefineVerbatimEnvironment{Sinput}{Verbatim}{fontshape=sl,formatcom=\color{red}}
%\DefineVerbatimEnvironment{Soutput}{Verbatim}{formatcom=\color{MidnightBlue}}
%\DefineVerbatimEnvironment{Scode}{Verbatim}{fontshape=sl}



% <<knitr-setup, include=FALSE>>=
% opts_hooks$set(comment=function(x) return("comment"=NA))
% ##knit_hooks$set(inline = function(x) {
% ##    if(!require(highr)) {
% ##        install.packages("highr")
% ##    }
% ##    if (is.numeric(x)) return(knitr:::format_sci(x, 'latex'))
% ##    highr:::hi_latex(x)
% ##})
% @













%% New command for inline code that isn't to be evaluated
\definecolor{inlinecolor}{rgb}{0.878, 0.918, 0.933}
\newcommand{\inr}[1]{\colorbox{inlinecolor}{\texttt{#1}}}
\IfFileExists{upquote.sty}{\usepackage{upquote}}{}
\begin{document}

% This would affect all code boxes. Not a good idea.
% \setlength\fboxsep{0pt}



\begin{frame}[plain]
  \LARGE
%  \maketitle
  {\centering
%  \textcolor{RoyalBlue}{\huge \bf Lab 1 -- Introduction to \R} \\
  {\huge \bf Lab 1 -- Introduction to \R} \\
  \vspace{0.9cm}
%  \includegraphics[width=0.4\textwidth]{figs/Rlogo} \\
  \vspace{0.5cm}
%  August 13 \& 14, 2018 \\
  WILD 8390 \par
  \vfill
  \large
  Richard Chandler \\
  University of Georgia \\
  }
\end{frame}




\section{Why?}


\begin{frame}[plain]
  \frametitle{Today's Topics}
  \Large
  \only<1>{\tableofcontents}%[hideallsubsections]}
  \only<2 | handout:0>{\tableofcontents[currentsection]}%,hideallsubsections]}
\end{frame}



\begin{frame}
  \frametitle{Good and Not So Good Things About \R}
%  {\Large \textcolor{bb}{Good}}
  {\Large Good}
  \large
  \begin{itemize}%[<+->]
    \item<1-> Powerful platform for statistical analysis %Flexible %If you're going to use a stats program, might as well use one that does everything!
%    \item<1-> Many \inr{packages} \colorbox{BurntOrange}{written} for ecologists
    \item<1-> Many packages written for ecologists
    \item<1-> It's free
    \item<1-> Scripts save time
    \item<1-> \R~teaches you statistics
  \end{itemize}
  \vspace{0.5cm}
  \uncover<2->{
%  \pause
  {\Large Not so good??}}
  \begin{itemize}
    \item<2-> Steep learning curve
    \item<2-> Help pages written for people familiar with \R
    \item<2-> Developed by statisticians for statisticians
    \item<2-> Not as fast as some languages
  \end{itemize}
%  }
\end{frame}






\begin{frame}[fragile]
  \frametitle{Vectorized arithmetic}
  How could we calculate the body mass index (BMI = $\text{weight}/\text{height}^2$) from the following data:
  \begin{center}
    \begin{tabular}{ccccccc}
      \hline
      & \multicolumn{6}{c}{Individual} \\
      \cline{2-7}
      & 1 & 2 & 3 & 4 & 5 & 6 \\
      \hline
      Weight & 60 & 72 & 57 & 90 & 95 & 72 \\
      Height & 1.8 & 1.8 & 1.7 & 1.9 & 1.7 & 1.9 \\
      \hline
    \end{tabular}
  \end{center}
  \pause
  First, create the vectors:
\begin{knitrout}
\definecolor{shadecolor}{rgb}{0.878, 0.918, 0.933}\color{fgcolor}\begin{kframe}
\begin{alltt}
\hlstd{weight} \hlkwb{<-} \hlkwd{c}\hlstd{(}\hlnum{60}\hlstd{,} \hlnum{72}\hlstd{,} \hlnum{57}\hlstd{,} \hlnum{90}\hlstd{,} \hlnum{95}\hlstd{,} \hlnum{72}\hlstd{)}
\hlstd{height} \hlkwb{<-} \hlkwd{c}\hlstd{(}\hlnum{1.8}\hlstd{,} \hlnum{1.8}\hlstd{,} \hlnum{1.7}\hlstd{,} \hlnum{1.9}\hlstd{,} \hlnum{1.7}\hlstd{,} \hlnum{1.9}\hlstd{)}
\end{alltt}
\end{kframe}
\end{knitrout}
  \pause
  Then, evaluate the equation in just one line:
\begin{knitrout}\footnotesize
\definecolor{shadecolor}{rgb}{0.878, 0.918, 0.933}\color{fgcolor}\begin{kframe}
\begin{alltt}
\hlstd{BMI} \hlkwb{<-} \hlstd{weight}\hlopt{/}\hlstd{height}\hlopt{^}\hlnum{2}
\hlstd{BMI}
\end{alltt}
\begin{verbatim}
## [1] 18.51852 22.22222 19.72318 24.93075 32.87197 19.94460
\end{verbatim}
\end{kframe}
\end{knitrout}
\end{frame}










\end{document}
