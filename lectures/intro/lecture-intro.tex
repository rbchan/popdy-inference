\documentclass[color=usenames,dvipsnames]{beamer}\usepackage[]{graphicx}\usepackage[]{color}
% maxwidth is the original width if it is less than linewidth
% otherwise use linewidth (to make sure the graphics do not exceed the margin)
\makeatletter
\def\maxwidth{ %
  \ifdim\Gin@nat@width>\linewidth
    \linewidth
  \else
    \Gin@nat@width
  \fi
}
\makeatother

\definecolor{fgcolor}{rgb}{0, 0, 0}
\newcommand{\hlnum}[1]{\textcolor[rgb]{0.69,0.494,0}{#1}}%
\newcommand{\hlstr}[1]{\textcolor[rgb]{0.749,0.012,0.012}{#1}}%
\newcommand{\hlcom}[1]{\textcolor[rgb]{0.514,0.506,0.514}{\textit{#1}}}%
\newcommand{\hlopt}[1]{\textcolor[rgb]{0,0,0}{#1}}%
\newcommand{\hlstd}[1]{\textcolor[rgb]{0,0,0}{#1}}%
\newcommand{\hlkwa}[1]{\textcolor[rgb]{0,0,0}{\textbf{#1}}}%
\newcommand{\hlkwb}[1]{\textcolor[rgb]{0,0.341,0.682}{#1}}%
\newcommand{\hlkwc}[1]{\textcolor[rgb]{0,0,0}{\textbf{#1}}}%
\newcommand{\hlkwd}[1]{\textcolor[rgb]{0.004,0.004,0.506}{#1}}%
\let\hlipl\hlkwb

\usepackage{framed}
\makeatletter
\newenvironment{kframe}{%
 \def\at@end@of@kframe{}%
 \ifinner\ifhmode%
  \def\at@end@of@kframe{\end{minipage}}%
  \begin{minipage}{\columnwidth}%
 \fi\fi%
 \def\FrameCommand##1{\hskip\@totalleftmargin \hskip-\fboxsep
 \colorbox{shadecolor}{##1}\hskip-\fboxsep
     % There is no \\@totalrightmargin, so:
     \hskip-\linewidth \hskip-\@totalleftmargin \hskip\columnwidth}%
 \MakeFramed {\advance\hsize-\width
   \@totalleftmargin\z@ \linewidth\hsize
   \@setminipage}}%
 {\par\unskip\endMakeFramed%
 \at@end@of@kframe}
\makeatother

\definecolor{shadecolor}{rgb}{.97, .97, .97}
\definecolor{messagecolor}{rgb}{0, 0, 0}
\definecolor{warningcolor}{rgb}{1, 0, 1}
\definecolor{errorcolor}{rgb}{1, 0, 0}
\newenvironment{knitrout}{}{} % an empty environment to be redefined in TeX

\usepackage{alltt}
%\documentclass[color=usenames,dvipsnames,handout]{beamer}

%\usepackage[roman]{../lectures}
\usepackage[sans]{../lectures}


\hypersetup{pdfpagemode=UseNone,pdfstartview={FitV}}



\title{Lecture 1 -- Introduction to }
\author{Richard Chandler}
%\date{January 14, 2019}

%\newcommand{\R}{{\bf R}}



% Load function to compile and open PDF


% Compile and open PDF







%<<knitr-setup,include=FALSE,purl=FALSE>>=
%##opts_chunk$set(comment=NA)
%@


%% New command for inline code that isn't to be evaluated
\definecolor{inlinecolor}{rgb}{0.878, 0.918, 0.933}
\newcommand{\inr}[1]{\colorbox{inlinecolor}{\texttt{#1}}}
\IfFileExists{upquote.sty}{\usepackage{upquote}}{}
\begin{document}

% This would affect all code boxes. Not a good idea.
% \setlength\fboxsep{0pt}



\begin{frame}[plain]
  \LARGE
%  \maketitle
  {\centering
%  \textcolor{RoyalBlue}{\huge \bf Lab 1 -- Introduction to \R} \\
  {\huge \bf Lecture 1 -- Overview} \\
  \vspace{0.9cm}
  \vspace{0.5cm}
  WILD(FISH) 8390 \par
  \vfill
  \large
  Richard Chandler \\
  University of Georgia \\
  }
\end{frame}




\section{Overview}


\begin{frame}[plain]
  \frametitle{Today's Topics}
  \Large
  \only<1>{\tableofcontents}%[hideallsubsections]}
  \only<2 | handout:0>{\tableofcontents[currentsection]}%,hideallsubsections]}
\end{frame}


\section{Motivation}

\begin{frame}
  \frametitle{Overview}
  {\Large Themes}
  \begin{itemize}
    \item<2-> We need models
    \item<3-> In the old days, models were used strictly for theory
    \item<4-> Now we can fit models to data. 
    \item<5-> Confronting models with data has ushered in a revolution. Let's us:
      \begin{itemize}
        \item Evaluate hypotheses
        \item Make predictions
        \item Inform management decisions
        \item Forecast future outcomes
      \end{itemize}
   \item<6-> Hierarchical models are the key
  \end{itemize}
\end{frame}






\section{Examples}


\subsection{Central Georgia Black Bears}


\begin{frame}
  \frametitle{Bears}
  \begin{enumerate}
    \item Black
      \begin{enumerate}
        \item LA
        \item GA
      \end{enumerate}
    \item Brown
  \end{enumerate}
  \[
    \prod_{i=1}^N \frac{\lambda({\bm s}_i)}{\int_{\mathcal{S}} \lambda({\bm s}) \mathrm{d}{\bm s}}
  \]
\end{frame}



\subsection{Chiricahua Leopard Frogs}



\section{Syllabus}






    
    





\end{document}
