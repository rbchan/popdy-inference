\documentclass[color=usenames,dvipsnames]{beamer}\usepackage[]{graphicx}\usepackage[]{color}
% maxwidth is the original width if it is less than linewidth
% otherwise use linewidth (to make sure the graphics do not exceed the margin)
\makeatletter
\def\maxwidth{ %
  \ifdim\Gin@nat@width>\linewidth
    \linewidth
  \else
    \Gin@nat@width
  \fi
}
\makeatother

\definecolor{fgcolor}{rgb}{0, 0, 0}
\newcommand{\hlnum}[1]{\textcolor[rgb]{0.69,0.494,0}{#1}}%
\newcommand{\hlstr}[1]{\textcolor[rgb]{0.749,0.012,0.012}{#1}}%
\newcommand{\hlcom}[1]{\textcolor[rgb]{0.514,0.506,0.514}{\textit{#1}}}%
\newcommand{\hlopt}[1]{\textcolor[rgb]{0,0,0}{#1}}%
\newcommand{\hlstd}[1]{\textcolor[rgb]{0,0,0}{#1}}%
\newcommand{\hlkwa}[1]{\textcolor[rgb]{0,0,0}{\textbf{#1}}}%
\newcommand{\hlkwb}[1]{\textcolor[rgb]{0,0.341,0.682}{#1}}%
\newcommand{\hlkwc}[1]{\textcolor[rgb]{0,0,0}{\textbf{#1}}}%
\newcommand{\hlkwd}[1]{\textcolor[rgb]{0.004,0.004,0.506}{#1}}%
\let\hlipl\hlkwb

\usepackage{framed}
\makeatletter
\newenvironment{kframe}{%
 \def\at@end@of@kframe{}%
 \ifinner\ifhmode%
  \def\at@end@of@kframe{\end{minipage}}%
  \begin{minipage}{\columnwidth}%
 \fi\fi%
 \def\FrameCommand##1{\hskip\@totalleftmargin \hskip-\fboxsep
 \colorbox{shadecolor}{##1}\hskip-\fboxsep
     % There is no \\@totalrightmargin, so:
     \hskip-\linewidth \hskip-\@totalleftmargin \hskip\columnwidth}%
 \MakeFramed {\advance\hsize-\width
   \@totalleftmargin\z@ \linewidth\hsize
   \@setminipage}}%
 {\par\unskip\endMakeFramed%
 \at@end@of@kframe}
\makeatother

\definecolor{shadecolor}{rgb}{.97, .97, .97}
\definecolor{messagecolor}{rgb}{0, 0, 0}
\definecolor{warningcolor}{rgb}{1, 0, 1}
\definecolor{errorcolor}{rgb}{1, 0, 0}
\newenvironment{knitrout}{}{} % an empty environment to be redefined in TeX

\usepackage{alltt}
%\documentclass[color=usenames,dvipsnames,handout]{beamer}

\usepackage[roman]{../lectures}
%\usepackage[sans]{../lectures}


\hypersetup{pdfpagemode=UseNone,pdfstartview={FitV}}


\title{Lecture 4 -- Occuancy models: simulation and fitting }
\author{Richard Chandler}


% Load function to compile and open PDF


% Compile and open PDF







%<<knitr-setup,include=FALSE,purl=FALSE>>=
%##opts_chunk$set(comment=NA)
%@


%% New command for inline code that isn't to be evaluated
\definecolor{inlinecolor}{rgb}{0.878, 0.918, 0.933}
\newcommand{\inr}[1]{\colorbox{inlinecolor}{\texttt{#1}}}
\IfFileExists{upquote.sty}{\usepackage{upquote}}{}
\begin{document}






\begin{frame}[plain]
  \LARGE
%  \maketitle
  \centering
  {\huge Lecture 4 -- Occupancy models: simulation and fitting} \\
  {\color{default} \rule{\textwidth}{0.1pt}}
  \vfill
  \large
  WILD(FISH) 8390 \\
  Estimation of Fish and Wildlife Population Parameters \\
  \vfill
  \large
  Richard Chandler \\
  University of Georgia \\
\end{frame}





\section{Motivation}



\begin{frame}[plain]
  \frametitle{Today's Topics}
  \Large
  \only<1>{\tableofcontents}%[hideallsubsections]}
  \only<2 | handout:0>{\tableofcontents[currentsection]}%,hideallsubsections]}
\end{frame}

\begin{frame}
  
\end{frame}


\begin{frame}
  \frametitle{Outline}
  \Large
  \tableofcontents[currentsection]
\end{frame}



\section{Simulation}



\section{Model fitting}


\subsection{Likelihood-based methods}


\subsection{Bayesian methods}



\section{Real data}







\begin{frame}
  \frametitle{Assignment}
  \small
  % \scriptsize
  Create a self-contained R script (or better yet an Rmarkdown file)
  to do the following:
  \begin{enumerate}
    \small
    % \footnotesize
    % \scriptsize
    \item Simulate logistic regression data according to
      $y_i \sim \mathrm{Bern}(p_i)$ and $\mathrm{logit}(p_i) = \beta_0
      + \beta_1 x_i$ with $\beta_0=-1$ and $\beta_1=1$. Generate the
      covariate using \inr{x <- rnorm(100)}.
    \item Fit a logistic regression model to the simulated data ($y$
      and $x$) using the \inr{glm} function. Create a figure showing
      $x$ and $y$ with the fitted regression line. Use
      \inr{predict} to get the fitted line.
    \item Fit 4 Poisson regression models to the Canada Warbler
      data. Try to explain as much variation in abundance as you can
      using the two explanatory variables: {\tt elevation} and
      {\tt year}. You can use quadratic effects and
      interactions. Interpret the estimates from the model with the 
      lowest AIC, and create a graph depicting the estimated
      relationships. 
  \end{enumerate}
  Upload your {\tt .R} or {\tt .Rmd} file to ELC before Monday, Sept 7. 
\end{frame}





\end{document}

