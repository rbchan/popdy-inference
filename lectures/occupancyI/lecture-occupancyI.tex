\documentclass[color=usenames,dvipsnames]{beamer}\usepackage[]{graphicx}\usepackage[]{xcolor}
% maxwidth is the original width if it is less than linewidth
% otherwise use linewidth (to make sure the graphics do not exceed the margin)
\makeatletter
\def\maxwidth{ %
  \ifdim\Gin@nat@width>\linewidth
    \linewidth
  \else
    \Gin@nat@width
  \fi
}
\makeatother

\definecolor{fgcolor}{rgb}{0, 0, 0}
\newcommand{\hlnum}[1]{\textcolor[rgb]{0.69,0.494,0}{#1}}%
\newcommand{\hlsng}[1]{\textcolor[rgb]{0.749,0.012,0.012}{#1}}%
\newcommand{\hlcom}[1]{\textcolor[rgb]{0.514,0.506,0.514}{\textit{#1}}}%
\newcommand{\hlopt}[1]{\textcolor[rgb]{0,0,0}{#1}}%
\newcommand{\hldef}[1]{\textcolor[rgb]{0,0,0}{#1}}%
\newcommand{\hlkwa}[1]{\textcolor[rgb]{0,0,0}{\textbf{#1}}}%
\newcommand{\hlkwb}[1]{\textcolor[rgb]{0,0.341,0.682}{#1}}%
\newcommand{\hlkwc}[1]{\textcolor[rgb]{0,0,0}{\textbf{#1}}}%
\newcommand{\hlkwd}[1]{\textcolor[rgb]{0.004,0.004,0.506}{#1}}%
\let\hlipl\hlkwb

\usepackage{framed}
\makeatletter
\newenvironment{kframe}{%
 \def\at@end@of@kframe{}%
 \ifinner\ifhmode%
  \def\at@end@of@kframe{\end{minipage}}%
  \begin{minipage}{\columnwidth}%
 \fi\fi%
 \def\FrameCommand##1{\hskip\@totalleftmargin \hskip-\fboxsep
 \colorbox{shadecolor}{##1}\hskip-\fboxsep
     % There is no \\@totalrightmargin, so:
     \hskip-\linewidth \hskip-\@totalleftmargin \hskip\columnwidth}%
 \MakeFramed {\advance\hsize-\width
   \@totalleftmargin\z@ \linewidth\hsize
   \@setminipage}}%
 {\par\unskip\endMakeFramed%
 \at@end@of@kframe}
\makeatother

\definecolor{shadecolor}{rgb}{.97, .97, .97}
\definecolor{messagecolor}{rgb}{0, 0, 0}
\definecolor{warningcolor}{rgb}{1, 0, 1}
\definecolor{errorcolor}{rgb}{1, 0, 0}
\newenvironment{knitrout}{}{} % an empty environment to be redefined in TeX

\usepackage{alltt}
%\documentclass[color=usenames,dvipsnames,handout]{beamer}

\usepackage[roman]{../lectures}
%\usepackage[sans]{../lectures}

\mode<handout>{
  \usetheme{default}
%  \setbeamercolor{background canvas}{bg=black!5}
 \pgfpagesuselayout{4 on 1}[letterpaper,landscape,border shrink=2.5mm]
%  \pgfpagesuselayout{2 on 1}[letterpaper,border shrink=10mm]
}


\hypersetup{pdfpagemode=UseNone,pdfstartview={FitV}}


\title{Occuancy models: simulation and fitting }
\author{Richard Chandler}


% Load function to compile and open PDF


% Compile and open PDF








%% New command for inline code that isn't to be evaluated
\definecolor{inlinecolor}{rgb}{0.878, 0.918, 0.933}
\newcommand{\inr}[1]{\colorbox{inlinecolor}{\texttt{#1}}}
\IfFileExists{upquote.sty}{\usepackage{upquote}}{}
\begin{document}






\begin{frame}[plain]
  \LARGE
%  \maketitle
  \centering
  {\huge Lecture 3 -- Occupancy models: simulation and fitting} \\
  {\color{default} \rule{\textwidth}{0.1pt}}
  \vfill
  \large
  WILD(FISH) 8390 \\
%  Estimation of Fish and Wildlife Population Parameters \\
  Inference for Models of Fish and Wildlife Population Dynamics \\
  \vfill
  \large
  Richard Chandler \\
  University of Georgia \\
\end{frame}






\section{Motivation}



\begin{frame}[plain]
  \frametitle{Outline}
  \Large
  \only<1>{\tableofcontents}%[hideallsubsections]}
  \only<2 | handout:0>{\tableofcontents[currentsection]}%,hideallsubsections]}
\end{frame}



\begin{frame}
  \frametitle{Motivation}
  We often want to do logistic regression, but the data are subject
  to measurement error. \\  
  \pause
  \vfill
  Specifically, wildlife data are often fraught with false negatives
  (ie, the species was there but we didn't detect it). \\
  \pause
  \vfill
  Ignoring this source of measurement error can result in misleading
  inferences. \\
  \begin{itemize}
    \item Occupancy will be under-estimated
    \item Habitat associations may be incorrect
  \end{itemize}
\end{frame}


% \begin{frame}
%   \frametitle{Ignoring detection probability}
%   Show some examples
% \end{frame}



\begin{frame}
  \frametitle{Accounting for detection probability}
  \small
  The model for the state process is the same as logistic regression:
  \begin{gather*}
    \mathrm{logit}(\psi_i) = \beta_0 + \beta_1 x_{i1} + \beta_2 x_{i2} + \cdots \\
    z_i \sim \mathrm{Bern}(\psi_i)
  \end{gather*}
  \pause
  \vfill
  Model for the observation process conditional on the state process:
  \begin{gather*}
    \mathrm{logit}(p_{ij}) = \alpha_0 + \alpha_1 x_{ij1} + \alpha_2 x_{ij2} + \cdots \\
    y_{ij} \sim \mathrm{Bern}(z_i\times p_{ij})
  \end{gather*}
  \pause
  Definitions \\
  $\psi_i$ -- probability that the species occurs at site $i$ \\
  $z_i$ -- binary presence/absence variable at site $i$ \\
  $p_{ij}$ -- probability of detecting the species at site $i$ on occasion $j$ \\
  $y_{ij}$ -- binary detection/non-detection data
\end{frame}


\section{Simulation}



\begin{frame}
  \frametitle{Outline}
  \Large
  \tableofcontents[currentsection]
\end{frame}


\begin{frame}[fragile]
  \frametitle{Simulation}
  Simplest case with no covariates: \\
  \vfill
  \begin{enumerate}[<+->]
  \item Specify the parameters and dimensions
\begin{knitrout}\footnotesize
\definecolor{shadecolor}{rgb}{0.878, 0.918, 0.933}\color{fgcolor}\begin{kframe}
\begin{alltt}
\hldef{psi} \hlkwb{<-} \hlnum{0.5}       \hlcom{## Occurrence probability}
\hldef{p} \hlkwb{<-} \hlnum{0.2}         \hlcom{## Detection probability}
\hldef{nSites} \hlkwb{<-} \hlnum{20}
\hldef{nVisits} \hlkwb{<-} \hlnum{4}
\end{alltt}
\end{kframe}
\end{knitrout}
  \item Now, simulate presence-absence
\begin{knitrout}\footnotesize
\definecolor{shadecolor}{rgb}{0.878, 0.918, 0.933}\color{fgcolor}\begin{kframe}
\begin{alltt}
\hlkwd{set.seed}\hldef{(}\hlnum{3439}\hldef{)}    \hlcom{## Just to make it reproducible}
\hldef{z} \hlkwb{<-} \hlkwd{rbinom}\hldef{(nSites,} \hlkwc{size}\hldef{=}\hlnum{1}\hldef{, psi)} \hlcom{## pres/absence}
\end{alltt}
\end{kframe}
\end{knitrout}
  \item Simulate observations
\begin{knitrout}\footnotesize
\definecolor{shadecolor}{rgb}{0.878, 0.918, 0.933}\color{fgcolor}\begin{kframe}
\begin{alltt}
\hldef{y} \hlkwb{<-} \hlkwd{matrix}\hldef{(}\hlnum{NA}\hldef{,} \hlkwc{nrow}\hldef{=nSites,} \hlkwc{ncol}\hldef{=nVisits)}
\hlkwa{for}\hldef{(i} \hlkwa{in} \hlnum{1}\hlopt{:}\hldef{nSites) \{}
    \hldef{y[i,]} \hlkwb{<-} \hlkwd{rbinom}\hldef{(nVisits,} \hlkwc{size}\hldef{=}\hlnum{1}\hldef{,} \hlkwc{prob}\hldef{=z[i]}\hlopt{*}\hldef{p)}
\hldef{\}}
\end{alltt}
\end{kframe}
\end{knitrout}
\end{enumerate}
\end{frame}




\begin{frame}[fragile]
  \frametitle{Simulated data}
  \begin{columns}
    \begin{column}{0.4\textwidth}
      \small
      Observations
%      \tiny
\begin{knitrout}\scriptsize
\definecolor{shadecolor}{rgb}{0.878, 0.918, 0.933}\color{fgcolor}\begin{kframe}
\begin{alltt}
\hldef{y}
\end{alltt}
\begin{verbatim}
##       [,1] [,2] [,3] [,4]
##  [1,]    0    0    0    0
##  [2,]    0    0    0    0
##  [3,]    1    0    0    0
##  [4,]    0    0    0    0
##  [5,]    0    0    0    0
##  [6,]    0    0    0    0
##  [7,]    0    1    1    0
##  [8,]    0    0    0    0
##  [9,]    0    0    0    0
## [10,]    0    1    0    1
## [11,]    0    0    0    0
## [12,]    0    0    0    0
## [13,]    0    0    0    0
## [14,]    0    0    0    0
## [15,]    0    0    0    0
## [16,]    0    1    1    0
## [17,]    0    0    0    0
## [18,]    1    0    1    0
## [19,]    0    0    0    0
## [20,]    0    1    0    0
\end{verbatim}
\end{kframe}
\end{knitrout}
  \end{column}
  \begin{column}{0.6\textwidth}
    \pause
%    \scriptsize
    {\centering Summary stats \\}
    \vspace{24pt}
  Detections at each site \\
\begin{knitrout}\scriptsize
\definecolor{shadecolor}{rgb}{0.878, 0.918, 0.933}\color{fgcolor}\begin{kframe}
\begin{alltt}
\hldef{siteDets} \hlkwb{<-} \hlkwd{rowSums}\hldef{(y)} \hlcom{# Dets at each site}
\hlkwd{table}\hldef{(siteDets)}        \hlcom{# Frequency}
\end{alltt}
\begin{verbatim}
## siteDets
##  0  1  2 
## 14  2  4
\end{verbatim}
\end{kframe}
\end{knitrout}
\pause
\vfill
\small
Proportion of sites known to be occupied \\
\begin{knitrout}\scriptsize
\definecolor{shadecolor}{rgb}{0.878, 0.918, 0.933}\color{fgcolor}\begin{kframe}
\begin{alltt}
\hldef{naiveOccupancy} \hlkwb{<-} \hlkwd{sum}\hldef{(siteDets}\hlopt{>}\hlnum{0}\hldef{)}\hlopt{/}\hldef{nSites}
\hldef{naiveOccupancy}
\end{alltt}
\begin{verbatim}
## [1] 0.3
\end{verbatim}
\end{kframe}
\end{knitrout}
  \end{column}
  \end{columns}
\end{frame}



\section{Model fitting}



\begin{frame}
  \frametitle{Outline}
  \Large
  \tableofcontents[currentsection]
\end{frame}



\subsection{Likelihood-based methods}


% \begin{frame}
%   \frametitle{Likelihood-based methods}
% \end{frame}

\begin{frame}[fragile]
  \frametitle{R package `unmarked'}
  % \small
  \normalsize
  Install and load the package  %\\
  \vspace{-12pt}
\begin{knitrout}
\definecolor{shadecolor}{rgb}{0.878, 0.918, 0.933}\color{fgcolor}\begin{kframe}
\begin{alltt}
\hlcom{## install.packages("unmarked")  }
\hlkwd{library}\hldef{(unmarked)}
\end{alltt}
\end{kframe}
\end{knitrout}
\pause
\vfill
Documentation (help files) %\\
  \vspace{-12pt}
\begin{knitrout}
\definecolor{shadecolor}{rgb}{0.878, 0.918, 0.933}\color{fgcolor}\begin{kframe}
\begin{alltt}
\hlkwd{help}\hldef{(}\hlsng{"unmarked"}\hldef{)}
\end{alltt}
\end{kframe}
\end{knitrout}
\vfill
Documentation (vignettes) %\\
  \vspace{-12pt}
\begin{knitrout}
\definecolor{shadecolor}{rgb}{0.878, 0.918, 0.933}\color{fgcolor}\begin{kframe}
\begin{alltt}
\hlkwd{vignette}\hldef{(}\hlkwc{package}\hldef{=}\hlsng{"unmarked"}\hldef{)}
\hlcom{## vignette("spp-dist")}
\end{alltt}
\end{kframe}
\end{knitrout}

\end{frame}





\begin{frame}[fragile]
  \frametitle{R package `unmarked'}
%  \small
  Unlike standard model fitting functions such as \inr{lm} and \inr{glm} 
  that require data to be in \texttt{data.frames}, data for `unmarked'
  has to be formatted in an \inr{unmarkedFrame}.
\begin{knitrout}\small
\definecolor{shadecolor}{rgb}{0.878, 0.918, 0.933}\color{fgcolor}\begin{kframe}
\begin{alltt}
\hldef{umf} \hlkwb{<-} \hlkwd{unmarkedFrameOccu}\hldef{(}\hlkwc{y}\hldef{=y)}
\hlkwd{summary}\hldef{(umf)}
\end{alltt}
\begin{verbatim}
## unmarkedFrame Object
## 
## 20 sites
## Maximum number of observations per site: 4 
## Mean number of observations per site: 4 
## Sites with at least one detection: 6 
## 
## Tabulation of y observations:
##  0  1 
## 70 10
\end{verbatim}
\end{kframe}
\end{knitrout}
\end{frame}





\begin{frame}[fragile]
  \frametitle{R package `unmarked'}
  \small
Fit the single-season occupancy model
\begin{knitrout}\scriptsize
\definecolor{shadecolor}{rgb}{0.878, 0.918, 0.933}\color{fgcolor}\begin{kframe}
\begin{alltt}
\hldef{fm} \hlkwb{<-} \hlkwd{occu}\hldef{(}\hlopt{~}\hlnum{1} \hlopt{~}\hlnum{1}\hldef{, umf)}
\hlkwd{summary}\hldef{(fm)}
\end{alltt}
\begin{verbatim}
## 
## Call:
## occu(formula = ~1 ~ 1, data = umf)
## 
## Occupancy (logit-scale):
##  Estimate    SE      z P(>|z|)
##     -0.52 0.613 -0.849   0.396
## 
## Detection (logit-scale):
##  Estimate    SE     z P(>|z|)
##    -0.684 0.543 -1.26   0.208
## 
## AIC: 59.11882 
## Number of sites: 20
## optim convergence code: 0
## optim iterations: 12 
## Bootstrap iterations: 0
\end{verbatim}
\end{kframe}
\end{knitrout}
Estimates obtained using maximum likelihood
\end{frame}



\begin{frame}[fragile]
  \frametitle{R package `unmarked'}
Occupancy estimate ($\hat{\psi}$)
\begin{knitrout}\footnotesize
\definecolor{shadecolor}{rgb}{0.878, 0.918, 0.933}\color{fgcolor}\begin{kframe}
\begin{alltt}
\hlkwd{backTransform}\hldef{(fm,} \hlkwc{type}\hldef{=}\hlsng{"state"}\hldef{)}
\end{alltt}
\begin{verbatim}
## Backtransformed linear combination(s) of Occupancy estimate(s)
## 
##  Estimate    SE LinComb (Intercept)
##     0.373 0.143   -0.52           1
## 
## Transformation: logistic
\end{verbatim}
\end{kframe}
\end{knitrout}
\pause
\vfill
Detection probability estimate ($\hat{p}$)
\begin{knitrout}\footnotesize
\definecolor{shadecolor}{rgb}{0.878, 0.918, 0.933}\color{fgcolor}\begin{kframe}
\begin{alltt}
\hlkwd{backTransform}\hldef{(fm,} \hlkwc{type}\hldef{=}\hlsng{"det"}\hldef{)}
\end{alltt}
\begin{verbatim}
## Backtransformed linear combination(s) of Detection estimate(s)
## 
##  Estimate    SE LinComb (Intercept)
##     0.335 0.121  -0.684           1
## 
## Transformation: logistic
\end{verbatim}
\end{kframe}
\end{knitrout}
\end{frame}



%% \begin{frame}
%%   \frametitle{In-class exercise}
%%   \begin{enumerate}
%%     \item Simulate an occupancy dataset with:
%%       \begin{itemize}
%%         \item $\psi=0.6$
%%         \item $p=0.1$
%%         \item nSites=100
%%         \item nVisits=5
%%       \end{itemize}
%%     \item Fit the model and report the estimates of $\psi$ and $p$
%%   \end{enumerate}
%% \end{frame}



\begin{frame}[fragile]
  \frametitle{Two types of occupancy estimates}
  % There are two types of occupancy estimates. \\
  \begin{enumerate}%[<+->]
  \item<1-> Unconditional occupancy $\psi=\Pr(z=1)$
    \begin{itemize}
      \item<1-> The estimate $\hat{\psi}$ that we computed earlier can be
        thought of as a prediction that we could use for a new
        site or a new time period
    \end{itemize}
  \item<2-> Conditional occupancy $\psi^* = \Pr(z_i=1|y_{i})$
    \begin{itemize}
      \item<2-> Conditional occupancy estimates only apply to sites in the
        sample, at the time of sampling
      \item<3-> In `unmarked', conditional occupancy is computed using
        Empirical Bayes methods, with the \inr{ranef} and \inr{bup}
        functions. In JAGS, full Bayesian inference is used instead. % of
%        Empirical Bayes.
      \item<4-> Conditional occupancy estimates can be used to answer
        questions like: 
        \begin{itemize}
          \item Which sites were occupied?
          \item How many sites were occupied?
        \end{itemize}
    \end{itemize}
  \item<5->[]  
%\pause
\begin{knitrout}\scriptsize
\definecolor{shadecolor}{rgb}{0.878, 0.918, 0.933}\color{fgcolor}\begin{kframe}
\begin{alltt}
\hlcom{## Posterior probs Pr(z_i=1 | y_i)       }
\hldef{z.post} \hlkwb{<-} \hlkwd{ranef}\hldef{(fm)}
\hlcom{## Extract posterior means}
\hldef{psi.conditional} \hlkwb{<-}  \hlkwd{bup}\hldef{(z.post,} \hlkwc{stat}\hldef{=}\hlsng{"mean"}\hldef{)}
\end{alltt}
\end{kframe}
\end{knitrout}
  \end{enumerate}
\end{frame}


\begin{frame}[fragile]
  \frametitle{Occupancy estimates}
\begin{knitrout}\scriptsize
\definecolor{shadecolor}{rgb}{0.878, 0.918, 0.933}\color{fgcolor}\begin{kframe}
\begin{alltt}
\hlkwd{round}\hldef{(}\hlkwd{data.frame}\hldef{(}\hlkwc{y}\hldef{=y,} \hlkwc{psi.unconditional}\hldef{=}\hlkwd{predict}\hldef{(fm,} \hlkwc{type}\hldef{=}\hlsng{"state"}\hldef{)[,}\hlnum{1}\hldef{],}
                 \hldef{psi.conditional),} \hlnum{3}\hldef{)}
\end{alltt}
\begin{verbatim}
##    y.1 y.2 y.3 y.4 psi.unconditional psi.conditional
## 1    0   0   0   0             0.373           0.104
## 2    0   0   0   0             0.373           0.104
## 3    1   0   0   0             0.373           1.000
## 4    0   0   0   0             0.373           0.104
## 5    0   0   0   0             0.373           0.104
## 6    0   0   0   0             0.373           0.104
## 7    0   1   1   0             0.373           1.000
## 8    0   0   0   0             0.373           0.104
## 9    0   0   0   0             0.373           0.104
## 10   0   1   0   1             0.373           1.000
## 11   0   0   0   0             0.373           0.104
## 12   0   0   0   0             0.373           0.104
## 13   0   0   0   0             0.373           0.104
## 14   0   0   0   0             0.373           0.104
## 15   0   0   0   0             0.373           0.104
## 16   0   1   1   0             0.373           1.000
## 17   0   0   0   0             0.373           0.104
## 18   1   0   1   0             0.373           1.000
## 19   0   0   0   0             0.373           0.104
## 20   0   1   0   0             0.373           1.000
\end{verbatim}
\end{kframe}
\end{knitrout}
\end{frame}



\begin{frame}[fragile]
  \frametitle{Occupancy estimates}
  How many sites were occupied? \pause
  We can answer this using posterior prediction.
\begin{knitrout}\tiny
\definecolor{shadecolor}{rgb}{0.878, 0.918, 0.933}\color{fgcolor}\begin{kframe}
\begin{alltt}
\hldef{nsim} \hlkwb{<-} \hlnum{1000}
\hldef{sites.occupied.post} \hlkwb{<-} \hlkwd{predict}\hldef{(z.post,} \hlkwc{func}\hldef{=sum,} \hlkwc{nsim}\hldef{=nsim)}
\hlkwd{par}\hldef{(}\hlkwc{mai}\hldef{=}\hlkwd{c}\hldef{(}\hlnum{0.9}\hldef{,}\hlnum{0.9}\hldef{,}\hlnum{0.1}\hldef{,}\hlnum{0.1}\hldef{))}
\hlkwd{plot}\hldef{(}\hlkwd{table}\hldef{(sites.occupied.post)}\hlopt{/}\hldef{nsim,} \hlkwc{lwd}\hldef{=}\hlnum{5}\hldef{,} \hlkwc{xlab}\hldef{=}\hlsng{"Sites occupied"}\hldef{,}
    \hlkwc{ylab}\hldef{=}\hlsng{"Empirical Bayes posterior probability"}\hldef{)}
\hlkwd{abline}\hldef{(}\hlkwc{v}\hldef{=}\hlkwd{sum}\hldef{(z),} \hlkwc{col}\hldef{=}\hlsng{"red"}\hldef{)} \hlcom{## Actual number occupied}
\end{alltt}
\end{kframe}

{\centering \includegraphics[width=0.7\textwidth]{figure/bup-hist-1} 

}


\end{knitrout}
\end{frame}



\begin{frame}[fragile]
  \frametitle{Occupancy estimates}
  The posterior median (or mean or mode) can be used as a point estimate:
\begin{knitrout}
\definecolor{shadecolor}{rgb}{0.878, 0.918, 0.933}\color{fgcolor}\begin{kframe}
\begin{alltt}
\hldef{psi.cond.estimate} \hlkwb{<-} \hlkwd{median}\hldef{(sites.occupied.post)}
\hldef{psi.cond.estimate}
\end{alltt}
\begin{verbatim}
## [1] 7
\end{verbatim}
\end{kframe}
\end{knitrout}

\pause
\vfill
Quantiles can be used for the 95\% confidence interval:
\begin{knitrout}
\definecolor{shadecolor}{rgb}{0.878, 0.918, 0.933}\color{fgcolor}\begin{kframe}
\begin{alltt}
\hlkwd{quantile}\hldef{(sites.occupied.post,} \hlkwc{prob}\hldef{=}\hlkwd{c}\hldef{(}\hlnum{0.025}\hldef{,} \hlnum{0.975}\hldef{))}
\end{alltt}
\begin{verbatim}
##  2.5% 97.5% 
##     6    10
\end{verbatim}
\end{kframe}
\end{knitrout}
\end{frame}



\subsection{Bayesian methods}


\begin{frame}
  \frametitle{Bayesian methods}
  Bayesian methods incorporate prior distributions into inferences. \\
  \pause
  \vfill
  We often have prior information, and it's easy to use. \\
  \pause
  \vfill
  If we don't have prior information, we can use vague priors and
  obtain similar inferences as likelihood-based methods. \\
  \pause
  \vfill
  Ultimately, we want the posterior distribution of the model
  parameters. We'll use MCMC to obtain samples from this
  distribution. 
\end{frame}




\begin{frame}
  \frametitle{JAGS and the jagsUI package}
  A reminder\dots
  %% We will use JAGS and the R packages `rjags' and `jagsUI'.
  % JAGS can
  % be downloaded here: \url{
  %   https://sourceforge.net/projects/mcmc-jags/files/
  % }. \\
  % \pause
  \vfill
  The basic steps of a Bayesian analysis with JAGS are:
  \begin{enumerate}%[<+->]
    \item Write a model in the BUGS language in a text file
    \item Put data in a named list
    \item Create a function to generate random initial values
    \item Specify the parameters you want to monitor
    \item Use MCMC to draw posterior samples
    \item Use posterior samples for inference and prediction
  \end{enumerate}
\end{frame}



\begin{frame}[fragile]
  \frametitle{The BUGS language}
  The BUGS model file: {\tt occupancy-model.jag}
\begin{knitrout}\scriptsize
\definecolor{shadecolor}{rgb}{0.961, 0.961, 0.863}\color{fgcolor}\begin{kframe}
\begin{verbatim}
model {

psi ~ dunif(0,1)            # Prior for occupancy parameter
p ~ dunif(0,1)              # Prior for detection probability

for(i in 1:nSites) {
  z[i] ~ dbern(psi)         # Latent presence/absence
  for(j in 1:nOccasions) {
    y[i,j] ~ dbern(z[i]*p)  # Model for the data
  }
}

sitesOccupied <- sum(z)     # Total number of sites occupied

}
\end{verbatim}
\end{kframe}
\end{knitrout}
\end{frame}





\begin{frame}[fragile]
  \frametitle{Data, inits, and parameters}
  Put data in a named list
  \vspace{-12pt}
\begin{knitrout}\small
\definecolor{shadecolor}{rgb}{0.878, 0.918, 0.933}\color{fgcolor}\begin{kframe}
\begin{alltt}
\hldef{jags.data} \hlkwb{<-} \hlkwd{list}\hldef{(}\hlkwc{y}\hldef{=y,} \hlkwc{nSites}\hldef{=nSites,} \hlkwc{nOccasions}\hldef{=nVisits)}
\end{alltt}
\end{kframe}
\end{knitrout}
\pause
\vfill
  Initial values
  \vspace{-12pt}
\begin{knitrout}\small
\definecolor{shadecolor}{rgb}{0.878, 0.918, 0.933}\color{fgcolor}\begin{kframe}
\begin{alltt}
\hldef{jags.inits} \hlkwb{<-} \hlkwa{function}\hldef{() \{}
    \hlkwd{list}\hldef{(}\hlkwc{psi}\hldef{=}\hlkwd{runif}\hldef{(}\hlnum{1}\hldef{),} \hlkwc{p}\hldef{=}\hlkwd{runif}\hldef{(}\hlnum{1}\hldef{),} \hlkwc{z}\hldef{=}\hlkwd{rep}\hldef{(}\hlnum{1}\hldef{, nSites))}
\hldef{\}}
\end{alltt}
\end{kframe}
\end{knitrout}
\pause
\vfill
  Parameters to monitor
  \vspace{-12pt}
\begin{knitrout}\small
\definecolor{shadecolor}{rgb}{0.878, 0.918, 0.933}\color{fgcolor}\begin{kframe}
\begin{alltt}
\hldef{jags.pars} \hlkwb{<-} \hlkwd{c}\hldef{(}\hlsng{"psi"}\hldef{,} \hlsng{"p"}\hldef{,} \hlsng{"sitesOccupied"}\hldef{)}
\end{alltt}
\end{kframe}
\end{knitrout}
\end{frame}





\begin{frame}[fragile]
  \frametitle{MCMC}
  \small
  This function will compile the model, run adaptive MCMC, and then
  draw posterior samples using 3 Markov chains run in parallel. \\
\begin{knitrout}\scriptsize
\definecolor{shadecolor}{rgb}{0.878, 0.918, 0.933}\color{fgcolor}\begin{kframe}
\begin{alltt}
\hlcom{## install.packages("jagsUI")}
\hlkwd{library}\hldef{(jagsUI)}
\hldef{jags.post.samples} \hlkwb{<-} \hlkwd{jags.basic}\hldef{(}\hlkwc{data}\hldef{=jags.data,} \hlkwc{inits}\hldef{=jags.inits,}
                                \hlkwc{parameters.to.save}\hldef{=jags.pars,}
                                \hlkwc{model.file}\hldef{=}\hlsng{"occupancy-model.jag"}\hldef{,}
                                \hlkwc{n.chains}\hldef{=}\hlnum{3}\hldef{,} \hlkwc{n.adapt}\hldef{=}\hlnum{100}\hldef{,} \hlkwc{n.burnin}\hldef{=}\hlnum{0}\hldef{,}
                                \hlkwc{n.iter}\hldef{=}\hlnum{2000}\hldef{,} \hlkwc{parallel}\hldef{=}\hlnum{TRUE}\hldef{)}
\end{alltt}
\end{kframe}
\end{knitrout}
\end{frame}



\begin{frame}[fragile]
  \frametitle{Summarize output}
\begin{knitrout}\scriptsize
\definecolor{shadecolor}{rgb}{0.878, 0.918, 0.933}\color{fgcolor}\begin{kframe}
\begin{alltt}
\hlkwd{summary}\hldef{(jags.post.samples[,jags.pars])}
\end{alltt}
\begin{verbatim}
## 
## Iterations = 101:2100
## Thinning interval = 1 
## Number of chains = 3 
## Sample size per chain = 2000 
## 
## 1. Empirical mean and standard deviation for each variable,
##    plus standard error of the mean:
## 
##                 Mean     SD Naive SE Time-series SE
## psi           0.4349 0.1627 0.002101       0.005900
## p             0.3275 0.1116 0.001441       0.003578
## sitesOccupied 8.5223 2.7909 0.036030       0.116918
## 
## 2. Quantiles for each variable:
## 
##                 2.5%    25%    50%     75%   97.5%
## psi           0.1814 0.3170 0.4131  0.5228  0.8356
## p             0.1279 0.2454 0.3231  0.4040  0.5512
## sitesOccupied 6.0000 7.0000 8.0000 10.0000 17.0000
\end{verbatim}
\end{kframe}
\end{knitrout}
\end{frame}




\begin{frame}[fragile]
  \frametitle{Traceplots and density plots}
\begin{knitrout}\footnotesize
\definecolor{shadecolor}{rgb}{0.878, 0.918, 0.933}\color{fgcolor}\begin{kframe}
\begin{alltt}
\hlkwd{plot}\hldef{(jags.post.samples[,jags.pars])}
\end{alltt}
\end{kframe}

{\centering \includegraphics[width=0.7\textwidth]{figure/bugs-plot-1} 

}


\end{knitrout}
\end{frame}


\section{Assignment}




% \begin{frame}
%   \frametitle{Outline}
%   \Large
%   \tableofcontents[currentsection]
% \end{frame}







\begin{frame}
  \frametitle{Assignment}
  \small
  % \scriptsize
  Create a self-contained R script or Rmarkdown file
  to do the following:
  \begin{enumerate}
    \small
    \item Import the Canada Warbler data ({\tt
      "cawa\_data\_2017\_occu.csv"}) and compute basic summary  stats
      like we did earlier. 
    \begin{itemize}
      \item Data are ``point count data'' from North Carolina
      \item Each 10-min survey was divided into 4, 2.5-min intervals 
    \end{itemize}
    \item Fit the single-season (static) occupancy model using
      `unmarked' and `JAGS'. 
    \item Create a \inr{data.frame} to compare the estimates of
      $\psi$, $p$, and ``nSitesOccupied'' from the two
      analyses. Compare point estimates and 95\% CIs. Describe any
      differences you see in the estimates.
  \end{enumerate}
  Upload your {\tt .R} or {\tt .Rmd} file to ELC by 5:00pm on Monday. 
\end{frame}





\end{document}

