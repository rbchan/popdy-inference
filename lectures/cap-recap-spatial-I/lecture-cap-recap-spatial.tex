\documentclass[color=usenames,dvipsnames]{beamer}\usepackage[]{graphicx}\usepackage[]{xcolor}
% maxwidth is the original width if it is less than linewidth
% otherwise use linewidth (to make sure the graphics do not exceed the margin)
\makeatletter
\def\maxwidth{ %
  \ifdim\Gin@nat@width>\linewidth
    \linewidth
  \else
    \Gin@nat@width
  \fi
}
\makeatother

\definecolor{fgcolor}{rgb}{0, 0, 0}
\newcommand{\hlnum}[1]{\textcolor[rgb]{0.69,0.494,0}{#1}}%
\newcommand{\hlsng}[1]{\textcolor[rgb]{0.749,0.012,0.012}{#1}}%
\newcommand{\hlcom}[1]{\textcolor[rgb]{0.514,0.506,0.514}{\textit{#1}}}%
\newcommand{\hlopt}[1]{\textcolor[rgb]{0,0,0}{#1}}%
\newcommand{\hldef}[1]{\textcolor[rgb]{0,0,0}{#1}}%
\newcommand{\hlkwa}[1]{\textcolor[rgb]{0,0,0}{\textbf{#1}}}%
\newcommand{\hlkwb}[1]{\textcolor[rgb]{0,0.341,0.682}{#1}}%
\newcommand{\hlkwc}[1]{\textcolor[rgb]{0,0,0}{\textbf{#1}}}%
\newcommand{\hlkwd}[1]{\textcolor[rgb]{0.004,0.004,0.506}{#1}}%
\let\hlipl\hlkwb

\usepackage{framed}
\makeatletter
\newenvironment{kframe}{%
 \def\at@end@of@kframe{}%
 \ifinner\ifhmode%
  \def\at@end@of@kframe{\end{minipage}}%
  \begin{minipage}{\columnwidth}%
 \fi\fi%
 \def\FrameCommand##1{\hskip\@totalleftmargin \hskip-\fboxsep
 \colorbox{shadecolor}{##1}\hskip-\fboxsep
     % There is no \\@totalrightmargin, so:
     \hskip-\linewidth \hskip-\@totalleftmargin \hskip\columnwidth}%
 \MakeFramed {\advance\hsize-\width
   \@totalleftmargin\z@ \linewidth\hsize
   \@setminipage}}%
 {\par\unskip\endMakeFramed%
 \at@end@of@kframe}
\makeatother

\definecolor{shadecolor}{rgb}{.97, .97, .97}
\definecolor{messagecolor}{rgb}{0, 0, 0}
\definecolor{warningcolor}{rgb}{1, 0, 1}
\definecolor{errorcolor}{rgb}{1, 0, 0}
\newenvironment{knitrout}{}{} % an empty environment to be redefined in TeX

\usepackage{alltt}
%\documentclass[color=usenames,dvipsnames,handout]{beamer}

\usepackage[roman]{../lectures}
%\usepackage[sans]{../lectures}


\hypersetup{pdfpagemode=UseNone,pdfstartview={FitV}}

\mode<handout>{
  \usetheme{default}
%  \setbeamercolor{background canvas}{bg=black!5}
% \pgfpagesuselayout{4 on 1}[letterpaper,landscape,border shrink=2.5mm]
%  \pgfpagesuselayout{2 on 1}[letterpaper,border shrink=10mm]
}


% Load function to compile and open PDF




% Compile and open PDF





% New command for inline code that isn't to be evaluated
\definecolor{inlinecolor}{rgb}{0.878, 0.918, 0.933}
\newcommand{\inr}[1]{\colorbox{inlinecolor}{\texttt{#1}}}
\IfFileExists{upquote.sty}{\usepackage{upquote}}{}
\begin{document}
    



\begin{frame}[plain]
  \LARGE
  \centering
  {
    \LARGE Spatial capture-recapture %\\
    for \\ closed populations \\
%    \Large simulation, fitting, and prediction \\
  }
  {\color{default} \rule{\textwidth}{0.1pt} }
  \vfill
  \large
  WILD(FISH) 8390 \\
%  Estimation of Fish and Wildlife Population Parameters \\
  Inference for Models of Fish and Wildlife Population Dynamics \\
  \vfill
  \large
  Richard Chandler \\
  University of Georgia \\
\end{frame}






\section{Overview}



\begin{frame}[plain]
  \frametitle{Outline}
  \Large
  \only<1>{\tableofcontents}%[hideallsubsections]}
  \only<2 | handout:0>{\tableofcontents[currentsection]}%,hideallsubsections]}
\end{frame}



\begin{frame}
  \frametitle{SCR overview}
  {\centering \large Two motivations for SCR \\}
  \vfill
  \begin{enumerate}
    \item Improved inference
    \begin{itemize}
      \item<1-> Non-spatial models don't account for important sources
        of variation in $p$. %that can cause bias.
        \begin{itemize}
          \item<1-> Distance between animals and traps
          \item<1-> Trap-specific covariates
        \end{itemize}
      \item<2-> SCR makes it possible to estimate \alert{density}, not
        just $N$ in an unknown region. 
    \end{itemize}
    \pause
    \vfill
  \item<3-> Improved science
  \begin{itemize}
    \item<3-> We can ask new questions, such as:
      \begin{itemize}
        \item<3-> What influences spatial variation in density?
        \item<4-> How do survival and recruitment vary in space and time?
        \item<5-> How does movement influence density and detectability?
      \end{itemize}
    \item<6-> Rather than think of SCR as a new estimation tool, you
      can think of it as an individual-based framework for inference on
      spatial population dynamics.
    \end{itemize}
  \end{enumerate}
  % \vfill
  % \centering
  % \footnotesize
  % \uncover<6->{
  % Rather than think of SCR as a new estimation tool, you can think of
  % it as an individual-based framework for inference on spatial
  % population dynamics. \\}
\end{frame}



% \begin{frame}
%   \frametitle{Comparison to mark-recapture}
%   The simplest estimator of abundance is 
%   \[
%     \hat{N} = \frac{n}{\hat{p}}
%   \]
%   where $n$ is the number of individuals detected, $p$ is detection
%   probability, and $E(n)=Np$. \\
%   \pause
%   \vfill
%   In distance sampling, we modeled detection probability as a
%   function of distance, and we replaced $p$ with average detection
%   probability. \\ 
%   \pause
%   \vfill
%   Spatial capture recapture can be thought of in a similar way, but we 
%   average detection probability over the spatial distribution of
%   individuals, not just over distance. 
% \end{frame}





\begin{frame}
  \frametitle{SCR data}
  \small
  % SCR capture histories histories have 3 dimensions instead of two. \\
  % \pause
  % \vfill
  SCR capture histories can be organized as a 3D array where $y_{ijk}$
  indicates if individual $i=1,\dots,n$, was captured in trap
  $j=1,\dots,J$ on occasion $k=1,\dots,K$. \\
  \pause \vfill
  Here's an example of a ``flattened array'' with $n=4$ animals captured  
  at $J=3$ traps on $K=2$ occasions. \\
  \centering
  \vfill  
  \begin{tabular}{lccccccc}
    \hline
    & \multicolumn{7}{c}{Occasion} \\
    \cline{2-8}
    & \multicolumn{3}{c}{1} & & \multicolumn{3}{c}{2} \\
    \cline{2-4} \cline{6-8}
    & \multicolumn{3}{c}{Trap} & & \multicolumn{3}{c}{Trap} \\
    \cline{2-4} \cline{6-8}
    Individual & 1 & 2 & 3 & & 1 & 2 & 3 \\
    \hline
    1 & 0 & 0 & 0 & & 1 & 0 & 1 \\
    2 & 1 & 1 & 1 & & 0 & 1 & 1 \\
    3 & 0 & 1 & 0 & & 0 & 0 & 0 \\
    4 & 0 & 1 & 1 & & 1 & 0 & 0 \\
    \hline
  \end{tabular}
  \pause
  \vfill
  \flushleft
  Because we know the coordinates of the traps, we also know when and
  where each individual was detected. \\
  \pause
  \vfill
  This spatial information has been available all along, but it wasn't
  utilized to estimate density until Efford (2004, Oikos). \\
\end{frame}




% \begin{frame}
%   \frametitle{In-class exercise}
%   Building off the previous example\dots
%   \begin{enumerate}
%     \item Compute $\bar{p}$ for line-transect sampling when
%       $\sigma=50, 100, \mathrm{and}\, 200$, instead of $\sigma=25$.  
%     \item Repeat, but for point-transect sampling. 
%   \end{enumerate}
% \end{frame}





\begin{frame}
  \frametitle{\large Closed population model ($N$ known hypothetically) }
  \footnotesize
  State model (a spatial point process model) %\\
  \begin{gather*}
    \lambda(\bs) = \exp(\beta_0 + \beta_1 w_1(\bs) + \beta_2 w_2(\bs) + \cdots) \\
    \Lambda = \int_{\mathcal{S}} \lambda(\bs) \; \mathrm{d}\bs \\
    N \sim \mathrm{Pois}(\Lambda) \\
    \bsi \propto \lambda(\bs) \;\; \mathrm{for}\; i=1,\dots,N 
  \end{gather*}
  \pause
%  \vfill
  Observation model (supposing $N$ was known)
  \begin{gather*}
    p_{ij} = g_0\exp(-\|\bsi - \bxj\|^2/(2\sigma^2))  \;\; \mathrm{for}\, j=1,\dots,J  \\
    y_{ijk} \sim \mathrm{Bernoulli}(p_{ij})
  \end{gather*}
  \pause
%  \vfill
%  \footnotesize
  \scriptsize
  Definitions \\
  \hangindent=0.9cm $\lambda(\bs)$ -- ``Intensity function'' %or ``density surface''
  describing expected density of individuals at location $\bs$ \\ 
  $\Lambda$ -- Expected number of individuals in the spatial region $\mathcal{S}$ \\
  $N$ -- Realized number of individuals (ie, population size) \\
  $\bsi$ -- Location of the $i$th activity center \\
  $\bxj$ -- Location of trap $j$ \\
  $\dsixj$ -- Euclidean distance between $\bsi$ and $\bxj$ \\
  $g_0$ -- Capture probability when distance between activity centers
  and traps is 0 \\
  $\sigma$ -- Scale parameter of encounter function \\
  $p_{ij}$ -- Capture probability \\
  $y_{ijk}$ -- Spatial capture histories \\
  % \pause
  % \vfill
  % The problem with this formulation is that we don't observe the ``all
  % zero'' encounter histories (and thus we don't know $N$). 
\end{frame}







\begin{frame}
  \frametitle{\large Closed population model ($N$ known hypothetically) }
  \footnotesize
  Many variations and extensions are possible for both the state and
  observation models, but today we will focus on the simplest case
  with no spatial variation in the expected value of density.
  \begin{gather*}
    N \sim \mathrm{Pois}(\lambda) \\
    \bsi \sim \mathrm{Unif}(\cal S) \\
    %% p_{ij} = g_0\exp(-\|\bsi - \bxj\|^2/(2\sigma^2))  \;\; \mathrm{for}\, j=1,\dots,J  \\
    p_{ij} = g_0\exp\left(\frac{-\|\bsi - \bxj\|^2}{2\sigma^2}\right)  \\
    y_{ijk} \sim \mathrm{Bernoulli}(p_{ij})
  \end{gather*}
  \pause
  \vfill
  Just remember, this formulation isn't exactly correct because we
  can't observe the ``all zero'' encounter histories (and thus we
  don't know $N$).
\end{frame}







% \begin{frame}
%   \frametitle{Spatial point processes}
%   {\centering \large
%     The state model of SCR is a spatial (or spatio-temporal) point process \\}
%   \vfill
%   \pause
%   There are many varieties of spatial point processes, including \\
%   \begin{itemize}
%     \item (In)homogeneous binomial point process
%     \item (In)homogeneous Poisson point process
%     \item Cox process
%     \item Gibbs process
%     \item Markov point process
% %    \item Among others
%   \end{itemize}
% \end{frame}





% \begin{frame}
%   \frametitle{Spatial point processes}
% %  \large
%   All share a few properties \\
%   \begin{itemize}%[<+->]
%     \item<1-> The data are a collection of points called a ``point pattern''
%     \item<2-> Points are in an area called the state-space
%       ($\mathcal{S}$), or observation window, which is usually two
%       dimensional  
%     \item<3-> An intensity function ($\lambda(\bs)$) describes spatial
%       variation in the density of points
%     \item<4-> The area under this function is the expected number of
%       points (a.k.a, $N$) in the region:
%   \end{itemize}
%   \vfill
% %  \Large
%   \uncover<5->{
% \[
%   E(N) = \Lambda = \int_{\mathcal{S}} \lambda(\bs) \;\mathrm{d}\bs
% \]
% }
% \end{frame}



% \begin{frame}
%   \frametitle{Poisson point process}
%   Properties
%   \begin{itemize}
%     \item If density is constant throughout the state-space:
%       $\lambda(\bs) = \lambda$, the process is said to be
%       ``homogeneous''
%     \item Otherwise, the process is ``inhomogeneous''
%     \item The number of points in any region of the state-space is
%       Poisson distributed
%     \item Points are independent of one another (no attraction or
%       repulsion) 
%   \end{itemize}
%   \pause
%   \vfill
% %   \centering {\bf Note:} For more complicated models, it's easier to
% %   work in discrete space than continuous space. Plus, spatial
% %   covariates always come to us in the form of raster. \\
%   An inhomogeneous point process allows for spatial variation
%   in density, which can be modeled using \alert{spatial covariates},
%   such as raster layers. \\
% \end{frame}






\begin{frame}
  \frametitle{Shiny app}
  \LARGE
  \centering
  % \href{
  %   https://richard-chandler.shinyapps.io/scr-cap-prob/
  % }{
  %   Shiny App 
  % } \\
  Shiny App \\
  \vfill
  \normalsize
  \color{blue}
  \url{
    https://richard-chandler.shinyapps.io/scr-cap-prob/
  } \\
\end{frame}




% \begin{frame}
%   \frametitle{Observation models}
%   The SCR model assumes that capture probability decreases with
%   distance between activity centers and traps. \\
%   \pause
%   \vfill
%   We can use encounter rate functions similar to those from distance 
%   sampling for this purpose. \\
%   \pause
%   \vfill
%   The most common encounter rate function that we'll consider is based
%   on the Gaussian distribution:
%   \begin{gather*}
%     p_{ij} = g_0\exp(-\dsixj/(2\sigma^2)) \\
%     y_{ijk} \sim \mathrm{Bernoulli}(p_{ij}) 
%   \end{gather*}
%   \pause
% %  \vfill
%   However, there are many options and we aren't restricted to the
%   Bernoulli observation model. 
% \end{frame}







% \begin{frame}
%   \frametitle{Observation models}
% %  \begin{itemize}
% %  \item
%   Observation models are not the same as encounter rate functions. \\
%   \pause \vfill
%   % \item
%   Observation models are chosen with respect to the sampling
%       method (mist-net, camera trap, hair snare, etc\dots) \\
% %  \end{itemize}
%   \pause \vfill
%   % \begin{center}
%   \centering
%     \begin{tabular}{lll}
%       \hline
%       Model       & Detector/Trap     & Examples           \\
%       \hline
%       Bernoulli   & Proximity    & Hair-snares        \\
%       Poisson     & Count        & Camera trap        \\
%       Multinomial/Categorical & Multi-catch  & Mist net, crab pot \\
%       ---          & Single-catch & Sherman trap       \\
%       \hline
%     \end{tabular}
%     % \end{center}
%   \pause \vfill
%   There are other observation models for data from area searches,
%   transects, or acoustic recording units.
% \end{frame}








\section{Simulation}


\begin{frame}
  \frametitle{Outline}
  \Large
%  \tableofcontents[currentsection,currentsubsection]
  \tableofcontents[currentsection]
\end{frame}




% \begin{frame}[fragile]
%   \frametitle{Homogeneous binomial point process}
% <<bpp1,size='footnotesize',out.width="50%",fig.align="center">>=
% N <- 25
% s <- cbind(runif(N, 0, 1), runif(N, 0, 1))
% plot(s, pch=16, col="blue", xlab="Easting", ylab="Northing",
%      xlim=c(0,1), ylim=c(0,1), cex.lab=1.5, asp=1)
% @
% \end{frame}






\begin{frame}[fragile]
  \frametitle{Homogeneous Poisson point process}
\begin{knitrout}\footnotesize
\definecolor{shadecolor}{rgb}{0.878, 0.918, 0.933}\color{fgcolor}\begin{kframe}
\begin{alltt}
\hldef{lambda1} \hlkwb{<-} \hlnum{25}\hldef{; A} \hlkwb{<-} \hlnum{1}     \hlcom{## lambda1=density, A=area}
\hldef{N} \hlkwb{<-} \hlkwd{rpois}\hldef{(}\hlnum{1}\hldef{, lambda1}\hlopt{*}\hldef{A)}
\hldef{s} \hlkwb{<-} \hlkwd{cbind}\hldef{(}\hlkwd{runif}\hldef{(N,} \hlnum{0}\hldef{,} \hlnum{1}\hldef{),} \hlkwd{runif}\hldef{(N,} \hlnum{0}\hldef{,} \hlnum{1}\hldef{))}
\hlkwd{plot}\hldef{(s,} \hlkwc{pch}\hldef{=}\hlnum{16}\hldef{,} \hlkwc{col}\hldef{=}\hlsng{"blue"}\hldef{,} \hlkwc{xlab}\hldef{=}\hlsng{"Easting"}\hldef{,} \hlkwc{ylab}\hldef{=}\hlsng{"Northing"}\hldef{,}
     \hlkwc{xlim}\hldef{=}\hlkwd{c}\hldef{(}\hlnum{0}\hldef{,}\hlnum{1}\hldef{),} \hlkwc{ylim}\hldef{=}\hlkwd{c}\hldef{(}\hlnum{0}\hldef{,}\hlnum{1}\hldef{),} \hlkwc{cex.lab}\hldef{=}\hlnum{1.5}\hldef{,} \hlkwc{asp}\hldef{=}\hlnum{1}\hldef{)}
\end{alltt}
\end{kframe}

{\centering \includegraphics[width=0.5\linewidth]{figure/ppp1-1} 

}


\end{knitrout}
\end{frame}












% \begin{frame}[fragile]
%   \frametitle{Inhomogeneous Poisson point process}
%   First, let's import a raster layer
% <<ippp1,size='footnotesize',fig.width=7.2,out.width="60%",fig.align="center",results="hide">>=
% library(raster)
% elevation <- raster("elevation.tif")
% plot(elevation, col=topo.colors(100), main="Elevation")
% @
% \end{frame}




% \begin{frame}
%   \frametitle{Closed population estimation options}
%   Conditional likelihood \\
%   \begin{itemize}
%     \item Estimate $\tilde{p}$\footnote{$\tilde{p}$ depends on $\lambda(s), g_0,
%         \sigma, \mathcal{S}, x$}, and then compute $\hat{N}=n/\hat{\tilde{p}}$
%   \end{itemize}
%   \pause
%   \vfill
%   Joint likelihood \\
%   \begin{itemize}
%     \item Estimate $N$ and $\tilde{p}$ jointly
%     \item Joint likelihood can be written as
%       \begin{itemize}
%       \item $L(N,g_0,\sigma;y,n) = p(y|n,p)p(n|N,p)$ or
%       \item $L(N,g_0,\sigma;y,n) = p(y,n|N,p)$
%       \end{itemize}
%   \end{itemize}
%   \pause \vfill
%   Data augmentation \\
%   \begin{itemize}
%     \item Tack on many ``all zero'' encounter histories and estimate
%       how many of them actually occurred
%     \item Usually, but not necessarily, used in Bayesian inference
%   \end{itemize}
% \end{frame}






% \begin{frame}[fragile]
%   \frametitle{Inhomogeneous Poisson point process}
%   \small
%   Second, let's pick some coefficients and create a density surface
% <<ippp2,size='footnotesize',fig.width=7.2,out.width="60%",fig.align="center">>=
% beta0 <- -15
% beta1 <- 0.01 #0.005
% lambda <- exp(beta0 + beta1*elevation)  # Intensity function
% plot(lambda, col=terrain.colors(100), main="Density surface")
% @
% \end{frame}




% \begin{frame}[fragile]
%   \frametitle{Inhomogeneous Poisson point process}
%   \small
%   Third, simulate $N$
%   \vspace{-6pt}
% <<ippp3,size='footnotesize'>>=
% set.seed(538)  
% ds <- 1                            ## Pixel area is 1 ha
% lambda.values <- values(lambda)    ## Convert raster to vector
% Lambda <- sum(lambda.values*ds)    ## E(N)
% (N <- rpois(1, Lambda))            ## Realized N
% @
% \pause
% \vfill
% Fourth, simulate and $\bs_1, \dots, \bs_N$. To do this, we'll pick
% pixels proportional to density. Then we'll jitter each point
% inside its pixel. 
%   \vspace{-6pt}
% <<ipp4,size='footnotesize'>>=
% n.pixels <- length(lambda)
% jitter <- 0.005                    ## Half width of pixel 
% s.pixels <- sample(n.pixels, size=N, replace=TRUE,
%                    prob=lambda.values/Lambda)
% elevation.xyz <- as.data.frame(elevation, xy=TRUE)
% s <- elevation.xyz[s.pixels,c("x","y")] +
%     cbind(runif(N, -jitter, jitter),runif(N, -jitter, jitter))
% @
% \end{frame}







% \begin{frame}[fragile]
%   \frametitle{Inhomogeneous Poisson point process}
% <<ippp5,size='scriptsize',fig.width=7.2,out.width="70%",fig.align="center">>=
% plot(lambda, col=terrain.colors(100),
%      main="Density surface with activity centers")
% points(s, pch=16, cex=1, col="blue")
% @
% \end{frame}






\begin{frame}[fragile]
  \frametitle{Traps}
\begin{knitrout}\scriptsize
\definecolor{shadecolor}{rgb}{0.878, 0.918, 0.933}\color{fgcolor}\begin{kframe}
\begin{alltt}
\hldef{x} \hlkwb{<-} \hlkwd{cbind}\hldef{(}\hlkwd{rep}\hldef{(}\hlkwd{seq}\hldef{(}\hlnum{0.15}\hldef{,} \hlnum{0.85}\hldef{,} \hlkwc{by}\hldef{=}\hlnum{0.1}\hldef{),} \hlkwc{each}\hldef{=}\hlnum{8}\hldef{),}
           \hlkwd{rep}\hldef{(}\hlkwd{seq}\hldef{(}\hlnum{0.15}\hldef{,} \hlnum{0.85}\hldef{,} \hlkwc{by}\hldef{=}\hlnum{0.1}\hldef{),} \hlkwc{times}\hldef{=}\hlnum{8}\hldef{))}  \hlcom{## Trap locations}
\hlkwd{plot}\hldef{(s,} \hlkwc{pch}\hldef{=}\hlnum{16}\hldef{,} \hlkwc{col}\hldef{=}\hlsng{"blue"}\hldef{,} \hlkwc{xlab}\hldef{=}\hlsng{"Easting"}\hldef{,} \hlkwc{ylab}\hldef{=}\hlsng{"Northing"}\hldef{,} \hlkwc{asp}\hldef{=}\hlnum{1}\hldef{,}
     \hlkwc{main}\hldef{=}\hlsng{"Activity centers and traps"}\hldef{)}
\hlkwd{points}\hldef{(x,} \hlkwc{pch}\hldef{=}\hlnum{3}\hldef{)}              \hlcom{## Trap locations}
\end{alltt}
\end{kframe}

{\centering \includegraphics[width=0.6\linewidth]{figure/traps1-1} 

}


\end{knitrout}
\end{frame}





\begin{frame}[fragile]
  \frametitle{Distance between traps and activity centers}
  Compute distances between activity centers ($\bs_1, \dots, \bs_N$)
  and traps ($\bx_1, \dots, \bx_J$).
\begin{knitrout}\footnotesize
\definecolor{shadecolor}{rgb}{0.878, 0.918, 0.933}\color{fgcolor}\begin{kframe}
\begin{alltt}
\hldef{J} \hlkwb{<-} \hlkwd{nrow}\hldef{(x)}                 \hlcom{## nTraps}
\hldef{dist.sx} \hlkwb{<-} \hlkwd{matrix}\hldef{(}\hlnum{NA}\hldef{, N, J)}
\hlkwa{for}\hldef{(i} \hlkwa{in} \hlnum{1}\hlopt{:}\hldef{N) \{}
    \hldef{dist.sx[i,]} \hlkwb{<-} \hlkwd{sqrt}\hldef{((s[i,}\hlnum{1}\hldef{]}\hlopt{-}\hldef{x[,}\hlnum{1}\hldef{])}\hlopt{^}\hlnum{2} \hlopt{+} \hldef{(s[i,}\hlnum{2}\hldef{]}\hlopt{-}\hldef{x[,}\hlnum{2}\hldef{])}\hlopt{^}\hlnum{2}\hldef{)}
\hldef{\}}
\end{alltt}
\end{kframe}
\end{knitrout}
\pause
\vfill
  Look at distances between first 4 individuals and first 5 traps.
\begin{knitrout}\footnotesize
\definecolor{shadecolor}{rgb}{0.878, 0.918, 0.933}\color{fgcolor}\begin{kframe}
\begin{alltt}
\hlkwd{round}\hldef{(dist.sx[}\hlnum{1}\hlopt{:}\hlnum{4}\hldef{,}\hlnum{1}\hlopt{:}\hlnum{5}\hldef{],} \hlkwc{digits}\hldef{=}\hlnum{2}\hldef{)}
\end{alltt}
\begin{verbatim}
##      [,1] [,2] [,3] [,4] [,5]
## [1,] 0.88 0.85 0.82 0.81 0.81
## [2,] 0.88 0.79 0.71 0.63 0.56
## [3,] 0.27 0.33 0.41 0.50 0.59
## [4,] 0.79 0.80 0.83 0.87 0.92
\end{verbatim}
\end{kframe}
\end{knitrout}

\end{frame}






\begin{frame}[fragile]
  \frametitle{Capture probability}
  Compute capture probability
\begin{knitrout}\footnotesize
\definecolor{shadecolor}{rgb}{0.878, 0.918, 0.933}\color{fgcolor}\begin{kframe}
\begin{alltt}
\hldef{g0} \hlkwb{<-} \hlnum{0.2}
\hldef{sigma} \hlkwb{<-} \hlnum{0.05}
\hldef{p} \hlkwb{<-} \hldef{g0}\hlopt{*}\hlkwd{exp}\hldef{(}\hlopt{-}\hldef{dist.sx}\hlopt{^}\hlnum{2}\hlopt{/}\hldef{(}\hlnum{2}\hlopt{*}\hldef{sigma}\hlopt{^}\hlnum{2}\hldef{))}
\end{alltt}
\end{kframe}
\end{knitrout}
\pause
\vfill
  Look at capture probs for first 4 individuals and first 5 traps.
\begin{knitrout}\footnotesize
\definecolor{shadecolor}{rgb}{0.878, 0.918, 0.933}\color{fgcolor}\begin{kframe}
\begin{alltt}
\hlkwd{round}\hldef{(p[}\hlnum{1}\hlopt{:}\hlnum{4}\hldef{,}\hlnum{1}\hlopt{:}\hlnum{5}\hldef{],} \hlkwc{digits}\hldef{=}\hlnum{3}\hldef{)}
\end{alltt}
\begin{verbatim}
##      [,1] [,2] [,3] [,4] [,5]
## [1,]    0    0    0    0    0
## [2,]    0    0    0    0    0
## [3,]    0    0    0    0    0
## [4,]    0    0    0    0    0
\end{verbatim}
\end{kframe}
\end{knitrout}

\end{frame}





\begin{frame}[fragile]
  \frametitle{Capture histories}
  Simulate capture histories for all $N$ individuals
\begin{knitrout}\footnotesize
\definecolor{shadecolor}{rgb}{0.878, 0.918, 0.933}\color{fgcolor}\begin{kframe}
\begin{alltt}
\hldef{K} \hlkwb{<-} \hlnum{5}                          \hlcom{# nOccasions}
\hldef{y.all} \hlkwb{<-} \hlkwd{array}\hldef{(}\hlnum{NA}\hldef{,} \hlkwd{c}\hldef{(N, J, K))}
\hlkwa{for}\hldef{(i} \hlkwa{in} \hlnum{1}\hlopt{:}\hldef{N) \{}
    \hlkwa{for}\hldef{(j} \hlkwa{in} \hlnum{1}\hlopt{:}\hldef{J) \{}
        \hldef{y.all[i,j,]} \hlkwb{<-} \hlkwd{rbinom}\hldef{(K,} \hlnum{1}\hldef{,} \hlkwc{prob}\hldef{=p[i,j])}
    \hldef{\}}
\hldef{\}}
\end{alltt}
\end{kframe}
\end{knitrout}
\pause
\vfill
  Discard individuals not captured
\begin{knitrout}\footnotesize
\definecolor{shadecolor}{rgb}{0.878, 0.918, 0.933}\color{fgcolor}\begin{kframe}
\begin{alltt}
\hldef{captured} \hlkwb{<-} \hlkwd{rowSums}\hldef{(y.all)}\hlopt{>}\hlnum{0}
\hldef{y} \hlkwb{<-} \hldef{y.all[captured,,]}
\end{alltt}
\end{kframe}
\end{knitrout}
\pause
\vfill
  Capture histories for first 2 individuals and first 5 traps
  on first occasion.
\begin{knitrout}\footnotesize
\definecolor{shadecolor}{rgb}{0.878, 0.918, 0.933}\color{fgcolor}\begin{kframe}
\begin{alltt}
\hldef{y[}\hlnum{1}\hlopt{:}\hlnum{2}\hldef{,}\hlnum{1}\hlopt{:}\hlnum{5}\hldef{,}\hlnum{1}\hldef{]}
\end{alltt}
\begin{verbatim}
##      [,1] [,2] [,3] [,4] [,5]
## [1,]    0    0    0    0    0
## [2,]    0    0    0    0    0
\end{verbatim}
\end{kframe}
\end{knitrout}

\end{frame}







\begin{frame}[fragile]
  \frametitle{Summary stats}
  \small
  Individuals captured
\begin{knitrout}\footnotesize
\definecolor{shadecolor}{rgb}{0.878, 0.918, 0.933}\color{fgcolor}\begin{kframe}
\begin{alltt}
\hldef{(n} \hlkwb{<-} \hlkwd{nrow}\hldef{(y))}
\end{alltt}
\begin{verbatim}
## [1] 18
\end{verbatim}
\end{kframe}
\end{knitrout}
\pause \vfill
  Capture frequencies
  \vspace{-6pt}  
\begin{knitrout}\footnotesize
\definecolor{shadecolor}{rgb}{0.878, 0.918, 0.933}\color{fgcolor}\begin{kframe}
\begin{alltt}
\hldef{y.tilde} \hlkwb{<-} \hlkwd{rowSums}\hldef{(y)}
\hlkwd{table}\hldef{(y.tilde)}
\end{alltt}
\begin{verbatim}
## y.tilde
## 1 2 3 4 
## 9 7 1 1
\end{verbatim}
\end{kframe}
\end{knitrout}
\pause
\vfill
Spatial recaptures
  \vspace{-6pt}  
\begin{knitrout}\footnotesize
\definecolor{shadecolor}{rgb}{0.878, 0.918, 0.933}\color{fgcolor}\begin{kframe}
\begin{alltt}
\hldef{y.nok} \hlkwb{<-} \hlkwd{apply}\hldef{(y,} \hlkwd{c}\hldef{(}\hlnum{1}\hldef{,} \hlnum{2}\hldef{), sum)}
\hldef{y.nojk} \hlkwb{<-} \hlkwd{apply}\hldef{(y.nok}\hlopt{>}\hlnum{0}\hldef{,} \hlnum{1}\hldef{, sum)}
\hlkwd{table}\hldef{(y.nojk)}
\end{alltt}
\begin{verbatim}
## y.nojk
##  1  2  3 
## 13  4  1
\end{verbatim}
\end{kframe}
\end{knitrout}
\end{frame}




\begin{frame}[fragile]
  \frametitle{Spider plot}
\begin{knitrout}\scriptsize
\definecolor{shadecolor}{rgb}{0.878, 0.918, 0.933}\color{fgcolor}\begin{kframe}
\begin{alltt}
\hlkwd{plot}\hldef{(}\hlnum{0}\hldef{,} \hlkwc{xlim}\hldef{=}\hlkwd{c}\hldef{(}\hlnum{0}\hldef{,}\hlnum{1}\hldef{),} \hlkwc{ylim}\hldef{=}\hlkwd{c}\hldef{(}\hlnum{0}\hldef{,}\hlnum{1}\hldef{),} \hlkwc{asp}\hldef{=}\hlnum{1}\hldef{,} \hlkwc{xlab}\hldef{=}\hlsng{""}\hldef{,} \hlkwc{ylab}\hldef{=}\hlsng{""}\hldef{,}
     \hlkwc{main}\hldef{=}\hlsng{"Activity centers, traps, and capture locs"}\hldef{)}
\hldef{s.cap} \hlkwb{<-} \hldef{s[captured,]}
\hlkwa{for}\hldef{(i} \hlkwa{in} \hlnum{1}\hlopt{:}\hldef{n) \{}
    \hldef{traps.i} \hlkwb{<-} \hlkwd{which}\hldef{(}\hlkwd{rowSums}\hldef{(y[i,,])}\hlopt{>}\hlnum{0}\hldef{)}
    \hlkwa{for}\hldef{(j} \hlkwa{in} \hlnum{1}\hlopt{:}\hlkwd{length}\hldef{(traps.i)) \{}
        \hlkwd{segments}\hldef{(s.cap[i,}\hlnum{1}\hldef{], s.cap[i,}\hlnum{2}\hldef{],}
                 \hldef{x[traps.i[j],}\hlnum{1}\hldef{], x[traps.i[j],}\hlnum{2}\hldef{],} \hlkwc{col}\hldef{=}\hlkwd{gray}\hldef{(}\hlnum{0.3}\hldef{))}
    \hldef{\}}
\hldef{\}}
\hlkwd{points}\hldef{(s[captured,],} \hlkwc{pch}\hldef{=}\hlnum{16}\hldef{,} \hlkwc{col}\hldef{=}\hlsng{"blue"}\hldef{)} \hlcom{## Activity center locations}
\hlkwd{points}\hldef{(s[}\hlopt{!}\hldef{captured,],} \hlkwc{pch}\hldef{=}\hlnum{1}\hldef{,} \hlkwc{col}\hldef{=}\hlsng{"blue"}\hldef{)} \hlcom{## Activity center locations}
\hlkwd{points}\hldef{(x,} \hlkwc{pch}\hldef{=}\hlnum{3}\hldef{)}                         \hlcom{## Trap locations}
\end{alltt}
\end{kframe}
\end{knitrout}
\end{frame}


\begin{frame}
  \frametitle{Spider plot}
  \centering
  \includegraphics[width=0.85\textwidth]{figure/spider-1} \\
\end{frame}






% \begin{frame}
%   \frametitle{Joint likelihood}
%   % \footnotesize
%   \small
%   The joint likelihood looks similar to the nonspatial likelihood,
%   except that we have a third dimension for $y$ and we have to
%   integrate out the latent activity center $s_i$. 
%   \pause
%   \vfill
%   \flushleft
%   \begin{equation*}
% %  \begin{multline*}
% %    L(N,p; y,n) =                                          \\
%     L(N,p; y,n) = \left\{\prod_{i=1}^n \prod_{j=1}^J \prod_{k=1}^K p_{ij}^{y_{ijj}}(1-p_{ij})^{1-y_{ijk}}\right\}
% %    \left\{\frac{N!}{(N-n)!}  \left(\prod_{j=1}^J(1-p)\right)^{N-n} \right\}
%     \frac{N!}{(N-n)!}  \left(q^*\right)^{N-n}
% %  \end{multline*}
%   \end{equation*}
% \end{frame}



% \begin{frame}
%   \frametitle{Model variations}
%   Aside from the approach to estimation, the key consideration
%   concerns the sources of variation in capture probability ($p$). \\
%   \pause
%   \vfill
%   Otis et al. (1978, Wildlife Monographs) identified several model variations
%   \begin{itemize}
%     \small
%     \item $M_0$ -- $p$ is constant
%     \item $M_t$ -- unique $p$ for each capture occasion
%     \item \hangindent=0.8cm $M_b$ -- behavioral response with $p$ different than
%       recapture probability $c$
%     \item $M_h$ -- individual heterogeneity in $p$
%   \end{itemize}
%   \pause \vfill
%   These can be combined, but beware of identifiability issues. See
%   Otis et al. (1978) for details.  \\
%   \pause \vfill
%   Later we'll talk about another important class of models, the
%   ``individual covariate'' models.  
% \end{frame}






%\section{Model $M_0$}


%\subsection{Simulation}


%\subsection{Model $M_0$}






% \begin{frame}[fragile]
%   \frametitle{Summary stats}
%   Capture history frequencies
% <<M0-hist,size='scriptsize'>>=
% histories <- apply(y, 1, paste, collapse="")
% sort(table(histories))
% @
% \pause
% \vfill
%   Detection frequencies
% <<M0-freq,size='scriptsize'>>=
% y.tilde <- rowSums(y)
% sort(table(y.tilde))
% @   
% \end{frame}



%\subsection{Model $M_t$}


% \begin{frame}[fragile]
%   \frametitle{Model $M_t$ -- Temporal variation}
%   Capture probability for each occasion
% <<sim-Mt-pars,size='scriptsize'>>=
% p.t <- c(0.3, 0.5, 0.2, 0.4)
% @
%   \pause
%   \vfill
%   All capture histories (for captured and uncaptured individuals)
%   \vspace{-6pt}
% <<sim-Mt-ch,size='scriptsize'>>=
% y.all.Mt <- matrix(NA, N, J)
% for(i in 1:N) {
%     y.all.Mt[i,] <- rbinom(J, 1, p.t) }
% @
%   \pause
%   \vfill
%   Observed capture histories (data)
%   \vspace{-6pt}
% <<sim-Mt-y1,size='scriptsize'>>=
% captured.Mt <- rowSums(y.all.Mt)>0
% n.Mt <- sum(captured.Mt)
% y.Mt <- y.all.Mt[captured.Mt,]
% y.Mt[1:3,]
% colSums(y.Mt)
% @ 
% \end{frame}









\section{Likelihood}



\begin{frame}
  \frametitle{Outline}
  \Large
  \tableofcontents[currentsection]
\end{frame}




\begin{frame}
  \frametitle{Software options}
%  \small
  Program DENSITY
  \begin{itemize}
%  \footnotesize
    \item Windows program with GUI
  \end{itemize}
  \vfill
  R package `secr'
  \begin{itemize}
%  \footnotesize
    \item The oldest R package with the most options
  \end{itemize}
  \vfill
  R package `oSCR'
  \begin{itemize}
%  \footnotesize
    \item A newer R package with similar functionality
  \end{itemize}
\end{frame}



% %% p(y,n|N,p)
% \begin{frame}[fragile]
%   \frametitle{Joint likelihood for $M_0$}
%   The joint likelihood has a multinomial form:
%   \begin{multline*}
%     L(N,p; y,n) = \\
%     \left\{\prod_{i=1}^n \prod_{j=1}^J p^{y_{ij}}(1-p)^{1-y_{ij}}\right\}
%     \left\{\frac{N!}{(N-n)!}  \left(\prod_{j=1}^J(1-p)\right)^{N-n} \right\}
%   \end{multline*}
%   \pause
%   \vfill
% <<nll-M0,echo=TRUE,size='scriptsize'>>=
% nll.M0 <- function(pars, y) {           ## Negative log-likelihood
%     n <- nrow(y);       J <- ncol(y)
%     N <- exp(pars[1])
%     n0 <- N-n
%     if(n0<0) return(NA)
%     p <- plogis(pars[2])
%     ld.y1 <- sum(dbinom(y, 1, p, log=TRUE))
%     p0 <- (1-p)^J
%     ld.n0 <- lgamma(N+1)-lgamma(n0+1)+n0*log(p0)
%     nll <- -(ld.y1+ld.n0)
%     return(nll)
% }
% @
% \end{frame}



% \begin{frame}[fragile]
%   \frametitle{Maximize joint likelihood for $M_0$}
% Minimized the negative log-likelihood
% <<opt-nll-M0, size='scriptsize'>>=
% fm.M0 <- optim(c(log.N=4,logit.p=0), nll.M0, y=y, hessian=TRUE)
% fm.M0.est <- data.frame(Estimate=c(fm.M0$par[1], fm.M0$par[2]),
%                         SE=sqrt(diag(solve(fm.M0$hessian))))
% fm.M0.est
% @
% \pause
% \vfill
% Back-transform the estimates
% <<opt-nll-M0-back, size='scriptsize'>>=
% c(N.hat=exp(fm.M0$par[1]), p.hat=plogis(fm.M0$par[2]))
% @
% \pause
% \vfill
% Compare to data-generating values
% <<dg,size='scriptsize'>>=
% c(N=N, p=p)
% @ 
% \end{frame}





\begin{frame}[fragile]
  \frametitle{R package `secr'}
  Import data from two text files
\begin{knitrout}\tiny
\definecolor{shadecolor}{rgb}{0.878, 0.918, 0.933}\color{fgcolor}\begin{kframe}
\begin{alltt}
\hlkwd{library}\hldef{(secr)}
\hldef{sch} \hlkwb{<-} \hlkwd{read.capthist}\hldef{(}\hlkwc{captfile}\hldef{=}\hlsng{"encounter_data_file.csv"}\hldef{,}
                     \hlkwc{trapfile}\hldef{=}\hlsng{"trap_data_file.csv"}\hldef{,}
                     \hlkwc{detector}\hldef{=}\hlsng{"proximity"}\hldef{,} \hlkwc{fmt}\hldef{=}\hlsng{"trapID"}\hldef{)}
\end{alltt}


{\ttfamily\noindent\itshape\color{messagecolor}{\#\# No errors found :-)}}\begin{alltt}
\hlkwd{summary}\hldef{(sch)}
\end{alltt}
\begin{verbatim}
## Object class       capthist 
## Detector type      proximity 
## Detector number    64 
## Average spacing    100 m 
## x-range            150 850 m 
## y-range            150 850 m 
## 
## Counts by occasion 
##                    1  2  3  4  5 Total
## n                  9  4  4  6  6    29
## u                  9  1  2  3  3    18
## f                  9  7  2  0  0    18
## M(t+1)             9 10 12 15 18    18
## losses             0  0  0  0  0     0
## detections         9  5  4  6  6    30
## detectors visited  9  5  4  6  6    30
## detectors used    64 64 64 64 64   320
\end{verbatim}
\end{kframe}
\end{knitrout}
\end{frame}



\begin{frame}[fragile]
  \frametitle{R package `secr'}
\begin{knitrout}
\definecolor{shadecolor}{rgb}{0.878, 0.918, 0.933}\color{fgcolor}\begin{kframe}
\begin{alltt}
\hlkwd{plot}\hldef{(sch)}
\end{alltt}
\end{kframe}

{\centering \includegraphics[width=0.7\linewidth]{figure/secr-plot-1} 

}


\end{knitrout}
\end{frame}




\begin{frame}[fragile]
  \frametitle{R package `secr'}
  Fit the SCR equivalent of Model $M_0$
\begin{knitrout}\scriptsize
\definecolor{shadecolor}{rgb}{0.878, 0.918, 0.933}\color{fgcolor}\begin{kframe}
\begin{alltt}
\hldef{fm.M0} \hlkwb{<-} \hlkwd{secr.fit}\hldef{(sch,} \hlkwc{model}\hldef{=}\hlkwd{list}\hldef{(}\hlkwc{D}\hldef{=}\hlopt{~}\hlnum{1}\hldef{,} \hlkwc{g0}\hldef{=}\hlopt{~}\hlnum{1}\hldef{,} \hlkwc{sigma}\hldef{=}\hlopt{~}\hlnum{1}\hldef{),}
    \hlkwc{details} \hldef{=} \hlkwd{list}\hldef{(}\hlkwc{fastproximity} \hldef{=} \hlnum{FALSE}\hldef{),} \hlcom{## For AIC comparisons}
    \hlkwc{buffer}\hldef{=}\hlnum{150}\hldef{,} \hlkwc{trace}\hldef{=}\hlnum{FALSE}\hldef{,} \hlkwc{ncores}\hldef{=}\hlnum{3}\hldef{)}
\hlkwd{coef}\hldef{(fm.M0)}
\end{alltt}
\begin{verbatim}
##            beta   SE.beta       lcl        ucl
## D     -1.050968 0.2738747 -1.587753 -0.5141834
## g0    -1.276753 0.4489717 -2.156721 -0.3967843
## sigma  3.789683 0.1342465  3.526565  4.0528010
\end{verbatim}
\end{kframe}
\end{knitrout}
\pause
\vfill
Estimates on original scale
\begin{knitrout}\scriptsize
\definecolor{shadecolor}{rgb}{0.878, 0.918, 0.933}\color{fgcolor}\begin{kframe}
\begin{alltt}
\hlkwd{predict}\hldef{(fm.M0)}
\end{alltt}
\begin{verbatim}
##        link   estimate SE.estimate        lcl        ucl
## D       log  0.3495992  0.09757018  0.2043844  0.5979887
## g0    logit  0.2181035  0.07656510  0.1037048  0.4020852
## sigma   log 44.2423619  5.96624128 34.0069365 57.5584511
\end{verbatim}
\end{kframe}
\end{knitrout}
\inr{D} is density ($\lambda$, animals/ha), \inr{g0} is $g_0$, and \inr{sigma} is 
$\sigma$.   
\end{frame}



\begin{frame}[fragile]
  \frametitle{R package `secr'}
  Fit the SCR equivalent of Model $M_t$
\begin{knitrout}\scriptsize
\definecolor{shadecolor}{rgb}{0.878, 0.918, 0.933}\color{fgcolor}\begin{kframe}
\begin{alltt}
\hldef{fm.Mt} \hlkwb{<-} \hlkwd{secr.fit}\hldef{(sch,} \hlkwc{model}\hldef{=}\hlkwd{list}\hldef{(}\hlkwc{D}\hldef{=}\hlopt{~}\hlnum{1}\hldef{,} \hlkwc{g0}\hldef{=}\hlopt{~}\hldef{t,} \hlkwc{sigma}\hldef{=}\hlopt{~}\hlnum{1}\hldef{),}
                  \hlkwc{buffer}\hldef{=}\hlnum{150}\hldef{,} \hlkwc{trace}\hldef{=}\hlnum{FALSE}\hldef{,} \hlkwc{ncores}\hldef{=}\hlnum{3}\hldef{)}
\hlkwd{coef}\hldef{(fm.Mt)}
\end{alltt}
\begin{verbatim}
##             beta   SE.beta       lcl        ucl
## D     -1.0603936 0.2725340 -1.594550 -0.5262367
## g0    -0.6434653 0.6623739 -1.941694  0.6547637
## g0.t2 -0.8125256 0.7056696 -2.195613  0.5705614
## g0.t3 -1.1182751 0.7397397 -2.568138  0.3315881
## g0.t4 -0.6023297 0.6829484 -1.940884  0.7362246
## g0.t5 -0.6045682 0.6819724 -1.941210  0.7320733
## sigma  3.7799754 0.1330984  3.519107  4.0408435
\end{verbatim}
\end{kframe}
\end{knitrout}
\pause
\vfill
Estimates on original scale
\begin{knitrout}\scriptsize
\definecolor{shadecolor}{rgb}{0.878, 0.918, 0.933}\color{fgcolor}\begin{kframe}
\begin{alltt}
\hlkwd{predict}\hldef{(fm.Mt)}
\end{alltt}
\begin{verbatim}
##        link   estimate SE.estimate        lcl        ucl
## D       log  0.3463195  0.09616385  0.2029998  0.5908242
## g0    logit  0.3444636  0.14956962  0.1254618  0.6580832
## sigma   log 43.8149643  5.85762566 33.7542824 56.8742975
\end{verbatim}
\end{kframe}
\end{knitrout}
\end{frame}



\begin{frame}[fragile]
  \frametitle{R package `secr'}
  Fit the SCR equivalent of Model $M_b$
\begin{knitrout}\scriptsize
\definecolor{shadecolor}{rgb}{0.878, 0.918, 0.933}\color{fgcolor}\begin{kframe}
\begin{alltt}
\hldef{fm.Mb} \hlkwb{<-} \hlkwd{secr.fit}\hldef{(sch,} \hlkwc{model}\hldef{=}\hlkwd{list}\hldef{(}\hlkwc{D}\hldef{=}\hlopt{~}\hlnum{1}\hldef{,} \hlkwc{g0}\hldef{=}\hlopt{~}\hldef{b,} \hlkwc{sigma}\hldef{=}\hlopt{~}\hlnum{1}\hldef{),}
                  \hlkwc{buffer}\hldef{=}\hlnum{150}\hldef{,} \hlkwc{trace}\hldef{=}\hlnum{FALSE}\hldef{,} \hlkwc{ncores}\hldef{=}\hlnum{3}\hldef{)}
\hlkwd{coef}\hldef{(fm.Mb)}
\end{alltt}
\begin{verbatim}
##                beta   SE.beta       lcl       ucl
## D        -0.9355618 0.5587887 -2.030768 0.1596439
## g0       -1.5401416 1.0918175 -3.680065 0.5997815
## g0.bTRUE  0.2900354 1.0711536 -1.809387 2.3894579
## sigma     3.7945281 0.1361532  3.527673 4.0613835
\end{verbatim}
\end{kframe}
\end{knitrout}
\pause
\vfill
Estimates on original scale
\begin{knitrout}\scriptsize
\definecolor{shadecolor}{rgb}{0.878, 0.918, 0.933}\color{fgcolor}\begin{kframe}
\begin{alltt}
\hlkwd{predict}\hldef{(fm.Mb)}
\end{alltt}
\begin{verbatim}
##        link   estimate SE.estimate         lcl        ucl
## D       log  0.3923654   0.2375315  0.13123476  1.1730930
## g0    logit  0.1765147   0.1587036  0.02460088  0.6456063
## sigma   log 44.4572512   6.0811580 34.04464338 58.0545716
\end{verbatim}
\end{kframe}
\end{knitrout}
\end{frame}


\begin{frame}[fragile]
  \frametitle{R package `secr'}
  What about $N$? \pause \inr{E.N} is the expected value of
  $N$. \inr{R.N} is the realized value of $N$. 
  \vfill
\begin{knitrout}\footnotesize
\definecolor{shadecolor}{rgb}{0.878, 0.918, 0.933}\color{fgcolor}\begin{kframe}
\begin{alltt}
\hlkwd{region.N}\hldef{(fm.M0)}
\end{alltt}
\begin{verbatim}
##     estimate SE.estimate      lcl      ucl  n
## E.N 34.24296    9.556923 20.01929 58.57253 18
## R.N 34.24297    7.555912 24.82226 56.67252 18
\end{verbatim}
\end{kframe}
\end{knitrout}
  \pause
\begin{knitrout}\footnotesize
\definecolor{shadecolor}{rgb}{0.878, 0.918, 0.933}\color{fgcolor}\begin{kframe}
\begin{alltt}
\hlkwd{region.N}\hldef{(fm.Mt)}
\end{alltt}
\begin{verbatim}
##     estimate SE.estimate      lcl      ucl  n
## E.N 33.92172    9.419174 19.88367 57.87077 18
## R.N 33.92173    7.402642 24.69007 55.89223 18
\end{verbatim}
\end{kframe}
\end{knitrout}
  \pause
\begin{knitrout}\footnotesize
\definecolor{shadecolor}{rgb}{0.878, 0.918, 0.933}\color{fgcolor}\begin{kframe}
\begin{alltt}
\hlkwd{region.N}\hldef{(fm.Mb)}
\end{alltt}
\begin{verbatim}
##     estimate SE.estimate      lcl      ucl  n
## E.N 38.43188    23.26603 12.85434 114.9035 18
## R.N 38.43189    22.42490 21.57669 134.7173 18
\end{verbatim}
\end{kframe}
\end{knitrout}
%\pause
%\vfill
\end{frame}


\begin{frame}[fragile]
  \frametitle{R package `secr'}
  \small
  Was \inr{buffer} big enough?
\begin{knitrout}\scriptsize
\definecolor{shadecolor}{rgb}{0.878, 0.918, 0.933}\color{fgcolor}\begin{kframe}
\begin{alltt}
\hlkwd{predict}\hldef{(}\hlkwd{update}\hldef{(fm.M0,} \hlkwc{buffer}\hldef{=}\hlnum{100}\hldef{))[}\hlnum{1}\hldef{,]}
\end{alltt}
\begin{verbatim}
##   link  estimate SE.estimate       lcl       ucl
## D  log 0.3554085  0.09872472 0.2082871 0.6064475
\end{verbatim}
\end{kframe}
\end{knitrout}
\pause
\vspace{-12pt}
\begin{knitrout}\scriptsize
\definecolor{shadecolor}{rgb}{0.878, 0.918, 0.933}\color{fgcolor}\begin{kframe}
\begin{alltt}
\hlkwd{predict}\hldef{(}\hlkwd{update}\hldef{(fm.M0,} \hlkwc{buffer}\hldef{=}\hlnum{150}\hldef{))[}\hlnum{1}\hldef{,]}
\end{alltt}
\begin{verbatim}
##   link estimate SE.estimate       lcl       ucl
## D  log 0.349599  0.09757013 0.2043843 0.5979884
\end{verbatim}
\end{kframe}
\end{knitrout}
\pause
\vspace{-12pt}
\begin{knitrout}\scriptsize
\definecolor{shadecolor}{rgb}{0.878, 0.918, 0.933}\color{fgcolor}\begin{kframe}
\begin{alltt}
\hlkwd{predict}\hldef{(}\hlkwd{update}\hldef{(fm.M0,} \hlkwc{buffer}\hldef{=}\hlnum{200}\hldef{))[}\hlnum{1}\hldef{,]}
\end{alltt}
\begin{verbatim}
##   link  estimate SE.estimate      lcl       ucl
## D  log 0.3494254  0.09755105 0.204251 0.5977846
\end{verbatim}
\end{kframe}
\end{knitrout}
\pause
\vspace{-12pt}
\begin{knitrout}\scriptsize
\definecolor{shadecolor}{rgb}{0.878, 0.918, 0.933}\color{fgcolor}\begin{kframe}
\begin{alltt}
\hlkwd{predict}\hldef{(}\hlkwd{update}\hldef{(fm.M0,} \hlkwc{buffer}\hldef{=}\hlnum{250}\hldef{))[}\hlnum{1}\hldef{,]}
\end{alltt}
\begin{verbatim}
##   link  estimate SE.estimate       lcl       ucl
## D  log 0.3494237  0.09755102 0.2042495 0.5977831
\end{verbatim}
\end{kframe}
\end{knitrout}
\pause
\vspace{-12pt}
\begin{knitrout}\scriptsize
\definecolor{shadecolor}{rgb}{0.878, 0.918, 0.933}\color{fgcolor}\begin{kframe}
\begin{alltt}
\hlkwd{predict}\hldef{(}\hlkwd{update}\hldef{(fm.M0,} \hlkwc{buffer}\hldef{=}\hlnum{300}\hldef{))[}\hlnum{1}\hldef{,]}
\end{alltt}
\begin{verbatim}
##   link  estimate SE.estimate       lcl       ucl
## D  log 0.3494236    0.097551 0.2042495 0.5977829
\end{verbatim}
\end{kframe}
\end{knitrout}
\end{frame}



\begin{frame}[fragile]
  \frametitle{R package `secr'}
  AIC
\begin{knitrout}\tiny
\definecolor{shadecolor}{rgb}{0.878, 0.918, 0.933}\color{fgcolor}\begin{kframe}
\begin{alltt}
\hlkwd{AIC}\hldef{(fm.M0, fm.Mt, fm.Mb)}
\end{alltt}
\begin{verbatim}
##                  model   detectfn npar    logLik     AIC    AICc  dAIC  AICwt
## fm.M0 D~1 g0~1 sigma~1 halfnormal    3 -117.3269 240.654 242.368 0.000 0.6867
## fm.Mb D~1 g0~b sigma~1 halfnormal    4 -117.2848 242.570 245.646 1.916 0.2635
## fm.Mt D~1 g0~t sigma~1 halfnormal    7 -115.9497 245.899 257.099 5.245 0.0499
\end{verbatim}
\end{kframe}
\end{knitrout}
\end{frame}


\section{Data augmentation}


%\section{Prediction}
%\subsection{Likelihood-based inference}


\begin{frame}
  \frametitle{Outline}
  \Large
  \tableofcontents[currentsection]
\end{frame}





\begin{frame}
  \frametitle{Data augmentation model}
  The DA version of a basic SCR model is:
  \begin{gather*}
    \bsi \sim \mathrm{Unif}(\mathcal{S}) \\
    z_i \sim \mathrm{Bern}(\psi) \\
    %% p_{ij} = g_0\exp(-\|\bsi-\bxj\|^2/(2\sigma^2)) \\
    p_{ij} = g_0\exp\left(\frac{-\|\bsi-\bxj\|^2}{2\sigma^2}\right) \\
    y_{ijk} \sim \mathrm{Bern}(z_i \times p_{ij}) \\
    N=\sum_{i=1}^M z_i
  \end{gather*}
  % A uniform prior on $\psi$ results in a discrete uniform prior on
  % $N$. We can change the prior for $N$ by changing the prior on
  % $\psi$, recognizing that $E(N)=M\psi$.
  % But why bother with augmentation?
  % \begin{itemize}
  %   \item DA works for \alert{all} varieties of mark-recapture models
  %   \item Make it easy to incorporate
  %     individual-covariates\dots\pause including distance and
  %     location!   
  % \end{itemize}
\end{frame}






\begin{frame}[fragile]
  \frametitle{Model SCR$_0$ -- data augmentation}
\vspace{-3pt}
\begin{knitrout}\scriptsize
\definecolor{shadecolor}{rgb}{0.878, 0.918, 0.933}\color{fgcolor}\begin{kframe}
\begin{alltt}
\hlkwd{writeLines}\hldef{(}\hlkwd{readLines}\hldef{(}\hlsng{"SCR0.jag"}\hldef{))}
\end{alltt}
\end{kframe}
\end{knitrout}
\begin{knitrout}\scriptsize
\definecolor{shadecolor}{rgb}{0.961, 0.961, 0.863}\color{fgcolor}\begin{kframe}
\begin{verbatim}
model {
g0 ~ dunif(0, 1)       ## Baseline capture probability
sigma ~ dunif(0, 0.5)  ## Scale parameter of encounter function
psi ~ dunif(0, 1)      ## Data augmentation parameter

for(i in 1:M) {
  s[i,1] ~ dunif(xlim[1], xlim[2]) ## x-coord of activity center
  s[i,2] ~ dunif(ylim[1], ylim[2]) ## y-coord of activity center
  z[i] ~ dbern(psi)                ## Is individual real?
  for(j in 1:J) {
    dist[i,j] <- sqrt((s[i,1]-x[j,1])^2 + (s[i,2]-x[j,2])^2)
    p[i,j] <- g0*exp(-dist[i,j]^2/(2*sigma^2))
    for(k in 1:K) {
      y[i,j,k] ~ dbern(z[i]*p[i,j]) ## Model for the capture histories
    }
  }
}

EN <- M*psi  ## Expected value of N
N <- sum(z)  ## Realized value of N
}
\end{verbatim}
\end{kframe}
\end{knitrout}
\end{frame}





\begin{frame}[fragile]
  \frametitle{Model SCR$_0$ -- data augmentation}
  Data
  \vspace{-6pt}
\begin{knitrout}\scriptsize
\definecolor{shadecolor}{rgb}{0.878, 0.918, 0.933}\color{fgcolor}\begin{kframe}
\begin{alltt}
\hldef{M} \hlkwb{<-} \hlnum{150}
\hldef{y.aug} \hlkwb{<-} \hlkwd{array}\hldef{(}\hlnum{0}\hldef{,} \hlkwd{c}\hldef{(M, J, K))}
\hldef{y.aug[}\hlnum{1}\hlopt{:}\hlkwd{nrow}\hldef{(y),,]} \hlkwb{<-} \hldef{y}
\hldef{jags.data.SCR0} \hlkwb{<-} \hlkwd{list}\hldef{(}\hlkwc{y}\hldef{=y.aug,} \hlkwc{M}\hldef{=M,} \hlkwc{J}\hldef{=J,} \hlkwc{K}\hldef{=K,}
                       \hlkwc{x}\hldef{=x,} \hlkwc{xlim}\hldef{=}\hlkwd{c}\hldef{(}\hlnum{0}\hldef{,}\hlnum{1}\hldef{),} \hlkwc{ylim}\hldef{=}\hlkwd{c}\hldef{(}\hlnum{0}\hldef{,}\hlnum{1}\hldef{))}
\end{alltt}
\end{kframe}
\end{knitrout}
\pause
\vfill
  Inits and parameters
  \vspace{-6pt}
\begin{knitrout}\scriptsize
\definecolor{shadecolor}{rgb}{0.878, 0.918, 0.933}\color{fgcolor}\begin{kframe}
\begin{alltt}
\hldef{ji.SCR0} \hlkwb{<-} \hlkwa{function}\hldef{() \{}
    \hlkwd{list}\hldef{(}\hlkwc{z}\hldef{=}\hlkwd{rep}\hldef{(}\hlnum{1}\hldef{,M),} \hlkwc{psi}\hldef{=}\hlkwd{runif}\hldef{(}\hlnum{1}\hldef{),}
         \hlkwc{s}\hldef{=}\hlkwd{cbind}\hldef{(}\hlkwd{runif}\hldef{(M),} \hlkwd{runif}\hldef{(M)),}
         \hlkwc{g0}\hldef{=}\hlkwd{runif}\hldef{(}\hlnum{1}\hldef{),} \hlkwc{sigma}\hldef{=}\hlkwd{runif}\hldef{(}\hlnum{1}\hldef{,} \hlnum{0.05}\hldef{,} \hlnum{0.1}\hldef{)) \}}
\hldef{jp.SCR0} \hlkwb{<-} \hlkwd{c}\hldef{(}\hlsng{"g0"}\hldef{,} \hlsng{"sigma"}\hldef{,} \hlsng{"EN"}\hldef{,} \hlsng{"N"}\hldef{)}
\hlkwd{library}\hldef{(jagsUI)}
\end{alltt}
\end{kframe}
\end{knitrout}
\pause
\vfill
MCMC
  \vspace{-6pt}
\begin{knitrout}\scriptsize
\definecolor{shadecolor}{rgb}{0.878, 0.918, 0.933}\color{fgcolor}\begin{kframe}
\begin{alltt}
\hldef{jags.post.SCR0} \hlkwb{<-} \hlkwd{jags.basic}\hldef{(}\hlkwc{data}\hldef{=jags.data.SCR0,} \hlkwc{inits}\hldef{=ji.SCR0,}
                             \hlkwc{parameters.to.save}\hldef{=jp.SCR0,}
                             \hlkwc{model.file}\hldef{=}\hlsng{"SCR0.jag"}\hldef{,}
                             \hlkwc{n.chains}\hldef{=}\hlnum{3}\hldef{,} \hlkwc{n.adapt}\hldef{=}\hlnum{100}\hldef{,} \hlkwc{n.burnin}\hldef{=}\hlnum{0}\hldef{,}
                             \hlkwc{n.iter}\hldef{=}\hlnum{2000}\hldef{,} \hlkwc{parallel}\hldef{=}\hlnum{TRUE}\hldef{)}
\end{alltt}
\end{kframe}
\end{knitrout}
\end{frame}




\begin{frame}[fragile]
  \frametitle{Posterior summaries}
\begin{knitrout}\tiny
\definecolor{shadecolor}{rgb}{0.878, 0.918, 0.933}\color{fgcolor}\begin{kframe}
\begin{alltt}
\hlkwd{summary}\hldef{(jags.post.SCR0)}
\end{alltt}
\begin{verbatim}
## 
## Iterations = 101:2100
## Thinning interval = 1 
## Number of chains = 3 
## Sample size per chain = 2000 
## 
## 1. Empirical mean and standard deviation for each variable,
##    plus standard error of the mean:
## 
##               Mean        SD  Naive SE Time-series SE
## EN        38.34185 10.509583 1.357e-01      0.4260859
## N         37.79533  9.281117 1.198e-01      0.4100381
## deviance 212.78601 17.273841 2.230e-01      0.7465667
## g0         0.22329  0.079472 1.026e-03      0.0036405
## sigma      0.04474  0.006116 7.895e-05      0.0002632
## 
## 2. Quantiles for each variable:
## 
##               2.5%       25%      50%       75%     97.5%
## EN        22.32660  31.02414  36.8426  44.06243  62.78484
## N         24.97500  31.00000  36.0000  42.00000  60.00000
## deviance 182.60146 200.56035 211.5269 223.63981 250.24868
## g0         0.09674   0.16522   0.2130   0.27192   0.40764
## sigma      0.03455   0.04061   0.0441   0.04813   0.05929
\end{verbatim}
\end{kframe}
\end{knitrout}
\end{frame}



\begin{frame}[fragile]
  \frametitle{Traceplots and density plots}
\begin{knitrout}\footnotesize
\definecolor{shadecolor}{rgb}{0.878, 0.918, 0.933}\color{fgcolor}\begin{kframe}
\begin{alltt}
\hlkwd{plot}\hldef{(jags.post.SCR0[,jp.SCR0])}
\end{alltt}
\end{kframe}

{\centering \includegraphics[width=0.7\textwidth]{figure/plot-mcmc-SCR0-1} 

}


\end{knitrout}
\end{frame}


\begin{frame}
  \frametitle{Easy tricks to speed up Bayesian inference}
  \begin{itemize}
    % \item Use joint likelihood $p(y|n)p(0|n,N)p(N)$ approach as shown in
    %   non-spatial lecture.
    %   \begin{itemize}
    %     \item This can work well, but not when there are other
    %       individual-level covariates.
    %   \end{itemize}
    \item Treat $z_i=1$ as data for first $n$ individuals.
    \item If there are no occasion-specific covariates, collapse data
      and use binomial instead of Bernoulli distribution.
    \item Use a single zero for each augmented individual, instead of
      an array of zeros. Then compute probability of detecting an
      individual at least once.
  \end{itemize}
\end{frame}




\begin{frame}[fragile]
  \frametitle{Model SCR$_0$ -- data augmentation}
\vspace{-3pt}
\begin{knitrout}\tiny
\definecolor{shadecolor}{rgb}{0.878, 0.918, 0.933}\color{fgcolor}\begin{kframe}
\begin{alltt}
\hlkwd{writeLines}\hldef{(}\hlkwd{readLines}\hldef{(}\hlsng{"SCR0-faster.jag"}\hldef{))}
\end{alltt}
\end{kframe}
\end{knitrout}
\begin{knitrout}\tiny
\definecolor{shadecolor}{rgb}{0.961, 0.961, 0.863}\color{fgcolor}\begin{kframe}
\begin{verbatim}
model {
g0 ~ dunif(0, 1)       ## Baseline capture probability
sigma ~ dunif(0, 0.5)  ## Scale parameter of encounter function
psi ~ dunif(0, 1)      ## Data augmentation parameter

for(i in 1:M) {
  s[i,1] ~ dunif(xlim[1], xlim[2])
  s[i,2] ~ dunif(ylim[1], ylim[2])
  z[i] ~ dbern(psi)
  for(j in 1:J) {
    dist[i,j] <- sqrt((s[i,1]-x[j,1])^2 + (s[i,2]-x[j,2])^2)
    p[i,j] <- g0*exp(-dist[i,j]^2/(2*sigma^2))
  }
}
for(i in 1:n) {  ## Model for observed capture histories
  for(j in 1:J) {
    y.tilde[i,j] ~ dbinom(p[i,j], K)
  }
}
for(i in (n+1):M) { ## Model for augmented guys
  PrAtLeastOneCap[i] <- 1-prod(1-p[i,])^K
  zero[i] ~ dbern(PrAtLeastOneCap[i]*z[i])
}

EN <- M*psi  ## Expected value of N
N <- sum(z)  ## Realized value of N
}
\end{verbatim}
\end{kframe}
\end{knitrout}
\end{frame}




\begin{frame}[fragile]
  \frametitle{Model SCR$_0$ -- faster}
  Data
  \vspace{-6pt}
\begin{knitrout}\scriptsize
\definecolor{shadecolor}{rgb}{0.878, 0.918, 0.933}\color{fgcolor}\begin{kframe}
\begin{alltt}
\hldef{y.tilde} \hlkwb{<-} \hlkwd{apply}\hldef{(y,} \hlkwd{c}\hldef{(}\hlnum{1}\hldef{,}\hlnum{2}\hldef{), sum)}
\hldef{n} \hlkwb{<-} \hlkwd{nrow}\hldef{(y)}
\hldef{jags.data.SCR0.faster} \hlkwb{<-} \hlkwd{list}\hldef{(}\hlkwc{y.tilde}\hldef{=y.tilde,} \hlkwc{n}\hldef{=n,} \hlkwc{M}\hldef{=M,} \hlkwc{J}\hldef{=J,}
                              \hlkwc{z}\hldef{=}\hlkwd{c}\hldef{(}\hlkwd{rep}\hldef{(}\hlnum{1}\hldef{, n),} \hlkwd{rep}\hldef{(}\hlnum{NA}\hldef{, M}\hlopt{-}\hldef{n)),}
                              \hlkwc{K}\hldef{=K,} \hlkwc{zero}\hldef{=}\hlkwd{rep}\hldef{(}\hlnum{0}\hldef{, M),} \hlkwc{x}\hldef{=x,}
                              \hlkwc{xlim}\hldef{=}\hlkwd{c}\hldef{(}\hlnum{0}\hldef{,}\hlnum{1}\hldef{),} \hlkwc{ylim}\hldef{=}\hlkwd{c}\hldef{(}\hlnum{0}\hldef{,}\hlnum{1}\hldef{))}
\end{alltt}
\end{kframe}
\end{knitrout}
\pause
\vfill
  Inits and parameters (same as before)
\pause
\vfill
\begin{knitrout}\scriptsize
\definecolor{shadecolor}{rgb}{0.878, 0.918, 0.933}\color{fgcolor}\begin{kframe}
\begin{alltt}
\hldef{ji.SCR0.faster} \hlkwb{<-} \hlkwa{function}\hldef{() \{}
    \hlkwd{list}\hldef{(}\hlkwc{z}\hldef{=}\hlkwd{c}\hldef{(}\hlkwd{rep}\hldef{(}\hlnum{NA}\hldef{, n),} \hlkwd{rep}\hldef{(}\hlnum{0}\hldef{,M}\hlopt{-}\hldef{n)),} \hlkwc{psi}\hldef{=}\hlkwd{runif}\hldef{(}\hlnum{1}\hldef{),}
         \hlkwc{s}\hldef{=}\hlkwd{cbind}\hldef{(}\hlkwd{runif}\hldef{(M),} \hlkwd{runif}\hldef{(M)),}
         \hlkwc{g0}\hldef{=}\hlkwd{runif}\hldef{(}\hlnum{1}\hldef{),} \hlkwc{sigma}\hldef{=}\hlkwd{runif}\hldef{(}\hlnum{1}\hldef{,} \hlnum{0.05}\hldef{,} \hlnum{0.1}\hldef{)) \}}
\end{alltt}
\end{kframe}
\end{knitrout}
MCMC
  \vspace{-6pt}
\begin{knitrout}\scriptsize
\definecolor{shadecolor}{rgb}{0.878, 0.918, 0.933}\color{fgcolor}\begin{kframe}
\begin{alltt}
\hldef{jags.post.SCR0.faster} \hlkwb{<-} \hlkwd{jags.basic}\hldef{(}\hlkwc{data}\hldef{=jags.data.SCR0.faster,}
                                    \hlkwc{inits}\hldef{=ji.SCR0.faster,}
                                    \hlkwc{parameters.to.save}\hldef{=jp.SCR0,}
                                    \hlkwc{model.file}\hldef{=}\hlsng{"SCR0-faster.jag"}\hldef{,}
                                    \hlkwc{n.chains}\hldef{=}\hlnum{3}\hldef{,} \hlkwc{n.adapt}\hldef{=}\hlnum{100}\hldef{,} \hlkwc{n.burnin}\hldef{=}\hlnum{0}\hldef{,}
                                    \hlkwc{n.iter}\hldef{=}\hlnum{2000}\hldef{,} \hlkwc{parallel}\hldef{=}\hlnum{TRUE}\hldef{)}
\end{alltt}
\end{kframe}
\end{knitrout}
\end{frame}




\begin{frame}[fragile]
  \frametitle{Posterior summaries}
\begin{knitrout}\tiny
\definecolor{shadecolor}{rgb}{0.878, 0.918, 0.933}\color{fgcolor}\begin{kframe}
\begin{alltt}
\hlkwd{summary}\hldef{(jags.post.SCR0.faster)}
\end{alltt}
\begin{verbatim}
## 
## Iterations = 101:2100
## Thinning interval = 1 
## Number of chains = 3 
## Sample size per chain = 2000 
## 
## 1. Empirical mean and standard deviation for each variable,
##    plus standard error of the mean:
## 
##               Mean        SD  Naive SE Time-series SE
## EN        37.91885 10.538384 1.360e-01      0.4262260
## N         37.42983  9.184596 1.186e-01      0.3924181
## deviance 177.85757 15.153944 1.956e-01      0.5693382
## g0         0.22650  0.081995 1.059e-03      0.0038030
## sigma      0.04485  0.006296 8.128e-05      0.0003032
## 
## 2. Quantiles for each variable:
## 
##               2.5%       25%       50%       75%     97.5%
## EN        21.42326  30.30816  36.64816  43.85615  62.06503
## N         24.00000  31.00000  36.00000  42.00000  59.00000
## deviance 150.96169 167.13560 176.77262 187.73161 210.38517
## g0         0.09792   0.16590   0.21589   0.27430   0.41153
## sigma      0.03459   0.04038   0.04422   0.04853   0.05935
\end{verbatim}
\end{kframe}
\end{knitrout}
\end{frame}



\begin{frame}[fragile]
  \frametitle{Traceplots and density plots}
\begin{knitrout}\footnotesize
\definecolor{shadecolor}{rgb}{0.878, 0.918, 0.933}\color{fgcolor}\begin{kframe}
\begin{alltt}
\hlkwd{plot}\hldef{(jags.post.SCR0.faster[,jp.SCR0])}
\end{alltt}
\end{kframe}

{\centering \includegraphics[width=0.7\textwidth]{figure/plot-mcmc-SCR0-faster-1} 

}


\end{knitrout}
\end{frame}


\section{Summary}



\begin{frame}
  \frametitle{Alternative observation models}
%  \begin{itemize}
%  \item
  % Observation models are not the same as encounter rate functions. \\
  % \pause \vfill
  % % \item
  Observation models are chosen with respect to the sampling
      method (mist-net, camera trap, hair snare, etc\dots) \\
%  \end{itemize}
  \pause \vfill
  % \begin{center}
  Common options include:
  \centering
    \begin{tabular}{lll}
      \hline
      Model       & Detector/Trap     & Examples           \\
      \hline
      Bernoulli   & Proximity    & Hair-snares        \\
      Poisson     & Count        & Camera trap        \\
      Multinomial/Categorical & Multi-catch  & Mist net, crab pot \\
      ---          & Single-catch & Sherman trap       \\
      \hline
    \end{tabular}
    % \end{center}
  \pause \vfill
  There are other observation models for data from area searches,
  transects, or acoustic recording units.
\end{frame}








\begin{frame}
  \frametitle{SCR study design}
  SCR uses model-based, rather than design-based,
  inference (see Ch. 10). \\ 
  \pause
  \vfill
  Random placement of traps is not required, but it's a good idea to
  randomly sample locations along the environmental gradients you're
  interested in. \\
  \pause \vfill
  The other two key design considerations are:
  \begin{enumerate}
    \item<3-> Capture as many individuals as you can (i.e., maximize $n$)
    \item<4-> Obtain as many {\it spatial} recaptures as possible
  \end{enumerate}
  \vfill
  \uncover<5->{
    There is a tradeoff between these two objectives. Simulation is
    often the best option for finding the right balance.
  }
\end{frame}

%\section{Summary}


\begin{frame}
  \frametitle{SCR summary}
  We assume that variation in $p$ arises from distance between animals
  and traps. \\
  \pause \vfill
  We can estimate abundance and model distribution (i.e., spatial
  variation in density) \\
  \pause \vfill
  Next time, we'll see how to do that using secr and JAGS, and we'll
  make a bunch of maps. \\
\end{frame}




%\section{Assignment}




\begin{frame}[fragile]
  \frametitle{Assignment}
  Create a self-contained R script or Rmarkdown file to do the
  following: 
  \vfill
  \begin{enumerate}
    \item Import the bear data formatted for `secr' and fit a model
      where $g_0$ varies among the 8 occasions. Use `secr' to
      determine what an appropriate buffer should be.
    \item Import the bear data formatted for JAGS and tabulate the
      capture frequencies and the number of spatial recapture. Fit the
      same model in JAGS that you fit in secr. Compare estimates and 95\% CIs
      for $g_{0,t}$, $\sigma$, $E(N)$ and $N$.  
  \end{enumerate}
  \vfill
  Upload your work to ELC by 5:00pm on Tuesday. 
\end{frame}





\end{document}

