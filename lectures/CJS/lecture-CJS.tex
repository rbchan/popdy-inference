\documentclass[color=usenames,dvipsnames]{beamer}\usepackage[]{graphicx}\usepackage[]{xcolor}
% maxwidth is the original width if it is less than linewidth
% otherwise use linewidth (to make sure the graphics do not exceed the margin)
\makeatletter
\def\maxwidth{ %
  \ifdim\Gin@nat@width>\linewidth
    \linewidth
  \else
    \Gin@nat@width
  \fi
}
\makeatother

\definecolor{fgcolor}{rgb}{0, 0, 0}
\newcommand{\hlnum}[1]{\textcolor[rgb]{0.69,0.494,0}{#1}}%
\newcommand{\hlsng}[1]{\textcolor[rgb]{0.749,0.012,0.012}{#1}}%
\newcommand{\hlcom}[1]{\textcolor[rgb]{0.514,0.506,0.514}{\textit{#1}}}%
\newcommand{\hlopt}[1]{\textcolor[rgb]{0,0,0}{#1}}%
\newcommand{\hldef}[1]{\textcolor[rgb]{0,0,0}{#1}}%
\newcommand{\hlkwa}[1]{\textcolor[rgb]{0,0,0}{\textbf{#1}}}%
\newcommand{\hlkwb}[1]{\textcolor[rgb]{0,0.341,0.682}{#1}}%
\newcommand{\hlkwc}[1]{\textcolor[rgb]{0,0,0}{\textbf{#1}}}%
\newcommand{\hlkwd}[1]{\textcolor[rgb]{0.004,0.004,0.506}{#1}}%
\let\hlipl\hlkwb

\usepackage{framed}
\makeatletter
\newenvironment{kframe}{%
 \def\at@end@of@kframe{}%
 \ifinner\ifhmode%
  \def\at@end@of@kframe{\end{minipage}}%
  \begin{minipage}{\columnwidth}%
 \fi\fi%
 \def\FrameCommand##1{\hskip\@totalleftmargin \hskip-\fboxsep
 \colorbox{shadecolor}{##1}\hskip-\fboxsep
     % There is no \\@totalrightmargin, so:
     \hskip-\linewidth \hskip-\@totalleftmargin \hskip\columnwidth}%
 \MakeFramed {\advance\hsize-\width
   \@totalleftmargin\z@ \linewidth\hsize
   \@setminipage}}%
 {\par\unskip\endMakeFramed%
 \at@end@of@kframe}
\makeatother

\definecolor{shadecolor}{rgb}{.97, .97, .97}
\definecolor{messagecolor}{rgb}{0, 0, 0}
\definecolor{warningcolor}{rgb}{1, 0, 1}
\definecolor{errorcolor}{rgb}{1, 0, 0}
\newenvironment{knitrout}{}{} % an empty environment to be redefined in TeX

\let\hlstd\hldef
\let\hlstr\hlsng
\usepackage{alltt}
%\documentclass[color=usenames,dvipsnames,handout]{beamer}

\usepackage[roman]{../lectures}
%\usepackage[sans]{../lectures}


\hypersetup{pdfpagemode=UseNone,pdfstartview={FitV}}

\mode<handout>{
  \usetheme{default}
%  \setbeamercolor{background canvas}{bg=black!5}
%  \pgfpagesuselayout{4 on 1}[letterpaper,landscape,border shrink=2.5mm]
%  \pgfpagesuselayout{2 on 1}[letterpaper,border shrink=10mm]
}

% Compile and open PDF






%% New command for inline code that isn't to be evaluated
\definecolor{inlinecolor}{rgb}{0.878, 0.918, 0.933}
\newcommand{\inr}[1]{\colorbox{inlinecolor}{\texttt{#1}}}





\IfFileExists{upquote.sty}{\usepackage{upquote}}{}
\begin{document}





\begin{frame}[plain]
  \centering
  \LARGE
  % Lecture 15:
  Open Populations and the \\
  Spatial Cormack-Jolly-Seber Model \\
  \vfill
  \large
  WILD(FISH) 8390 \\
%%  Estimation of Fish and Wildlife Population Parameters \\
  Inference for Models of Fish and Wildlife Population Dynamics \\
  \vfill
  Richard Chandler \\
  University of Georgia \\
\end{frame}




%\section{Intro}



\begin{frame}[plain]
  \frametitle{Topics}
  \Large
  \only<1>{\tableofcontents}%[hideallsubsections]}
%  \only<2 | handout:0>{\tableofcontents[currentsection]}%,hideallsubsections]}
\end{frame}




\section{Intro}





\begin{frame}
  \frametitle{Open populations}
  \large
  Overview
  \begin{itemize}[<+->]
    \normalsize
    \item Population closure is an assumption we make to estimate
      density in short time periods.
    \item If we want to study the processes governing spatial
      {and temporal} variation in density, we need to move
      beyond closed population models.
    \item Open population SCR models allow us to model spatial and
      temporal variation in survival, recruitment, and movement.
  \end{itemize}
\end{frame}



\begin{frame}
  \frametitle{Types of closure}
  \large
  Geographic closure
  \begin{itemize}
    \normalsize
    \item Animals cannot enter or leave the study area
    \item But what is the study area in non-spatial settings?
    \item What is a non-spatial setting?
  \end{itemize}
  \pause
  \vfill
  \large
  Demographic closure
  \begin{itemize}
    \normalsize
    \item No births or deaths
    \item Reasonable over short time periods
    \item But temporal dynamics are often of chief interest
  \end{itemize}
\end{frame}





\begin{frame}
  \frametitle{Two varieties of open models}
  \large
  Cormack-Jolly-Seber Model
  \begin{itemize}%[<+->]
    \normalsize
    \item<1-> Primarily used to estimate survival
    \item<2-> When permanent emigration is possible, estimate is of
      ``apparent survival'' ($\phi$)
    \item<3-> Apparent survival is the probability of
      surviving and not leaving the study area permanently
    \item<4-> CJS models are not used to estimate abundance or recruitment
  \end{itemize}
  \pause
  \vfill
  \large
  \uncover<5->{Jolly-Seber model}
  \normalsize
  \begin{itemize}
    \item<6-> Traditionally used to estimate abundance, (apparent)
      survival, and recruitment  
  \end{itemize}
  \pause
  \vfill
  \centering %\bf
  \uncover<7->{
    Both classes of models can be made spatially
    explicit to study movement, among other things \\
  }
\end{frame}






\section{Non-spatial CJS}







\begin{frame}
  \frametitle{Non-spatial CJS model}
  \large
  Notes
  \begin{itemize}[<+->]
    \normalsize
    \item Typical scenario involves ``releasing'' indviduals and recapturing them over time
    \item We don't model initial capture process
    \item Instead, we ``condition on initial capture''
    \item As an example, imagine releasing a bunch of tagged animals
      in a translocation study
    \item Because we aren't estimating $N$, no data augmentation required
    \item It is possible to estimate parameters using 1 sampling occasion per season
    \item Robust design provides more information about $p$
  \end{itemize}
\end{frame}




\begin{frame}
  \frametitle{Non-spatial CJS model}
  State model
  \[
    z_{i,t} \sim \mbox{Bernoulli}(z_{i,t-1} \times \phi)
  \]
  \vfill
  Observation model
  \[
    y_{i,t} \sim \mbox{Bernoulli}(z_{i,t} \times p)
  \]
  \pause
  \vfill
  \small
  where
  \begin{itemize}
    \item $z_{i,t}$ is ``alive state'' of individual $i$ at time $t$
    \item $\phi$ is ``apparent survival''. Probability of being alive
      and not permanently emigrating. 
    \item $y_{i,t}=1$ if individual was encountered, $y_{i,t}=0$ otherwise.
  \end{itemize}
\end{frame}









\begin{frame}[fragile]
  \frametitle{Simulating CJS data without robust design}
  {%\bf
    Parameters and dimensions}
  %\small
\begin{knitrout}\footnotesize
\definecolor{shadecolor}{rgb}{0.878, 0.918, 0.933}\color{fgcolor}\begin{kframe}
\begin{alltt}
\hldef{T} \hlkwb{<-} \hlnum{10}               \hlcom{## primary periods (eg, years)}
\hldef{K} \hlkwb{<-} \hlnum{1}                \hlcom{## only 1 secondary sampling occasion}
\hldef{n} \hlkwb{<-} \hlnum{25}
\hldef{phi} \hlkwb{<-} \hlnum{0.7}
\hldef{p} \hlkwb{<-} \hlnum{0.4}
\hldef{z} \hlkwb{<-} \hlkwd{matrix}\hldef{(}\hlnum{NA}\hldef{, n, T)}
\hldef{y} \hlkwb{<-} \hlkwd{matrix}\hldef{(}\hlnum{NA}\hldef{, n, T)}
\hldef{first} \hlkwb{<-} \hlkwd{rpois}\hldef{(n,} \hlnum{1}\hldef{)}\hlopt{+}\hlnum{1} \hlcom{## random release dates}
\end{alltt}
\end{kframe}
\end{knitrout}
\pause
\vfill
{%\bf
  Generate $z$ and $y$}
\begin{knitrout}\footnotesize
\definecolor{shadecolor}{rgb}{0.878, 0.918, 0.933}\color{fgcolor}\begin{kframe}
\begin{alltt}
\hlkwa{for}\hldef{(i} \hlkwa{in} \hlnum{1}\hlopt{:}\hldef{n) \{}
    \hldef{z[i,first[i]]} \hlkwb{<-} \hlnum{1}  \hlcom{## Known alive at release}
    \hldef{y[i,first[i]]} \hlkwb{<-} \hlnum{1}
    \hlkwa{for}\hldef{(t} \hlkwa{in} \hldef{(first[i]}\hlopt{+}\hlnum{1}\hldef{)}\hlopt{:}\hldef{T) \{}
        \hldef{z[i,t]} \hlkwb{<-} \hlkwd{rbinom}\hldef{(}\hlnum{1}\hldef{,} \hlnum{1}\hldef{, z[i,t}\hlopt{-}\hlnum{1}\hldef{]}\hlopt{*}\hldef{phi)} \hlcom{## Alive/dead state}
        \hldef{y[i,t]} \hlkwb{<-} \hlkwd{rbinom}\hldef{(}\hlnum{1}\hldef{,} \hlnum{1}\hldef{, z[i,t]}\hlopt{*}\hldef{p)}     \hlcom{## Data}
    \hldef{\}}
\hldef{\}}
\end{alltt}
\end{kframe}
\end{knitrout}
\end{frame}




\begin{frame}[fragile]
  \frametitle{Simulated CJS data}
  \begin{columns}
    \begin{column}{0.5\textwidth}
      Latent alive/dead state
\begin{knitrout}\tiny
\definecolor{shadecolor}{rgb}{0.878, 0.918, 0.933}\color{fgcolor}\begin{kframe}
\begin{alltt}
\hldef{z[,}\hlnum{1}\hlopt{:}\hlnum{7}\hldef{]}
\end{alltt}
\begin{verbatim}
##       [,1] [,2] [,3] [,4] [,5] [,6] [,7]
##  [1,]    1    0    0    0    0    0    0
##  [2,]   NA    1    1    1    1    0    0
##  [3,]   NA    1    1    0    0    0    0
##  [4,]    1    0    0    0    0    0    0
##  [5,]    1    1    1    0    0    0    0
##  [6,]   NA    1    1    1    1    0    0
##  [7,]   NA   NA    1    0    0    0    0
##  [8,]   NA   NA   NA    1    1    1    1
##  [9,]    1    1    0    0    0    0    0
## [10,]    1    0    0    0    0    0    0
## [11,]    1    1    1    1    0    0    0
## [12,]    1    0    0    0    0    0    0
## [13,]   NA   NA    1    0    0    0    0
## [14,]    1    0    0    0    0    0    0
## [15,]   NA    1    1    1    1    0    0
## [16,]    1    1    1    1    1    0    0
## [17,]    1    0    0    0    0    0    0
## [18,]    1    1    1    0    0    0    0
## [19,]    1    0    0    0    0    0    0
## [20,]   NA    1    1    0    0    0    0
## [21,]   NA   NA    1    1    0    0    0
## [22,]   NA    1    1    1    0    0    0
## [23,]   NA    1    0    0    0    0    0
## [24,]    1    0    0    0    0    0    0
## [25,]   NA    1    1    1    0    0    0
\end{verbatim}
\end{kframe}
\end{knitrout}
    \end{column}
    \begin{column}{0.5\textwidth}
      Encounter histories (data)
\begin{knitrout}\tiny
\definecolor{shadecolor}{rgb}{0.878, 0.918, 0.933}\color{fgcolor}\begin{kframe}
\begin{alltt}
\hldef{y[,}\hlnum{1}\hlopt{:}\hlnum{7}\hldef{]}
\end{alltt}
\begin{verbatim}
##       [,1] [,2] [,3] [,4] [,5] [,6] [,7]
##  [1,]    1    0    0    0    0    0    0
##  [2,]   NA    1    0    0    1    0    0
##  [3,]   NA    1    0    0    0    0    0
##  [4,]    1    0    0    0    0    0    0
##  [5,]    1    0    1    0    0    0    0
##  [6,]   NA    1    0    0    1    0    0
##  [7,]   NA   NA    1    0    0    0    0
##  [8,]   NA   NA   NA    1    0    0    1
##  [9,]    1    0    0    0    0    0    0
## [10,]    1    0    0    0    0    0    0
## [11,]    1    1    0    0    0    0    0
## [12,]    1    0    0    0    0    0    0
## [13,]   NA   NA    1    0    0    0    0
## [14,]    1    0    0    0    0    0    0
## [15,]   NA    1    0    0    0    0    0
## [16,]    1    0    1    0    1    0    0
## [17,]    1    0    0    0    0    0    0
## [18,]    1    1    0    0    0    0    0
## [19,]    1    0    0    0    0    0    0
## [20,]   NA    1    0    0    0    0    0
## [21,]   NA   NA    1    1    0    0    0
## [22,]   NA    1    1    1    0    0    0
## [23,]   NA    1    0    0    0    0    0
## [24,]    1    0    0    0    0    0    0
## [25,]   NA    1    1    1    0    0    0
\end{verbatim}
\end{kframe}
\end{knitrout}
    \end{column}
  \end{columns}
\end{frame}



\subsection{Likelihood}


\begin{frame}[plain]
  \frametitle{Topics}
  \Large
  \tableofcontents[currentsection,currentsubsection]
\end{frame}






\begin{frame}[fragile]
  \frametitle{Likelihood analysis}
  \small
  We can fit CJS models in R with the `RMark' or `marked'
  packages. We'll use `marked', which requires the capture histories
  to be formatted as a character vector named \inr{ch}. 
\begin{knitrout}\footnotesize
\definecolor{shadecolor}{rgb}{0.878, 0.918, 0.933}\color{fgcolor}\begin{kframe}
\begin{alltt}
\hlkwd{library}\hldef{(marked)}
\hldef{y.marked} \hlkwb{<-} \hlkwd{ifelse}\hldef{(}\hlkwd{is.na}\hldef{(y),} \hlnum{0}\hldef{, y)}
\hldef{cap.histories} \hlkwb{<-} \hlkwd{data.frame}\hldef{(}
    \hlkwc{ch}\hldef{=}\hlkwd{apply}\hldef{(y.marked,} \hlnum{1}\hldef{, paste,} \hlkwc{collapse}\hldef{=}\hlsng{""}\hldef{))}
\hlkwd{head}\hldef{(cap.histories,} \hlkwc{n}\hldef{=}\hlnum{3}\hldef{)}
\end{alltt}
\begin{verbatim}
##           ch
## 1 1000000000
## 2 0100100000
## 3 0100000000
\end{verbatim}
\end{kframe}
\end{knitrout}
  \pause
  \vfill
  You can use formulas to model covariate effects.
\begin{knitrout}\footnotesize
\definecolor{shadecolor}{rgb}{0.878, 0.918, 0.933}\color{fgcolor}\begin{kframe}
\begin{alltt}
\hldef{fm0} \hlkwb{<-} \hlkwd{crm}\hldef{(}\hlkwc{data}\hldef{=cap.histories,} \hlkwc{model}\hldef{=}\hlsng{"CJS"}\hldef{,} \hlkwc{hessian}\hldef{=}\hlnum{TRUE}\hldef{,}
           \hlkwc{model.parameters}\hldef{=}\hlkwd{list}\hldef{(}\hlkwc{Phi}\hldef{=}\hlkwd{list}\hldef{(}\hlkwc{formula}\hldef{=}\hlopt{~}\hlnum{1}\hldef{),}
                                 \hlkwc{p}\hldef{=}\hlkwd{list}\hldef{(}\hlkwc{formula}\hldef{=}\hlopt{~}\hlnum{1}\hldef{)))}
\end{alltt}
\end{kframe}
\end{knitrout}
\end{frame}





\begin{frame}[fragile]
  \frametitle{Likelihood analysis}
  Estimates of apparent survival and capture probability.
\begin{knitrout}\footnotesize
\definecolor{shadecolor}{rgb}{0.878, 0.918, 0.933}\color{fgcolor}\begin{kframe}
\begin{alltt}
\hlkwd{predict}\hldef{(fm0)}
\end{alltt}
\begin{verbatim}
## $Phi
##   occ estimate         se       lcl       ucl
## 1   4 0.617999 0.08624809 0.4415182 0.7680138
## 
## $p
##   occ  estimate        se       lcl       ucl
## 1   5 0.3242274 0.1095067 0.1526448 0.5609906
\end{verbatim}
\end{kframe}
\end{knitrout}
\end{frame}




\subsection{Bayesian}



\begin{frame}[plain]
  \frametitle{Topics}
  \Large
  \tableofcontents[currentsection,currentsubsection]
\end{frame}


\begin{frame}[fragile]
  \frametitle{Non-spatial CJS model in JAGS}
%  \scriptsize \fbox{\parbox{\linewidth}{\verbatiminput{CJS-nonspatial.jag}}}
\begin{knitrout}\scriptsize
\definecolor{shadecolor}{rgb}{0.878, 0.918, 0.933}\color{fgcolor}\begin{kframe}
\begin{alltt}
\hlkwd{writeLines}\hldef{(}\hlkwd{readLines}\hldef{(}\hlsng{"CJS-nonspatial.jag"}\hldef{))}
\end{alltt}
\end{kframe}
\end{knitrout}
\begin{knitrout}\scriptsize
\definecolor{shadecolor}{rgb}{0.961, 0.961, 0.863}\color{fgcolor}\begin{kframe}
\begin{verbatim}
model {

phi ~ dunif(0,1)
p ~ dunif(0,1)

for(i in 1:n) {
  z[i,first[i]] <- 1             ## Known alive at time of release
  for(t in (first[i]+1):T) {
    z[i,t] ~ dbern(z[i,t-1]*phi) ## Survives with probability phi
    y[i,t] ~ dbern(z[i,t]*p)     ## Detected with probability p
    }
  }

}
\end{verbatim}
\end{kframe}
\end{knitrout}
\end{frame}







\begin{frame}[fragile]
  \frametitle{JAGS}
  Data
\begin{knitrout}\scriptsize
\definecolor{shadecolor}{rgb}{0.878, 0.918, 0.933}\color{fgcolor}\begin{kframe}
\begin{alltt}
\hldef{jags.data.nonsp} \hlkwb{<-} \hlkwd{list}\hldef{(}\hlkwc{y}\hldef{=y,} \hlkwc{n}\hldef{=n,} \hlkwc{first}\hldef{=first,} \hlkwc{T}\hldef{=T)}
\end{alltt}
\end{kframe}
\end{knitrout}
  \pause
  \vfill
  Initial values for $z$ matrix
\begin{knitrout}\scriptsize
\definecolor{shadecolor}{rgb}{0.878, 0.918, 0.933}\color{fgcolor}\begin{kframe}
\begin{alltt}
\hldef{zi.nonsp} \hlkwb{<-} \hlkwd{matrix}\hldef{(}\hlnum{NA}\hldef{, n, T)}
\hlkwa{for}\hldef{(i} \hlkwa{in} \hlnum{1}\hlopt{:}\hldef{n) \{}
    \hldef{zi.nonsp[i,(first[i]}\hlopt{+}\hlnum{1}\hldef{)}\hlopt{:}\hldef{T]} \hlkwb{<-} \hlnum{1}
    \hldef{\}}
\hldef{jags.inits.nonsp} \hlkwb{<-} \hlkwa{function}\hldef{()} \hlkwd{list}\hldef{(}\hlkwc{phi}\hldef{=}\hlkwd{runif}\hldef{(}\hlnum{1}\hldef{),} \hlkwc{p}\hldef{=}\hlkwd{runif}\hldef{(}\hlnum{1}\hldef{),} \hlkwc{z}\hldef{=zi.nonsp)}
\hldef{jags.pars.nonsp} \hlkwb{<-} \hlkwd{c}\hldef{(}\hlsng{"phi"}\hldef{,} \hlsng{"p"}\hldef{)}
\end{alltt}
\end{kframe}
\end{knitrout}
  \pause
  \vfill
  {%\bf \normalsize
    Fit the model}
\begin{knitrout}\scriptsize
\definecolor{shadecolor}{rgb}{0.878, 0.918, 0.933}\color{fgcolor}\begin{kframe}
\begin{alltt}
\hlkwd{library}\hlstd{(jagsUI)}
\hlstd{jags.post.samples.nonsp} \hlkwb{<-} \hlkwd{jags.basic}\hlstd{(}\hlkwc{data}\hlstd{=jags.data.nonsp,}
                                      \hlkwc{inits}\hlstd{=jags.inits.nonsp,}
                                      \hlkwc{parameters.to.save}\hlstd{=jags.pars.nonsp,}
                                      \hlkwc{model.file}\hlstd{=}\hlstr{"CJS-nonspatial.jag"}\hlstd{,}
                                      \hlkwc{n.chains}\hlstd{=}\hlnum{3}\hlstd{,} \hlkwc{n.adapt}\hlstd{=}\hlnum{100}\hlstd{,} \hlkwc{n.burnin}\hlstd{=}\hlnum{0}\hlstd{,}
                                      \hlkwc{n.iter}\hlstd{=}\hlnum{2000}\hlstd{,} \hlkwc{parallel}\hlstd{=}\hlnum{TRUE}\hlstd{)}
\end{alltt}
\end{kframe}
\end{knitrout}
\end{frame}









\begin{frame}[fragile]
  \frametitle{Posterior distributions}
\begin{knitrout}\scriptsize
\definecolor{shadecolor}{rgb}{0.878, 0.918, 0.933}\color{fgcolor}\begin{kframe}
\begin{alltt}
\hlkwd{plot}\hldef{(jags.post.samples.nonsp)}
\end{alltt}
\end{kframe}

{\centering \includegraphics[width=0.65\linewidth]{figure/jc1-plot-1} 

}


\end{knitrout}
\end{frame}








\section{Spatial CJS}



\begin{frame}[plain]
  \frametitle{Topics}
  \Large
  \tableofcontents[currentsection]
\end{frame}


\begin{frame}
  \frametitle{Spatial CJS}
  \large
  Why spatial?
  \begin{itemize}
    \normalsize
    \item Comes with all the benefits of other SCR models
    \item Allows for movement between primary periods, making
      it possible to separate survival from permanent emigration
    \item Movement itself may be of interest, such as in studies of
      dispersal
  \end{itemize}
  \pause
  \vfill
  \normalsize
  {\centering %\bf
    We begin by ignoring movement, but we use the robust design \\ }
\end{frame}




\begin{frame}[fragile]
  \frametitle{Simulating spatial CJS data with robust design}
  \scriptsize %\footnotesize
  \begin{columns}
%      \column{\dimexpr\paperwidth-10pt}
    \begin{column}{0.5\textwidth}%{0.5\dimexpr\paperwidth-10pt}
      {%\bf
        Parameters and data dimensions}
\begin{knitrout}\tiny
\definecolor{shadecolor}{rgb}{0.878, 0.918, 0.933}\color{fgcolor}\begin{kframe}
\begin{alltt}
\hldef{T} \hlkwb{<-} \hlnum{10}       \hlcom{# primary occasions}
\hldef{K} \hlkwb{<-} \hlnum{3}        \hlcom{# secondary occasions}
\hldef{n} \hlkwb{<-} \hlnum{25}       \hlcom{# nIndividuals}
\hldef{phi} \hlkwb{<-} \hlnum{0.7}
\hldef{p0} \hlkwb{<-} \hlnum{0.4}     \hlcom{# baseline capture prob}
\hldef{sigma} \hlkwb{<-} \hlnum{0.1}  \hlcom{# scale parameter}
\end{alltt}
\end{kframe}
\end{knitrout}
\pause
{%\bf
  Traps, activity centers, and \\ detection probability}
\begin{knitrout}\tiny
\definecolor{shadecolor}{rgb}{0.878, 0.918, 0.933}\color{fgcolor}\begin{kframe}
\begin{alltt}
\hldef{co} \hlkwb{<-} \hlkwd{seq}\hldef{(}\hlnum{0.25}\hldef{,} \hlnum{0.75}\hldef{,} \hlkwc{length}\hldef{=}\hlnum{5}\hldef{)}
\hldef{x} \hlkwb{<-} \hlkwd{cbind}\hldef{(}\hlkwd{rep}\hldef{(co,} \hlkwc{each}\hldef{=}\hlnum{5}\hldef{),}
           \hlkwd{rep}\hldef{(co,} \hlkwc{times}\hldef{=}\hlnum{5}\hldef{))}
\hldef{J} \hlkwb{<-} \hlkwd{nrow}\hldef{(x)}
\hldef{xlim} \hlkwb{<-} \hldef{ylim} \hlkwb{<-} \hlkwd{c}\hldef{(}\hlnum{0}\hldef{,}\hlnum{1}\hldef{)}
\hldef{s} \hlkwb{<-} \hlkwd{cbind}\hldef{(}\hlkwd{runif}\hldef{(n, xlim[}\hlnum{1}\hldef{], xlim[}\hlnum{2}\hldef{]),}
           \hlkwd{runif}\hldef{(n, ylim[}\hlnum{1}\hldef{], ylim[}\hlnum{2}\hldef{]))}
\hldef{d} \hlkwb{<-} \hldef{p.sp} \hlkwb{<-} \hlkwd{matrix}\hldef{(}\hlnum{NA}\hldef{, n, J)}
\hlkwa{for}\hldef{(i} \hlkwa{in} \hlnum{1}\hlopt{:}\hldef{n) \{}
    \hldef{d[i,]} \hlkwb{<-} \hlkwd{sqrt}\hldef{((s[i,}\hlnum{1}\hldef{]}\hlopt{-}\hldef{x[,}\hlnum{1}\hldef{])}\hlopt{^}\hlnum{2} \hlopt{+}
                  \hldef{(s[i,}\hlnum{2}\hldef{]}\hlopt{-}\hldef{x[,}\hlnum{2}\hldef{])}\hlopt{^}\hlnum{2}\hldef{)}
    \hldef{p.sp[i,]} \hlkwb{<-} \hldef{p0}\hlopt{*}\hlkwd{exp}\hldef{(}\hlopt{-}\hldef{d[i,]}\hlopt{^}\hlnum{2}\hlopt{/}\hldef{(}\hlnum{2}\hlopt{*}\hldef{sigma}\hlopt{^}\hlnum{2}\hldef{))}
\hldef{\}}
\end{alltt}
\end{kframe}
\end{knitrout}
    \end{column}
\pause
%\vfill
\begin{column}{0.5\textwidth}%{0.5\dimexpr\paperwidth-10pt}
\scriptsize %\tiny
{%\bf
  Generate $z$ and $y$}
\begin{knitrout}\tiny
\definecolor{shadecolor}{rgb}{0.878, 0.918, 0.933}\color{fgcolor}\begin{kframe}
\begin{alltt}
\hldef{z} \hlkwb{<-} \hlkwd{matrix}\hldef{(}\hlnum{NA}\hldef{, n, T)}
\hldef{y.sp} \hlkwb{<-} \hlkwd{array}\hldef{(}\hlnum{NA}\hldef{,} \hlkwd{c}\hldef{(n, J, K, T))} \hlcom{## 4D array}
\hldef{first} \hlkwb{<-} \hlkwd{rpois}\hldef{(n,} \hlnum{1}\hldef{)}\hlopt{+}\hlnum{1} \hlcom{## random release dates}
\hlkwa{for}\hldef{(i} \hlkwa{in} \hlnum{1}\hlopt{:}\hldef{n) \{}
    \hldef{z[i,first[i]]} \hlkwb{<-} \hlnum{1}
    \hlkwa{for}\hldef{(t} \hlkwa{in} \hldef{(first[i]}\hlopt{+}\hlnum{1}\hldef{)}\hlopt{:}\hldef{T) \{}
        \hldef{z[i,t]} \hlkwb{<-} \hlkwd{rbinom}\hldef{(}\hlnum{1}\hldef{,} \hlnum{1}\hldef{,}
                         \hldef{z[i,t}\hlopt{-}\hlnum{1}\hldef{]}\hlopt{*}\hldef{phi)}
        \hlkwa{for}\hldef{(j} \hlkwa{in} \hlnum{1}\hlopt{:}\hldef{J) \{}
            \hldef{y.sp[i,j,}\hlnum{1}\hlopt{:}\hldef{K,t]} \hlkwb{<-} \hlkwd{rbinom}\hldef{(K,} \hlnum{1}\hldef{,}
                \hldef{z[i,t]}\hlopt{*}\hldef{p.sp[i,j])}
        \hldef{\}}
    \hldef{\}}
\hldef{\}}
\end{alltt}
\end{kframe}
\end{knitrout}
  \end{column}
\end{columns}
\end{frame}






\subsection{Likelihood}


\begin{frame}[plain]
  \frametitle{Topics}
  \Large
  \tableofcontents[currentsection,currentsubsection]
\end{frame}



\begin{frame}[fragile]
  \frametitle{Likelihood analysis of spatial CJS model}
  We'll use the R package `openpopscr'\footnote{Another option is the
`openCR' package.} for model fitting. It isn't on  
  CRAN, but you can install it from github using the following
  commands.
\begin{knitrout}\scriptsize
\definecolor{shadecolor}{rgb}{0.878, 0.918, 0.933}\color{fgcolor}\begin{kframe}
\begin{alltt}
\hlkwd{library}\hldef{(remotes)}
\hlkwd{install_github}\hldef{(}\hlsng{"r-glennie/openpopscr"}\hldef{,} \hlkwc{build} \hldef{=} \hlnum{TRUE}\hldef{,}
               \hlkwc{build_opts} \hldef{=} \hlkwd{c}\hldef{(}\hlsng{"--no-resave-data"}\hldef{,} \hlsng{"--no-manual"}\hldef{),}
               \hlkwc{build_vignettes} \hldef{=} \hlnum{TRUE}\hldef{)}
\hlkwd{library}\hldef{(openpopscr)}
\hlkwd{library}\hldef{(secr)}
\end{alltt}
\end{kframe}
\end{knitrout}
\vfill
If you're on Windows, you might have to install Rtools first. \\
\centering
\url{https://cran.r-project.org/bin/windows/Rtools/} \\

\end{frame}



\begin{frame}[fragile]
  \frametitle{Likelihood analysis of spatial CJS model}
Begin by making a mask in `secr'
\begin{knitrout}\scriptsize
\definecolor{shadecolor}{rgb}{0.878, 0.918, 0.933}\color{fgcolor}\begin{kframe}
\begin{alltt}
\hldef{trap.df} \hlkwb{<-} \hlkwd{data.frame}\hldef{(x}\hlopt{*}\hlnum{1000}\hldef{);} \hlkwd{colnames}\hldef{(trap.df)} \hlkwb{<-} \hlkwd{c}\hldef{(}\hlsng{"x"}\hldef{,}\hlsng{"y"}\hldef{)}
\hldef{traps} \hlkwb{<-} \hlkwd{read.traps}\hldef{(}\hlkwc{data}\hldef{=trap.df,} \hlkwc{detector}\hldef{=}\hlsng{"proximity"}\hldef{)}
\hldef{mask} \hlkwb{<-} \hlkwd{make.mask}\hldef{(}\hlkwc{traps}\hldef{=traps,} \hlkwc{buffer}\hldef{=}\hlnum{250}\hldef{)}
\hlkwd{plot}\hldef{(mask);} \hlkwd{points}\hldef{(traps,} \hlkwc{pch}\hldef{=}\hlnum{3}\hldef{,} \hlkwc{col}\hldef{=}\hlsng{"blue"}\hldef{,} \hlkwc{lwd}\hldef{=}\hlnum{2}\hldef{)}
\end{alltt}
\end{kframe}

{\centering \includegraphics[width=0.6\linewidth]{figure/mask-1} 

}


\end{knitrout}
\end{frame}


\begin{frame}[fragile]
  \frametitle{Likelihood analysis of spatial CJS model}
Format for `secr'
\begin{knitrout}\scriptsize
\definecolor{shadecolor}{rgb}{0.878, 0.918, 0.933}\color{fgcolor}\begin{kframe}
\begin{alltt}
\hldef{y.sp.secr} \hlkwb{<-} \hldef{y.sp}
\hldef{y.sp.secr[}\hlkwd{is.na}\hldef{(y.sp)]} \hlkwb{<-} \hlnum{0}
\hldef{caps} \hlkwb{<-} \hlkwd{data.frame}\hldef{(}\hlkwc{session}\hldef{=}\hlnum{1}\hldef{,}
                   \hlkwc{animal}\hldef{=}\hlkwd{rep}\hldef{(}\hlkwd{slice.index}\hldef{(y.sp.secr,} \hlnum{1}\hldef{), y.sp.secr),}
                   \hlkwc{occasion}\hldef{=}\hlkwd{rep}\hldef{(}\hlkwd{slice.index}\hldef{(y.sp.secr,} \hlnum{3}\hlopt{:}\hlnum{4}\hldef{), y.sp.secr),}
                   \hlkwc{trap}\hldef{=}\hlkwd{rep}\hldef{(}\hlkwd{slice.index}\hldef{(y.sp.secr,} \hlnum{2}\hldef{), y.sp.secr))}
\hldef{capthist} \hlkwb{<-} \hlkwd{make.capthist}\hldef{(}\hlkwc{captures}\hldef{=caps,} \hlkwc{traps}\hldef{=traps,} \hlkwc{noccasions}\hldef{=}\hlnum{30}\hldef{)}
\end{alltt}
\end{kframe}
\end{knitrout}
\vfill
Then format for `openpopscr'
\begin{knitrout}\scriptsize
\definecolor{shadecolor}{rgb}{0.878, 0.918, 0.933}\color{fgcolor}\begin{kframe}
\begin{alltt}
\hldef{cjs.data} \hlkwb{<-} \hldef{ScrData}\hlopt{$}\hlkwd{new}\hldef{(capthist, mask,} \hlkwc{primary}\hldef{=}\hlkwd{rep}\hldef{(}\hlnum{1}\hlopt{:}\hlnum{10}\hldef{,} \hlkwc{each}\hldef{=}\hlnum{3}\hldef{))}
\end{alltt}
\end{kframe}
\end{knitrout}
\end{frame}


\begin{frame}[fragile]
  \frametitle{Likelihood analysis}
Create the model object and then fit it
\begin{knitrout}\scriptsize
\definecolor{shadecolor}{rgb}{0.878, 0.918, 0.933}\color{fgcolor}\begin{kframe}
\begin{alltt}
\hlstd{mod} \hlkwb{<-} \hlstd{CjsModel}\hlopt{$}\hlkwd{new}\hlstd{(}\hlkwd{list}\hlstd{(lambda0}\hlopt{~}\hlnum{1}\hlstd{, sigma}\hlopt{~}\hlnum{1}\hlstd{, phi}\hlopt{~}\hlnum{1}\hlstd{), cjs.data,}
                    \hlkwc{start}\hlstd{=}\hlkwd{list}\hlstd{(}\hlkwc{lambda0}\hlstd{=}\hlnum{1.5}\hlstd{,} \hlkwc{sigma}\hlstd{=}\hlnum{50}\hlstd{,} \hlkwc{phi}\hlstd{=}\hlnum{0.5}\hlstd{))}
\hlstd{mod}\hlopt{$}\hlkwd{fit}\hlstd{()}
\hlstd{mod}
\end{alltt}
\end{kframe}
\end{knitrout}
Back-transform
\begin{knitrout}\scriptsize
\definecolor{shadecolor}{rgb}{0.878, 0.918, 0.933}\color{fgcolor}\begin{kframe}
\begin{alltt}
\hldef{mod}\hlopt{$}\hlkwd{get_par}\hldef{(}\hlsng{"lambda0"}\hldef{,} \hlkwc{k} \hldef{=} \hlnum{1}\hldef{,} \hlkwc{j} \hldef{=} \hlnum{1}\hldef{)}
\end{alltt}
\begin{verbatim}
## [1] 0.3824823
\end{verbatim}
\begin{alltt}
\hldef{mod}\hlopt{$}\hlkwd{get_par}\hldef{(}\hlsng{"sigma"}\hldef{,} \hlkwc{k} \hldef{=} \hlnum{1}\hldef{,} \hlkwc{j} \hldef{=} \hlnum{1}\hldef{)}
\end{alltt}
\begin{verbatim}
## [1] 102.0998
\end{verbatim}
\begin{alltt}
\hldef{mod}\hlopt{$}\hlkwd{get_par}\hldef{(}\hlsng{"phi"}\hldef{,} \hlkwc{k} \hldef{=} \hlnum{1}\hldef{,} \hlkwc{m}\hldef{=}\hlnum{1}\hldef{)}
\end{alltt}
\begin{verbatim}
## [1] 0.5708015
\end{verbatim}
\end{kframe}
\end{knitrout}
\end{frame}





\subsection{Bayesian}



\begin{frame}[plain]
  \frametitle{Topics}
  \Large
  \tableofcontents[currentsection,currentsubsection]
\end{frame}




\begin{frame}[fragile]
  \frametitle{Spatial CJS model in JAGS}
%  \vspace{-5mm}
%  \scriptsize \fbox{\parbox{\linewidth}{\verbatiminput{CJS-spatial.jag}}}
\begin{knitrout}\tiny
\definecolor{shadecolor}{rgb}{0.878, 0.918, 0.933}\color{fgcolor}\begin{kframe}
\begin{alltt}
\hlkwd{writeLines}\hldef{(}\hlkwd{readLines}\hldef{(}\hlsng{"CJS-spatial.jag"}\hldef{))}
\end{alltt}
\end{kframe}
\end{knitrout}
\begin{knitrout}\tiny
\definecolor{shadecolor}{rgb}{0.961, 0.961, 0.863}\color{fgcolor}\begin{kframe}
\begin{verbatim}
model {

phi ~ dunif(0,1)     ## Apparent survival
p0 ~ dunif(0,1)      ## Baseline capture prob
sigma ~ dunif(0, 2)  ## Scale parameter of detection function

for(i in 1:n) {      ## Loop over individuals
  s[i,1] ~ dunif(xlim[1], xlim[2]) # static activity centers
  s[i,2] ~ dunif(ylim[1], ylim[2])
  for(j in 1:J) {    ## Loop over traps
    d[i,j] <- sqrt((s[i,1] - x[j,1])^2 + (s[i,2] - x[j,2])^2)
    p[i,j] <- p0*exp(-d[i,j]^2/(2*sigma^2)) ## Capture prob
    }
  z[i,first[i]] <- 1 ## Condition on first encounter
  for(t in (first[i]+1):T) {
    z[i,t] ~ dbern(z[i,t-1]*phi) ## Model subsequent encounters
    for(j in 1:J) {
      for(k in 1:K) {
        y[i,j,k,t] ~ dbern(z[i,t]*p[i,j])
        }
      }
    }
  }

}
\end{verbatim}
\end{kframe}
\end{knitrout}
\end{frame}





\begin{frame}[fragile]
  \frametitle{JAGS}
  Data
\begin{knitrout}\scriptsize
\definecolor{shadecolor}{rgb}{0.878, 0.918, 0.933}\color{fgcolor}\begin{kframe}
\begin{alltt}
\hldef{jags.data.sp} \hlkwb{<-} \hlkwd{list}\hldef{(}\hlkwc{y}\hldef{=y.sp,} \hlkwc{n}\hldef{=n,} \hlkwc{first}\hldef{=first,} \hlkwc{x}\hldef{=x,}
                     \hlkwc{J}\hldef{=J,} \hlkwc{K}\hldef{=K,} \hlkwc{T}\hldef{=T,} \hlkwc{xlim}\hldef{=xlim,} \hlkwc{ylim}\hldef{=ylim)}
\end{alltt}
\end{kframe}
\end{knitrout}
  Initial values for $z$ matrix
  \scriptsize
\begin{knitrout}\scriptsize
\definecolor{shadecolor}{rgb}{0.878, 0.918, 0.933}\color{fgcolor}\begin{kframe}
\begin{alltt}
\hldef{zi.sp} \hlkwb{<-} \hlkwd{matrix}\hldef{(}\hlnum{NA}\hldef{, n, T)}
\hlkwa{for}\hldef{(i} \hlkwa{in} \hlnum{1}\hlopt{:}\hldef{n) \{}
    \hldef{zi.sp[i,(first[i]}\hlopt{+}\hlnum{1}\hldef{)}\hlopt{:}\hldef{T]} \hlkwb{<-} \hlnum{1}
    \hldef{\}}
\hldef{jags.inits.sp} \hlkwb{<-} \hlkwa{function}\hldef{()} \hlkwd{list}\hldef{(}\hlkwc{phi}\hldef{=}\hlkwd{runif}\hldef{(}\hlnum{1}\hldef{),} \hlkwc{z}\hldef{=zi.sp)}
\hldef{jags.pars.sp} \hlkwb{<-} \hlkwd{c}\hldef{(}\hlsng{"phi"}\hldef{,} \hlsng{"p0"}\hldef{,} \hlsng{"sigma"}\hldef{)}
\end{alltt}
\end{kframe}
\end{knitrout}
\pause
\vfill
  {\normalsize Fit the model}
\begin{knitrout}\scriptsize
\definecolor{shadecolor}{rgb}{0.878, 0.918, 0.933}\color{fgcolor}\begin{kframe}
\begin{alltt}
\hlstd{jags.post.samples.sp} \hlkwb{<-} \hlkwd{jags.basic}\hlstd{(}\hlkwc{data}\hlstd{=jags.data.sp,}
                                   \hlkwc{inits}\hlstd{=jags.inits.sp,}
                                   \hlkwc{parameters.to.save}\hlstd{=jags.pars.sp,}
                                   \hlkwc{model.file}\hlstd{=}\hlstr{"CJS-spatial.jag"}\hlstd{,}
                                   \hlkwc{n.chains}\hlstd{=}\hlnum{3}\hlstd{,} \hlkwc{n.adapt}\hlstd{=}\hlnum{100}\hlstd{,} \hlkwc{n.burnin}\hlstd{=}\hlnum{0}\hlstd{,}
                                   \hlkwc{n.iter}\hlstd{=}\hlnum{2000}\hlstd{,} \hlkwc{parallel}\hlstd{=}\hlnum{TRUE}\hlstd{)}
\end{alltt}
\end{kframe}
\end{knitrout}
\end{frame}









\begin{frame}[fragile]
  \frametitle{Posterior distributions}
\begin{knitrout}\scriptsize
\definecolor{shadecolor}{rgb}{0.878, 0.918, 0.933}\color{fgcolor}\begin{kframe}
\begin{alltt}
\hlkwd{plot}\hldef{(jags.post.samples.sp)}
\end{alltt}
\end{kframe}

{\centering \includegraphics[width=0.6\linewidth]{figure/jc2-plot-1} 

}


\end{knitrout}
\end{frame}










\section{Spatial CJS with movement}









\begin{frame}
  \frametitle{Modeling dispersal among years}
  {\large What if activity centers move between years?}
  \begin{itemize}
    \normalsize
    \item Could affect estimates of detection parameters
    \item More importantly, might provide opportunity to study dispersal
  \end{itemize}
  \pause
  \vfill
  {\large Potential movement models}
  \begin{itemize}
    \normalsize
    \item Random walk: ${\bm s}_t \sim \mbox{Normal}({\bm s}_{t-1}, \tau^2)$
    \item Random walk with attraction point: ${\bm s}_t \sim \mbox{Normal}({\bm  s}_{t-1} + \rho(\bar{\bm s} - {\bm s}_{t-1}), \tau^2)$
    \item Extensions can be made to allow for resource selection,
      landscape resistance, etc.%\dots\footnote{https://esajournals.onlinelibrary.wiley.com/doi/full/10.1002/ecy.3473}. 
  \end{itemize}
\end{frame}








\begin{frame}[fragile]
  \frametitle{Simulating spatial CJS data with dispersal}
  \begin{columns}
    \footnotesize %\small
    \begin{column}{0.5\textwidth}%{0.5\dimexpr\paperwidth-10pt}
    Parameters and data dimensions
\begin{knitrout}\tiny
\definecolor{shadecolor}{rgb}{0.878, 0.918, 0.933}\color{fgcolor}\begin{kframe}
\begin{alltt}
\hldef{T} \hlkwb{<-} \hlnum{10}\hldef{; K} \hlkwb{<-} \hlnum{5}\hldef{; n} \hlkwb{<-} \hlnum{25} \hlcom{## same as before}
\hldef{phi} \hlkwb{<-} \hlnum{0.9}\hldef{; p0} \hlkwb{<-} \hlnum{0.4}\hldef{; sigma} \hlkwb{<-} \hlnum{0.05}
\hlcom{## dispersal parameter (of random walk)}
\hldef{tau} \hlkwb{<-} \hlnum{0.05}
\end{alltt}
\end{kframe}
\end{knitrout}
\pause
Activity centers take a random walk
\begin{knitrout}\tiny
\definecolor{shadecolor}{rgb}{0.878, 0.918, 0.933}\color{fgcolor}\begin{kframe}
\begin{alltt}
\hlcom{## state-space should be bigger}
\hldef{xlim} \hlkwb{<-} \hldef{ylim} \hlkwb{<-} \hlkwd{c}\hldef{(}\hlnum{0}\hldef{,} \hlnum{1}\hldef{)}
\hldef{first} \hlkwb{<-} \hlkwd{rpois}\hldef{(n,} \hlnum{1}\hldef{)}\hlopt{+}\hlnum{1}
\hldef{s} \hlkwb{<-} \hlkwd{array}\hldef{(}\hlnum{NA}\hldef{,} \hlkwd{c}\hldef{(n,} \hlnum{2}\hldef{, T))}
\hlkwa{for}\hldef{(i} \hlkwa{in} \hlnum{1}\hlopt{:}\hldef{n) \{}
    \hldef{s[i,,first[i]]} \hlkwb{<-}
        \hlkwd{cbind}\hldef{(}\hlkwd{runif}\hldef{(}\hlnum{1}\hldef{, xlim[}\hlnum{1}\hldef{], xlim[}\hlnum{2}\hldef{]),}
              \hlkwd{runif}\hldef{(}\hlnum{1}\hldef{, ylim[}\hlnum{1}\hldef{], ylim[}\hlnum{2}\hldef{]))}
    \hlkwa{for}\hldef{(t} \hlkwa{in} \hldef{(first[i]}\hlopt{+}\hlnum{1}\hldef{)}\hlopt{:}\hldef{T) \{}
        \hldef{s[i,}\hlnum{1}\hldef{,t]} \hlkwb{<-} \hlkwd{rnorm}\hldef{(}\hlnum{1}\hldef{, s[i,}\hlnum{1}\hldef{,t}\hlopt{-}\hlnum{1}\hldef{], tau)}
        \hldef{s[i,}\hlnum{2}\hldef{,t]} \hlkwb{<-} \hlkwd{rnorm}\hldef{(}\hlnum{1}\hldef{, s[i,}\hlnum{2}\hldef{,t}\hlopt{-}\hlnum{1}\hldef{], tau)}
    \hldef{\}}
\hldef{\}}
\hldef{d} \hlkwb{<-} \hldef{p} \hlkwb{<-} \hlkwd{array}\hldef{(}\hlnum{NA}\hldef{,} \hlkwd{c}\hldef{(n, J, T))}
\hlkwa{for}\hldef{(i} \hlkwa{in} \hlnum{1}\hlopt{:}\hldef{n) \{}
    \hlkwa{for}\hldef{(t} \hlkwa{in} \hlnum{1}\hlopt{:}\hldef{T) \{}
        \hldef{d[i,,t]} \hlkwb{<-} \hlkwd{sqrt}\hldef{((s[i,}\hlnum{1}\hldef{,t]}\hlopt{-}\hldef{x[,}\hlnum{1}\hldef{])}\hlopt{^}\hlnum{2} \hlopt{+}
                        \hldef{(s[i,}\hlnum{2}\hldef{,t]}\hlopt{-}\hldef{x[,}\hlnum{2}\hldef{])}\hlopt{^}\hlnum{2}\hldef{)}
        \hldef{p[i,,t]} \hlkwb{<-} \hldef{p0}\hlopt{*}\hlkwd{exp}\hldef{(}\hlopt{-}\hldef{d[i,,t]}\hlopt{^}\hlnum{2} \hlopt{/}
                          \hldef{(}\hlnum{2}\hlopt{*}\hldef{sigma}\hlopt{^}\hlnum{2}\hldef{))}
    \hldef{\}}
\hldef{\}}
\end{alltt}
\end{kframe}
\end{knitrout}
    \end{column}
    \pause
    \begin{column}{0.5\textwidth}%{0.5\dimexpr\paperwidth-10pt}
Generate $z$ and $y$
\begin{knitrout}\tiny
\definecolor{shadecolor}{rgb}{0.878, 0.918, 0.933}\color{fgcolor}\begin{kframe}
\begin{alltt}
\hldef{z} \hlkwb{<-} \hlkwd{matrix}\hldef{(}\hlnum{NA}\hldef{, n, T)}
\hldef{y.move} \hlkwb{<-} \hlkwd{array}\hldef{(}\hlnum{NA}\hldef{,} \hlkwd{c}\hldef{(n, J, T))}
\hlkwa{for}\hldef{(i} \hlkwa{in} \hlnum{1}\hlopt{:}\hldef{n) \{}
    \hldef{z[i,first[i]]} \hlkwb{<-} \hlnum{1}
    \hlkwa{for}\hldef{(t} \hlkwa{in} \hldef{(first[i]}\hlopt{+}\hlnum{1}\hldef{)}\hlopt{:}\hldef{T) \{}
        \hldef{z[i,t]} \hlkwb{<-} \hlkwd{rbinom}\hldef{(}\hlnum{1}\hldef{,} \hlnum{1}\hldef{, z[i,t}\hlopt{-}\hlnum{1}\hldef{]}\hlopt{*}\hldef{phi)}
        \hlkwa{for}\hldef{(j} \hlkwa{in} \hlnum{1}\hlopt{:}\hldef{J) \{}
            \hldef{y.move[i,j,t]} \hlkwb{<-}
                \hlkwd{rbinom}\hldef{(}\hlnum{1}\hldef{, K, z[i,t]}\hlopt{*}\hldef{p[i,j,t])}
        \hldef{\}}
    \hldef{\}}
\hldef{\}}
\end{alltt}
\end{kframe}
\end{knitrout}
\end{column}
  \end{columns}
\end{frame}





\begin{frame}[fragile]
  \frametitle{Movement of activity center 1}
\begin{knitrout}\scriptsize
\definecolor{shadecolor}{rgb}{0.878, 0.918, 0.933}\color{fgcolor}\begin{kframe}
\begin{alltt}
\hlkwd{plot}\hldef{(}\hlkwd{t}\hldef{(s[}\hlnum{1}\hldef{,,]),} \hlkwc{pch}\hldef{=}\hlnum{16}\hldef{,} \hlkwc{type}\hldef{=}\hlsng{"o"}\hldef{,} \hlkwc{xlab}\hldef{=}\hlsng{"x"}\hldef{,} \hlkwc{ylab}\hldef{=}\hlsng{"y"}\hldef{,}
     \hlkwc{xlim}\hldef{=}\hlkwd{c}\hldef{(}\hlnum{0}\hldef{,} \hlnum{1}\hldef{),} \hlkwc{ylim}\hldef{=}\hlkwd{c}\hldef{(}\hlnum{0}\hldef{,} \hlnum{1}\hldef{),} \hlkwc{asp}\hldef{=}\hlnum{1}\hldef{,} \hlkwc{col}\hldef{=}\hlsng{"blue"}\hldef{)}
\hlkwd{points}\hldef{(x,} \hlkwc{pch}\hldef{=}\hlnum{3}\hldef{)}
\end{alltt}
\end{kframe}

{\centering \includegraphics[width=0.65\linewidth]{figure/s1-1} 

}


\end{knitrout}
\end{frame}





\begin{frame}[fragile]
  \frametitle{Movement of all activity centers}
\begin{knitrout}\scriptsize
\definecolor{shadecolor}{rgb}{0.878, 0.918, 0.933}\color{fgcolor}

{\centering \includegraphics[width=0.65\linewidth]{figure/s-all-1} 

}


\end{knitrout}
\end{frame}




\begin{frame}[fragile]
  \frametitle{Spatial CJS model with dispersal in JAGS}
  \vspace{-2mm}
  \tiny %\fbox{\parbox{\linewidth}{\verbatiminput{CJS-spatial-move.jag}}}
\begin{knitrout}\tiny
\definecolor{shadecolor}{rgb}{0.878, 0.918, 0.933}\color{fgcolor}\begin{kframe}
\begin{alltt}
\hlkwd{writeLines}\hldef{(}\hlkwd{readLines}\hldef{(}\hlsng{"CJS-spatial-move.jag"}\hldef{))}
\end{alltt}
\end{kframe}
\end{knitrout}
\begin{knitrout}\tiny
\definecolor{shadecolor}{rgb}{0.961, 0.961, 0.863}\color{fgcolor}\begin{kframe}
\begin{verbatim}
model {

phi ~ dunif(0,1)
p0 ~ dunif(0,1)
sigma ~ dunif(0, 2)
tau ~ dunif(0, 2)
tauprec <- 1/tau^2 # must convert tau to precision

for(i in 1:n) {
  s[i,1,first[i]] ~ dunif(xlim[1], xlim[2])
  s[i,2,first[i]] ~ dunif(ylim[1], ylim[2])
  for(j in 1:J) {
    d[i,j,first[i]] <- sqrt((s[i,1,first[i]] - x[j,1])^2 + (s[i,2,first[i]] - x[j,2])^2)
    p[i,j,first[i]] <- p0*exp(-d[i,j,first[i]]^2/(2*sigma^2))
    }
  z[i,first[i]] <- 1
  for(t in (first[i]+1):T) {
    s[i,1,t] ~ dnorm(s[i,1,t-1], tauprec)#T(xlim[1], xlim[2])
    s[i,2,t] ~ dnorm(s[i,2,t-1], tauprec)#T(ylim[1], ylim[2])
    for(j in 1:J) {
      d[i,j,t] <- sqrt((s[i,1,t] - x[j,1])^2 + (s[i,2,t] - x[j,2])^2)
      p[i,j,t] <- p0*exp(-d[i,j,t]^2/(2*sigma^2))
      }
    z[i,t] ~ dbern(z[i,t-1]*phi)
    for(j in 1:J) {
      y[i,j,t] ~ dbin(z[i,t]*p[i,j,t], K)
#      for(k in 1:K) { y[i,j,k,t] ~ dbern(z[i,t]*p[i,j,t]) }
  } } } 

}
\end{verbatim}
\end{kframe}
\end{knitrout}
\end{frame}





\begin{frame}[fragile]
  \frametitle{JAGS}
  \small
  Data
\begin{knitrout}\tiny
\definecolor{shadecolor}{rgb}{0.878, 0.918, 0.933}\color{fgcolor}\begin{kframe}
\begin{alltt}
\hldef{jd.sp.move} \hlkwb{<-} \hlkwd{list}\hldef{(}\hlkwc{y}\hldef{=y.move,} \hlkwc{n}\hldef{=n,} \hlkwc{first}\hldef{=first,} \hlkwc{x}\hldef{=x,}
                   \hlkwc{J}\hldef{=J,} \hlkwc{K}\hldef{=K,} \hlkwc{T}\hldef{=T,} \hlkwc{xlim}\hldef{=xlim,} \hlkwc{ylim}\hldef{=ylim)}
\end{alltt}
\end{kframe}
\end{knitrout}
  Initial values for $z$, and parameters to monitor
\begin{knitrout}\tiny
\definecolor{shadecolor}{rgb}{0.878, 0.918, 0.933}\color{fgcolor}\begin{kframe}
\begin{alltt}
\hldef{zi} \hlkwb{<-} \hlkwd{matrix}\hldef{(}\hlnum{NA}\hldef{, n, T)}
\hlkwa{for}\hldef{(i} \hlkwa{in} \hlnum{1}\hlopt{:}\hldef{n) \{}
    \hldef{zi[i,(first[i]}\hlopt{+}\hlnum{1}\hldef{)}\hlopt{:}\hldef{T]} \hlkwb{<-} \hlnum{1}
    \hldef{\}}
\hldef{ji.sp.move} \hlkwb{<-} \hlkwa{function}\hldef{()} \hlkwd{list}\hldef{(}\hlkwc{phi}\hldef{=}\hlkwd{runif}\hldef{(}\hlnum{1}\hldef{),} \hlkwc{z}\hldef{=zi,} \hlkwc{s}\hldef{=s)} \hlcom{## Cheating with s}
\hldef{jp.sp.move} \hlkwb{<-} \hlkwd{c}\hldef{(}\hlsng{"phi"}\hldef{,} \hlsng{"p0"}\hldef{,} \hlsng{"sigma"}\hldef{,} \hlsng{"tau"}\hldef{)}
\hldef{good1} \hlkwb{<-} \hlkwd{which.max}\hldef{(}\hlkwd{rowSums}\hldef{(z,} \hlkwc{na.rm}\hldef{=}\hlnum{TRUE}\hldef{)}\hlopt{+}\hlkwd{rowSums}\hldef{(y.move,} \hlkwc{na.rm}\hldef{=}\hlnum{TRUE}\hldef{))}
\hldef{time1} \hlkwb{<-} \hldef{first[good1]}\hlopt{:}\hldef{T}
\hldef{jp.s1x} \hlkwb{<-} \hlkwd{paste0}\hldef{(}\hlsng{"s["}\hldef{, good1,} \hlsng{",1,"}\hldef{, time1,} \hlsng{"]"}\hldef{)}
\hldef{jp.s1y} \hlkwb{<-} \hlkwd{paste0}\hldef{(}\hlsng{"s["}\hldef{, good1,} \hlsng{",2,"}\hldef{, time1,} \hlsng{"]"}\hldef{)}
\hldef{jp.z1} \hlkwb{<-} \hlkwd{paste0}\hldef{(}\hlsng{"z["}\hldef{, good1,} \hlsng{","}\hldef{, time1,} \hlsng{"]"}\hldef{)}
\end{alltt}
\end{kframe}
\end{knitrout}
  \pause
  \vfill
  Fit the model
\begin{knitrout}\tiny
\definecolor{shadecolor}{rgb}{0.878, 0.918, 0.933}\color{fgcolor}\begin{kframe}
\begin{alltt}
\hlstd{jps.sp.move} \hlkwb{<-} \hlkwd{jags.basic}\hlstd{(}\hlkwc{data}\hlstd{=jd.sp.move,}
                          \hlkwc{inits}\hlstd{=ji.sp.move,}
                          \hlkwc{parameters.to.save}\hlstd{=}
                              \hlkwd{c}\hlstd{(jp.sp.move, jp.s1x, jp.s1y, jp.z1),}
                          \hlkwc{model.file}\hlstd{=}\hlstr{"CJS-spatial-move.jag"}\hlstd{,}
                          \hlkwc{n.chains}\hlstd{=}\hlnum{3}\hlstd{,} \hlkwc{n.adapt}\hlstd{=}\hlnum{100}\hlstd{,} \hlkwc{n.burnin}\hlstd{=}\hlnum{0}\hlstd{,}
                          \hlkwc{n.iter}\hlstd{=}\hlnum{2000}\hlstd{,} \hlkwc{parallel}\hlstd{=}\hlnum{TRUE}\hlstd{)}
\end{alltt}
\end{kframe}
\end{knitrout}
\end{frame}



\begin{frame}[fragile]
  \frametitle{Posterior distributions}
\begin{knitrout}\scriptsize
\definecolor{shadecolor}{rgb}{0.878, 0.918, 0.933}\color{fgcolor}\begin{kframe}
\begin{alltt}
\hlkwd{plot}\hldef{(jps.sp.move[,jp.sp.move])}
\end{alltt}
\end{kframe}

{\centering \includegraphics[width=0.6\linewidth]{figure/jc3-plot-1} 

}


\end{knitrout}
\end{frame}



\begin{frame}[fragile]
  \frametitle{Posterior movement path}
\begin{knitrout}\scriptsize
\definecolor{shadecolor}{rgb}{0.878, 0.918, 0.933}\color{fgcolor}

{\centering \includegraphics[width=0.8\linewidth]{figure/jc3-s-1} 

}


\end{knitrout}
\end{frame}





\begin{frame}
  \frametitle{Summary}
  \large
%  Key points
%  \begin{itemize}[<+->]
%  \item
  Spatial CJS models allow for inference about survival,
  movement, and capture probability. \\
  \pause \vfill
  % \item
  We could have considered many other movement models. \\
  \pause \vfill
  % \item
  Next, we'll use Jolly-Seber models to study spatio-temporal
  variation in abundance and recruitment. \\
%  \end{itemize}
\end{frame}



% \begin{frame}
%   \frametitle{Assignment}
%   {\bf \large For next week}
%   \begin{enumerate}[\bf (1)]
%     \item Modify {\tt CJS-spatial.jag} to estimate yearly survival
%     \item Work on analysis of your own data
%     \item Prepare a 2-min presentation on your planned analysis.
%     \item Paper should be a minimum of 4 pages, single-spaced, 12-pt
%       font, including:
%       \begin{itemize}
%         \item Introduction
%         \item Methods (including model description)
%         \item Results (with figures)
%         \item Discussion
%       \end{itemize}
%   \end{enumerate}
% \end{frame}


\end{document}








