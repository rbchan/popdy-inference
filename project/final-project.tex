\documentclass[12pt]{article}

\usepackage[vmargin=1in,hmargin=1in]{geometry}
\usepackage{setspace}
\usepackage{parskip}

\title{Final Project Guidelines \\ Estimation of Fish and Wildlife Population Parameters \\ WILD 8390}
\date{}


\begin{document}

%\vspace{-72pt}

\maketitle

\vspace{-48pt}

\section*{Overview}

Students will write a short (3--5 page) scientific paper supported by
an analysis of either real or simulated data using one of the models
covered during this semester. The paper will be submitted
along with a self-contained R script (or .Rmd file), and data.


\section*{Paper structure}

\subsubsection*{\it Abstract \normalfont (1 paragraph, $<$500 words)}
\vspace{-6pt}
Briefly describe the motivation, methods, results, and
implications. %Explain how the study can inform conservation efforts.

\subsubsection*{\it Introduction \normalfont ($>$2 paragraphs)}
\vspace{-6pt}
Describe motivation for the study and background information. Cite
prior research. 

\subsubsection*{\it Methods \normalfont ($>$2 paragraphs)}
\vspace{-6pt}
Describe study area, study species, study design, data collection, and
data analysis. Emphasize the model description and include equations.

\subsubsection*{\it Results \normalfont ($>$1 paragraph)}
\vspace{-6pt}
Describe key findings, and support them with references to figures and tables.

\subsubsection*{\it Discussion \normalfont ($>$1 paragraph)}
\vspace{-6pt}
Explain how your results advance knowledge and inform conservation
efforts. Connect your results to previously published work.

\subsubsection*{\it Literature Cited \normalfont ($>$3 citations)}
\vspace{-6pt}
Use any peer-reviewed journal format you like, and be consistent.

\subsubsection*{\it Tables and Figures}
\vspace{-6pt}
Include at least 1 table and 2 figures, along with
clear, informative captions. 


% \begin{enumerate}
%   \item Abstract
%   \begin{itemize}
%     \item 1 paragraph ($<$500 words)
%   \end{itemize}
%   \item Introduction
%   \begin{itemize}
%     \item $\ge 3$ paragraphs
%   \end{itemize}
%   \item Methods
%   \begin{itemize}
%     \item $\ge 3$ paragraphs
%     \item Emphasize the model description and include equations
% %    \item Include equations
%   \end{itemize}
%   \item Results
%   \begin{itemize}
%     \item $\ge 2$ paragraphs
%     \item Reference tables and figures
%   \end{itemize}
%   \item Discussion
%   \begin{itemize}
%     \item $\ge 2$ paragraphs
%   \end{itemize}
%   \item Literature Cited
%   \begin{itemize}
%     \item $\ge 3$ citations
%   \end{itemize}
%   \item Tables and figures
%   \begin{itemize}
%     \item $\ge 1$ table
%     \item $\ge 2$ figures
%   \end{itemize}
% %  \item Figures
% %  \begin{itemize}
% %    \item $\ge 2$ figures
% %  \end{itemize}
% \end{enumerate}


\section*{Timeline}

\begin{itemize}
  \item Oct 12: Email instructors with brief (1 paragraph) description
    of project idea
  \item Nov 17: Submit first draft
  \item Nov 24: Submit peer review
  \item Nov 1 \& Dec 3: Presentations
  \item Dec 8: Submit final paper, code, and data
\end{itemize}


\section*{Grading}

The project comprises 30\% of your overall grade, and will be graded as follows:

\begin{itemize}
  \item First draft: 10\%
    \begin{itemize}
      \item No Results or Discussion required
      \item The analysis does not need to be complete, but you should make substantial progress.
    \end{itemize}
  \item Peer review: 10\%
    \begin{itemize}
      \item Write your review as you would for a peer-reviewed journal.
    \end{itemize}
  \item Final paper: 60\%
    \begin{itemize}
      \item Will be graded based on criteria listed above under
        ``Paper structure''. It should also incorporate suggestions from
        peer-review and from the presentations.
    \end{itemize}
  \item Analysis: 20\%
    \begin{itemize}
      \item R code should be easy to follow and annotated
      \item We should be able to reproduce results reported in the paper
      \item Script should not throw errors
    \end{itemize}
\end{itemize}

The 10-min presentation counts towards an additional 20\% of your
overall grade. The presentation should include background information,
methods, results, and conclusions.   


\end{document}
