\documentclass[12pt]{article}

\usepackage[vmargin=1in,hmargin=1in]{geometry}
\usepackage{setspace}
\usepackage{parskip}
\usepackage[hidelinks=true]{hyperref}
\usepackage{xcolor}

\title{Final Project Guidelines \\ %Estimation of Fish and Wildlife Population Parameters
  Inference for Models of Fish and Wildlife \\ Population Dynamics \\ WILD 8390}
\date{}


\begin{document}

\maketitle

\vspace{-48pt}

\section*{Overview}

Each student will write a short (3--5 page) scientific paper supported by
an analysis of either real or simulated data using one of the models
covered during this semester. The paper will be submitted
along with a self-contained R script (or .Rmd file), and data.


\section*{Paper structure}

Abstract (1 paragraph, $<$500 words) \\
Briefly describe the motivation, methods, results, and
implications. %Explain how the study can inform conservation efforts.
\vspace{6pt}

Introduction ($>$2 paragraphs) \\
Describe motivation for the study and background information. Cite
prior research. 
\vspace{6pt}

Methods  ($>$2 paragraphs) \\
Describe study area, study species, study design, data collection, and
data analysis. Emphasize the model description and include equations.
\vspace{6pt}

Results  ($>$1 paragraph) \\
Describe key findings, and support them with references to figures and tables.
\vspace{6pt}

Discussion  ($>$1 paragraph) \\
Explain how your results advance knowledge and inform conservation
efforts. Connect your results to previously published work.
\vspace{6pt}

Literature Cited  ($>$3 citations) \\
Use any peer-reviewed journal format you like, and be consistent.

Tables and Figures \\
Include at least 1 table and 2 figures, along with
clear, informative captions. 


\section*{Timeline}

\begin{itemize}
  \item Sept 26: Email instructors with brief (1 paragraph) description
    of project idea
  \item Oct 28: Submit first draft (later than the syllabus says)
  \item Nov 5: Submit peer review
  \item Nov 21 \& Nov 26: Presentations
  \item Dec 3: Submit final paper, code, and data
\end{itemize}


\section*{Grading}

The project comprises 30\% of your overall grade, and will be graded as follows:

\begin{itemize}
  \item First draft: 10\%
    \begin{itemize}
      \item No Results or Discussion required, but you must submit an R script.
      \item The analysis does not need to be complete, but you should make substantial progress.
    \end{itemize}
  \item Peer review: 10\%
    \begin{itemize}
      \item Write your review as you would for a peer-reviewed journal. Good advice can be found \textcolor{blue}{\href{https://www.britishecologicalsociety.org/wp-content/uploads/2017/06/BES-Peer-Review-Guide-2017_web.pdf}{here}} and \textcolor{blue}{\href{https://www.nature.com/articles/d41586-018-06991-0}{here}}.
      \item Each of your two reviews should be 1-2 pages, including sections for ``general comments'' and ``specific comments.''
    \end{itemize}
  \item Final paper: 60\%
    \begin{itemize}
      \item Will be graded based on criteria listed above under
        ``Paper structure''. It should also incorporate suggestions from
        peer-review and from the presentations.
    \end{itemize}
  \item Analysis: 20\%
    \begin{itemize}
      \item R code should be easy to follow and annotated
      \item We should be able to reproduce results reported in the paper
      \item Script should not throw errors
      \item Script should not include lengthy data formatting code. Format the data elsewhere, and then import it ready for analysis.
    \end{itemize}
\end{itemize}

The 10-min presentation counts towards an additional 20\% of your
overall grade. The presentation should include background information,
methods, results, and conclusions.   


\end{document}
