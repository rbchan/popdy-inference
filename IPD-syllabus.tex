%% UGA guidelines:
%% https://curriculumsystems.uga.edu/curriculum/courses/syllabus/

\documentclass[12pt]{article}

\usepackage[top=2.5cm,left=2.5cm,right=2.5cm,bottom=2cm]{geometry}
\usepackage[pdftex,hidelinks,pdfstartview={Fit}]{hyperref}
\usepackage{parskip}
\usepackage{setspace}
\usepackage{tocloft}
\usepackage{verbatim}
\usepackage{xcolor}
 

\begin{document}


%\begin{center}

{\centering

{\Large \bf \sc
  Estimation of Fish and Wildlife \\ Population Parameters \\
}
\vspace{12pt}

WILD(FISH) 8390  \\
Lecture: Tues, Thurs 12:45--2:00 PM \\
Room 1057-C130 Life Sciences \\


\normalsize

\vspace{24pt}

\begin{tabular}[h!]{ccc}
\textbf{Instructor}                 & \hspace{0.01cm} & \textbf{Teaching Assistant} \\
Dr. Richard Chandler                & & Heather Gaya\\
% Office: 3-409-B                     & & Office: 3-402& Office: 4-306B   \\
%Phone: 706-542-5818                 & & --- \\
email: rchandler@warnell.uga.edu    & & Heather.Gaya@uga.edu \\
Virtual office hours: Mon 10:00-11:00 \&  & & Fri 10:00-11:00 \\
   Thurs 2:00--3:00 & & \\
\end{tabular}


}


%\vspace{0.1cm}

\normalsize


\vspace{-2mm}
\section*{\normalsize Course Description}
\vspace{-4mm}
Classical courses on population dynamics focus on theoretical models 
of how populations should behave under a set of assumptions.
Evaluating how actual populations behave requires confronting models
with data. Recently developed statistical tools make it possible to fit
models of population dynamics to field data while accounting for
observation error. This course will explain how to use these tools for
statistical inference and prediction.  


\vspace{-2mm}
\section*{\normalsize Course Objectives and Learning Outcomes}
\vspace{-4mm}
By the end of the course, students will know how to (1) design studies
of wildlife population dynamics, (2) fit statistical
models to data collected in the field, (3) interpret results, and (4)
make predictions useful for informing conservation decisions.  


\vspace{-2mm}
\section*{\normalsize Textbooks}
\vspace{-4mm}

\begin{itemize}
  \item (AHM) K\'ery, M. and J.A. Royle. 2016. Applied Hierarchical Modeling
    in Ecology: Analysis of Distribution, Abundance and Species Richness
    in R and BUGS : Volume 1: Prelude and Static Models. Academic Press.
    Download chapters of the eBook from the UGA library
        \href{http://eds.b.ebscohost.com/eds/detail/detail?vid=1&sid=23324ccb-5b1f-4eb6-b8b7-005232b9b48f%40pdc-v-sessmgr05&bdata=JnNpdGU9ZWRzLWxpdmU%3d#AN=1098832&db=nlebk}{\textcolor{blue}{here}}.
  \item (SCR) Royle, J.A., R.B. Chandler, R. Sollmann, and
    B. Gardner. 2013. Spatial Capture-Recapture. Academic Press.
    Download chapters of the eBook from the UGA library
        \href{http://eds.b.ebscohost.com/eds/detail/detail?vid=1&sid=44b8c9d7-8190-4990-8419-610cbd7b7fa9%40pdc-v-sessmgr01&bdata=JnNpdGU9ZWRzLWxpdmU%3d#AN=501310&db=nlebk}{\textcolor{blue}{here}}.
%    \end{itemize}
\end{itemize}

  

\vspace{-2mm}
\section*{\normalsize Grading}
\vspace{-4mm}
\begin{center}
%  \small
  \begin{tabular}[h!]{lrr}
    \hline
                           & Quantity & Grade percentage         \\
    \hline
    Weekly assignments$^*$ & 15       & 45\%                     \\
    Final presentation     & 1        & 20\%                     \\
    Final paper$^{*}$      & 1        & 30\%                     \\
    Class participation    &          & 5\%                      \\
    \hline
  \end{tabular}                                                  \\
  \small
\hspace{0mm} $^*$Late assignments will be penalized 3 points/day \\ 
\end{center}
The plus/minus grading system will be used, according to UGA policy,
and assigned following this plus/minus grading scale: A = 93-100, A- =
90-92.9, B+ = 87-89.9, B = 83-86.9, B- = 80-82.9, C+ = 77-79.9, C =
73-76.9, C-= 70-72.9, D = 60-69.9, F = $<$60. 


\vspace{-2mm}
\section*{\normalsize Attendance}
\vspace{-4mm}
Attendance is optional, but it will be impossible to succeed in the
class if you don't participate in either in-person or online
lectures. 

\vspace{-2mm}
\section*{\normalsize COVID-19}
\vspace{-4mm}
Per USG guidelines, students and instructors are required to wear a
face mask while inside instructional buildings. Face covering use is
in addition to, and is not a substitute for, appropriate
distancing. Anyone not wearing a face covering when required will be
asked to wear one or must leave the area. Repeated refusal to comply
with the requirement may result in discipline through the applicable
conduct code for faculty, staff or students.

Reasonable accommodations may be made (through DRC) for those who are
unable to wear a face covering for documented health reasons. 

Prior to Thanksgiving Break, lectures and labs will be delivered
in-person, and they will be recorded and posted online. After
Thanksgiving Break, all lectures and labs will be delivered online. 

Online (remote) learning will involve both synchronous and
asynchronous activities.

%{\it DawgCheck}

Please perform a quick symptom check each weekday on DawgCheck—on the
UGA app or website—whether you feel sick or not. It will help health
providers monitor the health situation on campus:
\url{https://dawgcheck.uga.edu/}

%{\it What do I do if I have symptoms?}

Students showing symptoms should self-isolate and schedule an
appointment with the University Health Center by calling 706-542-1162
(Monday-Friday, 8 a.m.-5 p.m.). Please DO NOT walk-in. For emergencies
and after-hours care, see
\url{https://www.uhs.uga.edu/info/emergencies}.  



\vspace{-2mm}
\section*{\normalsize Academic Honesty}
\vspace{-4mm}
As a University of Georgia student, you have agreed to abide by the
UGA academic honesty policy. UGA Student Honor code: “I will be
academically honest in all of my academic work and will not tolerate
academic dishonesty of others”. A Culture of Honesty, the University's
policy and procedures for handling cases of suspected dishonesty, can
be found at \url{https://honesty.uga.edu/}. You are responsible for
informing yourself about the university’s standards before performing
any academic work. Lack of knowledge of the academic honesty policy is
not a reasonable explanation for a violation. Please ask if you have
questions related to course assignments and the academic honesty
policy. Any form of possible academic dishonesty will be reported to
the UGA Office of the Vice President for Instruction.


\vspace{-2mm}
\section*{\normalsize Cell Phones and Laptops}
\vspace{-4mm}
Cell phones are not allowed during class unless explicit permission is
granted. Laptop computers should be brought to class for daily
exercises.


%\newpage

\section*{\normalsize Tentative Course Outline}
\vspace{-6mm}

\begin{center}
\begin{tabular}[c]{lll}
\hline \hline
{\bf Date} & {\bf Lecture}                                & {\bf Reading}                    \\
\hline
           \multicolumn{3}{c}{PART 1 -- BACKGROUND}                                          \\
\hline
Aug 20     & Introduction                                 &                                  \\
\hline
Aug 25     & Classical models of population dynamics      & AHM Foreward, Preface, Chapter 1     \\
Aug 27     & Classical statistical models                 & AHM Chapters 2-3                 \\
\hline
           \multicolumn{3}{c}{PART 2 -- STATIC MODELS}                                      \\
           \multicolumn{3}{c}{\it PART 2a -- Spatially-referenced, temporally replicated binary data}             \\
\hline
Sept 1     & Occupancy models -- simulation               & AHM Ch 4 and Ch 10 (pp 551--564) \\
Sept 3     & Occupancy models -- fitting                  &                                  \\
\hline
Sept 8     & Occupancy models -- covariates               & AHM Chapter 10 (pp 564--590)     \\
Sept 10    & Occupancy models -- predication              & AHM Chapter 10 (pp 591--600)     \\
\hline
           \multicolumn{3}{c}{\it PART 2b -- Spatially-referenced, temporally replicated count data}             \\
\hline
Sept 15    & Binomial $N$-mixture models -- simulation    & AHM Chapter 6 (pp 219--228)      \\
Sept 17    & Binomial $N$-mixture models -- fitting       & AHM Chapter 6 (pp 229--245)      \\
\hline
Sept 22    & Binomial $N$-mixture models -- GoF           & AHM Chapter 6 (pp 245--254)      \\
Sept 24    & Binomial $N$-mixture models -- prediction    & AHM Chapter 6 (pp 254--282)      \\
\hline
Sept 29    & Binomial $N$-mixture models -- extensions    & AHM Chapter 6 (pp 282--312)      \\
\hline
           \multicolumn{3}{c}{\it PART 2c -- Spatially-referenced, aggregated mark-recapture data} \\
\hline
Oct 1      & Multinomial $N$-mixture models -- simulation & AHM Chapter 7 (pp 313--323)      \\
\hline
Oct 6      & Multinomial $N$-mixture models -- fitting    & AHM Chapter 7 (pp 323--349)      \\
Oct 8      & Multinomial $N$-mixture models -- fitting    & AHM Chapter 7 (pp 349--366)      \\
\hline
Oct 13     & Multinomial $N$-mixture models -- prediction & AHM Chapter 7 (pp 367--392)      \\
\hline
           \multicolumn{3}{c}{\it PART 2d -- Distance sampling and mark-recapture data} \\
\hline
Oct 15     & Distance sampling -- conventional            & AHM Chapter 8 (pp 393--426)      \\
\hline
Oct 20     & Distance sampling -- hierarchical            & AHM Chapter 8 (pp 426--461)      \\
Oct 22     & Non-spatial capture-recapture                & SCR Chapters 1 and 4             \\
\hline
Oct 27     & Spatial capture-recapture                    & SCR Chapter 5                    \\
Oct 29     & Spatial capture-recapture                    & SCR Chapters 6 and 7             \\
\hline
           \multicolumn{3}{c}{PART 3 -- DYNAMIC MODELS}                                    \\
           \multicolumn{3}{c}{\it PART 3a -- Modeling local colonization and extinction} \\
\hline
Nov 3      & Dynamic occupancy models                     & TBD                              \\
Nov 5      & Dynamic occupancy models                     & TBD                              \\
\hline
           \multicolumn{3}{c}{\it PART 3b -- Modeling local recruitment and apparent survival} \\
\hline
Nov 10     & Dynamic $N$-mixture models                   & Due: first draft                 \\
Nov 12     & Dynamic $N$-mixture models                   & TBD                              \\
\hline
           \multicolumn{3}{c}{\it PART 3c -- Individual-level recruitment, survival, and movement} \\
\hline
Nov 17     & Dynamic capture-recapture models             & TBD                              \\
Nov 19     & Dynamic spatial capture-recapture models     & SCR Chapter 5                    \\
\hline
           \multicolumn{3}{c}{PART 4 -- STUDENT PRESENTATIONS}                                    \\
\hline
Nov 24     & Student presentations                        & Due: peer review                 \\
Nov 26     & Thanksgiving Break                           &                                  \\
\hline
Dec 1      & Student presentations                        &                                  \\
Dec 3      & Student presentations                        &                                  \\
\hline
Dec 8      & No class (Friday schedule)                   & Due: Final paper                 \\
% \hline
% Dec 17     & Final exam (12:00-3:00)                      &                                  \\
\hline \hline
\end{tabular}
\end{center}


\clearpage


\vspace{-2mm}
\section*{\normalsize Disclaimer}
\vspace{-4mm}

The course syllabus is a general plan for the course; deviations
announced to the class by the instructor may be necessary. 


\vspace{-2mm}
\section*{\normalsize Academic Coaching}
\vspace{-4mm}

Assistance with time management, test and performance anxiety,
notetaking, motivation, text comprehension, test preparation, and
other barriers to success at UGA. Link for the Office of Academic
Enhancement (\url{https://dae.uga.edu/services/academic-coaching/}). 

\vspace{-2mm}
\section*{\normalsize Accommodations for Disabilities}
\vspace{-4mm}

If you require a disability-required accommodation, it is essential
that you register with the Disability Resource Center (Clark Howell
Hall; \url{https://drc.uga.edu}; 706-542-8719 [voice]; 706-542-8778 [TTY])
and notify me of your eligibility for reasonable accommodations. We
can then plan how best to coordinate your accommodations. Please note
that accommodations cannot be provided retroactively.



\vspace{-2mm}
\section*{\normalsize Mental Health and Wellness Resources}
\vspace{-4mm}

\begin{itemize}
  \item If you or someone you know needs assistance, you are
    encouraged to contact Student Care and Outreach in the Division of
    Student Affairs at 706-542-7774 or visit \url{https://sco.uga.edu}. They
    will help you navigate any difficult circumstances you may be facing
    by connecting you with the appropriate resources or services.
  \item UGA has several resources for a student seeking mental health
    services (\url{https://www.uhs.uga.edu/bewelluga/bewelluga}) or crisis
    support (\url{https://www.uhs.uga.edu/info/emergencies}).
  \item If you need help managing stress anxiety, relationships, etc.,
    please visit BeWellUGA (\url{https://www.uhs.uga.edu/bewelluga/bewelluga})
    for a list of FREE workshops, classes, mentoring, and health
    coaching led by licensed clinicians and health educators in the
    University Health Center.
  \item Additional resources can be accessed through the UGA App.
\end{itemize}




\vspace{-2mm}
\section*{\normalsize FERPA Notice}
\vspace{-4mm}

The Federal Family Educational Rights and Privacy Act (FERPA) grants
students certain information privacy rights. To comply with FERPA, all
communication that refers to individual students must be through a
secure medium (UGAMail or eLC) or in person. Instructors are not
allowed to respond to messages that refer to individual students or
student progress in the course through non-UGA accounts, phone calls,
or other types of electronic media. For details, please visit
\url{https://apps.reg.uga.edu/FERPA}. 


\end{document}
