%% UGA guidelines:
%% https://curriculumsystems.uga.edu/curriculum/courses/syllabus/

\documentclass[12pt]{article}

\usepackage[top=2.5cm,left=2.5cm,right=2.5cm,bottom=2cm]{geometry}
\usepackage[pdftex,hidelinks=true,pdfstartview={Fit}]{hyperref}
\usepackage{parskip}
\usepackage{setspace}
\usepackage{tocloft}
\usepackage{verbatim}
\usepackage{xcolor}
 

\begin{document}



{\centering

{\Large \sc
  % Estimation of Fish and Wildlife \\ Population Parameters \\
  Inference for Models of Fish and Wildlife \\ Population Dynamics  
}
\vspace{6pt}

WILD(FISH) 8390  \\
Lecture: Tues, Thurs 12:45--2:00 PM, Fall 2025 \\
Room 4-516, Warnell \\


\normalsize

\vspace{12pt}

\begin{tabular}[h!]{lccc}
                     & \textbf{Instructor}       & \hspace{0.01cm} & \textbf{Teaching Assistant} \\
                     & Dr. Richard Chandler      &                 & Jen Brown                \\
Office room:         & 3-407                     &                 & 4-219                       \\
%Phone: 706-542-5818 &                           & ---                                           \\
Email:               & rchandler@warnell.uga.edu &                 & jennifer.brown1@uga.edu       \\
Office hours:        & Thur 2:00--3:00           &                 & Fri 11:30--1:30             \\
\end{tabular}


}


%\vspace{0.1cm}

\normalsize


\vspace{-2mm}
\section*{\normalsize Course Description}
\vspace{-4mm}
Recently developed statistical tools make it possible to fit models of
population dynamics to field data while accounting for observation
error. This course will explain how to use these tools for statistical
inference and prediction about animal population parameters including
abundance, occupancy, survival, recruitment, and dispersal.   


\vspace{-2mm}
\section*{\normalsize Course Objectives and Learning Outcomes}
\vspace{-4mm}
By the end of the course, students will know how to (1) design studies
of wildlife population dynamics, (2) fit statistical
models to data collected in the field, (3) interpret results, and (4)
make predictions useful for informing conservation decisions.  


\vspace{-2mm}
\section*{\normalsize Textbooks}
\vspace{-4mm}

\begin{itemize}
  \setlength\itemsep{-6pt}
  \item (AHM1) K\'ery, M. and J.A. Royle. 2016. Applied Hierarchical Modeling
    in Ecology: Analysis of Distribution, Abundance and Species Richness
    in R and BUGS : Volume 1: Prelude and Static Models. Academic Press.
    \href{https://www.sciencedirect.com/book/9780128013786/applied-hierarchical-modeling-in-ecology}{\textcolor{blue}{eBook}}.
  \item (AHM2) K\'ery, M. and J.A. Royle. 2020. Applied Hierarchical Modeling
    in Ecology: Analysis of Distribution, Abundance and Species Richness
    in R and BUGS : Volume 2: Dynamic and Advanced Models. Academic Press.
    \href{https://www.sciencedirect.com/book/9780128237687/applied-hierarchical-modeling-in-ecology-analysis-of-distribution-abundance-and-species-richness-in-r-and-bugs}{\textcolor{blue}{eBook}}.
  \item (SCR) Royle, J.A., R.B. Chandler, R. Sollmann, and
    B. Gardner. 2013. Spatial Capture-Recapture. Academic Press.
    Download chapters of the eBook from the UGA library
    \href{https://www-sciencedirect-com.proxy-remote.galib.uga.edu/book/9780124059399/spatial-capture-recapture}{\textcolor{blue}{eBook}}.
\end{itemize}

  

\vspace{-2mm}
\section*{\normalsize Grading}
\vspace{-4mm}
\begin{center}
%  \small
  \begin{tabular}[h!]{lrr}
    \hline
                           & Quantity & Grade percentage         \\
    \hline
    Weekly assignments$^*$ & 15       & 45\%                     \\
    Final presentation     & 1        & 20\%                     \\
    Final paper$^{*}$      & 1        & 30\%                     \\
    Class participation    &          & 5\%                      \\
    \hline
  \end{tabular}                                                  \\
  \small
\hspace{0mm} $^*$Late assignments will be penalized 3 points/day, up to a maximum of 50 points off.  \\ 
\end{center}
The plus/minus grading system will be used, according to UGA policy,
and assigned following this plus/minus grading scale: A = $>$93-100,
A- = $>$90-93, B+ = $>$87-90, B = $>$83-87, B- = $>$80-83, C+ = $>$77-80,
C = $>$73-77, C-= $>$70-73, D = 60-70, F = $<$60. 


%\vspace{-2mm}
\section*{\normalsize Attendance}
%\vspace{-4mm}
Attendance is optional, but it will be difficult to succeed if you don't participate in lectures. 



%\vspace{-2mm}
\section*{\normalsize Academic Honesty}
%\vspace{-4mm}
As a University of Georgia student, you have agreed to abide by the
UGA academic honesty policy. UGA Student Honor code: ``I will be
academically honest in all of my academic work and will not tolerate
academic dishonesty of others''. A Culture of Honesty, the University's
policy and procedures for handling cases of suspected dishonesty, can
be found at \href{https://honesty.uga.edu/}{here}. You are responsible for
informing yourself about the university’s standards before performing
any academic work. Lack of knowledge of the academic honesty policy is
not a reasonable explanation for a violation. Please ask if you have
questions related to course assignments and the academic honesty
policy. Any form of possible academic dishonesty will be reported to
the UGA Office of the Vice President for Instruction.



%\vspace{-2mm}
\section*{\normalsize Generative AI}
%\vspace{-4mm}

Artificial intelligence (AI) tools, like chatGPT and other generative AI
chatbots, can be useful for improving text and computer code. However,
they can also hinder learning if used to generate answers and bypass
critical thinking. Students must disclose the use of generative AI for
completing assignments, including the final paper. If a
student decides to use generative AI, it can only be used to improve
text and code written by the student. AI tools cannot be used to create
answers by inputting assignment questions into prompts. Students
must report the prompts provided to AI tools, along with the original
text (or code) and the final text. Prompts and output can be disclosed
by providing a link to the AI chat.



%\vspace{-2mm}
\section*{\normalsize Academic Coaching}
%\vspace{-4mm}

Assistance with time management, test and performance anxiety,
notetaking, motivation, text comprehension, test preparation, and
other barriers to success at
UGA. \href{https://dae.uga.edu/services/academic-coaching/}{\color{blue}
  Link} for the Office of Academic Enhancement. 

%\vspace{-2mm}
\section*{\normalsize Accommodations for Disabilities}
%\vspace{-4mm}

If you require a disability-required accommodation, it is essential
that you register with the \href{https://drc.uga.edu}{Disability Resource Center} (Clark Howell
Hall; 706-542-8719)
and notify me of your eligibility for reasonable accommodations. We
can then plan how best to coordinate your accommodations. Please note
that accommodations cannot be provided retroactively.


\clearpage

% \vspace{-2mm}
% \section*{\normalsize Mental Health and Wellness Resources}
% \vspace{-4mm}

% \begin{itemize}
%   \setlength\itemsep{-6pt}
%   \item If you or someone you know needs assistance, you are
%     encouraged to contact Student Care and Outreach in the Division of
%     Student Affairs at 706-542-7774 or visit
%     \href{https://sco.uga.edu}{their website}. They
%     will help you navigate any difficult circumstances you may be facing
%     by connecting you with the appropriate resources or services.
%   \item UGA has several resources for a student seeking mental health
%     \href{https://www.uhs.uga.edu/bewelluga/bewelluga}{services} or
%     \href{https://www.uhs.uga.edu/info/emergencies}{crisis support}.
%   \item If you need help managing stress anxiety, relationships, etc.,
%     please visit \href{https://www.uhs.uga.edu/bewelluga/bewelluga}{BeWellUGA}
%     for a list of FREE workshops, classes, mentoring, and health
%     coaching led by licensed clinicians and health educators in the
%     University Health Center.
%   \item Additional resources can be accessed through the UGA App.
% \end{itemize}

\section*{\normalsize UGA Well-Being Resources}
%\vspace{-4mm}

UGA Well-being Resources promote student success by cultivating a
culture that supports a more active, healthy, and engaged student
community. Anyone needing assistance is encouraged to contact Student Care \&
Outreach (SCO) in the Division of Student Affairs at 706-542-8479 or
visit \url{sco.uga.edu}. Student Care \& Outreach helps students
navigate difficult circumstances by connecting them with the most
appropriate resources or services. They also administer the Embark@UGA
program which supports students experiencing, or who have experienced,
homelessness, foster care, or housing insecurity. 

UGA provides both clinical and non-clinical options to support student
well-being and mental health, any time, any place. Whether on campus,
or studying from home or abroad, UGA Well-being Resources are here to
help. 

% Well-being Resources: \url{well-being.uga.edu}

Student Care and Outreach: \url{sco.uga.edu}

University Health Center: \url{healthcenter.uga.edu}

Counseling \& Psychiatric Services: \url{caps.uga.edu} or CAPS 24/7
crisis support 706-542-2273 

Health Promotion/ Fontaine Center: \url{healthpromotion.uga.edu}

Accessibility and Testing: \url{accessibility.uga.edu}

Additional information, including free digital well-being resources,
can be accessed through the UGA app or by visiting
\url{https://well-being.uga.edu}. 

\clearpage

\section*{\normalsize Tentative Course Outline}
\vspace{-6mm}

\begin{center}
\begin{tabular}[c]{lll}
\hline \hline
{\bf Date} & {\bf Lecture}                                & {\bf Reading}                                         \\
\hline
           \multicolumn{3}{c}{PART 1 -- BACKGROUND}                                                               \\
\hline
Aug 14     & Introduction                                 &                                                       \\
\hline
%Aug 24    & Classical models of population dynamics      & AHM1 Foreword, Preface, Chapter 1                     \\
Aug 19     & Linear models                                & AHM1 Foreword, Preface, Chapter 1                     \\
Aug 21     & Generalized linear models                    & AHM1 Chapter 3                                        \\
\hline
           \multicolumn{3}{c}{PART 2 -- STATIC MODELS}                                                            \\
\hline
Aug 26     & Occupancy models                             & AHM1 Ch 4 and Ch 10 (pp 551--564)                     \\
Aug 28     & Occupancy models                             &                                                       \\
\hline
Sept 2     & Occupancy models                             & AHM1 Chapter 10 (pp 564--600)                         \\
Sept 4     & Occupancy models                             & AHM1 Chapter 10 (pp 591--600)                         \\
\hline
Sept 9    & Binomial $N$-mixture models                  & AHM1 Chapter 6 (pp 219--228)                          \\
Sept 11    & Prior predictive checks                      & AHM1 Chapter 6 (pp 229--245)                          \\
\hline
Sept 16    & Model selection                              & AHM1 Chapter 6 (pp 245--254)                          \\
Sept 18    & Goodness-of-fit                              & AHM1 Chapter 6 (pp 254--312)                          \\
\hline
Sept 23    & Multinomial $N$-mixture models               & AHM1 Chapter 7 (pp 313--334)                          \\
Sept 25    & Multinomial $N$-mixture models               & AHM1 Chapter 7 (pp 334--391)                          \\
\hline
Sept 30      & Hierarchical distance sampling               & AHM1 Chapter 8 (pp 426--460)                          \\
Oct 2      & Conventional distance sampling               & AHM1 Chapter 8 (pp 393--426)                          \\
\hline
Oct 7     & Non-spatial capture-recapture                & SCR Chapters 1--4                                     \\
Oct 9     & Spatial capture-recapture                    & SCR Chapter 5                                         \\
\hline
Oct 14     & Spatial capture-recapture                    & SCR Chapter 7                                         \\
Oct 16     & Spatial capture-recapture                    & SCR Chapters 9                                        \\
\hline
           \multicolumn{3}{c}{PART 3 -- DYNAMIC MODELS}                                                           \\
\hline
Oct 21     & Dynamic occupancy models                     & AHM2 Ch4 (Due: first draft)                           \\
Oct 23     & Dynamic $N$-mixture models                   & AHM2 Chapter 2                                        \\
\hline
Oct 28     & Survival models                               & AHM2 Chapter 3                                        \\
Oct 30      & Cormack-Jolly-Seber models                   & SCR Chapter 16                                        \\
\hline
Nov 4      & Jolly-Seber models                           & SCR Chapter 16 (Due: peer review)                     \\
Nov 6      & Jolly-Seber models                           & SCR Chapter 16                                        \\
\hline
Nov 11     & Dynamic spatial capture-recapture models     & SCR Chapter 16                                        \\
Nov 13     & Dynamic spatial capture-recapture models     & SCR Chapter 16                                        \\
\hline
Nov 18     & Dynamic spatial capture-recapture models     & SCR Ch 16                                             \\
Nov 20     & Student presentation                           &                                                       \\
\hline
%            \multicolumn{3}{c}{PART 4 -- STUDENT PRESENTATIONS}                                                    \\
% \hline
Nov 25     & Student presentations                        &                                                       \\
Nov 27     & Thanksgiving Break                        &                                                       \\
\hline
Dec 2      & No class (Friday schedule)                   & Due: Final paper                                      \\
% \hline
% Dec 17   & Final exam (12:00-3:00)                      &                                                       \\
\hline \hline
\end{tabular}
\end{center}

The course syllabus is a general plan for the course; deviations announced to the class by the instructor may be necessary.


\end{document}
